
\documentclass[12pt,letterpaper,reqno,openany,oneside,spanish]{book}
\usepackage[T1]{fontenc}
\usepackage[utf8]{inputenc}
\usepackage{amsmath,amssymb,latexsym,amsfonts,times,ifthen,amsthm,enumerate,euler}
\usepackage{boiboites}
\usepackage{calligra}
\usepackage[spanish,activeacute]{babel}
\usepackage{latexsym} % Símbolos
\usepackage{fancybox}
\usepackage{fancyhdrtesis}
\usepackage{showkeys}
\usepackage{shortvrb}

%\usepackage[latin1]{inputenc} % Caracteres con acentos.
%\usepackage[activeacute,spanish]{babel} % paquete para utilizar español
\usepackage{verbatim}
\usepackage{graphics}
\usepackage{multicol}
\usepackage{graphicx} % Inclusión de gráficos.
\usepackage{makeidx} %para el indice
\usepackage{colortbl}
\usepackage{color}
\usepackage{multicol}
\usepackage{graphpap}
\usepackage{losymbol} % lista de simbolos
\usepackage{Tesis}
\usepackage{titlesec}
\usepackage[mathscr]{eucal}

\numberwithin{equation}{chapter}
\renewcommand{\theequation}{\arabic{chapter}.\arabic{equation}}
% Margenes
\usepackage[text={15.5cm,20.5cm},centering]{geometry}
 % \usepackage[left=3cm,right=2.5cm]{geometry}


% Estilo de página
% http://www.fceia.unr.edu.ar/lcc/cdrom/Instalaciones/LaTex/latex.html#SECTION01020000000000000000
\pagestyle{headings}
%.....................................................................................................
\frenchspacing
%.....................................................................................................
% \newboxedtheorem[boxcolor=orange, background=blue!5, titlebackground=blue!20,titleboxcolor = black]{thm}{Teorema}{thCounter}
%\newboxedtheorem[boxcolor=orange, background=blue!5, titlebackground=blue!20,titleboxcolor = black]{defn}{Definici\'on}{thCounter}
%\newboxedtheorem[boxcolor=orange, background=blue!5, titlebackground=blue!20,titleboxcolor = black]{prop}{Proposición}{thCounter}
%\newtheorem{proposition}{ Proposici\'on  \newline}
%%\newtheorem{definition}{ Definici\'on\newline}[chapter]

%\numberwithin{equation}{chapter}

\frenchspacing % usar espaciado normal después de '.'

%<<< °°° >>> <<< *** >>> <<< °°° >>> <<< *** >>> <<< °°° >>> <<< *** >>> <<< °°° >>> <<< *** >>> <<< °°° >>>
% Definicion de los encabezamientos de t_encab.sty
\pagestyle{fancyplain}
%\pagestyle{headings}
\lhead[\fancyplain{}{\normalsize\thepage}]
          {\fancyplain{}{\normalsize\rightmark}}
\rhead[\fancyplain{}{\normalsize\leftmark}]
         {\fancyplain{}{\normalsize\thepage}} \chead{}\lfoot{}\rfoot{}\cfoot{}
\newcommand{\clearemptydoublepage}{\newpage\thispagestyle{empty}\cleardoublepage}
%....................................................................................................
\newcommand{\bigrule}{\titlerule[0.5mm]}

\titleformat{\chapter}[display] % cambiamos el formato de los capítulos
{\bfseries\Large} % por defecto se usarán caracteres de tamaño \Huge en negrita
{% contenido de la etiqueta
 \titlerule % línea horizontal
 \filleft % texto alineado a la derecha
 \Huge\chaptertitlename\ % "Capítulo" o "Apéndice" en tamaño \Large en lugar de \Huge
 \Huge\thechapter} % número de capítulo en tamaño \Large
{0mm} % espacio mínimo entre etiqueta y cuerpo
{\filleft} % texto del cuerpo alineado a la derecha
[\vspace{1mm} \bigrule] % después del cuerpo, dejar espacio vertical y trazar línea horizontal gruesa


%\definecolor{myblue}{RGB}{84, 177, 184}
%\definecolor{myblue1}{RGB}{49, 133, 156}
%\usepackage{hyperref}
%\hypersetup{
%	colorlinks   = true, %Colours links instead of ugly boxes
%	urlcolor     = myblue1, %Colour for external hyperlinks
%	linkcolor    = myblue1, %Colour of internal links
%	citecolor   = red %Colour of citations
%}
%\hypersetup{colorlinks=true,allcolors=cyan}
%\usepackage{hypcap}

%\usepackage{mathptmx}
%\renewcommand{\rmdefault}{pifont}
%\usepackage{helvet}

\hyphenation{con-ve-xi-dad}
\hyphenation{mid-con-ve-xi-dad}



\interdisplaylinepenalty=0
\renewcommand{\baselinestretch}{1.5}
\footskip 0.4in


\newcommand{\R}{\mathbb{R}}
\newcommand{\Q}{\mathbb{Q}}

\newcommand{\N}{\mathbb{N}}
\newcommand{\Z}{\mathbb{Z}}
\newcommand{\C}{\mathbb{C}}
\newcommand{\D}{\mathbb{D}}
\newcommand{\HH}{\mathbb{H}}

\renewcommand{\P}{\mathscr{P}}
\newcommand{\U}{\mathscr{U}}
\newcommand{\B}{\mathscr{B}}
\newcommand{\Tau}{\mathscr{T}}


\newcommand{\ds}{\displaystyle}
\newcommand{\pr}[1]{\left( #1\right) }
\newcommand{\ch}[1]{\left[ #1\right] }

\newcommand{\dist}{\mathop{\hbox{\rm dist}}\nolimits}
%\newcommand{\d}{\mathop{\hbox{\rm d}}\nolimits}

\newcommand{\conv}{\mathop{\hbox{\rm conv}}\nolimits}
\newcommand{\rec}{\mathop{\hbox{\rm rec}}\nolimits}
\newcommand{\rsg}{\mathop{\hbox{\rm rsg}}\nolimits}
\newcommand{\cl}{\mathop{\hbox{\rm cl}}\nolimits}
\newcommand{\IIm}{\mathop{\hbox{\rm Im}}\nolimits}

\newcommand{\PV}{\mathop{\hbox{\rm PV}}\nolimits}
\newcommand{\Dom}{\mathop{\hbox{\rm Dom}}\nolimits}
\newcommand{\n}{\mathop{\hbox{\rm n}}\nolimits}
\newcommand{\Gr}{\mathop{\hbox{\rm Gr}}\nolimits}

\newtheorem{theorem}{Teorema}
\renewcommand{\thetheorem}{\textrm{\thesection.\arabic{theorem}}}
%\newtheorem*{theorem*}{Teorema}

\def\Thm#1#2{\ifthenelse{\equal{#1}{*}}{\begin{theorem*}#2\end{theorem*}}
  {\begin{theorem}\label{T#1}#2\end{theorem}}}
	
\newtheorem{Atheorem}{Teorema}
\renewcommand{\theAtheorem}{\Alph{Atheorem}}

\def\THM#1#2{\begin{Atheorem}\label{T#1}#2\end{Atheorem}}
\def\thm#1{Teorema~\ref{T#1}}

\newtheorem{proposition}[theorem]{Proposición}
%\newtheorem*{proposition*}{Proposición}
\def\Prp#1#2{\ifthenelse{\equal{#1}{*}}{\begin{proposition*}#2\end{proposition*}}
             {\begin{proposition}\label{P#1}#2\end{proposition}}}
\def\prp#1{Proposición~\ref{P#1}}

\newtheorem{corollary}[theorem]{Corolario}
%\newtheorem*{corollary*}{Corolario}
\def\Cor#1#2{\ifthenelse{\equal{#1}{*}}{\begin{corollary*}#2\end{corollary*}}
             {\begin{corollary}\label{C#1}#2\end{corollary}}}
\def\cor#1{Corolario~\ref{C#1}}

\newtheorem{lemma}[theorem]{Lema}
%\newtheorem*{lemma*}{Lema}
\def\Lem#1#2{\ifthenelse{\equal{#1}{*}}{\begin{lemma*}#2\end{lemma*}}
             {\begin{lemma}\label{L#1}#2\end{lemma}}}
\def\lem#1{Lema~\ref{L#1}}

\newtheorem{example}[theorem]{Ejemplo}
%\newtheorem*{example*}{Ejemplo}
\long\def\Exa#1#2{\ifthenelse{\equal{#1}{*}}{\begin{example*}\rm #2\end{example*}}
            {\begin{example}\label{Ex#1}\rm #2\end{example}}}
\def\exa#1{Example~\ref{E#1}}

\newtheorem{problem}[subsection]{Problem}
\def\Prob#1#2{\begin{problem}\label{Prob#1}\rm #2\end{problem}}
\def\prob#1{Problem~\ref{Prob#1}}
%\theoremstyle{definition}
\newtheorem{definition}[theorem]{Definición}
%\newtheorem*{definition*}{Definición}
\def\Defi#1#2{\ifthenelse{\equal{#1}{*}}{\begin{definition*}#2\end{definition*}}
      {\begin{definition}\label{D#1}#2\end{definition}}}
\def\defi#1{Definición~\ref{D#1}}

\newtheorem{remark}[theorem]{Observación}
%\newtheorem*{remark*}{Remark}
\long\def\Rem#1#2{\ifthenelse{\equal{#1}{*}}{\begin{remark*}#2\end{remark*}}
             {\begin{remark}\label{R#1}#2\end{remark}}}
\def\rem#1{Observación~\ref{R#1}}
\def\eq#1{{\rm(\ref{E#1})}}
%\renewcommand{\theequation}{\thesection.\arabic{equation}}
\def\Eq#1#2{\ifthenelse{\equal{#1}{*}}
  {\begin{equation*}\begin{aligned}[]#2\end{aligned}\end{equation*}}
  {\begin{equation}\begin{aligned}[]\label{E#1}#2\end{aligned}\end{equation}}}

\makeindex
%%%%%%%%%%%%%%%%%%%%%%%%%%%%%%
\begin{document}

%\pagenumbering{Roman}

\begin{titlepage}
%

%%
%\begin{floatingfigure}[l]{4cm}
 %\scalebox{0.15}{\includegraphics{ucv.jpg}}
%\end{floatingfigure}

\begin{center}

UNIVERSIDAD CENTRAL DE VENEZUELA.

FACULTAD DE CIENCIAS.

POSTGRADO EN MATEM\'ATICA.

\end{center}

\begin{figure}[h]
	\centering
		\includegraphics[scale=0.3]{ucv.jpg}
\end{figure}


\vspace{1cm}

\begin{large}

\begin{center}

\parbox{12cm}{
\begin{center} 
\textsc{\textbf{Teoremas de tipo Bernstein-Doetsch para
								multifunciones convexas y c\'oncavas.}}
\end{center}
}
\end{center}

\vspace{1cm}

\begin{description}
\item[Autor:] Msc. Carlos Gonz\'alez.
\item[Tutor:] Dr. Nelson Merentes.
\end{description}

\begin{flushright}
\begin{minipage}[b][5cm][b]{0,45\textwidth}
Trabajo de Grado de Maestría presentado
ante la ilustre Universidad Central de 
Venezuela para optar al título de Doctor en Matemática.
\end{minipage}
\end{flushright}

\begin{center}

Caracas, Venezuela \\ Enero, 2018

\end{center}

\end{large}

\end{titlepage}

 %\thispagestyle{empty} % esto quita el número de la página actual
 %\begin{center}
%\includegraphics[width=3cm,height=3cm]{ucv.eps}
%
 %{\bf UNIVERSIDAD CENTRAL DE VENEZUELA\\
%FACULTAD DE CIENCIAS\\
%POSTGRADO EN MATEM\'ATICA}
%
%
%
%%\textbf{\large{  TESIS DOCTORAL.}}
 %\vspace*{4cm}
%
%\textbf{  {\emph{Teoremas de tipo Bernstein-Doetsch para
								%multifunciones convexas y concavas.}}}
%\end{center}
%
%
%
%
%
%\vspace*{3.5cm}
%\normalsize
%
%
%\noindent \textbf{Autor}: Lic. Odalis M. Mejía G.\\
%%\small
%\vspace{0.5cm}
%\hspace{-0.12cm}{\bf Tutor:} Dr. Nelson J. Merentes D.\\
%\vspace{0.5cm}
%\hspace{7cm}
%\begin{minipage}[t]{8cm}
%Trabajo de Grado de Maestr\'{\i}a Presentado ante la ilustre Universidad Central de Venezuela para optar al
%t\'{\i}tulo de Magister Scientiarum, Menci\'{o}n Matem\'{a}tica.
%\end{minipage}
 %\vspace*{1.5cm}
%
%
%\begin{center}
%Caracas, Enero de 2013
%\end{center}
%\newpage
%
%%\maketitle
%\parskip=5mm




\thispagestyle{empty}
Nosotros, los abajo firmantes, designados por la Universidad Central
de Venezuela como integrantes del Jurado Examinador del Trabajo de 
Grado de Maestr\'{\i}a titulado ``\textbf{Teoremas de tipo Bernstein-Doetsch para
multifunciones convexas y c\'oncavas.}'', presentado por el
\textbf{Lic. Carlos L. Gonz\'alez R.}, titular de la C\'edula de Identidad
\textbf{17.980.310}, certificamos que este trabajo cumple con los
requisitos exigidos por nuestra Magna Casa de Estudios para optar al
t\'{\i}tulo de \textbf{Magister Scientiarum mención Matem\'atica}. \vspace{1.4cm}
\begin{center}
\vspace{1cm}
\underline{\hspace{8cm}}

\vspace{0.5cm}

\textbf{Dr. Nelson J. Merentes D.\\Tutor}

\vspace{2cm}

\underline{\hspace{8cm}}

\vspace{0.3cm}

\textbf{ \\Dr. Teodoro Lara\\Jurado}

\vspace{2cm}

\underline{\hspace{8cm}}

\vspace{0.3cm}

\textbf{ \\Dra. Yamilet Quintana\\Jurado}


\end{center}

\newpage


\thispagestyle{empty}
\vspace*{2.5cm}


\begin{flushright}

A mis abuelas y a mis padres...
\end{flushright}



\newpage 
\chapter*{Agradecimientos}
\vspace{1cm}

En primer lugar le quiero dar gracias a DIOS por haberme
puesto en el camino de las  matem\'aticas y por haberme permitido
llegar a donde estoy. Gracias a mis abuelas Miriam y Silveria por 
darme su apoyo incondicional en todo momento y a mis abuelos que aunque
no esten presentes fisicamente, lo están en mi corazón. 
Gracias a mis Padres Aris y Luis
y a mi hermanita Daniela por todo su amor. Quiero tambi\'en manifestar mi
gratitud a la srta Freysimar, mi novia y mi compañera incondicional
que sin duda me ha apoyado en todo lo que hago. Gracias también a su 
familia por hacerme sentir uno más de ellos, en especial a Isa y a Freddy
por estar siempre pendientes de mi. 

Gracias a todos y cada uno de los profesores que he tenido a lo largo
de mi carrera y que me han ayudado en mi formación como matemático durante
todo este tiempo, en especial al profesor Zsolt Páles por su valiosa ayuda 
en la dirección de esta investigación y a mi favorito, mi padre el profesor 
Luís González por su apoyo a lo largo de toda mi vida. 

Mi agradecimiento al profesor Nelson Merentes por todo el apoyo brindado 
para poder llevar a feliz término la elaboración de este trabajo, por
haber confiado en mi y por no permitir que dejára de lado el fascinante mundo de
las matemáticas.

Para finalizar, gracias a mis amigos y compañeros de trabajo por todo el apoyo,
Padilla, Diosa, Mildred, Andrés Perez, Daniela, Manuel, Mairene, Jean Carlos,
Kenyer, Hugo y Tomás.    



 

  



 \newpage

\tableofcontents
\newpage 

\addcontentsline{toc}{chapter}{Resumen}
\chapter*{Resumen}
\markright{RESUMEN}


En general, no toda función midconvexa es convexa. Sin embargo, en la clase de  
funciones continuas, es bien conocido que una función es midconvexa y continua si 
y sólo si ella es convexa \cite{Kuc09}. El teorema de Bernstein-Doetsch publicado hace 100 años da una condición
más débil que la continuidad, para que una función midconvexa definida en un subconjunto
convexo de la recta real sea continua y por lo tanto convexa. 
Este teorema fue establecido en 1915, y es uno de los resultados clásicos más importantes 
obtenidos en la teoría de funciones convexas. Desde entonces, ha sido generalizado de varias 
maneras diferentes y por muchos autores. 

Los teoremas principales en esta monografía generalizan algunos de los resultados obtenidos por Ng, Nikodem
Averna, Cardinali, Papalini, y otros, relacionados con funciones fuertemente y aproximadamente convexas,
puede revisar \cite{MakPal10b,MakPal13b,Nik89} y las referencias que allí los autores mencionan.
Aquí, consideramos una multifunción $F(\cdot)$ definida en un subconjunto convexo $D$ de
un espacio topológico lineal $X$ la cual toma valores en la clase de subconjuntos no-vacíos 
de otro espacio topológico lineal $Y$ y que satisface la siguiente inclusión de tipo Jensen:
$$
\frac{F(x)+F(y)}2 + A(x-y) \subseteq \overline{\bigg(F\bigg(\frac{x+y}2\bigg)+B(x-y)\bigg)},
\qquad
(x,y\in D)
$$
donde, $A(\cdot)$ y $B(\cdot)$ son multifunciones definidas en $D-D$ y que además satisfacen que para todo $x\in D-D$,
$0\in A(x)\cap B(x)$. Ahora bien, asumiendo algunas condiciones de regularidad sobre la multifunción $F$  
se puede demostrar que la multifunción $F(\cdot)$ satisface para $x,y\in D$ y $t\in[0,1]$
las siguientes inclusiones:
$$
tF(x)+(1-t)F(y) + A^T(t,x-y) \subseteq \overline{\bigg(F(tx+(1-t)y)+B^T(t,x-y)\bigg)}.
$$
y
$$
tF(x)+(1-t)F(y) + A^\perp(t,x-y) \subseteq \overline{\bigg(F(tx+(1-t)y)+B^\perp(t,x-y)\bigg)}.
$$

Aquí, $A^T$ denota la transformación de Takagi-Hazy-P\'ales asociada a la multifunción $A$, 
mientras que $A^\perp$ denota la transformaci\'on de Takagi-Tabor asociada a la multifunci\'on
$A$. Estas transformaciones, estan definidas para 
 $t\in[0,1]$ y $x\in D- D$ mediante las siguientes fórmulas:
$$
A^T(t,x) = \overline{\bigcup_{n=0}^{\infty}\sum_{k=0}^n{\frac1{2^k}A(2\mbox{dist}(2^kt,\Z)x)}}
$$
y
$$
A^\perp(t,x) = \overline{\bigcup_{n=0}^{\infty}\sum_{k=0}^n{2\mbox{dist}(2^kt,\Z)A\bigg(\frac{x}{2^k}\bigg)}}
$$
respectivamente

{\bf Palabras Claves:} K-Jensen convexidad/concavidad, multifunción, Transformación de Takagi, convexidad aproximada,
convexidad fuerte.
%\addcontentsline{toc}{chapter}{Abstract}
\chapter*{Abstract}
\markright{ABSTRACT}
In general, every midconvex function is not necessarly convex. However in the class of 
continuous functions, it is well know that a function is midconvex and  continuous 
if and only if it is convex \cite{Kuc09}. The Bernstein-Doetsch Theorem published 100 years ago, gives some regularity 
conditions weaker than continuity, for a real valued  midconvex function defined on a 
convex subset of the real line to be convex, and hence continuous. 
This theorem was stablished  in 1915, and it is one of the most important and classical 
results obtained in convexity theory. Since then, it has been generalized in many 
different ways and by many authors. 

The main theorem of this paper generalizes some of the results  obtained by Ng, Nikodem,
Averna, Cardinali, Papalini, and others, related to strongly and approximately convex
functions, see for instance \cite{MakPal10b,MakPal13b,Nik89} and the references therein. 
Here, we consider a set valued map $F(\cdot)$ defined on a convex subset $D$ of a 
topological Hausdorff space X, which takes his values at the class of nonempty subsets 
of another topological Hausdorff space Y and satisfies the following Jensen type inclusion:
$$
\frac{F(x)+F(y)}2 + A(x-y) \subseteq \overline{\bigg(F\bigg(\frac{x+y}2\bigg)+B(x-y)\bigg)},
\qquad
(x,y\in D)
$$
where, $A(\cdot)$ and $B(\cdot)$ are set valued maps defined on $D-D$ and for all $x\in D-D$,
$0\in A(x)\cap B(x)$. Now, under some regularity assumptions  
one can prove that the set valued map $F(\cdot)$ satisfies for $x,y\in D$ and $t\in[0,1]$
the following convexity type inclusion:
$$
tF(x)+(1-t)F(y) + A^T(t,x-y) \subseteq \overline{\bigg(F(tx+(1-t)y)+B^T(t,x-y)\bigg)}.
$$
$A^T$ denotes the Takagi transformation associated to the set valued map $A$, 
and it is defined for $t\in[0,1]$ and $x\in D- D$ by the following formula:
$$
A^T(t,x) = \overline{\bigcup_{n=0}^{\infty}\sum_{k=0}^n{\frac1{2^k}A(2\mbox{dist}(2^kt,\Z)x)}}.
$$


{\bf Key words:} K-Jensen convexity/concavity, set-valued map, Takagi transformation, approximate convexity,
strong convexity.

\addcontentsline{toc}{chapter}{Introducci\'on}
\chapter*{Introducción}
\markright{INTRODUCCIÓN}


El Teorema de Bernstein y Doetsch \cite{BerDoe15} publicado hace 100 años, ha sido uno de los
resultados fundamentales en la teoría de convexidad
\cite{Kuc09}. Este teorema asegura que si la funci\'on $f:I\to\R$ es 
Jensen convexa (donde $I$ es un intervalo de la recta real), i.e.,
\Eq{JC}{
  f\Big(\frac{x + y}{2}\Big)\leq \frac{f(x) + f(y)}{2} \qquad(x,y\in I)
}
y también es localmente acotada superior, entonces $f$ debe ser continua
sobre $I$ y por lo tanto convexa en $I$. 
Si $-f$ es Jensen convexa, entonces se dice que $f$ es Jensen cóncava y los resultados obtenidos
para esta clase de funciones son análogos bajo la hipótesis de que $f$ debe ser localmente
acotada inferior. Este teorema es muy importante y ha sido aplicado y generalizado de muchas maneras
las cuales describiremos brevemente a continuación.

Cuando el co-dominio $Y$ de la función $f$ es un espacio vectorial ordenado, 
i.e, el conjunto $K$ de elementos
no-negativos de $Y$ forma un cono convexo, entonces se puede definir la convexidad de tipo Jensen 
con respecto al cono $K$ (frecuentemente conocida como Jensen $K$-convexidad) de la función
$f:I\to Y$ por
\Eq{JCK}{
  \frac{f(x) + f(y)}{2}\in f\Big(\frac{x + y}{2}\Big)+K \qquad(x,y\in I).
}
En particular, si $Y=\R$ y $K=\R_+$, entonces \eq{JCK} es equivalente a \eq{JC}. 
Las extensiones del teorema de Bernstein--Doetsch a esta clase de funciones fueron formuladas por
Trudzik \cite{Tru84}. Las funciones $f:I\to Y$ que satisfacen la inclusión
\Eq{JCVK}{
  f\Big(\frac{x + y}{2}\Big)\in \frac{f(x) + f(y)}{2} + K \qquad(x,y\in I)
}
se denominan $K$-Jensen cóncavas. Evidentemente, esta última relación es válida si y sólo
si $(-f)$ es $K$-Jensen convexa (ó si $f$ es $(-K)$-Jensen convexa). 
Por lo tanto, los resultados relacionados con funciones $K$-Jensen cóncavas siempre pueden
ser obtenidos directamente de los resultados establecidos para funciones $K$-Jensen convexas.
%(Esto sin embargo, no ocurrirá para el caso de multifunciones.)

Generalizado un poco más, podemos considerar el caso de las multifunciones.
Una multifunci\'on no es m\'as que una funci\'on cuyas im\'agenes
son subconjuntos de un conjunto $Y$ cualquiera.
Si $X$ es un espacio normado, $D\subseteq X$ es un conjunto convexo y abierto y
$Y$ es un espacio vectorial ordenado,
entonces se dice que una multifunción $F:D\to \P_0(Y)$ es $K$-Jensen convexa si para todo 
$x,y\in D$ se cumple lo siguiente
\Eq{SVJCK}{
  \frac{F(x) + F(y)}{2}\subseteq F\Big(\frac{x + y}{2}\Big)+K. 
}
Observe que si $F$ es de la forma $F(x)=\{f(x)\}$ para alguna función $f:D\to Y$, entonces \eq{SVJCK}
es equivalente a \eq{JCK}, y así se evidencia como la inclusión \eq{SVJCK} generaliza \eq{JCK}.
Los resultados de tipo Bernstein--Doetsch  para este tipo de multifunciones han sido obtenidos 
durante las últimas cinco décadas por los profesores 
Averna, Cardinali, Nikodem, Papalini \cite{AveCar90,CarNikPap93,Nik86,Nik87a,Nik87c,Nik89,Pap90} 
y Borwein \cite{Bor77}. La noción de $K$-Jensen concavidad para una multifunción 
$F:D\to \P_0(Y)$, análoga a la inclusión \eq{JCVK}, se define por 
\Eq{SVJCVK}{
  F\Big(\frac{x + y}{2}\Big)\subseteq \frac{F(x) + F(y)}{2} + K \qquad(x,y\in D).
}
En el desarrollo de este trabajo, veremos que, en general, la $K$-Jensen concavidad 
de $F$ \textit{no es} equivalente a la $K$-Jensen 
convexidad de $-F$. Esto trae como consecuencia, que al hablar de multifunciones, 
los resultados relacionados a convexidad y concavidad necesitan ser tratados por separado. 

Otra cadena de generalizaciones del teorema de Bernstein--Doetsch emerge del artículo del profesor
K. Nikodem junto con el prof. Ng \cite{NgNik93} en el contexto de convexidad aproximada.
Allí, ellos demuestran que si $f:D\to\R$ es $\varepsilon$-Jensen convexa para algún 
$\varepsilon\geq0$, i.e.,
\Eq{JCe}{
  f\Big(\frac{x + y}{2}\Big)\leq \frac{f(x) + f(y)}{2} + \varepsilon \qquad(x,y\in D),
}
y si además $f$ es localmente acotada superior, entonces $f$ es $2\varepsilon$-convexa, i.e.,
\Eq{Ce}{
  f(tx+(1-t)y)\leq tf(x)+(1-t)f(y)+2\varepsilon \qquad(x,y\in D,\,t\in[0,1]).
}
%La versión correspondiente del resultado anterior para el caso de multifunciones, será formulado 
%como una consecuencia de los resultados principales obtenidos en nuestra investigación.

Considerando un término de error más general, Házy y Páles \cite{HazPal04} investigaron la 
siguiente desigualdad de tipo Jensen aproximada
\Eq{JCHP}{
  f\Big(\frac{x + y}{2}\Big)\leq \frac{f(x) + f(y)}{2} + \varepsilon\|x-y\| \qquad(x,y\in D),
}
asumiendo que $D$ es un subconjunto de un espacio normado $X$ y $f$ es una función a valores reales.
En dicho artículo, los autores demostraron que bajo la hipótesis usual de que $f$ es localmente 
acotada superior, la desigualdad \eq{JCHP} implica
\Eq{CHP}{
  f(tx+(1-t)y)\leq tf(x)+(1-t)f(y)+2\varepsilon T(t)\|x-y\| \qquad(x,y\in D,\,t\in[0,1]),
}
donde la función $T:\R\to[0,1]$, es la conocida función de Takagi, y se define por
\Eq{*}{
  T(t):=\sum_{n=0}^\infty \frac1{2^n}\dist(2^nt,\Z).
}
Algunos resultados, que extienden este tipo de nociones a términos más generales de errores y también 
a conceptos de convexidad relacionados con sistemas de Chebyshev, han sido obtenidos recientemente
por Házy, Makó and Páles \cite{Haz05a,Haz07b,HazPal05,HazPal09,MakPal10b,MakPal12b,MakPal12c,MakPal13b}
y por Mureńko, Ja. Tabor, Jó. Tabor, and Żoldak 
\cite{MurTabTab12,TabTab09b,TabTab09a,TabTabZol10b,TabTabZol10a}.

%
%Para el caso de multifunciones $K$-Jensen convexas y $K$-Jensen cóncavas, haremos formulaciones
%más generales como consecuencia directa de los resultados principales de nuestra investigación.

Finalmente, haremos mención a la noción de convexidad fuerte, la cual en cierto sentido, es lo contrario 
a convexidad aproximada. Siguiendo a Polyak \cite{Pol66}, una función $f:D\to\R$ es 
fuertemente Jensen convexa con módulo $\varepsilon\geq0$ si
\Eq{SJC}{
  f\Big(\frac{x + y}{2}\Big)\leq \frac{f(x) + f(y)}{2} - \frac{\varepsilon}{4}\|x-y\|^2 \qquad(x,y\in D).
}
Asumiendo que $f$ es localmente acotada superior, los profesores Azócar, Gimenez, Nikodem y Sánchez en
 \cite{AzoGimNikSan11} demostraron que si $f$ es fuertemente Jensen convexa 
entonces $f$ es fuertemente convexa con módulo $\varepsilon$, i.e., 
\Eq{SC}{
  f(tx+(1-t)y)\leq tf(x)+(1-t)f(y) - \varepsilon t(1-t)\|x-y\|^2 \qquad(x,y\in D,\,t\in[0,1]).
}
La versión conjunto valuada de este resultado fue establecida por Leiva, Merentes, Nikodem, y Sánchez 
en \cite{LeiMerNikSan13}.

En este trabajo, mostraremos una generalización del teorema de Bernstein--Doetsch 
para multifunciones que satisfacen una inclusión general de tipo Jensen
de la forma
\Eq{intro1}{
\frac{F(x)+F(y)}2 + A(x-y) \subseteq F\bigg(\frac{x+y}{2}\bigg) + B(x-y)
\qquad (x,y\in D),
}
donde $A$ y $B$ son multifunciones definidas en $D-D$. Bajo ciertas condiciones
de regularidad sobre la multifunción $F$ demostraremos que ella satisface
la inclusión
\Eq{intro2}{
tF(x)+(1-t)F(y)+S_A(t,x-y)\subseteq F(tx+(1-t)y)+S_B(t,x-y)
} 
para todo $t\in [0,1]$ y para todo $x,y\in D$.
Donde $S_A$ y $S_B$ son ciertas multifunciones definidas en
$[0,1]\times D-D$ que surgen de forma natural como una 
generalización conjunto-valuada de la función de Takagui cuya 
importancia se ve reflejada en los artículos de Z. Páles, J. Makó,
A. Hazy, Jo. Tabor, Ja. Tabor \cite{MakPal10b,MakPal12a,TabTab09b,TabTab09a}
como veremos en el Capítulo 2 de esta monografía.
Los resultados que presentaremos aquí generalizan
 a la mayoría  de los resultados
obtenidos en esta línea de investigación desde el año 1915 y además
proporcionan nuevos Teoremas de tipo Bernstein--Doetsch
para multifunciones aproximadamente midconvexas y midcóncavas.

Note que la inclusión \eq{intro1} es una combinación de 
midconvexidad aproximada y midconvexidad fuerte para la multifunción
$F$, mientras que la inclusión \eq{intro2} combina convexidad fuerte y 
aproximada. De hecho, si $A=\{0\}$, entonces \eq{intro1} corresponde
a un tipo de midconvexidad aproximada mientras que por el contrario
si $B=\{0\}$ entonces la misma inclusión corresponde a un tipo de 
midconvexidad fuerte. Por otra parte, podemos decir que en este
trabajo no se demuestra que la inclusión \eq{intro2} es óptima
y por lo tanto queda como problema abierto.

Debido a la observación hecha previamente con respecto a la naturaleza
de las multifunciones convexas y cóncavas, estableceremos también, 
un Teorema de tipo Bernstein--Doetsch para multifunciones midcóncavas. Los resultados principales
de esta investigación que  serán presentados en el Capítulo 4 de esta monografía,
se pueden encontrar en \cite{GonNikPalRoa}.
%\pagenumbering{Arabic}
%\addtocounter{page}{13}

%\chapter*{Introducción}
\markright{INTRODUCCIÓN}


El Teorema de Bernstein y Doetsch \cite{BerDoe15} publicado hace 100 años, ha sido uno de los
resultados fundamentales en la teoría de convexidad
\cite{Kuc09}. Este teorema asegura que si la funci\'on $f:I\to\R$ es 
Jensen convexa (donde $I$ es un intervalo de la recta real), i.e.,
\Eq{JC}{
  f\Big(\frac{x + y}{2}\Big)\leq \frac{f(x) + f(y)}{2} \qquad(x,y\in I)
}
y también es localmente acotada superior, entonces $f$ debe ser continua
sobre $I$ y por lo tanto convexa en $I$. 
Si $-f$ es Jensen convexa, entonces se dice que $f$ es Jensen cóncava y los resultados obtenidos
para esta clase de funciones son análogos bajo la hipótesis de que $f$ debe ser localmente
acotada inferior. Este teorema es muy importante y ha sido aplicado y generalizado de muchas maneras
las cuales describiremos brevemente a continuación.

Cuando el co-dominio $Y$ de la función $f$ es un espacio vectorial ordenado, 
i.e, el conjunto $K$ de elementos
no-negativos de $Y$ forma un cono convexo, entonces se puede definir la convexidad de tipo Jensen 
con respecto al cono $K$ (frecuentemente conocida como Jensen $K$-convexidad) de la función
$f:I\to Y$ por
\Eq{JCK}{
  \frac{f(x) + f(y)}{2}\in f\Big(\frac{x + y}{2}\Big)+K \qquad(x,y\in I).
}
En particular, si $Y=\R$ y $K=\R_+$, entonces \eq{JCK} es equivalente a \eq{JC}. 
Las extensiones del teorema de Bernstein--Doetsch a esta clase de funciones fueron formuladas por
Trudzik \cite{Tru84}. Las funciones $f:I\to Y$ que satisfacen la inclusión
\Eq{JCVK}{
  f\Big(\frac{x + y}{2}\Big)\in \frac{f(x) + f(y)}{2} + K \qquad(x,y\in I)
}
se denominan $K$-Jensen cóncavas. Evidentemente, esta última relación es válida si y sólo
si $(-f)$ es $K$-Jensen convexa (ó si $f$ es $(-K)$-Jensen convexa). 
Por lo tanto, los resultados relacionados con funciones $K$-Jensen cóncavas siempre pueden
ser obtenidos directamente de los resultados establecidos para funciones $K$-Jensen convexas.
%(Esto sin embargo, no ocurrirá para el caso de multifunciones.)

Generalizado un poco más, podemos considerar el caso de las multifunciones.
Una multifunci\'on no es m\'as que una funci\'on cuyas im\'agenes
son subconjuntos de un conjunto $Y$ cualquiera.
Si $X$ es un espacio normado, $D\subseteq X$ es un conjunto convexo y abierto y
$Y$ es un espacio vectorial ordenado,
entonces se dice que una multifunción $F:D\to \P_0(Y)$ es $K$-Jensen convexa si para todo 
$x,y\in D$ se cumple lo siguiente
\Eq{SVJCK}{
  \frac{F(x) + F(y)}{2}\subseteq F\Big(\frac{x + y}{2}\Big)+K. 
}
Observe que si $F$ es de la forma $F(x)=\{f(x)\}$ para alguna función $f:D\to Y$, entonces \eq{SVJCK}
es equivalente a \eq{JCK}, y así se evidencia como la inclusión \eq{SVJCK} generaliza \eq{JCK}.
Los resultados de tipo Bernstein--Doetsch  para este tipo de multifunciones han sido obtenidos 
durante las últimas cinco décadas por los profesores 
Averna, Cardinali, Nikodem, Papalini \cite{AveCar90,CarNikPap93,Nik86,Nik87a,Nik87c,Nik89,Pap90} 
y Borwein \cite{Bor77}. La noción de $K$-Jensen concavidad para una multifunción 
$F:D\to \P_0(Y)$, análoga a la inclusión \eq{JCVK}, se define por 
\Eq{SVJCVK}{
  F\Big(\frac{x + y}{2}\Big)\subseteq \frac{F(x) + F(y)}{2} + K \qquad(x,y\in D).
}
En el desarrollo de este trabajo, veremos que, en general, la $K$-Jensen concavidad 
de $F$ \textit{no es} equivalente a la $K$-Jensen 
convexidad de $-F$. Esto trae como consecuencia, que al hablar de multifunciones, 
los resultados relacionados a convexidad y concavidad necesitan ser tratados por separado. 

Otra cadena de generalizaciones del teorema de Bernstein--Doetsch emerge del artículo del profesor
K. Nikodem junto con el prof. Ng \cite{NgNik93} en el contexto de convexidad aproximada.
Allí, ellos demuestran que si $f:D\to\R$ es $\varepsilon$-Jensen convexa para algún 
$\varepsilon\geq0$, i.e.,
\Eq{JCe}{
  f\Big(\frac{x + y}{2}\Big)\leq \frac{f(x) + f(y)}{2} + \varepsilon \qquad(x,y\in D),
}
y si además $f$ es localmente acotada superior, entonces $f$ es $2\varepsilon$-convexa, i.e.,
\Eq{Ce}{
  f(tx+(1-t)y)\leq tf(x)+(1-t)f(y)+2\varepsilon \qquad(x,y\in D,\,t\in[0,1]).
}
%La versión correspondiente del resultado anterior para el caso de multifunciones, será formulado 
%como una consecuencia de los resultados principales obtenidos en nuestra investigación.

Considerando un término de error más general, Házy y Páles \cite{HazPal04} investigaron la 
siguiente desigualdad de tipo Jensen aproximada
\Eq{JCHP}{
  f\Big(\frac{x + y}{2}\Big)\leq \frac{f(x) + f(y)}{2} + \varepsilon\|x-y\| \qquad(x,y\in D),
}
asumiendo que $D$ es un subconjunto de un espacio normado $X$ y $f$ es una función a valores reales.
En dicho artículo, los autores demostraron que bajo la hipótesis usual de que $f$ es localmente 
acotada superior, la desigualdad \eq{JCHP} implica
\Eq{CHP}{
  f(tx+(1-t)y)\leq tf(x)+(1-t)f(y)+2\varepsilon T(t)\|x-y\| \qquad(x,y\in D,\,t\in[0,1]),
}
donde la función $T:\R\to[0,1]$, es la conocida función de Takagi, y se define por
\Eq{*}{
  T(t):=\sum_{n=0}^\infty \frac1{2^n}\dist(2^nt,\Z).
}
Algunos resultados, que extienden este tipo de nociones a términos más generales de errores y también 
a conceptos de convexidad relacionados con sistemas de Chebyshev, han sido obtenidos recientemente
por Házy, Makó and Páles \cite{Haz05a,Haz07b,HazPal05,HazPal09,MakPal10b,MakPal12b,MakPal12c,MakPal13b}
y por Mureńko, Ja. Tabor, Jó. Tabor, and Żoldak 
\cite{MurTabTab12,TabTab09b,TabTab09a,TabTabZol10b,TabTabZol10a}.

%
%Para el caso de multifunciones $K$-Jensen convexas y $K$-Jensen cóncavas, haremos formulaciones
%más generales como consecuencia directa de los resultados principales de nuestra investigación.

Finalmente, haremos mención a la noción de convexidad fuerte, la cual en cierto sentido, es lo contrario 
a convexidad aproximada. Siguiendo a Polyak \cite{Pol66}, una función $f:D\to\R$ es 
fuertemente Jensen convexa con módulo $\varepsilon\geq0$ si
\Eq{SJC}{
  f\Big(\frac{x + y}{2}\Big)\leq \frac{f(x) + f(y)}{2} - \frac{\varepsilon}{4}\|x-y\|^2 \qquad(x,y\in D).
}
Asumiendo que $f$ es localmente acotada superior, los profesores Azócar, Gimenez, Nikodem y Sánchez en
 \cite{AzoGimNikSan11} demostraron que si $f$ es fuertemente Jensen convexa 
entonces $f$ es fuertemente convexa con módulo $\varepsilon$, i.e., 
\Eq{SC}{
  f(tx+(1-t)y)\leq tf(x)+(1-t)f(y) - \varepsilon t(1-t)\|x-y\|^2 \qquad(x,y\in D,\,t\in[0,1]).
}
La versión conjunto valuada de este resultado fue establecida por Leiva, Merentes, Nikodem, y Sánchez 
en \cite{LeiMerNikSan13}.

En este trabajo, mostraremos una generalización del teorema de Bernstein--Doetsch 
para multifunciones que satisfacen una inclusión general de tipo Jensen
de la forma
\Eq{intro1}{
\frac{F(x)+F(y)}2 + A(x-y) \subseteq F\bigg(\frac{x+y}{2}\bigg) + B(x-y)
\qquad (x,y\in D),
}
donde $A$ y $B$ son multifunciones definidas en $D-D$. Bajo ciertas condiciones
de regularidad sobre la multifunción $F$ demostraremos que ella satisface
la inclusión
\Eq{intro2}{
tF(x)+(1-t)F(y)+S_A(t,x-y)\subseteq F(tx+(1-t)y)+S_B(t,x-y)
} 
para todo $t\in [0,1]$ y para todo $x,y\in D$.
Donde $S_A$ y $S_B$ son ciertas multifunciones definidas en
$[0,1]\times D-D$ que surgen de forma natural como una 
generalización conjunto-valuada de la función de Takagui cuya 
importancia se ve reflejada en los artículos de Z. Páles, J. Makó,
A. Hazy, Jo. Tabor, Ja. Tabor \cite{MakPal10b,MakPal12a,TabTab09b,TabTab09a}
como veremos en el Capítulo 2 de esta monografía.
Los resultados que presentaremos aquí generalizan
 a la mayoría  de los resultados
obtenidos en esta línea de investigación desde el año 1915 y además
proporcionan nuevos Teoremas de tipo Bernstein--Doetsch
para multifunciones aproximadamente midconvexas y midcóncavas.

Note que la inclusión \eq{intro1} es una combinación de 
midconvexidad aproximada y midconvexidad fuerte para la multifunción
$F$, mientras que la inclusión \eq{intro2} combina convexidad fuerte y 
aproximada. De hecho, si $A=\{0\}$, entonces \eq{intro1} corresponde
a un tipo de midconvexidad aproximada mientras que por el contrario
si $B=\{0\}$ entonces la misma inclusión corresponde a un tipo de 
midconvexidad fuerte. Por otra parte, podemos decir que en este
trabajo no se demuestra que la inclusión \eq{intro2} es óptima
y por lo tanto queda como problema abierto.

Debido a la observación hecha previamente con respecto a la naturaleza
de las multifunciones convexas y cóncavas, estableceremos también, 
un Teorema de tipo Bernstein--Doetsch para multifunciones midcóncavas. Los resultados principales
de esta investigación que  serán presentados en el Capítulo 4 de esta monografía,
se pueden encontrar en \cite{GonNikPalRoa}.
\chapter{Espacios Topológicos Lineales.}
\setcounter{theorem}{0}
En este capítulo se hará un breve desarrollo de algunos conceptos 
de topología que serán usados a lo largo de todo el trabajo.
Los resultados que mencionaremos en este capítulo se encuentran
desarrollados en el libro del profesor W. Rudin\cite{Rud91}.

\section{Espacios topológicos.}
Sea $X$ un conjunto, se denota por $\P(X)$ al conjunto potencia de $X$. 
\Defi{espTop}{
Un \textbf{espacio topológico} consiste en una dupla $(X,\Tau )$,
donde $X$ es un conjunto y $\Tau $ es un subconjunto de 
$\P(X)$ que cumple con las siguientes propiedades:
\begin{enumerate}[i.]
\item $\emptyset \in \Tau $ y $X\in\Tau $.
\item Si $\{U_\alpha\}_{\alpha\in\Delta}\subseteq\Tau $
es una colección de elementos de $\Tau $, entonces:
\Eq{*}{
\bigcup_{\alpha\in\Delta} U_\alpha \in \Tau.
}
\item Si 
$\{U_k\}_{k=1}^n\subseteq\Tau $
es una colección finita de elementos de $\Tau $, entonces:
\Eq{*}{
\bigcap_{k=1}^n U_k \in \Tau.
}
\end{enumerate}
}
Cuando $(X,\Tau)$ es un espacio topológico, los elementos de $\Tau $ 
se llaman \textbf{conjuntos abiertos}. En lo que sigue a continuación,
el par $(X,\Tau )$ denotará un espacio topológico. Cuando no haya lugar
a confusión se omitirá la topología.

\Defi{Haus}{
Decimos que el espacio topológico $(X,\Tau)$, es un espacio
de Hausdorff, si para cada par de puntos diferentes, $x,y\in X$
existen abiertos disjuntos $U_x, U_y\in\Tau$ tales que
$x\in U_x$ y $y\in U_y$.
}

\Defi{basetop}{
Dado un espacio topológico $(X,\Tau  )$, y un subconjunto 
$\mathcal{B}\subseteq\Tau  $, se dice que $\mathcal{B}$ es una base 
para la topología $\Tau  $ si todo conjunto abierto $U \in\Tau $ es la 
unión de elementos de $\mathcal{B}$. De manera equivalente,
$\mathcal{B}$ es una base para $\Tau$ si y sólo si, para todo punto
$p$ perteneciente a cualquier abierto $O\in\Tau$ existe 
$B\in\mathcal{B}$, tal que $p\in B\subseteq O$.
}

\Defi{baseLocal}{
Dado un espacio topológico $(X,\Tau)$ y un punto $p\in X$.
Una base local en $p$, consiste en una colección de abiertos
$\mathcal{B}_p$, tal que, cualquier entorno abierto 
$U\subseteq X$ de $p$, contiene al menos un miembro de 
$\mathcal{B}_p$.
}

\Defi{acumPoints}{
Dado un conjunto $A\subseteq X$ y un punto $p\in X$, se dice que $p$
es un \textbf{punto de acumulación} del conjunto $A$, si para todo abierto $U\in\Tau $ que contiene a $p$, se tiene que 
\Eq{acumP}{
A \cap (U\setminus \{p\}) \neq \emptyset
}
}
\Defi{Closed}{
Un conjunto $V\subseteq X$ es \textbf{cerrado}, si el conjunto $X\setminus V$ es
abierto.
}
Los conjuntos cerrados también se pueden caracterizar de la siguiente
manera:
\Prp{acumP}{
Un subconjunto $V\subseteq X$ es cerrado, si y sólo si $V$ contiene 
todos sus puntos de acumulación.
}

\begin{proof}
Sea $V\subseteq X$ un conjunto cerrado y sea $p\in X$ un punto de acumulación
de $V$. Supongamos que $p\notin V$. Como $V$ es cerrado, entonces por definición
su complemento $U=X\setminus V$ es abierto y como $p\in U$, existe
un elemento básico $B\in\mathcal{B}$, tal que
$p\in B\subseteq U$. En consecuencia, $B\cap V=\emptyset$,
pero esto contradice el hecho de que $p$ es un punto de acumulación de $V$.
Luego, es falso suponer que $p\notin V$ y así, queda establecido que el
conjunto $V$ posee a todos sus puntos de acumulación.

Supongamos ahora que el conjunto $V$ posee a todos sus puntos de acumulación y
sea $U=X\setminus V$. Veamos que $U$ es abierto. Sea $p\in U$, por definición
se tiene que $p$ no es punto de acumulación de $V$, en consecuencia,
existe un entorno abierto $B\subseteq X$ de $p$ tal que $V\cap (B\setminus\{p\})=\emptyset$.
Por lo tanto, $p\in B\subseteq U$ y así, $U$ es abierto.
\end{proof}

Es posible definir una topología en términos de conjuntos cerrados.
\Thm{topClosed}{
Sea $X$ un espacio topológico, la clase de conjuntos cerrados en $X$,
posee las siguientes propiedades:
\begin{enumerate}[i.]
	\item $X$ y $\emptyset$ son conjuntos cerrados.
	\item La intersección arbitraria de conjuntos cerrados, es cerrado.
	\item La unión finita de conjuntos cerrados, es cerrado.
\end{enumerate}
}
\Defi{cl}{
Sea $A\subseteq X$. La \textbf{clausura} de $A$, se denota por
$\cl(A)$ y se define como la intersección de todos los conjuntos 
cerrados que contienen al conjunto $A$. Esto es,
\Eq{*}{
\cl(A) := \bigcap\{V\subseteq X: A\subseteq V \quad\mbox{y}
\quad V\quad\mbox{es cerrado}\}.
}
}
Se puede ver que $\cl(A)$ es un conjunto cerrado ya que es la 
intersección de conjuntos cerrados. Además, $\cl(A)$
 es el conjunto cerrado más pequeño que contiene al conjunto $A$. 
Es decir, si $V$ es un conjunto cerrado que contiene al conjunto $A$, 
entonces:
\Eq{*}{
A\subseteq\cl(A) \subseteq V.
}
Además, $V\subseteq X$ es un conjunto cerrado si y sólo si $V=\cl(V)$.
Todo esto se resume en la siguiente proposición.
\Prp{clcl}{
Sea $A\subseteq X$. Entonces,
\begin{enumerate}[i.]
	\item $\cl(A)$ es cerrado y  $A\subseteq\cl(A)$ .
	\item Si $V$ es un conjunto cerrado que contiene al conjunto $A$, 
	entonces $\cl(A)\subseteq V.$
	\item $A$ es cerrado si y sólo si $A=\cl(A)$.
\end{enumerate}
}

\begin{proof} Sea $A\subseteq X$.
\begin{enumerate}[i.]
	\item Por definición, el conjunto $\cl(A)$ es la intersección arbitraria
	de conjuntos cerrados y por el \thm{topClosed} se tiene que es un conjunto cerrado.
	Además, por definición $A\subseteq\cl(A)$.
	\item Es consecuencia directa de la definición de clausura.
	\item Supongamos que $A$ es cerrado. Entonces, por el item II 
	se tiene que $\cl(A)\subseteq A$ y en consecuencia $\cl(A)=A$.
	Recíprocamente, si $A=\cl(A)$, entonces, evidentemente  $A$ es cerrado, 
	lo que completa la demostración.
\end{enumerate}
\end{proof}

\Prp{cl2}{
Sea $A\subseteq X$. La clausura de $A$ es la unión de $A$ con 
su conjunto de puntos de acumulación $A'$. Esto es,
$\cl(A) = A \cup A'.$
}

\begin{proof}
Veamos que $\cl(A)\subseteq A\cup A'$. Sea $p\in\cl(A)$, y supongamos
que $p\notin A\cup A'$, esto es $p\notin A$ y $p\notin A'$.
Como $p\notin A'$, por definición existe un entorno abierto $B\subseteq X$
de $p$, tal que $A\cap(B\setminus\{p\})=\emptyset$, por lo tanto 
$B\setminus\{p\}\subseteq X\setminus A$ y en consecuencia $B\subseteq X\setminus A$.
Pero esto último equivale a $A\subseteq X\setminus B,$ y siendo $B$
un conjunto abierto, se tiene que $X\setminus B$ es cerrado. Como además,
este conjunto contiene al conjunto $A$, entonces por la definición de clausura
se tiene que $\cl(A)\subseteq X\setminus B$ y por lo tanto
$B\subseteq X\setminus \cl(A).$ Como $p\in B$, debe ocurrir entonces
que $p\notin\cl(A)$, pero esto es una contradicción. Luego, es falso 
suponer que $p\notin A\cup A'$ y así $\cl(A)\subseteq A\cup A'$.

Veamos ahora que $A\cup A'\subseteq\cl(A)$. Sea $p\in A\cup A'$. 
Si $p\in A$, entonces $p\in\cl(A)$ y en este caso la demostración es 
directa. Si por el contrario, $p\in A'$, entonces supongamos que 
$p\notin\cl(A)$. Como $\cl(A)$ es un conjunto cerrado entonces su 
complemento es abierto y por lo tanto existe un entorno abierto $B$
de $p$ tal que $p\in B\subseteq X\setminus\cl(A)$. En consecuencia,
$B\cap\cl(A)=\emptyset$ y como $A\subseteq\cl(A)$ entonces 
$B\cap A=\emptyset$, pero esto es una contradicción ya que $p$ es punto de
acumulación del conjunto $A$. Luego, es falso suponer que $p\notin\cl(A)$
y finalmente se tiene que $A\cup A'\subseteq\cl(A)$. Esto completa la demostración.
\end{proof}

\Prp{Kura}{Sean $A,B\subseteq X$, el operador $\cl(\cdot)$ cumple
las siguientes propiedades
\begin{multicols}{2}
\begin{enumerate}[i.]
	%\item $\cl(\emptyset) = \emptyset.$
	%\item $A\subseteq\cl(A).$
	\item $\cl(A\cup B) = \cl(A)\cup\cl(B).$
	\item $\cl(\cl(A)) = \cl(A).$
\end{enumerate}
\end{multicols}
}
\begin{proof}
La demostración de esta proposición es consecuencia directa
de la \prp{clcl} y de la \prp{cl2}.
\end{proof}
%%%%%%%%%%%%%%%%%%%%%%%%%%%%%%%%%%%%%%%%%%%%%%%%%%%%%%%%%
%%%%%%%%%%%%%%%%%%%%%%%%%%%%%%%%%%%%%%%%%%%%%%%%%%%%%%%%%%%
\section{Espacios vectoriales.}
\setcounter{theorem}{0}
Denótese por $\R$ al cuerpo de los números reales y por 
$\C$ al cuerpo de los números complejos. Por ahora,
se usará $\Phi$ para denotar tanto a $\R$ como a $\C$.
Un escalar, es un elemento del cuerpo $\Phi$.
\Defi{VecSp}{
Un \textit{espacio vectorial} sobre $\Phi$ es un conjunto
$X$, junto con dos operaciones, suma y producto por un escalar
las cuales satisfacen las siguientes propiedades:
\begin{enumerate}[i.]
	\item A cada par de vectores $x,y\in X$ les corresponde 
	un vector $x+y\in X$, de manera que 
	\Eq{*}{
	x+y=y+x\qquad\mbox{y}\qquad x+(y+z) = (x+y)+z.
	}
	\item $X$ contiene un único vector $0$ el cual se llamará 
	con frecuencia el origen de $X$, tal que $x+0=x,$ para 
	todo $x\in X$. Además a cada vector $x\in X$ le corresponde
	un vector $-x\in X$ tal que $x+(-x)=0$.
	\item A cada par $(\alpha,x)$ con $\alpha\in\Phi$, y 
	$x\in X$ le corresponde un vector $\alpha x\in X,$
	de manera que
	\Eq{*}{
	1x=x,\qquad \alpha(\beta x) = (\alpha\beta)x,
	}
	y además se cumplen las siguientes leyes distributivas
	\Eq{*}{
	\alpha(x+y) = \alpha x+\alpha y, \qquad
	(\alpha + \beta) x = \alpha x + \beta x.
	}
\end{enumerate}
}
\Rem{vect1}{
Un espacio vectorial real, es un espacio vectorial donde 
$\Phi = \R,$ y un espacio vectorial complejo, es un espacio
vectorial donde $\Phi=\C$. Cualquier afirmación sobre
un espacio vectorial, donde no se especifique explícitamente
el cuerpo, será válida para ambos casos.
}
\Defi{sumSet}{
Sea $X$ un espacio vectorial y sean $A,B\subseteq X$, $x\in X$ y
$\lambda\in\Phi$, entonces
\Eq{*}{
x+A &= \{x+a : a\in A\}, \\
A+B &= \{a+b : a\in A, b\in B\}, \\
\lambda A &= \{\lambda a: a\in A\}\\
x-A &= \{x-a : a\in A\}.
} 
}
\Rem{A+A}{
Con estas operaciones así definidas, puede ocurrir que 
$A+A\neq 2A.$
}
\Defi{subsp}{
Un subconjunto $W\subseteq X$ es llamado subespacio de $X$
si el mismo es un espacio vectorial (con respecto a las mismas
operaciones). 
}
\Prp{subsp}{
Sea $X$ un espacio vectorial y $W\subseteq X$. $W$ es un 
subespacio de $X$ si y sólo si, $0\in W$ y además,
para $\alpha,\beta\in\Phi$ se tiene que
\Eq{*}{
\alpha W + \beta W \subseteq W.
}
}
\Defi{cvxSet}{
Un conjunto $D\subseteq X$ es convexo, si para todo $t\in[0,1]$
se tiene que 
\Eq{*}{
tD + (1-t)D\subseteq D.
}
En otras palabras, el conjunto $D$ es convexo si para 
cualesquiera par de puntos $x,y\in D$ se tiene que
\Eq{*}{
[x,y]:= \{ty+(1-t)x : t\in [0,1]\}\subseteq D.
} 
}


\Defi{starSet}{
Sea $H\subseteq X$ y sea $p\in H$. 
Se dice que el conjunto $H$ es estrellado con respecto al punto $p$
si para todo $h\in H$ se tiene que el segmento $[h,p]\subseteq H$.
}
Existe una conexión entre los conjuntos estrellados y los conjuntos 
convexos. La siguiente proposición muestra este hecho
\Prp{CvxStr}{
El conjunto $H\subseteq X$ es convexo si y sólo si $H$ es 
estrellado con respecto a cada uno de sus puntos.
}
\Prp{D-DStr}{
Sea $H\subseteq X$ un conjunto convexo no-vacío. 
Entonces, el conjunto $H-H$ es estrellado con respecto al origen.
}
\begin{proof}
Sea $p\in H-H$. Por definición existen $h_1,h_2\in H$ tales que 
$p=h_1-h_2$. Luego, si $t\in[0,1]$, y $h\in H$ es un elemento cualquiera
de $H$ entonces 
$$
tp = th_1-th_2= th_1-th_2+(1-t)(h-h)=th_1+(1-t)h-(th_2+(1-t)h).
$$ 
Como $H$ es un conjunto
convexo, se tiene que $tp\in H-H$ para todo $t\in[0,1]$. Por lo tanto
el conjunto $H-H$ es estrellado con respecto al origen.
\end{proof}

\Defi{BlSet}{
Un conjunto $D\subseteq X$ se dice que es balanceado, si para todo $\alpha\in\Phi,$
 con $|\alpha|\leq 1$, se tiene que $\alpha D\subseteq D,$
}
%\Defi{dimX}{
%El espacio vectorial $X,$ tiene dimensión $n$ $(\dim X=n),$
%si $X$ posee una base $\B=\{u_1,\ldots,u_n\}$, de 
%$n$ vectores linealmente independientes. Esto quiere
%decir que todo $x\in X$, tiene una representación única
%de la forma
%\Eq{*}{
%x=\alpha_1u_1+\cdots+\alpha_nu_n, \quad \alpha_i\in\Phi,
%\quad i=1,\ldots,n.
%} 
%}
%\Rem{dim}{
%Si $\dim X = n$ para algún número natural $n$, se dice que 
%$X$ es de dimensión finita. Si $X=\{0\}$, entonces
%$\dim X = 0$.
%}


%%%%%%%%%%%%%%%%%%%%%%%%%%%%%%%%%%%%%%%%%%%%%%%%%%%%%%%%%%%
%%%%%%%%%%%%%%%%%%%%%%%%%%%%%%%%%%%%%%%%%%%%%%%%%%%%%%%%%%%

\section{Espacios topológicos lineales.}
\setcounter{theorem}{0}

Supongamos que $\Tau$ es una topología para un espacio vectorial $X$
que cumple con lo siguiente:
\begin{enumerate}[i.]
	\item Todo punto de $X$ es un conjunto cerrado.
	\item Las operaciones del espacio vectorial, son continuas 
	con respecto a $\Tau$.
\end{enumerate}

Bajo estas condiciones, se dice que $\Tau$ es una topología vectorial
en $X$ y $X$ es un \textit{espacio topológico lineal}.


\Defi{TaMl}{
Sea $X$ un espacio topológico lineal. Para cada $a\in X$
y para cada $\lambda\neq0$, se definen los operadores de 
traslación y multiplicación, $T_a(x)$ y $M_\lambda(x)$
respectivamente, de la siguiente manera
\Eq{TaMl}{
T_a(x) = x + a \qquad \mbox{y} \qquad
M_\lambda(x) = \lambda x, \quad (x\in X).
}
}

\Prp{hom}{
$T_a$ y $M_\lambda$ son homeomorfismos de $X$ en $X$.
}
\begin{proof}
Los axiomas de espacio vectorial, implican que estos
operadores son biyectivos, y además que sus inversas 
son $T_{-a}$ y $M_{1/\lambda}$, respectivamente. 
El hecho de que las operaciones, suma y producto por un 
escalar sean continuas, implica que tanto $T_a$ como 
$M_\lambda$, así como sus inversas sean continuas.
\end{proof}

Como consecuencia de esta proposición, se tiene que 
toda topología vectorial $\Tau$ es invariante. Es decir, 
un conjunto $E\subseteq X$ es abierto, si y sólo si, para todo $a\in X,$
el conjunto $a+E$ es abierto. Así, $\Tau$ está completamente
determinada por cualquier base local.

En este contexto, el término base local, se refiere a la base 
local en $0\in X$.
Se denotará por $\U(X)$ a dicha base.

\Defi{bddSet}{
Un subconjunto $H\subseteq X$ es acotado, si para todo entorno 
abierto $U$ de $0\in X,$ existe un número $s>0$ tal que 
$H\subseteq tU$ para todo $t>s$.
}
\Prp{bddtH}{
Sea $H\subseteq X$ un conjunto acotado. Para cada $\alpha\in[0,1]$
el conjunto $\alpha H$ es acotado y además $\bigcap_{\alpha\in[0,1]} \alpha H$
es acotado.
}
\begin{proof}
Sea $U\subseteq X$ un conjunto abierto tal que $0\in U$ y
sea $\alpha\in[0,1]$. 
Como $H$ es acotado, existe $s_\alpha>0$ tal que si $t>s_\alpha$, entonces
$H\subseteq t\big(\frac1\alpha U\big)$. En consecuencia, 
$\alpha H\subseteq \alpha t\big(\frac1\alpha U\big)=tU$ siempre y cuando
$t>s_\alpha$. Como $U$ es arbitrario se tiene que el conjunto
$\alpha H$ es acotado. 

Para ver que la intersección de los conjuntos de la forma $\alpha H$
con $\alpha\in[0,1]$ es un conjunto acotado, consideremos un abierto balanceado arbitrario
$U\subseteq X$ tal que $0\in U$. Como el conjunto $H$ es acotado, existe $s>0$
tal que si $t>s$ entonces, $H\subseteq tU$. Ahora bien, para $\alpha\in[0,1]$
se tiene que $\alpha H\subseteq \alpha tU$ y como 
$U$ es balanceado entonces $tU$ también lo es, luego,
como $\alpha\leq1$, entonces $\alpha tU\subseteq tU$. 
Por lo tanto, para $t>s$ se tiene lo siguiente
$$
\bigcap_{\alpha\in[0,1]}\alpha H\subseteq\bigcap_{\alpha\in[0,1]}\alpha tU
\subseteq\bigcap_{\alpha\in[0,1]}tU=tU.
$$
Como el abierto $U$ es arbitrario, entonces queda demostrado que el conjunto 
$\bigcap_{\alpha\in[0,1]} \alpha H$ es acotado.

\end{proof}

\Defi{symm}{
Sea $(X,\Tau)$ un espacio topológico lineal. Se dice 
que el abierto $U\subseteq X$, es simétrico 
si y sólo si $U=-U.$ 
}
\Prp{symm}{
Si $W\subseteq X$ es un entorno abierto de $0\in X$, entonces, 
existe un entorno abierto y simétrico $U\subseteq X$, de 
$0$ tal que $U+U\subseteq W.$
}
\begin{proof}
Sea $W\subseteq X$ un entorno abierto de $0$. Como la operación 
suma es continua y $0=0+0$, entonces, existen entornos abiertos 
$V_1,V_2$ de $0$, tales que
$V_1+V_2\subseteq W.$ 
Definamos $U:= V_1\cap V_2\cap (-V_1)\cap (-V_2).$ Es claro que 
$U$ es un abierto simétrico, y que además
$U+U\subseteq W.$
\end{proof}
\Prp{bddPrp}{
La familia de subconjuntos acotados es cerrada bajo la suma
y el producto por escalares positivos.
}
\begin{proof}
Sean $H_1, H_2\subseteq X$ conjuntos acotados. Veamos que la suma
$H:=H_1+H_2$ es un conjunto acotado. Sea $V\subseteq X$ un entorno abierto de $0\in X$
 y escojamos $U\in\Tau$ tal que $0\in U$ y $U+U\subseteq V$.
Como $H_1$ y $H_2$ son acotados, existen escalares positivos
$t_1$ y $t_2$ tales que
\Eq{*}{
H_1 &\subseteq s_1 U \qquad\mbox{si}\quad s_1>t_1, \\
H_2 &\subseteq s_2 U \qquad\mbox{si}\quad s_2>t_2.
}
Ahora bien, si $s>\max(t_1,t_2)$ entonces,
\Eq{*}{
H=H_1+H_2\subseteq sU + sU = s(U+U) \subseteq sV.
}
Por lo tanto, el conjunto $H$ es acotado. De manera similar
se demuestra que $tH_1$ es acotado para cualquier escalar
positivo $t$.
\end{proof}
\Thm{sep1}{
Suponga que $K$ y $C$ son subconjuntos de un espacio topológico
lineal $X$, $K$ es compacto, $C$ es cerrado y que 
$K\cap C=\emptyset$. Entonces, existe un entorno $V$ de $0\in X$ 
tal que 
\Eq{*}{
(K+V)\cap(C+V)=\emptyset.
}
}
\begin{proof}
Si $K=\emptyset$, entonces $K+V=\emptyset$ y la conclusión del 
teorema es directa. Por lo tanto, asuma que $K\neq\emptyset$.
Sea $x\in K$. Como $K$ y $C$ son disjuntos, se tiene que $x$
no está en $C$, por la \prp{symm}, existe un abierto simétrico 
$V_x\subseteq X$ tal que 
\Eq{*}{
(x+V_x+V_x+V_x)\cap C = \emptyset.
}
Como $V_x$ es simétrico, la condición anterior equivale a
\Eq{cond0}{
(x+V_x+V_x)\cap(C+V_x) = \emptyset.
}
Por otra parte, $K$ es compacto, es decir que admite un 
cubrimiento finito
\Eq{*}{
K\subseteq \bigcup_{i=1}^n(x_i + V_{x_i}).
}
Ahora bien, al considerar $V:= V_{x_1}\cap\cdots\cap V_{x_n},$
se tiene que 
\Eq{*}{
K+V \subseteq \bigcup_{i=1}^n(x_i + V_{x_i}+V)
    \subseteq \bigcup_{i=1}^n(x_i + V_{x_i}+V_{x_i}),
}
y por \eq{cond0}, ningún término en esta última unión
intersecta a $C+V$. Finalmente,
\Eq{*}{
(K+V)\bigcap(C+V) =\emptyset.
}
\end{proof}
Como consecuencia del resultado anterior, se tienen los siguientes teoremas
\Thm{clbase}{
Para todo elemento $B_1\in\U(X)$, existe otro elemento
$B_2\in\U(X)$, tal que
\Eq{*}{
\cl(B_2)\subseteq B_1.
}
}
\begin{proof}
Sea $B_1\in\U(X)$, definamos los conjuntos 
$C:=X\setminus B_1$ y $K:=\{0\}$. Es claro que $C$ es
cerrado, pues, es el complemento de un elemento en $\U(X)$,
y que además, $K$ es compacto. Por otra parte, 
$0\notin C$, es decir, $K\cap C=\emptyset$. Por el 
\thm{sep1}, existe un abierto $W\in\U(X)$, tal que
$W \cap (W+C)=\emptyset$. Ahora bien, esta última condición
implica que $W\cap C=\emptyset$, lo que se reduce a
$W\subseteq B_1$.

Aplicando la \prp{symm} al abierto $W$, se tiene que existe
un abierto simétrico, $B_2\in\U(X)$, tal que
$B_2+B_2\subseteq W$, por lo tanto, se tiene la siguiente
cadena de inclusiones
\Eq{*}{
\cl(B_2)\subseteq B_2+B_2 \subseteq W\subseteq B_1,
} 
la cual, finaliza la demostración.
\end{proof}
\Thm{HausTop}{
Todo espacio topológico lineal, es un espacio de Hausdorff.
}
\begin{proof}
Basta considerar el hecho de que los conjuntos unipuntuales
en un espacio topológico lineal, son cerrados y aplicar
el \thm{sep1}.
\end{proof}
\Thm{f1}{
Sea $X$ un espacio topológico lineal.
\begin{enumerate}[i.]
	\item Si $A\subseteq X$ entonces, 
	$\cl(A)=\bigcap_{V\in\U(X)}(A+V)$, 
	\item Si $A\subseteq X$ y $B\subseteq X$, entonces, 
	$\cl(A) +\cl(B) \subseteq \cl(A+B) = \cl(\cl(A)+\cl(B)).$
	\item Si $A\subseteq X$ es un conjunto acotado, entonces
	$\cl(A)$ también lo será.
\end{enumerate}
}
\begin{proof} Consideremos el espacio topológico lineal
$X$.
\begin{enumerate}[i.]
	\item Sea $A\subseteq X$, veamos que $\cl(A)=\bigcap(A+V)$, 
	donde $V$ recorre todos los entornos abiertos de $0$.
	Ahora bien, $x\in \cl(A)$ si y sólo si, 
	$(x+V)\cap A \neq\emptyset$ para todo entorno abierto 
	$V\in\U(X)$. 
	Pero esta condición se cumple si y sólo si 
	$x\in A+(-V)$, para todo
	abierto $V\in\U(X).$ Además, $V$ es un 
	entorno abierto de $0$, si y sólo si, $-V$ lo es, 
	por lo tanto, 
	\Eq{*}{
	\cl(A) = \bigcap_{V\in\U(X)}(A+V).
	}
	\item Sean $U,V\in\U(X)$ tales que $V+V\subseteq U$. 
	Sean $a\in\cl(A)$ y $b\in\cl(B)$, evidentemente,
	$a+b\in\cl(A)+\cl(B)$, y además,
	\Eq{*}{
	a+b\in\cl(A)+\cl(B)\subseteq A+V+B+V\subseteq A+B+U.
	}
	Como $U$ es arbitrario, se tiene lo siguiente
	\Eq{*}{
	a+b\in\bigcap_{U\in\U(X)}(A+B+U)=\cl(A+B).
	}
	Esto es,
	\Eq{*}{
	\cl(A)+\cl(B)\subseteq\cl(A+B).
	}
	\item Sea $A\subseteq X$ un conjunto acotado. Veamos que
	$\cl(A)$ también, es un conjunto acotado. 
	Considere $U\in \U(X)$, por el \thm{clbase} existe un 
	abierto $W\in\U(X)$ tal que, $\cl(W)\subseteq U$. Como
	$A$ es acotado, existe 
	un número real $s>0$, tal que $A\subseteq tW$,
	para todo número real $t>s$. Por lo tanto
	\Eq{*}{
	\cl(A)\subseteq t\cl(W)\subseteq tU,\qquad (t>s),
	}
	es decir, que $\cl(A)$ es acotado.
\end{enumerate}
\end{proof}

\Prp{sumSets}{
Si $\cl(A)$ es un conjunto compacto entonces
$\cl(A+B)=\cl(A)+\cl(B).$ Esto equivale
a decir que la suma de un conjunto compacto 
con un conjunto cerrado resulta ser un conjunto
cerrado.
}

%%%%%%%%%%%%%%%%%%%%%%%%%%%%%%%%%%%%%%%%%%%%%%%%%%%%%%%%
%%%%%%%%%%%%%%%%%%%%%%%%%%%%%%%%%%%%%%%%%%%%%%%%%%%%%%%%
\section{Conos convexos.}
\setcounter{theorem}{0}
A menos que se especifique otra cosa, $X$ denotará
un espacio topológico lineal.
\Defi{cono}{
El conjunto $K\subseteq X$
es un cono convexo, si 
$K+K\subseteq K, $ y $tK\subseteq K,$ para todo escalar 
$t>0$.
}
%\Defi{K<}{
%Dado un cono 
%$K\subseteq X$, se define la relación $\leq_K$,
%para $x,y\in X$, de la siguiente manera
%\Eq{<K}{
%x\leq_K y \iff y-x\in K\iff y\in x+K.
%}
%}
%De aquí en adelante, $K\subseteq X$ representa
%un cono convexo a menos que se especifique otra cosa.
%\Prp{<trans}{
%La relación $\leq_K$ es antisimétrica y transitiva.
%}
%\begin{proof}
%Veamos que $\leq_K$ es antisimétrica. Sean $x,y\in X$, 
%tales que
%$x\leq_K y$. Por definición esto significa que,
%$y-x\in K$. Pero, esto es equivalente a
%$-x-(-y) \in K$, por lo tanto, $-y\leq_K -x$.
%Veamos ahora que $\leq_K$ es transitiva, para ello,
%sean $a,b,c\in X$ tales que $a\leq_K b$ y $b\leq_K c$.
%Por definición, se tiene que $b-a\in K$ y que $c-b\in K$.
%Ahora bien, usando el hecho de que $K$ es cerrado bajo la 
%suma se sigue que 
%\Eq{*}{
%c-a = (c-b)+(b-a)\in K+K\subseteq K.
%}
%Esto es, precisamente $a\leq_K c$.
%\end{proof}
%\Prp{relOrd}{
%Si $0\in K$, entonces, la relación $\leq_K$,
%es reflexiva.
%}
%\begin{proof}
%La demostración es directa, pues $x-x=0\in K$ para todo
%$x\in X$. 
%\end{proof}
\Defi{Klbdd}{
Se dice que un conjunto $S\subseteq X$ es
$K$-acotado inferiormente, si existe un conjunto acotado
$H,$ tal que, $S\subseteq H+K$.
}
\Prp{KlbUn}{
La unión finita de conjuntos $K$-acotados inferiormente,
es de nuevo un conjunto $K$-acotado inferiormente.
}
\begin{proof}
Basta con probar que la unión de dos conjuntos
$K$-acotados inferiormente es de nuevo un conjunto 
$K$-acotado inferiormente. Sean $S_1,S_2\subseteq X$,
dos conjuntos tales que existen conjuntos acotados 
$H_1,H_2\subseteq X$ que satisfacen 
\Eq{*}{
S_1&\subseteq H_1+K \qquad\mbox{y}\\
S_2&\subseteq H_2+K.
}
Ahora bien, el resultado es consecuencia de la siguiente
cadena de inclusiones
\Eq{*}{
S_1\cup S_2 &\subseteq (H_1+K)\cup(H_2+K) 
								= \bigcup_{h\in H_1}(h+K)\cup
								  \bigcup_{h\in H_2}(h+K) \\
								&= \bigcup_{h\in H_1\cup H_2}(h+K)
								= (H_1\cup H_2) + K.									
}
\end{proof}
\Prp{Klbbop}{
La familia de subconjuntos $K$-acotados inferiormente 
es cerrada bajo la suma y el producto por escalares
positivos.
}
\begin{proof}
Sean $S_1, S_2\subseteq X$, dos conjuntos, $K$-acotados 
inferiormente.
Por definición, existen conjuntos acotados 
$H_1,H_2\subseteq X$ tales que 
\Eq{I0}{
S_1&\subseteq H_1+K \qquad\mbox{y}\qquad
S_2&\subseteq H_2+K. 
}
Por lo tanto
\Eq{*}{
S=S_1+S_2 \subseteq H_1+H_2+K+K \subseteq H_1+H_2+K,
}
pero por la \prp{bddPrp}, el conjunto $H:=H_1+H_2$
es acotado. Luego, $S\subseteq H+K$, es decir, que 
el conjunto $S$ es $K$-acotado inferiormente.

Para ver que $tS_1$ es $K$-acotado inferiormente, basta
con multiplicar la primera inclusión en \eq{I0}
por el escalar $t$ y aplicar de nuevo la \prp{bddPrp}.
\end{proof}
\Defi{sKlbdd}{
Se dice que un conjunto $S\subseteq X$ es
 semi-$K$-acotado inferiormente si existe un conjunto 
acotado $H,$ tal que, $S\subseteq\cl(H+K)$.
}
\Rem{KimpclK}{
De la definición se obtiene de inmediato que todo conjunto 
$S,$ $K$-acotado inferiormente, automáticamente es
semi-$K$-acotado inferiormente.
}
\Prp{sKlbddOp}{
La familia de conjuntos semi-$K$-acotados inferiormente
es cerrada bajo la suma y el producto por escalares 
positivos.
}
\begin{proof}
Sean $S_1,S_2\subseteq X$ conjuntos semi-$K$-acotados
inferiormente. Consideremos un abierto arbitrario 
$V\in\U(X)$. En vista de la \prp{symm} existe un abierto 
$U\in\U(X)$ tal que $U+U\subseteq V$ y
por definición, existen conjuntos acotados 
$H_1,H_2\subseteq X$ tales que 
\Eq{*}{
S_1&\subseteq\cl(H_1+K)\subseteq H_1+K+U,\qquad\mbox{y}\\
S_2&\subseteq\cl(H_2+K)\subseteq H_2+K+U.
}
Luego, usando la convexidad del cono $K$, y el hecho de que
$U+U\subseteq V$, se tiene que para todo abierto $V\in\U(X)$
\Eq{*}{
S_1+S_2\subseteq H_1+H_2+K+K+U+U
			 \subseteq H_1+H_2+K+V=H+K+V,
}
donde $H:=H_1+H_2$. Como $V$ es arbitrario, entonces
\Eq{*}{
S_1+S_2\subseteq\bigcap_{V\in\U(X)}(H+K+V)=\cl(H+K).
}
Finalmente, concluimos que $S_1+S_2$ es semi-$K$-acotado
inferiormente.

Veamos ahora que $tS_1$ también es semi-$K$-acotado
inferiormente, para cualquier escalar $t>0$. 
Sea $V\in\U(X)$ un abierto arbitrario. Por continuidad,
existe un abierto $U\subseteq X$ tal que $tU\subset V$.
Ahora bien, por definición $S_1\subseteq\cl(H_1+K)$, y 
por lo tanto
\Eq{*}{
tS_1\subseteq t\cl(H_1+K)\subseteq tH_1 + tK + tU
		\subseteq tH_1+K+V,
}
pero $H:=tH_1$ es acotado y $V$ es arbitrario, 
en consecuencia
\Eq{*}{
tS_1\subseteq\bigcap_{V\in\U(X)}(H+K+V) = \cl(H+K).
}
Es decir, $tS_1$ es semi-$K$-acotado inferiormente.
\end{proof}
\Prp{SKl=Kl-bdd}{
Si el espacio $X$ es localmente acotado, es decir, si 
existe un abierto $U$ que es acotado,
entonces la familia de conjuntos $K$-acotados inferiormente
y semi-$K$-acotados inferiormente coinciden.
}
\begin{proof}
Basta probar que si $X$ es localmente acotado, entonces
todo subconjunto $S$ de $X$ semi-$K$-acotado 
inferiormente es $K$-acotado inferiormente.
Sea $S\subseteq X$ un conjunto semi-$K$-acotado 
inferiormente y sea $U\in\U(X)$ un conjunto 
abierto y acotado. Por definición existe un conjunto
acotado $H_1\subseteq X$ tal que $S\subseteq \cl(H_1+K)$.
Sea $H:= H_1+U$. Es claro que el conjunto $H$ es acotado
y además,
\Eq{*}{
S\subseteq\cl(H_1+K)\subseteq H_1+K+U = H+K,
}  
en conclusión, el conjunto $S$ es $K$-acotado inferiormente.
\end{proof}
%%%%%%%%%%%%%%%%%%%%%%%%%%%%%%%%%%%%%%%%%%%%%%%%%%%%%%%%%%%
%%%%%%%%%%%%%%%%%%%%%%%%%%%%%%%%%%%%%%%%%%%%%%%%%%%%%%%%%%%

\section{Conjuntos $K$-convexos.}
\setcounter{theorem}{0}
\Defi{KCvx}{
Sea $D\subseteq X$. Se dice que $D$ es $K$-convexo si 
para todo $x,y\in X$ se tiene que $[x,y]\subseteq D+K.$
}
\Rem{0Cvx}{
Si $K=\{0\}$ la definición anterior se reduce
a la definición estandar de convexidad.
}
\Rem{1Cvx}{
Si $0\in K$, entonces $D\subseteq D+K$ para cualquier 
subconjunto $D\subseteq X$. 
}
Como consecuencia inmediata de la observación anterior,
se tiene la siguiente proposición.

\Prp{KCvx-1}{
Sea $K\subseteq X$ un cono convexo tal que $0\in K$. Si
el conjunto $D\subseteq X$ es convexo, entonces,
también es $K$-convexo.
}
En general, un conjunto $K$-convexo, no necesariamente
tiene que ser convexo. 
\Exa{Kcvx1}{
Consideremos $X=\R^2$, $K=[0,\infty)\times \{0\}$ y 
al subconjunto de $X$ dado en coordenadas polares por
 $D:=\{(r,\theta) : 0\leq r\leq1
\quad\mbox{y}\quad    \pi/4\leq\theta\leq7\pi/4\}$. 
El conjunto $D$ es $K$-convexo, pero sin embargo, 
no es convexo.
}
\Exa{Kcvx2}{
Si $0\notin K$ y $D$ es un conjunto convexo, entonces 
no necesariamente, $D$ es $K$-convexo. Para ver esto,
consideremos $X=\R^2,$ $K=(0,\infty)\times\{0\}$ y sea
$D=\{p\}$ con $p=(x_0,y_0)\in\R^2$. Es obvio que el conjunto 
$D$ es convexo, pero $D+K=(x_0,\infty)\times\{y_0\}$
y por lo tanto $D\not\subseteq D+K$, en consecuencia
el conjunto $D$ no es $K$-convexo.
}
\Prp{Kcvx-cvx}{
Supongamos que $0\in K$.
Un conjunto $D\subseteq X$ es $K$-convexo si y sólo si
el conjunto $D+K\subseteq X$ es convexo.
}
\begin{proof}
Sea $t\in[0,1]$. 
Supongamos que $D\subseteq X$ es $K$-convexo, entonces
\Eq{*}{
t(D+K)+(1-t)(D+K) \subseteq tD+(1-t)D + K 
\subseteq (D+K)+K \subseteq D+K,
} 
es decir, que el conjunto $D+K$ es convexo.
Supongamos ahora que el conjunto $D+K$ es convexo,
por lo tanto,
\Eq{*}{
tD+(1-t)D\subseteq tD+(1-t)D+K=t(D+K)+(1-t)(D+K)
\subseteq D+K.
}
Es decir, el conjunto $D$ es $K$-convexo.
\end{proof}
No necesariamente un cono convexo $K$ ha de tener al cero
como uno de sus elementos, sin embargo, 
\Prp{0inclK}{
$0\in\cl(K),$ para cualquier cono convexo $K$.
}
\begin{proof}
Si $0\in K$ la demostración es trivial. Supongamos que
$0\notin K$ y 
sean $V\in\U(X)$ un abierto simétrico y $k\in K$.
Consideremos la sucesión de números reales $(1/2^n)$,
cuyo límite es cero. Por lo tanto, existe un número
natural $N$, tal que si $n\geq N$ entonces
$k/2^n\in V.$ Lo cual es equivalente a
\Eq{*}{
0\in \frac{k}{2^n} + V \subseteq K + V.
}
Como $V$ es arbitrario, se tiene que 
$0\in\bigcap_{V\in\U(X)}(K+V) = \cl(K),$ lo que
completa la prueba. 
\end{proof}
\Defi{recH}{
Dado un subconjunto no vacío $H\subseteq X$, se define el cono
recesión de $H$, $(\rec(H))$ de la siguiente manera
\Eq{recH}{
\rec(H) = \{x\in X : tx+H\subseteq H, \mbox{ para todo }
	t\geq0\}
}
}
La siguiente proposición ilustra algunas propiedades 
del cono recesión asociado al conjunto $H$
\Prp{recH}{
Sea $H\subseteq X$ un conjunto no vacío. Entonces
\begin{enumerate}[i.]
	\item $0\in\rec(H)$ y $\rec(H)$ es un cono convexo.
	\item $K=\rec(H)$ es el cono convexo más grande 
	tal que $K+H\subseteq H$.
	\item $\cl(\rec(H)) \subseteq \rec(\cl(H)).$
	\item Para todo $x\in X$, $t>0$, $\rec(x+tH) = \rec(H)$.
	\item Para cualesquiera conjuntos no vacíos 
	$H_1,H_2\subseteq X$, 
	$$\rec(H_1)+\rec(H_2)\subseteq\rec(H_1+H_2).$$
\end{enumerate}
}
\begin{proof}$\quad$\vspace{-.2cm}
\begin{enumerate}[i.]
	\item En primer lugar, es evidente que $0\in\rec(H)$ pues, 
	$0t+H=H$. Para ver que $\rec(H)$ es convexo, sean
	$x,y\in\rec(H)$ y sea $s\in[0,1]$. Ahora bien, 
	como $x$ e $y$ están en $\rec(H)$, entonces
	$t(1-s)y + H \subseteq H$ y $tsx+H\subseteq H$, para
	cualquier número no-negativo $t$. Por lo tanto,
	\Eq{*}{
	t(sx+(1-s)y) + H = tsx + t(1-s)y + H 
	\subseteq tsx + H \subseteq H.
	}
	Es decir, que el segmento $[x,y]$ está contenido en 
	$\rec(H)$ para cualesquiera $x,y\in H$. De esta manera, 
	se ha demostrado que en efecto, $\rec(H)$ es un conjunto 
	convexo.
	
	\item Supongamos que $K\subseteq X$ es un cono convexo, 
	con la propiedad $K+H\subseteq H$, por lo tanto, 
	para cualquier $t\geq0$ se tiene que
	\Eq{*}{
	tK+H\subseteq K+H \subseteq H,
	}
	lo cual equivale a $K\subseteq\rec(H)$.
	
	\item En vista del \thm{f1}, se tiene que 
	$\cl(\rec(H))+\cl(H)\subseteq\cl(\rec(H)+H)$. 
	Además, de la definición se sigue el hecho de que 
	$\rec(H)+H\subseteq H$. Por lo tanto,
	\Eq{*}{
	\cl(\rec(H))+\cl(H)\subseteq\cl(\rec(H)+H)
	\subseteq\cl(H).
	}
	Como $\cl(\rec(H))$ es un cono convexo, pues 
	$\rec(H)$ lo es, entonces usando el numeral 2 de
	esta proposición, se tiene que 
	\Eq{*}{
	\cl(\rec(H)) \subseteq \rec(\cl(H)).
	}
	
	\item La demostración es directa de la definición.
	
	\item Es consecuencia directa del numeral 2.
\end{enumerate}
\end{proof}

\Prp{L2.2}{
Sean $(A_n)_{n\in\N},(B_n)_{n\in\N}$ dos sucesiones no-decrecientes
de subconjuntos de $X$, sea $H\subseteq X$ un conjunto
acotado, sea $K\subseteq\bigcap_{n\in\N}\cl(\rec(B_n))$ y sea
$(\epsilon_n)_{n\in\N}$ una sucesión de números reales que converge a 
cero. Asumamos que, para todo $n\geq0,$
\Eq{hip2.2}{
A_n\subseteq\cl(\epsilon_nH+K+B_n).
}
Entonces,
\Eq{Incn}{
\cl\Bigg(\bigcup_{n=0}^\infty A_n\Bigg) \subseteq
\cl\Bigg(\bigcup_{n=0}^\infty B_n\Bigg)
}
}
\begin{proof}
Sean $U\in\U(X)$ arbirario y sea $V\in\U(X)$ un abierto
balanceado tal que $V+V+V\subseteq U$. Como $H$ es acotado
y $(\epsilon_n)$ converge a cero, existe $N\in\N$
tal que si $n\geq N$ entonces, $\epsilon_nH\subseteq V$.
Además, como $K\subseteq\bigcap_{n\in\N}\cl(\rec(B_n))$,
se tiene que $K\subseteq\rec(B_n) + V$ para todo $n\in\N$.
De esta manera, para $n\geq N$
\Eq{*}{
A_n&\subseteq\cl(\epsilon_nH+K+B_n)
\subseteq\epsilon_nH+K+B_n+V \\
&\subseteq \rec(B_n)+B_n+V+V+V
\subseteq B_n + U \subseteq U+\bigcup_{n=0}^\infty B_n,
}
por lo tanto,
\Eq{*}{
\bigcup_{n\geq N}A_n \subseteq U+\bigcup_{n=0}^\infty B_n.
}
Como $U$ es arbitrario, y $(A_n)$ es no-decreciente
entonces
\Eq{*}{
\bigcup_{n=0}^\infty A_n\subseteq
\bigcap_{U\in\U(X)}\Bigg(U+\bigcup_{n=0}^\infty B_n\Bigg)
= \cl\Bigg(\bigcup_{n=0}^\infty B_n\Bigg).
}
El resultado se sigue de inmediato ya que el lado derecho de esta 
última inclusión es un conjunto cerrado.
\end{proof}
%%%%%%%%%%%%%%%%%%%%%%%%%%%%%%%%%%%%%%%%%%%%%%%%%%%%%%%%%%%%%%%%%%%%%%%%%%
%%%%%%%%%%%%%%%%%%%%%%%%%%%%%%%%%%%%%%%%%%%%%%%%%%%%%%%%%%%%%%%%%%%%%%%%%%%%%%%%%%%%%%%CHAPTER
%%%%%%%%%%%%%%%%%%%%%%%%%%%%%%%%%%%%%%%%%%%%%%%%%%%%%%%%%%%%%%%%%%%%%%%%%%

\chapter{El teorema de Bernstein--Doetsch}
\label{chapPrevio}

En este capítulo se presentarán las diferentes versiones 
del teorema de Bernstein--Doetsch que serán englobadas más 
adelante como consecuencia directa de los resultados principales que
obtuvimos al realizar este trabajo. Esta línea de investigación
comienza formalmente en el año 1905, con el artículo del matemático 
Danés, Johan Jensen \cite{Jen06}, luego en el año 1915, Bernstein y Doetsch 
publican un artículo donde demuestran el teorema que hoy en día lleva su nombre 
y que ha sido uno de los teoremas más importantes en la teoría de convexidad \cite{Kuc09}. 

Comenzaremos este capítulo dando dos definiciones que serán fundamentales
a lo largo del resto del trabajo.
\Defi{cvxFunc}{
Sea $X$ un espacio normado y sea $D$ un subconjunto convexo de $X$.
Una función $f:D\to\R$ es convexa si para todo $x,y\in D$ 
y para todo $t\in[0,1]$ se tiene que 
\Eq{cvxFunc}{
f(tx+(1-t)y)\leq tf(x)+(1-t)f(y).
}
}
\Defi{midCvx}{
Sea $X$ un espacio normado y sea $D$ un subconjunto convexo de $X$.
Una función $f:D\to\R$ es midconvexa si para todo $x,y\in D$ 
se tiene que
\Eq{midCvxFunc}{
f\bigg(\frac{x+y}{2}\bigg)\leq \frac{f(x)+f(y)}2.
}
}
Es elemental demostrar que si $f$ es una función midconvexa,
entonces $f$ satisface la desigualdad \eq{cvxFunc} para todo
$x,y\in D$ y para todo $t\in[0,1]\cap\Q$ (ver \cite{Kuc09}). 
Por lo tanto, en la clase de las funciones continuas, las nociones
de convexidad y midconvexidad son equivalentes entre sí.
%En este capítulo se pretende dar a conocer algunos resultados previos
%relacionados con funciones convexas, aproximadamente convexas y 
%fuertemente convexas los cuales han sido la motivación de nuestra investigación.

Antes de ir al teorema de Bernstein--Doetsch, no podemos dejar a un lado un resultado 
que se puede decir, sin dudas que dió comienzo a esta linea de investigación. 
El matemático Johan Jensen en 1905, 
publica un artículo en el cual demuestra el siguiente teorema. 

\begin{theorem}[\cite{Jen06}, pg. 188]
\label{TJensenFirst}
Sean $a,b\in\R$, $a<b$. 
Toda función $f:(a,b)\to\R$ acotada superiormente en $(a,b)$ que satisface 
la desigualdad
$$
f\bigg(\frac{x+y}{2}\bigg)\leq \frac{f(x)+f(y)}{2},
$$
para todo $x,y\in (a,b)$ necesariamente tiene que ser continua
y por lo tanto convexa.
\end{theorem}

\section{Versión Original.}
No es sino, diez años más tarde que Felix Bernstein y Gustav Doetsch publican,
en el año 1915, lo que será una generalización del resultado obtenido
por Jensen en 1905 y lo que hoy en día se conoce como el teorema de Bernstein--Doetsch. 
En dicho teorema, los autores debilitan la condición de que la función en cuestión
sea acotada en su dominio y la cambian por una condición más débil, a saber,
la condición de estar localmente acotada superior en solo un punto.
%
%En el libro verde del prof. Marek Kuczma \cite{Kuc09}, se pueden encontrar 
%dos demostraciones de dicho teorema. A continuación presentaremos una de ellas
%en la cual se usan tres resultados importantes que serán englobados en el siguiente 
%teorema
Debido a la importancia del teorema de Bernstein--Doetsch para el desarrollo
de esta investigación, haremos una demostración de dicho resultado. Para ello,
necesitaremos el siguiente lema auxiliar.
\begin{lemma}[\cite{Kuc09}, Teoremas 6.2.1, 6.2.2 y 6.2.3]
\label{LLocUpBdd}
Sean $D\subseteq\R^n$ un conjunto abierto y convexo. Sea $f:D\to\R$ una función 
midconvexa.
Si $f$ es localmente acotada superior en un punto $x_0\in D$, entonces
$f$ es localmente acotada en cada punto de $D$.
\end{lemma}

Vale la pena destacar que la demostración del lema anterior es la 
combinación de tres teoremas, cada uno de los cuales amerita
una demostración detallada. Con esta herramamienta, estaremos en la 
capacidad de demostrar el siguiente teorema.

\begin{theorem}[Teorema de Bernstein--Doetsch,\cite{BerDoe15}]
\label{TBD15}
Sea $N\in\N$ y sea $f:D\subseteq\R^N\to\R$ una función midconvexa. Si
$f$ es localmente acotada superior en un punto de $D$ entonces 
$f$ es continua en $D$.
\end{theorem}

\begin{proof}
El hecho de que la función midconvexa $f$ sea localmente acotada superior
en un punto, implica, según el \lem{LocUpBdd} que la función $f$ tiene que
ser localmente acotada en cada punto de $D$. Por lo tanto, a partir de $f$
podemos definir los siguientes números
\Eq{*}{
m_f(x_0)= \lim_{h\to0}\inf_{B(x_0,h)}{f(x)} \qquad\mbox{y}\qquad
M_f(x_0)= \lim_{h\to0}\sup_{B(x_0,h)}{f(x)}.
}
Note que tanto $m_f$ como $M_f$ son finitos y además, se tiene que para
todo $x\in D$
\Eq{mfM}{
m_f(x) \leq f(x) \leq M_f(x),
}
Ahora bien, sea $x\in D$ arbitrario. Podemos construir un par de sucesiones 
$(x_n)\subseteq D$ y $(z_n)\subseteq D$ tales que 
\Eq{limx}{
\lim_{n\to\infty}x_n=x, \qquad \lim_{n\to\infty}f(x_n) = m_f(x),
}
\Eq{limz}{
\lim_{n\to\infty}z_n=x, \qquad \lim_{n\to\infty}f(z_n) = M_f(x).
}
Para $n\in\N$, sea $y_n = 2z_n-x_n$. Es claro que $\lim y_n = x$, más aún
$z_n=(x_n+y_n)/2$ y por la midconvexidad de $f$ se tiene lo siguiente
\Eq{*}{
f(z_n) \leq \frac{f(x_n)+f(y_n)}2,
}
o equivalentemente 
$$
f(y_n) \geq 2f(z_n)-f(x_n).
$$
Si ahora tomamos el limite inferior en ambos lados de la desigualdad anterior, 
se obtiene lo siguiente
$$
\liminf_{n\to\infty} f(y_n) \geq 2M_f(x)-m_f(x),
$$
pero también,
$$
\liminf_{n\to\infty} f(y_n) \leq M_f(x).
$$
En consecuencia, se tiene que $M_f(x)\leq m_f(x)$. Al combinar esto 
con la desigualdad \eq{mfM} se tiene que $M_f(x)=m_f(x)$ y por lo tanto,
la función $f$ es continua en $x$. Como $x$ fue escogido de manera arbitraria
se tiene que la función $f$ es continua en $D$, y esto finaliza la demostración
del teorema.
\end{proof}



\section{Cambio en la estructura del espacio subyacente.}

Casi 50 años más tarde en 1964, el matemático, M.R. Mehdi \cite{Meh64} obtiene 
el siguiente resultado, que generaliza el Teorema \ref{TBD15}, 
al cambiar la estructura del espacio subyacente.

\Thm{Mehdi}{
Sean $X$ un espacio topológico lineal, $D\subseteq X$ un conjunto abierto
y convexo y sea $f:D\to\R$ una función midconvexa. Si $f$ es acotada superiormente
en un subconjunto abierto no vacío de $D$, entonces $f$ es una función continua.
}

Ahora el teorema de Bernstein--Doetsch es válido, para funciones cuyo
dominio esté sumergido en un espacio lineal topológico. Por otra parte, 
los matemáticos Z. Kominek y M. Kuczma, aseguran en su artículo \cite{KomKuc89b}
que los resultados obtenidos en \cite{Kom87a} y \cite{KomKuc89a} implican
el siguiente teorema y además de eso, ofrecen una generalización inmediata de 
éste y del Teorema \ref{TMehdi}.
\begin{theorem}[\cite{KomKuc89b}, Teorema C]
\label{KK1}
Sean $X$ un espacio lineal, $D\subseteq X$ un conjunto convexo algebraicamente abierto,
y sea $f:D\to\R$ una función midconvexa. Si $f$ es acotada superiormente, en un 
subconjunto no vacío y algebraicamente abierto de $D$, entonces $f$ es continua en $D$
con respecto a la topología de los conjuntos algebraicamente abiertos.
\end{theorem}
\begin{theorem}[\cite{KomKuc89b}, Teorema 1]
\label{KK2}
Sea $X$ un espacio lineal, dotado con una topología semilineal, sea $D\subseteq X$ un 
conjunto abierto y convexo. Sea $f:D\to\R$ una función midconvexa. Si $f$ es acotada
superiormente en un subconjunto abierto y no-vacío de $D$, entonces, $f$ es continua.
\end{theorem}

\section{Convexidad Aproximada.}

En el contexto de convexidad aproximada, D. H. Hyers y S. M. Ulam, en 1952
introducen la definición de $\varepsilon$-convexidad \cite{HyeUla52}, donde $\varepsilon$ es un número
real positivo. Allí los autores establecen que dados $n\in\N$, $\varepsilon>0$, y
una función $f$ a valores reales definida en un subconjunto $D\subseteq\R^n,$ se 
dice que $f$ es $\varepsilon$-convexa si y sólo si, para todo $x,y\in D$ se tiene que
\Eq{eCvx}{
f(tx+(1-t)y)\leq tf(x)+(1-t)f(y)+\varepsilon
}
para todo $t\in(0,1).$ En su artículo, demostraron que toda solución de la
desigualdad \eq{eCvx} está en correspondencia con una función convexa $g$,
tal que $|f-g|\leq\varepsilon$, este resultado es mejor conocido como el teorema
de estabilidad de Hyers y Ulam. 

Siguiendo esta dirección, K. Nikodem junto con Ng. en \cite{NgNik93}
demostraron la siguiente versión del teorema de Bernstein--Doetsch para
funciones $\varepsilon$-convexas que generaliza el Teorema \ref{TBD15}. 
Este resultado también fue establecido independientemente
por Lacskovich en \cite{Lac99}.
\begin{theorem}[\cite{NgNik93}, Teorema 1]
\label{TNgNik}
Sea $D\subseteq X$ un conjunto abierto y convexo de un espacio lineal topológico $X$. 
Si $f:D\to\R$ es una función localmente acotada superior en un punto de $D$, y
$\varepsilon$-midconvexa, i.e, para todo $x,y\in D$ se tiene que
$$
f\bigg(\frac{x+y}{2}\bigg)\leq\frac{f(x)+f(y)}2 + \varepsilon,
$$ 
entonces, $f$ es $2\varepsilon$-convexa.
\end{theorem}  

Diez años más tarde, Zolt Páles en \cite{Pal03a} formula la siguiente
definición: Una función $f$ definida en un subconjunto abierto
y convexo $D$ de un espacio normado real $X,$ es $(\delta,\varepsilon)$-convexa
si satisface
\Eq{edMCvx}{
f(tx+(1-t)y)\leq tf(x)+(1-t)f(y)+\delta t(1-t)\|x-y\|+\varepsilon
}
para todo $x,y\in D$ y $t\in(0,1)$. En su artículo, Z. Páles, obtiene
propiedades de estabilidad de tipo Hyers--Ulam asociadas
con la desigualdad \eq{edMCvx} y estudia las propiedades
que caracterizan este tipo de funciones. Un año depués,
en 2004, buscando más generalizaciones del \thm{BD15}, 
A. Házy y Z. Páles, en \cite{HazPal04} se plantean la siguiente interrogante:
¿Qué propiedades tendrán las funciones $(\delta,\varepsilon)$-midconvexas, 
localmente acotadas? La respuesta a esta interrogante, la encontramos
en los siguientes teoremas.

\begin{theorem}[\cite{HazPal04}, Teorema 3]
\label{THazPal1}
Sea $\delta$ un número no negativo. Si $f:D\to\R$ es $(\delta,0)$-midconvexa,
i.e, $f$ satisface la desigualdad
$$
f\bigg(\frac{x+y}{2}\bigg) \leq \frac{f(x)+f(y)}{2}+\delta\|x-y\|
$$
para todo $x,y\in D$, y si además $f$ es localmente acotada superior en un punto
de $D$. Entonces, $f$ es continua.
\end{theorem}

Además, llegan a la conclusión de que si $\varepsilon$ es un número positivo,
 no se puede garantizar que toda función $(\delta,\varepsilon)$-convexa
localmente acotada superior en un punto sea continua. Sin embargo, plantean
el siguiente teorema que generaliza el \thm{BD15} y el \thm{NgNik}
\begin{theorem}[\cite{HazPal04}, Teorema 4]
\label{THazPal2}
Sean $\delta$ y $\varepsilon$ dos números no negativos. Si $f:D\to\R$ es una función
$(\delta,\varepsilon)$-midconvexa acotada superiormente en un punto de $D$ entonces,
para todo $x,y\in D$ y para todo $t\in(0,1)$
\Eq{HazPal2}{
f(tx+(1-t)y) \leq tf(x)+(1-t)f(y)+2\delta \varphi(t)\|x-y\|+2\varepsilon
}
donde $\varphi$ es un punto fijo del operador $\mathcal{H}:\R^{[0,1]}\to\R^{[0,1]}$
definido por 
\Eq{HazPal3}{
(\mathcal{H}\varphi)(t):=
\left\{
\begin{array}{ll}
\dfrac{\varphi(2t)}{2} + t,& 0\leq t\leq \frac12, \\[.3cm]
\dfrac{\varphi(2t-1)}{2} + (1-t),& \frac12\leq t\leq1.
\end{array}
\right.
}
\end{theorem}
\Rem{takSol}{
Note que la función de Takagi, $T:\R\to[0,1]$ definida en la introducción
mediante la fórmula 
\Eq{Tak}{
T(t) = \sum_{n=0}^{\infty}\frac1{2^n}\dist(2^nt,\Z),\qquad(t\in\R),
}
satisface la siguiente cadena de igualdades
\Eq{*}{
T(2t) =\sum_{n=0}^{\infty}\frac1{2^n}\dist(2^{n+1}t,\Z)
			=2\sum_{n=1}^{\infty}\frac1{2^n}\dist(2^{n}t,\Z)
			=2(T(t)-\dist(t,\Z)).
}
Despejando $T(t)$ en la expresión anterior llegamos a la siguiene
relación
\Eq{Takrec}{
T(t)=\frac12T(2t)+\dist(t,\Z).
}
Ahora bien, si $t\in[0,1/2]$ entonces, $\dist(t,\Z)=t$, por lo tanto
\eq{Takrec} se convierte en 
\Eq{*}{
T(t)=\frac12T(2t)+ t,\qquad 0\leq t\leq\frac12
}
por otra parte, si $t\in(1/2,1]$ entonces, $\dist(t,\Z)=1-t$, y en este
caso \eq{Takrec} se convierte en
\Eq{*}{
T(t)=\frac12T(2t)+ 1-t,
}
además, como consecuencia de las propiedades elementales del ínfimo de un conjunto
se tiene que 
\Eq{*}{
\dist(2t,\Z)=\inf_{\alpha\in\Z}{|2t-\alpha|}
=\inf_{\alpha\in\Z}{|2t-1-(\alpha-1)|}
=\inf_{\beta\in\Z}{|2t-1-\beta|}
=\dist(2t-1,\Z)
}
lo que demuestra que la función de Takagi es 1-periódica y por lo tanto,
si $t\in(1/2,1],$ entonces
$$
T(t)=\frac12T(2t-1)+1-t.
$$ 
Esto significa que la función $T$ es solución
de la ecuación funcional $\mathcal{H}\varphi = \varphi$ y por lo tanto, 
ésta puede ser usada en la conclusión del \thm{HazPal2}.

Usualmente $T$ es conocida como función de ``Vander Waerden'' \cite{Wae30},
sin embargo Knoop \cite{Kno18} ya había descubierto que dicha función ya
había sido construida casi 30 años antes por T. Takagi
\cite{Tak03}. Para más detalles históricos, se pueden consultar los artículos de 
Billingsley \cite{Bil82}, Cater \cite{Cat84} y Kairies \cite{Kai98}.
}
Una pregunta que surge de manera natural luego de ver la conclusión
del \thm{HazPal2} es la siguiente, ¿Cúal será la mejor función que
puede ser colocada en el lugar de $\varphi$ en la desigualdad
\eq{HazPal2}? En respuesta a este planteamiento, A. Házy y Z. Páles en \cite{HazPal04}
conjeturan que la función óptima está dada por el límite de la
sucesión de funciones $\varphi_{n+1}=\mathcal{H}(\varphi_{n})$
$n\in\N$ con $\varphi_1(t)=1$ para todo $t\in(0,1)$, sin embargo,
no logran dar una demostración formal de ello para entonces. 

En 2008, Z. Boros logra resolver satisfactoriamente en su artículo \cite{Bor08},
el problema planteado por Z. Páles en \cite{ISFE41} sobre la 
$(1/2,0)$-midconvexidad de la función de Takagi. En el referido artículo
Boros demuestra que  
\Eq{*}{
T\bigg(\frac{x+y}2\bigg)\leq\frac{T(x)+T(y)}{2}+\frac12|x-y|,\qquad(x,y\in[0,1])
} 
y que por lo tanto, la función óptima que puede ser utilizada en la conclusión del
\thm{HazPal2} es la función de Takagi. 

Con base en lo que acabamos de desarrollar, el \thm{HazPal2}
puede ser reformulado de una manera un poco más simple
\begin{theorem}[\cite{HazPal04}, Teorema 4]
\label{THazPal3}
Sean $\delta$ y $\varepsilon$ dos números no negativos. Si $f:D\to\R$ es una función
$(\delta,\varepsilon)$-midconvexa acotada superiormente en un punto de $D$ entonces,
para todo $x,y\in D$ y para todo $t\in(0,1)$
\Eq{HazPal33}{
f(tx+(1-t)y) \leq tf(x)+(1-t)f(y)+2\delta T(t)\|x-y\|+2\varepsilon
}
donde $T$ es la función de Takagi definida en \eq{Tak}.
\end{theorem}

\subsection{$\alpha(\cdot)$-convexidad.}

A menos que se especifique otra cosa, a lo largo de esta sección
$D$ denotará un subconjunto abierto y convexo de un espacio normado
real $X$.

En el año 2005, A. Házy \cite{Haz05a} introduce el concepto de $(\delta,\varepsilon,p)$-convexidad.
Allí, el autor establece que
una función $f:D\to\R$ es $(\delta,\varepsilon,p)$-convexa, si 
\Eq{*}{
f(tx+(1-t)y)\leq tf(x)+(1-t)f(y)+\delta\|x-y\|^p+\varepsilon
}
para todo $x,y\in D$ y para todo $t\in[0,1]$. En su artículo, Házy
obtiene resultados análogos a los obtenidos por él y Z. Páles
en \cite{HazPal04} y que corresponden al \thm{HazPal1} y al \thm{HazPal3}
de esta sección.

Abriendo el abanico de posibilidades, Jacek Tabor y
Józef Tabor en \cite{TabTab09b}, generalizan las definiciones 
establecidas por Z. Páles y A. Házy anteriormente. En su artículo
ellos introducen la siguiente definición.
\Defi{alphaCvx}{
Dada una función 
$\alpha:[0,\infty)\to[0,\infty)$ 
no decreciente, se dice que una función $f:D\to\R$ es 
$\alpha(\cdot)$-midconvexa si
\Eq{*}{
f\bigg(\frac{x+y}{2}\bigg)\leq\frac{f(x)+f(y)}2+\alpha(\|x-y\|)
}
para todo $x,y\in D$.
}
Note que si $\alpha(u)=\varepsilon+\delta \|u\|^p$, la definición anterior
se reduce a la establecida por A. Házy en \cite{Haz05a}, mientras
que para $p=1$ se reduce a la definición establecida por Z. Páles 
en \cite{Pal03a}. De inmediato, Ja. Tabor y Jó. Tabor,
obtienen una adaptación del teorema de Bernstein--Doetsch
para esta nueva clase de funciones.

\begin{theorem}[\cite{TabTab09b}, Teorema 2.1]
\label{TTabTab1}
Sea $f:D\to\R$ una función $\alpha(\cdot)$-midconvexa y
localmente acotada superior en un punto. Entonces $f$ es localmente
acotada en cada punto de $\mbox{int}D$. Si además, 
$
\ds\lim_{r\to0^+}\alpha(r) = 0,
$
entonces $f$ es continua en $D$.
\end{theorem}
Motivado en los resultados obtenidos por Házy y Páles
en \cite{HazPal04}, Ja. Tabor y Jó. Tabor, establecieron los siguientes
resultados.

\begin{theorem}[\cite{TabTab09b}, Teorema 2.2]
\label{TTabTab2}
Sea $f:D\to\R$ una función $\alpha(\cdot)$-midconvexa. Entonces,
\Eq{TabTab2}{
f(tx+(1-t)y)\leq tf(x)+(1-t)f(y)
+\sum_{n=0}^\infty\frac{1}{2^k}\,\alpha\big(\dist(2^nt,\Z)\|x-y\|\big)
}
para todo $x,y\in D$, $t\in[0,1]\cap\Q$. Más aún, si $f$ es localmente 
acotada superior en un punto de $D$, entonces, la desigualdad
\eq{TabTab2} es válida para todo $t\in[0,1]$.
\end{theorem}

\begin{theorem}[\cite{TabTab09b}, Teorema 3.1]
\label{TTabTab3}
Sea $f:D\to\R$ una función $\alpha(\cdot)$-midconvexa. Entonces,
\Eq{TabTab3}{
f(tx+(1-t)y)\leq tf(x)+(1-t)f(y)
+\sum_{n=0}^\infty\dist(2^nt,\Z)\alpha\bigg(\frac{\|x-y\|}{2^k}\bigg)
}
para todo $x,y\in D$, $t\in[0,1]\cap\D$, donde $\D$ es el conjunto
de los racionales diádicos. Más aún, si $f$ es localmente 
acotada superior en un punto de $D$ y 
$$
\sum_{n=0}^\infty\alpha(1/2^n)<\infty
$$
entonces, $f$ es continua en $[0,1]$ y la desigualdad
\eq{TabTab3} es válida para todo $t\in[0,1]$.
\end{theorem}

Note que el \thm{TabTab2} generaliza el \thm{HazPal2}, mientras que
el \thm{TabTab3} introduce un nuevo término de error para la 
convexidad aproximada de la función $f$. Cabe destacar que 
J. Mako y Z. Páles en \cite[Teorema 26]{MakPal12a} establecen un teorema 
análogo al \thm{TabTab2} pero la demostración es completamente
diferente a la hecha por J. Tabor et. al. en \cite{TabTab09b}.

%la pregunta que surge de forma
%natural es: ¿Cúal de estas estimaciones es la más óptima?. 
%
%La respuesta a dicha interrogante, depende, por supuesto de la función 
%$\alpha$. Tabor et al. y Páles et al. en \cite{TabTab09a} y \cite{MakPal10b}
%respectivamente investigan al respecto enfocándose en el caso particular
%$\alpha(u)=\|u\|^p$ para $u\in D$. Por una parte, Ja. Tabor y Jó. Tabor
%en \cite{TabTab09a} demuestran que cuando $p\in[1,2]$, la estimación
%hecha en el \thm{TabTab2} es óptima. Por otra parte, J. Mako y Z. Páles 
%en \cite{MakPal10b} demuestran que en el caso $p\in(0,1)$

\section{Convexidad fuerte.}

En el año 1966, Boris Polyak en \cite{Pol66} establece la siguiente definición.
\Defi{strgPol}{
Sea $c$ una constante positiva. Se dice que una función $f:D\to\R$ 
es fuertemente convexa con módulo $c$, si satisface 
\Eq{strgPol}{
f(tx+(1-t)y)\leq tf(x)+(1-t)f(y)-ct(1-t)\|x-y\|^2
}
para todo $x,y\in D,$ $t\in[0,1]$.
}
Como consecuencia de la definición anterior, se tiene la siguiente
\Defi{StrMid}{
Sea $c$ una constante positiva. Se dice que una función $f:D\to\R$ 
es fuertemente midconvexa con módulo $c$, si satisface 
\Eq{strgMidPol}{
f\bigg(\frac{x+y}{2}\bigg)\leq \frac{f(x)+f(y)}2-\frac{c}4\|x-y\|^2
}
para todo $x,y\in D,$ $t\in[0,1]$.
}
En el año 2011, A. Azócar et. al. en \cite{AzoGimNikSan11} demuestran
que en la clase de funciones continuas, midconvexidad fuerte
y convexidad fuerte son equivalentes entre sí. 
Un resultado de tipo Bernstein--Doetsch para esta nueva 
clase de funciones fue establecido por A. Azócar et. al.
en \cite{AzoGimNikSan11}, allí los autores demostraron el 
siguiente teorema.
\begin{theorem}[\cite{AzoGimNikSan11}, Teorema 2.3]
\label{azoBD}
Sea $c>0$. Si $f:D\to\R$ es una función fuertemente midconvexa 
con módulo $c$ y acotada superiormente en un subconjunto de $D$ 
con interior no vacío, entonces, $f$ es una función continua
y además fuertemente convexa con módulo $c$.
\end{theorem}

Hasta ahora solo se han presentado resultados de tipo Bernstein--Doetsch
cuando el conjunto de llegada es $\R$. En el Capítulo 
\ref{ChapMultifunciones} mostraremos algunos resultados importantes
de este tipo cuando el codominio tiene una estructura más general.



%Resultados tipo B--D con convexidad aproximada. NgNik93
%Resultados tipo B--D con convexidad fuerte. AzoNikGimSan11
%Resultados tipo B--D para h-convexidad ??
%Resultados tipo B--D para convexidad aproximada generalizada HazPal, TabTab, 
%resultados de trudzik,
%quasy-convex
%algebraic version




\chapter{Multifunciones.}
\label{ChapMultifunciones}
\setcounter{theorem}{0}
En el siguiente capítulo, se establece el concepto
de multifunción o función conjunto-valuada junto con
algunas propiedades importantes. Las definiciones aquí establecidas
han sido tomadas del libro de análisis conjunto-valuado 
de Aubin et. al. \cite{AubPieFra09}.

\section{Definiciones Básicas.}
A menos que se especifique otra cosa, $X$ y $Y$ denotarán
espacios topológicos lineales.
\Defi{SVM}{
Si a cada $x\in X$ le corresponde un subconjunto
$F(x)\in\P(Y)$, se dice que $F$ es una multifunción
de $X$ en $Y$ y simplemente la denotaremos como 
$F:X\to\P(Y).$
}
\Defi{domSVM}{
El dominio de la multifunción $F:X\to\P(Y)$ es el conjunto
$$
\Dom(F):= \{x\in X \quad|\quad F(x)\neq\emptyset\}.
$$
}
\Exa{svm1}{
El primer ejemplo de una multifunción, surge naturalmente
a partir de una función dada $f:X\to Y$. Defínase 
$F:Y\to\P(X)$, tal que 
$$F(y) = f^{-1}(y)= \{x\in X \quad|\quad y=f(x)\}.$$
Note que $F(y)\neq\emptyset$ si y sólo si, $y\in f(X)$,
por lo tanto, $\Dom(F) = f(X)$.
}
A menos que se especifique otra cosa,
$F$ denotará a una multifunción de $X$ en $Y$ y al dominio
de $F$ lo denotaremos por $D$.
\Defi{grSVM}{
El gráfico de una multifunción lo denotaremos por $\Gr(F)$
y se define como
\Eq{*}{
\Gr(F):=\{(x,y)\in X\times Y \quad|\quad y\in F(x)\}.
}
}
El concepto de convexidad para una multifunción $F$
está relacionado con su gráfico, en este sentido, se tiene
la siguiente definición.
\Defi{cvxSVM}{
Se dice que la multifunción $F$ es convexa, si $\Gr(F)$ es 
un subconjunto convexo de $X\times Y$.
}
\Prp{cvxSVM1}{
La multifunción $F$ es convexa si y sólo si para todo $x_1,x_2\in X$
y para todo $t\in[0,1]$, se tiene que 
\Eq{cvxInc}{
tF(x_1) + (1-t)F(x_2)\subseteq F(tx_1+(1-t)x_2).
}
}
\begin{proof}
Supongamos que $F$ es convexa, ie, $\Gr(F)$ es un conjunto convexo de 
$X\times Y.$ Sean $x_1,x_2\in X$ y consideremos $z\in tF(x_1) + (1-t)F(x_2)$.
Por lo tanto, existen $y_1\in F(x_1)$, $y_2\in F(x_2)$ tales que, 
$
z=ty_1+(1-t)y_2.
$
Como $\Gr(F)$ es convexo, entonces, para todo $t\in[0,1]$
\Eq{*}{
t(x_1,y_1)+(1-t)(x_2,y_2)=(tx_1+(1-t)x_2,ty_1+(1-t)y_2)\in\Gr(F),
}
es decir,
$z=ty_1+(1-t)y_2\in F(tx_1+(1-t)x_2)$, para todo $t\in[0,1]$. Lo que
demuestra que la inclusión \eq{cvxInc} es válida. El recíproco, se demuestra
de manera análoga.
\end{proof}

\Prp{HcvxS}{
Si $H\subseteq X$ es un subconjunto estrellado de $X$ con respecto al
origen, entonces, para $0<a<b$ se tiene que $aH\subseteq bH$.
}
\begin{proof}
Como el conjunto $H$ es estrellado con respecto a $0\in H$, se tiene que para 
todo $\alpha\in[0,1]$ y para todo $h\in H$, $\alpha h + (1-\alpha)0\in H$.
Por lo tanto, para $\alpha\in[0,1]$ se tiene que $\alpha H\subseteq H$.
Ahora bien, como $0<a<b$ entonces $0<\frac{a}{b}<1$ y por lo tanto
$aH=b\bigg(\dfrac{a}{b}H\bigg)\subseteq bH,$ lo que completa la demostración. 
\end{proof}
\Rem{HcvxS}{
La proposición anterior es válida si $a<b<0$, pero no necesariamente
lo es cuando $a<0<b$. Note que si $a<b<0$ entonces, $0<\frac{a}{b}<1$
y por lo tanto se puede aplicar el mismo m\'etodo que se aplic\'o en
la demostraci\'on anterior. Para el caso $a<0<b$, basta considerar
$X=\R$, $H=[0,1]$, $a=-1$ y $b=1$.  
}

\Exa{B1}{
Sea $H\subseteq \R^3$ un conjunto convexo y no vacío que posee al origen de $\R^3.$
Sea $G:\R\to\P(\R^3)$ la multifunción definida por $G(x):= -x^2H$. 
Siendo $g(x) = -x^2$ una función cóncava, se tiene que para todo $t\in[0,1]$
y para todo $x,y\in\R$
$$
-(tx^2+(1-t)y^2)\leq -(tx+(1-t)y)^2.
$$ 
Como el conjunto $H$ es convexo y posee al origen, podemos aplicar la
\prp{HcvxS} para llegar a la siguiente inclusión
$$
-(tx^2+(1-t)y^2)H\subseteq -(tx+(1-t)y)^2H=G(tx+(1-t)y).
$$ 
De nuevo, usando la convexidad del conjunto $H$ se obtiene que 
$$
-(tx^2+(1-t)y^2)H=-tx^2H-(1-t)y^2H=tG(x)+(1-t)G(y),
$$
y por lo tanto
$$
tG(x)+(1-t)G(y)\subseteq G(tx+(1-t)y).
$$
}
\Defi{midCvxSVM}{
La multifunción $F$ es midconvexa o Jensen-convexa si satisface
la inclusión \eq{cvxInc} para $t=1/2$, es decir
\Eq{midCvxSVM}{
\frac{F(x)+F(y)}{2}\subseteq F\bigg(\frac{x+y}{2}\bigg).
}
}
\Exa{cvxSVM1}{
Sean $f,g:[a,b]\subseteq\R\to\R$ dos funciones tales que $f(x)<g(x)$
para todo $x\in[a,b]$ y que además $f$ y $-g$ son funciones convexas,
es decir que para todo $x,y\in[a,b]$ y $t\in[0,1]$ se satisface lo siguiente
\Eq{cvxF}{
f(tx+(1-t)y)&\leq tf(x) + (1-t)f(y) \\
tg(x) + (1-t)g(y)&\leq g(tx+(1-t)y).
}
Entonces, mediante un cálculo elemental, se puede ver que
la multifunción $F:[a,b]\to\mathcal{\R}$ definida por la fórmula 
$F(x):=[f(x),g(x)]$, para $x\in [a,b]$ es convexa. 
}
Así como en el caso de funciones a valores reales, también es posible generalizar
el concepto de concavidad para una multifunción. 
\Defi{ccvSVM}{
Se dice que la multifuncion $F$ es cóncava en $D$ si para todo $x,y\in D$
y para todo $t\in[0,1]$ se tiene que 
\Eq{ccvSVM}{
F(tx+(1-t)y)\subseteq tF(x)+(1-t)F(y)
}
}
Recordemos en el caso de las funciones, se tiene que una función $f$ es 
cóncava si y sólo si, la función $-f$ es convexa y esta caracterización
permite extender los resultados obtenidos para funciones convexas a funciones
cóncavas sin mayores dificultades. Lamentablemente, para el caso de multifunciones
esta caracterización no existe y el siguiente ejemplo lo ilustra de manera sencilla.

\Exa{Fcvx-Fcvx}{
Para $r>0$, se denotará por $D(r)$ al disco cerrado de radio $r$ centrado
en el origen, esto es 
$$
D(r):=\{(x,y)\in\R^2\quad|\quad x^2+y^2\leq r^2\}.
$$
Sea $F:[0,\infty)\to\P(\R^2)$ la multifunción definida por
$F(r):= D(r)$ para $r\geq0$. El gráfico de la multifunción $F$
es el conjunto
\Eq{*}{
\Gr(F)&=\{(r,(x,y))\in[0,\infty)\times\R^2\quad|\quad (x,y)\in F(r)\} \\
      &=\{(r,(x,y))\in[0,\infty)\times\R^2\quad|\quad x^2+y^2\leq r^2\}.
}
Es claro que al conjunto $\Gr(F)$ 
lo podemos representar como un cono en $\R^3$ con
vértice en el origen. Además, la multifunción $G=-F$ por
simetría posee el mismo gráfico de $F$ y esto significa
que sigue siendo una multifunción convexa.
}

Es decir que, si $F$ es convexa no necesariamente se tiene que
$-F$ es cóncava. Esto trae como consecuencia, que los resultados
obtenidos para multifunciones convexas no puedan extenderse directamente 
a resultados para multifunciones cóncavas y por lo tanto, ambas direcciones
deben tratarse por separado.

\Exa{cvxValuedNoCvx}{
Si las imagenes de una multifunción $F$ son conjuntos convexos,
no necesariamente ella es convexa. Si definimos
$F(r):=D(|r|)$ para $r\in\R$, es claro que $F(r)$
es un conjunto convexo para todo $r\in\R$, sin embargo,
el gráfico de dicha multifunción es un bicono en $\R^3$ con
vértice en el origen el cual no es un conjunto convexo.
}

\Rem{uniCase}{
Si $f:D\to Y$ es una función a valores en $Y$ y $F:D\to\P(Y)$
se define como $F(x):=\{f(x)\}$ para $x\in D$, entonces, la inclusión
\eq{cvxInc} se transforma en
$$
t\{f(x)\}+(1-t)\{f(y)\}\subseteq \{f(tx+(1-t)y)\}.
$$
Pero esto se cumple si y sólo si $f(tx+(1-t)y)=tf(x)+(1-t)f(y)$
para todo $t\in[0,1]$ y para todo $x,y\in D$. Note que 
las funciones lineales, tienen dicha propiedad. Aquí se evidencia
que la definición de convexidad dada por la inclusión 
\eq{cvxInc} no generaliza la \defi{cvxFunc}.
}

El siguiente Teorema es una generalización de un teorema 
clásico en la teoría de funciones convexas. Para su demostración
vamos a necesitar el siguiente lema, también conocido como
ley de cancelación de R\aa dstrom.

\begin{lemma}[\cite{Rad52}, Lema 1.]
\label{LRad52}
Sean $A,B$ y $C$ conjuntos dados de un espacio lineal topológico.
Suponga que $B$ es cerrado y convexo, $C$ es acotado y que además
$A+C\subseteq B+C$. Entonces, $A\subseteq B$.
\end{lemma}

\Thm{loc1}{
Sea $F:D\to \P_0(Y)$ una multifunción midconvexa tal que para todo
$x\in D$, el conjunto $F(x)$ es cerrado, convexo y acotado. Entonces,
para cualquier colección de puntos $x_1,\ldots,x_n$ en $D$
con $n\in\N$, se tiene que 
\Eq{loc1}{
\frac1n(F(x_1)+\cdots+F(x_n))\subseteq F\bigg(\frac{x_1+\cdots x_n}{n}\bigg).
}
}
\begin{proof}
Se puede demostrar fácilmente por inducción que para $p\in\N$, 
la multifunción $F$ satisface la inclusión
\Eq{loc11}{
\frac1{2^p}(F(x_1)+\cdots+F(x_{2^p}))\subseteq F\bigg(\frac{x_1+\cdots x_{2^p}}{2^p}\bigg).
}
Ahora bien, para $n\in\N$ fijo, consideremos $p\in\N$ tal que $n\leq2^p$.
Sean $x_1,\ldots,x_n$ puntos arbitrarios en $D$ y definamos
$x_k:=\bar{x}:=(x_1+\cdots+x_n)/n$ para $k=n+1,\ldots,2^p$.

Observe que con esta notación se tiene  
$\bar{x}=(x_1+\cdots+x_{2^p})/2^p$ y por lo tanto la inclusión \eq{loc11}
se convierte en
\Eq{loc12}{
\frac1{2^p}(F(x_1)+\cdots+F(x_n)+F(x_{n+1})+\cdots+F(x_{2^p}))\subseteq F(\bar{x}).
}
Como $x_k:=\bar{x}$ para $k=n+1,\ldots,2^p$ entonces, 
$F(x_{n+1})=\cdots=F(x_{2^p})=F(\bar{x})$ por lo tanto, 
la convexidad de las imágenes de la multifunción $F$ 
nos da la siguiente fórmula
\Eq{*}{
F(x_{n+1})+\cdots+F(x_{2^p})=(2^p-n)F(\bar{x}).
}
Luego, la inclusión \eq{loc12} equivale a
\Eq{loc13}{
\frac1{2^p}\big(F(x_1)+\cdots+F(x_n)+(2^p-n)F(\bar{x})\big)\subseteq F(\bar{x}).
}
Multiplicando \eq{loc13} por $2^p$ y usando de nuevo la convexidad de las imágenes de la 
multifunción $F$ se tiene lo siguiente
\Eq{*}{
F(x_1)+\cdots+F(x_n)+(2^p-n)F(\bar{x})\subseteq (2^p-n)F(\bar{x})+nF(\bar{x}).
}
Aplicando el \lem{Rad52} a la inclusión anterior se obtiene \eq{loc1} y esto
completa la demostración.
\end{proof}

\Cor{loc}{
Bajo las hipótesis del \thm{loc1} se cumple que para
todo $x,y\in D$ y para todo $t\in[0,1]\cap\Q$ 
\Eq{*}{
tF(x)+(1-t)F(y)\subseteq F(tx+(1-t)y).
}
}

\section{$K$-convexidad y $K$-concavidad de multifunciones.}

Sea $K\subseteq Y$ un cono convexo, las siguientes definiciones generalizan
las definiciones de convexidad y concavidad dadas en la sección anterior.
\Defi{K-cvxSVM}{
Se dice que la multifunción $F$ es $K$-convexa en $D$ si para todo $x,y\in D$ 
y para todo $t\in[0,1]$
\Eq{Kcvxsvm}{
tF(x)+(1-t)F(y)\subseteq F(tx+(1-t)y)+K
}
}
\Defi{K-ccvSVM}{
Se dice que la multifunción $F$ es $K$-cóncava en $D$ si para todo $x,y\in D$ 
y para todo $t\in[0,1]$
\Eq{Kccvsvm}{
F(tx+(1-t)y)\subseteq tF(x)+(1-t)F(y)+K
}
}

\Prp{nikHabil1}{
Si $K\subseteq Y$ es un cono que posee al origen, entonces,
una multifunción $F:X\to\P_0(Y)$ es $K$-convexa (resp. $K$-cóncava) si y sólo si
la multifunción $F+K:X\to \P_0(Y)$ definida por $(F+K)(x):=F(x)+K$
es convexa (resp. cóncava).
}
\begin{proof}
Supongamos que la multifunción $F$ es $K$-convexa, es decir, que para todo
$x,y\in D$ y para todo $t\in[0,1]$ se satisface la siguiente inclusión
\Eq{*}{
tF(x)+(1-t)F(y)\subseteq F(tx+(1-t)y)+K.
}
Ahora bien, para $x,y\in D$ y para $t\in[0,1]$ 
\Eq{*}{
t(F+K)(x)+(1-t)(F+K)(y) &= t(F(x)+K)+(1-t)(F(y)+K)\\
&=tF(x)+tK+(1-t)F(y)+(1-t)K
}
pero como $K$ es un cono convexo entonces, 
$tK+(1-t)K\subseteq K$ para todo $t\in[0,1]$ por lo tanto,
\Eq{*}{
t(F+K)(x)+(1-t)(F+K)(y)\subseteq tF(x)+(1-t)F(y)+K.
}
Finalmente, la $K$-convexidad de $F$ implica que 
\Eq{*}{
tF(x)+(1-t)F(y)+K&\subseteq F(tx+(1-t)y)+K+K \\
&\subseteq F(tx+(1-t)y)+K=(F+K)(tx+(1-t)y),
}
y así
\Eq{*}{
t(F+K)(x)+(1-t)(F+K)(y)\subseteq (F+K)(tx+(1-t)y),
}
lo que demuestra que la multifunción $(F+K)$ es convexa.

Recíprocamente, supongamos que la multifunción $(F+K)$ es
convexa. Como $0\in K$, entonces, para $t\in[0,1]$ y para 
$x,y\in D$,
\Eq{*}{
tF(x)+(1-t)F(y)\subseteq tF(x)+(1-t)F(y)+K.
}
Por otra parte, usando la convexidad del cono $K$, se sigue
que $K=tK+(1-t)K$ y así
\Eq{*}{
tF(x)+(1-t)F(y)+K &= t(F(x)+K)+(1-t)(F(y)+K) \\ 
&= t(F+K)(x)+(1-t)(F+K)(y).
}
Ahora bien, como $(F+K)$ es convexa entonces
\Eq{*}{
t(F+K)(x)+(1-t)(F+K)(y) \subseteq (F+K)(tx+(1-t)y).
}
Luego, para todo $t\in[0,1]$ y para todo $x,y\in D$
\Eq{*}{
tF(x)+(1-t)F(y)\subseteq F(tx+(1-t)y)+K,
}
lo que finaliza la demostración.
\end{proof}

\Defi{K-midCvxSVM}{
Se dice que la multifunción $F$ es $K$-midconvexa en $D$ si para todo $x,y\in D$ 
$F$ satisface la inclusión \eq{Kcvxsvm} para $t=\frac12$, i.e,
\Eq{Kmcvxsvm}{
\frac{F(x)+F(y)}2\subseteq F\bigg(\frac{x+y}2\bigg)+K.
}
}

\Defi{K-midCcvSVM}{
Se dice que la multifunción $F$ es $K$-midcóncava en $D$ si para todo $x,y\in D$ 
$F$ satisface la inclusión \eq{Kccvsvm} para $t=\frac12$, i.e,
\Eq{Kmccvsvm}{
F\bigg(\frac{x+y}2\bigg)\subseteq \frac{F(x)+F(y)}2+K.
}
}
\Rem{cvxSV}{
Dado un cono convexo $K\subseteq Y,$ definamos la relación $\leq_K$ en $Y$
de la siguiente manera
$$
x\leq_K y \iff y-x\in K.
$$
Si $F(x)=\{f(x)\}$, donde $f:X\to Y$ es una función cualquiera, entonces las 
inclusiones \eq{Kcvxsvm} y \eq{Kccvsvm} equivalen a 
\Eq{*}{
f(tx+(1-t)y)\leq_K tf(x)+(1-t)f(y)
\quad\mbox{y}\quad
tf(x)+(1-t)f(y)\leq_K f(tx+(1-t)y)
}
respectivamente. De hecho si $Y=\R$ y $K=\R_+$, entonces estas definiciones
coinciden con las definiciones de convexidad y concavidad de funciones 
respectivamente introducidas en el Capítulo 2.
}
%Muchos resultados importantes relacionados con multifunciones $K$-convexas 
%y $K$-cóncavas pueden ser encontrados en la tesis doctoral del
%profesor Kazimierz Nikodem \cite{Nik89}.
%
%En los resultados principales las siguientes definiciones serán importantes

Como es de esperarse, no toda multifunción $K$-midconvexa es convexa. Sin
embargo, K. Nikodem en \cite{Nik86} demuestra el siguiente resultado

\begin{theorem}
\label{TNik860}
Sea $F:D\to \P_0(Y)$ es una multifunción $K$-midconvexa, entonces
$F$ satisface la inclusión \eq{Kcvxsvm} para todo $x,y\in D$
y para todo $t\in\D\cap[0,1]$.
\end{theorem}

\Defi{Klbddsvm}{
Se dice que la multifunción $F$ es localmente $K$-acotada inferior en $x_0\in D$, si existe
un abierto $U\in\U(X)$ y un conjunto acotado $H\subseteq Y$ tales que 
\Eq{*}{
F(u)\subseteq H+K,\qquad\mbox{para todo }u\in (x_0+U)\cap D.
}
}
\Defi{Kubddsvm}{
Se dice que la multifunción $F$ es localmente $K$-acotada superior en $x_0\in D$, 
si $F$ es localmente $(-K)$-acotada inferior en dicho punto.
}
\Exa{Klwbdd}{
Definamos $F(x):=(\sin(x),\infty),$ $x\in\R$. Evidentemente, para todo $x\in\R$ 
el conjunto $F(x)$ no es acotado. Sin embargo, al considerar 
$K:=(0,\infty)$ se tiene que $F(x)\subseteq \{-1\}+K$ para todo $x\in \R$ por lo tanto,
esta multifunción así definida es $K$-acotada inferiormente en todo $\R$. 
}
\Defi{clKlbddsvm}{
Se dice que la multifunción $F$ es localmente semi-$K$-acotada inferior en $x_0\in D$, 
si existe un abierto $U\in\U(X)$ y un conjunto acotado $H\subseteq Y$ tales que 
\Eq{*}{
F(u)\subseteq \cl(H+K),\qquad\mbox{para todo }u\in (x_0+U)\cap D.
}
}
\Rem{equiv}{
Si $Y$ es un espacio localmente acotado, i.e., existe un abierto $V\in\U(Y)$ que es acotado,
entonces la \defi{Klbddsvm} y la \defi{clKlbddsvm} son equivalentes.
Basta ver que si $F(u)\subseteq\cl(H+K)$ entonces,
$F(u)\subseteq H+V+K$. Siendo $H+V$ un conjunto acotado,
se tiene que $F$ es localmente $K$-acotada inferior.
}
\Defi{weakKlbddsvm}{
Se dice que la multifunción $F$ es localmente débil-$K$-acotada superior en $x_0\in D$, si existe
un abierto $U\in\U(X)$ y un conjunto acotado $H\subseteq Y$ tales que 
\Eq{*}{
0\in F(u)+H+K,\qquad\mbox{para todo }u\in (x_0+U)\cap D.
}
}
\Defi{weakclKlbddsvm}{
Se dice que la multifunción $F$ es localmente débil-semi-$K$-acotada superior en $x_0\in D$, 
si existe un abierto $U\in\U(X)$ y un conjunto acotado $H\subseteq Y$ tales que 
\Eq{*}{
0\in\cl(F(u)+H+K),\qquad\mbox{para todo }u\in (x_0+U)\cap D.
}
}
\Prp{lem2.3}{
Supongamos que $F:D\to\P_0(Y)$ es una multifunción localmente semi-$K$-acotada inferior. 
Entonces, para cada subconjunto compacto $C\subseteq D$, existe un conjunto acotado 
$H\subseteq Y$ tal que, $F(x)\subseteq\cl(H+K),$ para todo $x\in C$.
}
\begin{proof}
Sea $C\subseteq D$ un conjunto compacto no-vacío.
Como la multifunción $F$ es localmente semi-$K$-acotada inferior, entonces, para cada
$x\in C$ existe un abierto $U_x\in \U(X)$ y un conjunto acotado $H_x\subseteq Y$ tales que
para todo $u\in (x+U_x)\cap D$, $F(u)\subseteq\cl(H_x+K)$. Ahora bien, es claro que 
$C\subseteq \bigcup_{x\in C}(x+U_x)$, es decir que la familia $\{x+U_x:x\in C\}$ es un 
cubrimiento por abiertos del compacto $C$, por lo tanto, existen $x_1,\ldots,x_n\in C$
tales que $C\subseteq\bigcup_{i=1}^n(x_i+U_{x_i})$. Sea $H:=H_{x_1}\cup\cdots\cup H_{x_n}$,
es claro que $H$ es un conjunto acotado, pues es la unión finita de conjuntos acotados. Además,
si $z\in C$ es un elemento arbitrario de $C$ entonces $z\in x_i+U_{x_i}$ para algún $i=0,\ldots,n$
y por lo tanto $F(z)\subseteq\cl(H_{x_i}+K)\subseteq\cl(H+K)$. Lo cual, finaliza la demostración. 
\end{proof}

\Prp{lem2.4}{
Supongamos que $F:D\to\P_0(Y)$ es una multifunción localmente débil-semi-$K$-acotada superior. 
Entonces, para cada subconjunto compacto $C\subseteq D$, existe un conjunto acotado 
$H\subseteq Y$ tal que, $0\in\cl(F(x)+H+K),$ para todo $x\in C$.
}
\begin{proof}
Sea $C\subseteq D$ un conjunto compacto no-vacío.
Como la multifunción $F$ es localmente débil-semi-$K$-acotada superior, entonces para cada
$x\in C$ existe un abierto $U_x\in \U(X)$ y un conjunto acotado $H_x\subseteq Y$ tales que
para todo $u\in (x+U_x)\cap D$, $0\in\cl(F(u)+H_x+K)$. Ahora bien, es claro que 
$C\subseteq \bigcup_{x\in C}(x+U_x)$, es decir que la familia $\{x+U_x:x\in C\}$ es un 
cubrimiento por abiertos del compacto $C$, por lo tanto, existen $x_1,\ldots,x_n\in C$
tales que $C\subseteq\bigcup_{i=1}^n(x_i+U_{x_i})$. Sea $H:=H_{x_1}\cup\cdots\cup H_{x_n}$,
es claro que $H$ es un conjunto acotado, pues es la unión finita de conjuntos acotados. Además,
si $z\in C$ es un elemento arbitrario de $C$ entonces $z\in x_i+U_{x_i}$ para algún $i=0,\ldots,n$
y por lo tanto, $0\in\cl(F(z)+H_{x_i}+K)\subseteq\cl(F(z)+H+K)$. Lo cual, finaliza la demostración. 
\end{proof}

\section{El teorema de Bernstein--Doetsch para multifunciones.}

Con el fin de establecer algunos resultados importantes relacionados con
el Teorema de Bernstein--Doetsch en el contexto de multifunciones convexas,
es necesario establecer las siguientes definiciones.

\Defi{KupSemConSVM}{
Sea $F:D\to\P_0(Y)$ una multifunción. Se dice que $F$ es $K$-semicontinua superior si
para todo abierto $V\in \U(Y)$, existe un abierto $U\in\U(X)$ tal que
\Eq{KupSemConSVM}{
F(x)\subseteq F(x_0)+V+K\qquad(x\in x_0+U).
}
}
\Defi{KloSemConSVM}{
Sea $F:D\to\P_0(Y)$ una multifunción. Se dice que $F$ es $K$-semicontinua inferior si
para todo abierto $V\in \U(Y)$, existe un abierto $U\in\U(X)$ tal que
\Eq{KloSemConSVM}{
F(x_0)\subseteq F(x)+V+K\qquad(x\in x_0+U).
}
}
\Defi{KConSVM}{
Sea $F:D\to\P_0(Y)$ una multifunción. Se dice que $F$ es $K$-continua si
$F$ es $K$-semicontinua superior e inferior al mismo tiempo.
}
\Defi{DirConSup}{ %continuidad direccional superior
	Sea $F:D\to\P_0(Y)$ una multifunci\'on. Decimos que $F$ es
	\emph{direccionalmente $K$-semicontinua superior en el punto $p\in D$} 
	si, para cualquier direcci\'on $h\in X$ 
	y para todo entorno abierto $U\in\U(Y)$, existe un n\'umero positivo
	$\delta$ tal que
	\Eq{*}{
		F(p+th)\subseteq F(p)+U+K
	}
	para todo $0<t<\delta$ con $p+th\in D$. En el caso particular cuando
	$K=\{0\}$, nos referiremos a esta definici\'on como 
	\textit{semicontinuidad direccional superior en $p$}. 
}
Claramente, toda multifunci\'on direccionalmente semicontinua superior
es direccionalmente $K$-semicontinua superior para cualquier
cono $K$. An\'alogamente, tenemos la siguiente 
\Defi{DirConInf} {
	Sea $F:D\to\P_0(Y)$ una multifunci\'on. Decimos que $F$ es
	\emph{direccionalmente $K$-semicontinua inferior en el punto $p\in D$} 
	si, para cualquier direcci\'on $h\in X$ 
	y para todo entorno abierto $U\in\U(Y)$, existe un n\'umero positivo
	$\delta$ tal que
	\Eq{*}{
		F(p)\subseteq F(p+th)+U+K
	}
	para todo $0<t<\delta$ con $p+th\in D$. En el caso particular cuando
	$K=\{0\}$, nos referiremos a esta definici\'on como 
	\emph{semicontinuidad direccional inferior en $p$}.
}
\Defi{ContDir} { % Continuidad direccional
	Si $F$ es al mismo tiempo direccionalmente $K$-semicontiua superior e inferior
	en $p$, entonces diremos que $F$ es \emph{direccionalmente $K$-continua en $p$.}	
}
	
\Lem{dusc}{
	Sea $K\subseteq Y$ un cono convexo y $S,T\subseteq Y$ conjuntos no-vac\'ios 
	y semi-$K$-acotados inferiormente que son semi-$K$-estrellados con respecto a algunos 
	elementos de $Y$. Entonces, la multifunci\'on $t\mapsto tS+(1-t)T$ 
	es direccionalmente $K$-continua en $[0,1]$.
}
\begin{proof}
	Sea $U\in\U(X)$. Es suficiente demostrar que existe un n\'umero real positivo
	$\delta$ tal que, para $s,t\in[0,1]$ con $|t-s|<\delta$,  
	\Eq{st}{
		tS+(1-t)T\subseteq sS+(1-s)T + U+K.
	}
	Primero, escojamos $V\in\U(Y)$ tal que $[3]V\subseteq U$.
	Obviamente, \eq{st} es v\'alida para $s=t$. Sin p\'erdida de generalidad,
	podemos asumir que $0\leq s<t\leq 1$. Supongamos que  $T$ es semi-$K$-estrellado con 
	respecto a $v\in Y$. Entonces, tenemos
	\Eq{est}{
		tS+(1-t)T
		&\subseteq sS + (t-s)S + (1-t)T + (t-s)v+(s-t)v \\
		&= sS + (t-s)S + (1-s)\bigg(\frac{1-t}{1-s}T+ \frac{t-s}{1-s}v\bigg)+(s-t)v \\
		&\subseteq sS + (t-s)(S-v) + (1-s)\cl(T+K) \\
		&\subseteq sS + (1-s)T + (t-s)(S-v) +V+K.
	}
	Como el conjunto $(S-v)$ es semi-$K$-acotado inferiormente, existe un subconjunto 
	acotado $H\subseteq Y$ tal que $S-v\subseteq\cl(H+K)$. Como el conjunto $H$
	es acotado, existe un n\'umero positivo $\delta\leq 1$ tal que $\delta H\subseteq V$.
	Por lo tanto, si $t-s<\delta$, entonces 	
	\Eq{*}{
		(t-s)(S-v)\subseteq (t-s)\cl(H+K)\subseteq (t-s)(H+V+K) \subseteq (t-s)H+V+K \subseteq [2]V+K.
	}
	Combinando estas estimaciones con \eq{est}, obtenemos que \eq{st} se cumple.
	La demostraci\'on para el caso $t<s$ es completamente an\'aloga y usa que $S$
	es $K$ estrellado y $T$ es semi-$K$-acotado inferiormente.
\end{proof}
	

\Rem{semCon}{
Cuando $K=\{0\}$, la $K$-continuidad de una multifunción equivale a la continuidad con 
respecto a la topología de Hausdorff, ver \cite{DeBPia83}. 

Por otra parte, si $F(x)={f(x)}$, para alguna función $f:D\to \R$
y $K=[0,\infty)$, entonces $K$-semicontinuidad superior e inferior
equivalen a semicontinuidad superior e inferior respectivamente.
}

\Exa{Kcon1}{
Si $A\subseteq Y$ es un conjunto acotado y $K\subseteq Y$ es un cono convexo. Entonces,
la multifunción $F:\R\to\P_0(Y)$ definida mediante la fórmula
$F(t):= tA+K$ es $K$-continua con respecto a la topología de Hausdorff.

Sea $t_0\in\R$ y sea $V\in\U(Y)$ un entorno abierto del origen en $Y$. 
Observemos lo siguiente
\Eq{*}{
F(t_0)=t_0A+K=(t_0-t+t)A+K\subseteq(t_0-t)A+tA+K=(t_0-t)A+F(t).
}
Ahora bien, como el conjunto $A$ es acotado, existe $\delta>0$ tal que 
si $0<s<\delta$ entonces $sA\subseteq V.$ Luego, si $0<t-t_0<\delta$
entonces $(t-t_0)A\subseteq V$ y por lo tanto,
\Eq{*}{
F(t_0)\subseteq F(t)+W.
}
Esto demuestra que la multifunción $F$ así definida es
semicontinua inferior con respecto a la topología de Hausdorff.
Para demostrar la semicontinuidad superior de $F$ se procede de manera
similar.
}

Es bien conocido que toda función midconvexa y continua es convexa.
A continuación, veremos un resultado análogo para la clase de multifunciones
midconvexas. Para un conjunto $A$ y $n\in\N$ se denotará 
$$
[n]A:=\{x_1+\cdots x_n | x_1,\ldots,x_n\in A\}.
$$ 
\Thm{midConSVM}{
Sean $X,Y$ espacios topológicos lineales.
Sea $F:D\to \P_0(Y)$ una multifunción $K$-semicontinua superior
tal que para todo $x\in D$ el conjunto
$F(x)\subseteq Y$  es compacto. Si $F$ es $K$-midconvexa y $K$-semicontinua superior
en $D$, entonces, $F$ es $K$-convexa.
}
\begin{proof}
Sean $x,y\in D$ y $t\in[0,1]$ fijos. Sea $(q_n)_n\subseteq\D$ una sucesión de números
diádicos racionales que converge a $t$. Sea $V\in \U(Y)$ un abierto 
arbitrario y consideremos $W\in\U(Y)$ tal que $[3]\,W\subseteq V.$ 
Por el \thm{Nik860} se tiene que para
todo $n\in\N$, la multifunción $F$ satisface la inclusión
\Eq{*}{
q_nF(x)+(1-q_n)F(y)\subseteq F(q_nx+(1-q_n)y)+K.
}
Como las imágenes de $F$ son conjuntos acotados, entonces existen
números naturales $n_1, n_2$ tales que para $n\geq n_1$, 
$$tF(x)\subseteq q_nF(x)+W$$
 y para $n\geq n_2$
$$(1-t)F(y)\subseteq(1-q_n)F(y)+W.$$ 
Por otra parte, la $K$-semicontinuidad superior de $F$ en 
el punto $tx+(1-t)y$ da como resultado que para $n\geq n_3$,
$n_3\in\N$ 
\Eq{*}{
F(q_nx+(1-q_n)y)\subseteq F(tx+(1-t)y)+W+K.
}
Ahora bien, si $n\geq\max\{n_1,n_2,n_3\}$ entonces
\Eq{*}{
tF(x)+(1-t)F(y)&\subseteq q_nF(x)+(1-q_n)F(y)+[2]W\\
&\subseteq F(q_nx+(1-q_n)y)+[2]W+K
 \subseteq F(tx+(1-t)y)+[3]W+K \\
&\subseteq F(tx+(1-t)y)+V+K.
}
Como el abierto $V$ es arbitrario, entonces
\Eq{*}{
tF(x)+(1-t)F(y)\subseteq \bigcap_{V\in\U(Y)}(F(tx+(1-t)y)+K+V)
=\cl(F(tx+(1-t)y)+K).
}
Como $K$ es un cono cerrado y $F(tx+(1-t)y)$ es un conjunto
compacto, entonces, por la \prp{sumSets} se tiene que
la suma de ellos es un conjunto cerrado y por lo tanto
$\cl(F(tx+(1-t)y)+K)=F(tx+(1-t)y)+K$, lo que completa la demostración.
\end{proof}


En este contexto, K. Nikodem en \cite{Nik86} establece el siguiente resultado, que 
generaliza el \thm{BD15} y además engloba parte de los resultados 
obtenidos por Trudzik en \cite{Tru84}.

\begin{theorem}[\cite{Nik86}, Teorema 1]
\label{Nik861}
Sean $X,Y$ espacios topológicos lineales, y sea $K\subseteq Y$ un cono convexo
tal que $0\in K$.
Sea $F:D\to \P_0(Y)$ una multifunción tal que para todo $x\in D$ el conjunto
$F(x)\subseteq Y$ es acotado. Si $F$ es $K$-midconvexa y $K$-acotada superior
en un subconjunto $H\subseteq D$ con interior no vacío, entonces,
$F$ es $K$-continua en $D$.
\end{theorem}

Cuando $K=\{0\}$, entonces, $K$-convexidad es simplemente convexidad
y K. Nikodem en 1987 obtuvo los siguientes resultados, 
\begin{theorem}[\cite{Nik87c}, Teorema 1]
\label{Nik87c1}
Sean $X,Y$ espacios topológicos lineales.
Sea $F:D\to \P_0(Y)$ una multifunción tal que para todo $x\in D$ el conjunto
$F(x)\subseteq Y$ es acotado. Si $F$ es midconvexa y acotada superiormente
en un subconjunto $H\subseteq D$ con interior no vacío, entonces,
$F$ es continua en $D$ con respecto a la topología de Hausdorff.
\end{theorem}

\begin{theorem}[\cite{Nik87a}, Teorema 1]
\label{Nik87a1}
Sean $X,Y$ espacios topológicos lineales.
Sea $F:D\to \P_0(Y)$ una multifunción tal que para todo $x\in D$ el conjunto
$F(x)\subseteq Y$ es acotado y convexo. Si $F$ es midcóncava y acotada 
en un subconjunto $H\subseteq D$ con interior no vacío, entonces,
$F$ es continua en $D$ con respecto a la topología de Hausdorff.
\end{theorem}

\section{Convexidad fuerte.}
Supongamos ahora que $(X,\|\cdot\|)$ y $(Y,\|\cdot\|)$
son espacios vectoriales normados
y sea $B_Y\subseteq Y$ el interior 
de la bola unitaria en $Y$. Huang en su artículo 
del año 2010 \cite{Hua10}, establece la siguiente 
definición, generalizando el concepto de convexidad fuerte
para funciones, introducido por Polyak en \cite{Pol66}. 

\Defi{StrgCvxSVM}{
Sea $F:D\to\P_0(Y)$ una multifunción y sea $c>0$. Se dice que 
$F$ es fuertemente convexa con módulo $c$
si 
\Eq{StrgCvxSVM}{
tF(x)+(1-t)F(y)+ct(1-t)\|x-y\|^2\overline{B_Y} \subseteq 
F(tx+(1-t)y)
}
para todo $x,y\in D$ y para todo $t\in[0,1]$.
}
En base a esta definición surge de forma natural la
definición de midconvexidad fuerte.
\Defi{StrgMidCvxSVM}{
Sea $F:D\to\P_0(Y)$ una multifunción y sea $c>0$. Se dice que 
$F$ es fuertemente midconvexa con módulo $c$ si 
\Eq{StrgMidCvxSVM}{
\frac{F(x)+F(y)}2+\frac{c}4\|x-y\|^2\overline{B_Y} \subseteq 
F\bigg(\frac{x+y}2\bigg).
}
}
El siguiente Teorema caracteriza a la familia de multifunciones
fuertemente convexas con m\'odulo $c$.
\begin{theorem}[\cite{LeiMerNikSan13}, Teorema 12]
Sean $(X,\|\cdot\|)$ un espacio con producto interno, $t$
un n\'umero fijo en $(0,1)$ y $D$ un subconjunto
convexo de $X$. Una multifunci\'on $F:D\to\P_0(Y)$ tal que $F(x)$
es un conjunto convexo y cerrado para todo $x\in D,$ es 
fuertemente convexa con m\'odulo $c$ si y s\'olo si
la multifunci\'on $G$ definida por $G(x):=F(x)+\|x\|^2\overline{B_Y}$
para $x\in D$ es convexa.
\end{theorem}

Es evidente que toda multifunción fuertemente convexa con módulo $c$
es convexa. Ahora bien, para esta nueva clase de multifunciones,
H. Leiva et. al. en \cite{LeiMerNikSan13} obtienen el siguiente
resultado de tipo Bernstein--Doetsch.

\begin{theorem}[\cite{LeiMerNikSan13}, Teorema 4]
Sea $F:D\to\P_0(Y)$ una multifunci\'on fuertemente midconvexa con m\'odulo
$c$ tal que para todo $x\in D$ el conjunto $F(x)$ es cerrado y acotado. Si
$F$ es semicontinua superior en $D$, entonces esta es fuertemente
convexa con m\'odulo $c$.
\end{theorem}   

Como consecuencia del teorema anterior surge el siguiente

\begin{corollary}
Supongamos que $D\subseteq X$ es un conjunto abierto y convexo. Si 
una multifunci\'on $F:D\to\P_0(Y)$ es fuertemente convexa con m\'odulo
$c$ y semicontinua inferior en un punto de $D$ entonces, $F$ es continua
y fuertemente convexa con m\'odulo $c$.
\end{corollary}

Hasta ahora hemos presentado la mayoría de los resultados de tipo Bernstein--Doetsch
que han sido obtenidos a lo largo del tiempo tanto para funciones como para
multifunciones. En el cap\'itulo 4 como se mencion\'o en la introducci\'on 
presentaremos dos Teoremas que pretenden englobar a gran parte de los resultados
de tipo Bernstein--Doetsch obtenidos hasta el momento.


\section{Transformación de Takagi.}

En el Capítulo \ref{chapPrevio} definimos la función de Takagi
$T$ para cada $x\in\R$, mediante la fórmula
\Eq{*}{
T(x):=\sum_{k=0}^\infty\frac{d_\Z(2^kt)}{2^k},
}
donde 
\Eq{dz}{
	d_\Z(x) := \dist(\Z,x):=\inf\{|z-x| : z\in\Z\},\quad x\in\R.	
}

Supongamos ahora que $D$ es un conjunto estrellado con respecto al origen, 
y consideremos una multifunción $S:D\to\P_0(Y)$ con la propiedad de 
que $0\in S(x)$ para todo $x\in D$.
Para dicha multifunción, se define $S^T:\R\times D\to Y$ de la siguiente manera
\Eq{TakTrS}{
 S^T(t,x):=\cl\bigg(\bigcup_{n=0}^{\infty} \sum_{k=0}^{n} 
                 \frac{1}{2^k}S\big(2d_{\Z}(2^kt)x\big)\bigg)\qquad(t\in\R,\,x\in D).
}
\Rem{TakTrans}{
	El hecho de que $0\in S(x)$ para todo $x\in D$ es crucial, ya que esto trae como consecuencia
	que la sucesión de conjuntos 
	\Eq{*}{
		\Bigg(\sum_{k=0}^{n}\frac{1}{2^k}S\big(2d_{\Z}(2^kt)x\big)\Bigg)_{n\in\N}
	}
	sea creciente. Por lo tanto, $S^T$ no es más que el límite inferior
	de dicha sucesión.
}
La multifunción $S^T$ será llamada transformación de Takagi de $S$ a lo largo del trabajo. 
\Defi{Recsvm}{
El cono recesión de la multifunción $S:D\to\P_0(Y)$ es el conjunto
\Eq{*}{
\rec(S):= \bigcap_{x\in D}\rec S(x)
}
}
De la definición se puede observar que para todo $x\in D$ se tiene que 
$$\rec(S)+S(x)\subseteq S(x).$$ 
A continuación, se establecerá la relación entre
una multifunción y su transformación de Takagi.
\Prp{SST}{
Sea $D\subseteq X$ un conjunto estrellado y sea $S:D\to\P_0(Y)$ una multifunción
tal que $0\in S(x)$ para todo $x\in D$. Entonces
\Eq{TT1}{
   \cl(S(x))\subseteq S^T\big(\tfrac12,x\big) \qquad (x\in D).
}
Si además, $S(0)\subseteq \overline\rec(S)$, entonces
\Eq{TT2}{
   \cl(S(x))= S^T\big(\tfrac12,x\big) \qquad (x\in D).
}
}
\begin{proof} 
Observe que $d_{\Z}\big(\tfrac12\big)=\tfrac12$ y $d_{\Z}\big(2^k\cdot\tfrac12\big)=0$ para $k\in\N$.
Por lo tanto,
\Eq{*}{
  S^T\big(\tfrac12,x\big)
   =\cl\bigg(\bigcup_{n=0}^{\infty} \sum_{k=0}^{n}\frac{1}{2^k}S\big(2d_{\Z}(2^k\cdot\tfrac12)x\big)\bigg)
   =\cl\bigg(S(x)+\bigcup_{n=0}^{\infty} \sum_{k=1}^{n}\frac{1}{2^k}S(0)\bigg).
}
Como $0\in S(0)$, entonces la inclusión \eq{TT1} se sigue inmediatamente. Para demostrar
\eq{TT2}, asuma que $S(0)\subseteq \overline\rec(S)$. Entonces, 
$S(0)\subseteq \overline\rec(S(x))\subseteq\rec(\overline{S(x)})$.
Como $\rec(\overline{S(x)})$ es un cono convexo, se tiene que este conjunto es cerrado
bajo la suma y bajo la multiplicación por escalares positivos.
Así, para todo $n\in\N$,
\Eq{*}{
  \sum_{k=1}^{n}\frac{1}{2^k}S(0)
  \subseteq\sum_{k=1}^{n}\frac{1}{2^k}\rec(\overline{S(x)})
  \subseteq\rec(\overline{S(x)}).
}
En consecuencia, 
\Eq{*}{
\bigcup_{n=0}^{\infty} \sum_{k=1}^{n}\frac{1}{2^k}S(0)\subseteq\rec(\overline{S(x)}).
}
Luego,
\Eq{*}{
  S^T\big(\tfrac12,x\big)
   =\cl\bigg(S(x)+\bigcup_{n=0}^{\infty}\sum_{k=1}^{n}\frac{1}{2^k}S(0)\bigg)
   \subseteq\cl\bigg(\overline{S(x)}+\rec(\overline{S(x)})\bigg)
   \subseteq \cl(S(x)),
}
lo cual completa la demostración de \eq{TT2}. 
\end{proof}

\Prp{Tak}{
Sea $D\subseteq X$ un conjunto estrellado, $S_0\subseteq Y$ un conjunto convexo  
que contiene a $0\in Y$ y $K\subseteq Y$ un cono convexo. Sea $\varphi:D\to \R_+$ 
una función localmente acotada superior y no negativa.
Definamos $S:D\to\P_0(Y)$ por $S(x):=K+\varphi(x)S_0$. Entonces
\Eq{Tak1}{
  S^T(t,x)=\cl\big(K+\varphi^T(t,x) S_0\big)
  \qquad(t\in\R,\,x\in D),
}
donde
\Eq{Tak2}{  
  \varphi^T(t,x)=\sum_{n=0}^{\infty}\frac{1}{2^n}\varphi\big(2d_{\Z}(2^nt)x\big)
  \qquad(t\in\R,\,x\in D).
}
Si además, $\varphi(0)=0$, entonces
\Eq{Tak+}{
  \varphi^T\big(\tfrac12,x\big)=\varphi(x) \qquad\mbox{y}\qquad 
  S^T\big(\tfrac12,x\big)=\cl(K+\varphi(x)S_0)=\cl(S(x))\qquad(x\in D).
}}

\begin{proof} 
Para $t\in\R$ y $n\geq 0$, se tiene que $0\leq 2d_{\Z}(2^nt)\leq 1$,
por lo tanto $2d_{\Z}(2^nt)x\in[0,x]$. Como la función $\varphi$ es localmente acotada 
superior en $D$, entonces es acotada superior en el segmento $[0,x]$ por alguna constante
$M(x)$. Luego
\Eq{*}{
  \varphi^T(t,x)=\sum_{n=0}^{\infty}\frac{1}{2^n}\varphi\big(2d_{\Z}(2^nt)x\big) 
    \leq \sum_{n=0}^{\infty}\frac{1}{2^n}M(x)=2M(x) \qquad(t\in\R).
}

Para probar \eq{Tak1}, fijemos $(t,x)\in\R\times D$. 

Para demostrar la inclusión $\subseteq$ en \eq{Tak1}, primero se demostrará que
\Eq{Tak3}{
\bigcup_{n=0}^{\infty} \sum_{k=0}^{n}\frac{1}{2^k}S\big(2d_{\Z}(2^kt)x\big)
\subseteq K+\varphi^T(t,x) S_0.
}
Usando la definición de $S$, la convexidad del conjunto $S_0$ y el hecho 
de que la función $\varphi$ es no-negativa, se tiene que para $n\geq 0$
\Eq{*}{
\sum_{k=0}^{n} \frac{1}{2^k}S\big(2d_{\Z}(2^kt)x\big) 
      &= \sum_{k=0}^{n} \bigg(K + \frac{1}{2^k}\varphi\big(2d_{\Z}(2^kt)x\big)S_0\bigg) 
      = K + \bigg(\sum_{k=0}^{n} \frac{1}{2^k}\varphi\big(2d_{\Z}(2^kt)x\big)\bigg)S_0 \\
      &\subseteq K + \bigg(\sum_{k=0}^{\infty} \frac{1}{2^k}\varphi\big(2d_{\Z}(2^kt)x\big)\bigg)S_0
      = K+\varphi^T(t,x) S_0.
}
Así, queda demostrada la inclusión \eq{Tak3}. Tomando la clausura en ambos lados, 
la inclusión $\subseteq$ en \eq{Tak1} se sigue de inmediato. 

Para la prueba de la inclusión $\supseteq$ en \eq{Tak1}, 
es suficiente mostrar que
\Eq{*}{
  K+\varphi^T(t,x)S_0\subseteq S^T(t,x).
}
Observe que para $n\geq0$ se tiene lo siguiente
\Eq{*}{
K&+\Bigg(\sum_{k=0}^{n} \frac{1}{2^k}\varphi\big(2d_{\Z}(2^kt)x\big)\Bigg)S_0
  \subseteq \Bigg(\sum_{k=0}^{n} \frac{1}{2^k}\varphi\big(2d_{\Z}(2^kt)x\big)\Bigg)(S_0+K)\\
	 &\subseteq \sum_{k=0}^{n} \frac{1}{2^k}\varphi\big(2d_{\Z}(2^kt)x\big)(S_0+K) 
	\subseteq \sum_{k=0}^{n} \frac{1}{2^k}\big(\varphi\big(2d_{\Z}(2^kt)x\big)S_0+K\big)\\
	&=\sum_{k=0}^{n} \frac{1}{2^k}S(2d_{\Z}(2^kt)x) 
	\subseteq \bigcup_{\ell=0}^{\infty} \sum_{k=0}^{\ell}\frac{1}{2^k}S\big(2d_{\Z}(2^kt)x\big).
}
Por lo tanto,
\Eq{*}{
K+\varphi^T(t,x)S_0 
	&=K+\Bigg(\sum_{k=0}^{\infty} \frac{1}{2^k}\varphi\big(2d_{\Z}(2^kt)x\big)\Bigg)S_0 \\
  &\subseteq \cl\Bigg(\bigcup_{\ell=0}^{\infty} \sum_{k=0}^{\ell}
								\frac{1}{2^k}S\big(2d_{\Z}(2^kt)x\big)\Bigg)
	= S^T(t,x).
}
Con lo cual se termina la demostración.

En el caso de que $\varphi(0)=0$, la primera igualdad en \eq{Tak+} es inmediata, la segunda
igualdad es consecuencia de \eq{Tak1}. 
\end{proof}

\Cor{Tak}{
Sea $X$ un espacio normado, $D\subseteq X$ un conjunto estrellado, $S_0\subseteq Y$ un 
conjunto convexo que contiene a $0\in Y$, $K\subseteq Y$ un cono convexo, 
y $\alpha>0$. Definamos $S:D\to\P_0(Y)$ por
$S(x):=K+\|x\|^\alpha S_0$. Entonces
\Eq{Tak1+}{
  S^T(t,x)=\cl\big(K+T_\alpha(t)\|x\|^\alpha S_0\big)
  \qquad(t\in\R,\,x\in D),
}
donde $T_\alpha:\R\to\R$ es la función de Takagi de orden $\alpha$
y es definida por
\Eq{Tak2+}{
  T_\alpha(t):=\sum_{n=0}^{\infty}2^{\alpha-n}(d_{\Z}(2^nt))^\alpha
  \qquad(t\in\R).
}}

\begin{proof}
Aplicando la \prp{Tak} con la función $\varphi$
definida mediante la f\'ormula
 $\varphi(x):=\|x\|^\alpha$, se observa que para
todo $t\in\R$ y para todo $x\in D$
\Eq{*}{
  \varphi^T(t,x)=\sum_{n=0}^{\infty}\frac{1}{2^n}\varphi\big(2d_{\Z}(2^nt)x\big)
  =\sum_{n=0}^{\infty}2^{\alpha-n} \big(d_{\Z}(2^nt)\big)^\alpha\|x\|^\alpha
  = T_\alpha(t)\|x\|^\alpha.
}
Por lo tanto, \eq{Tak1+} es consecuencia de \eq{Tak1}. 
\end{proof}

\Rem{Tak}{
Un caso particular importante es cuando $\alpha=1$, entonces $T_1=2T$, donde $T$ es la función de Takagi 
definida por \eq{Tak} en el Capítulo \ref{chapPrevio}. 
En el caso $\alpha=2$ un argumento interesante nos da una forma cerrada para $T_2$. 
Observe que $T_\alpha$ (para cualquier $\alpha>0$) satisface la ecuación funcional
\Eq{TT}{
  T_\alpha(t)=2^{\alpha}\big(d_{\Z}(t)\big)^\alpha+\frac12 T_\alpha(2t) \qquad(t\in\R).
}
Por el teorema de punto fijo de Banach, esta ecuación funcional tiene solución única en el espacio 
de Banach de las funciones reales, acotadas sobre la recta real (equipado con la norma
del supremo). Asi $T_\alpha$ es la única solución a \eq{TT}. Por otro lado, para $\alpha=2$ 
se puede verificar que la función periódica $T_2^*$ definida en $[0,1]$ por 
$T_2^*(t)=4t(1-t)$ también es solución de \eq{TT}, así, debe ocurrir que 
$T_2(t)=4t(1-t)$ para $t\in [0,1]$. Para mayores detalles, puede revisar \cite{MakPal13b}.}

\Cor{Tk}{
Sea $X$ un espacio normado, $D\subseteq X$ un conjunto estrellado, $S_0\subseteq Y$ un 
conjunto convexo que contiene a $0\in Y$, $K\subseteq Y$ un cono convexo. 
Definamos $S:D\to\P_0(Y)$ por $S(x):=K+S_0$. Entonces
\Eq{Tk}{
  S^T(t,x)=\cl\big(K+2S_0\big)  \qquad(t\in\R,\,x\in D).
}
}
\begin{proof} 
Aplicaremos la \prp{Tak} a la función constante $\varphi\equiv 1$.
Por lo tanto \eq{Tak2} nos dá $\varphi^T\equiv 2$, luego \eq{Tak1} es equivalente a lo
que queremos demostrar.
\end{proof}


\chapter{T\'ermino de error de tipo Takagi-Hazy-P\'ales.}
\setcounter{theorem}{0}
%Los resultados principales de esta investigación están concentrados en los siguientes
%dos teoremas. 
A lo largo de este capítulo, se asume que $X$ y $Y$ son espacios topológicos
lineales.

%El siguiente teorema teorema dá condiciones suficientes sobre una multifunción
%que satisface una inclusión de convexidad tipo Jensen, para que esta satisfaga
%una inclusión de tipo convexidad más general.
\section[Teorema de tipo Bernstein--Doetsch para multifunciones convexas]
{Teorema de tipo Bernstein--Doetsch con errores de tipo
THP para multifunciones convexas.}
\Thm{Convex}{
Sea $D\subseteq X$ un conjunto convexo no-vacío y $A,B:(D-D)\to\P_0(Y)$ multifunciones tales que
$0\in A(x)\cap B(x)$ para todo $x\in (D-D)$. Sea $K=\overline{\rec}(B)$, la clausura del
cono recesión de $B$. Sea $F:D\to\P_0(Y)$ una multifunción que satisface la inclusión de 
convexidad tipo Jensen 
\Eq{JCV}{
\dfrac{F(x) + F(y)}{2} + A(x-y) \subseteq \cl\bigg(F\pr{\dfrac{x+y}{2}} + B(x-y)\bigg) \qquad (x,y\in D).
}
Supongamos además, que $F$ tiene las siguientes propiedades:
\begin{enumerate}[i.]
 \item $F$ es puntualmente semi-$K$-acotada inferior.
 \item $F$ es localmente debíl-semi-$K$-acotada superior en $D$.
\end{enumerate}
Entonces $F$ satisface la siguiente inclusión 
\Eq{CV}{
 tF(x)+(1-t)F(y)+A^T(t,x-y)\subseteq \cl\big(F(tx+(1-t)y)+B^T(t,x-y)\big)
}
para todo $x,y\in D$ y $t\in[0,1].$
}

\begin{proof}
El primer paso en la demostración de \eq{CV}, será mostrar que, para todo $x,y\in D$
existe un conjunto acotado $H\subseteq Y$ tal que, para todo $n\geq 0$, $t\in[0,1]$, 
\Eq{CVn}{
 tF(x)&+(1-t)F(y) + \sum_{k=0}^{n-1}{\dfrac{1}{2^k}A\big(2 d_{\Z}(2^k t)(x-y)\big)} \\
 &\subseteq \cl\bigg(F(tx+(1-t)y) + \dfrac{1}{2^n} H + K +
  \sum_{k=0}^{n-1}{\dfrac{1}{2^k}B\big(2 d_{\Z}(2^k t)(x-y)\big)}\bigg). 
}

Fijemos $x,y\in D$ arbitrarios. Para verificar que \eq{CVn} se cumple, vamos a proceder aplicando
inducción sobre $n.$ 
Para el caso $n=0$, debemos demostrar que existe un conjunto acotado $H\subseteq Y$ tal que para todo
$t\in[0,1]$, 
\Eq{J0}{
tF(x)+(1-t)F(y)\subseteq \cl\big(F(tx+(1-t)y) + H + K\big).
}
Sea $U\in\U(Y)$ y escojamos un abierto balanceado $V\in\U(Y)$ tal que $V+V+V\subseteq U$. 
Como $F$ es puntualmente semi-$K$-acotada inferior, existen conjuntos acotados $H_x,H_y\subseteq Y$
tales que 
\Eq{*}{
F(x) \subseteq \cl(H_x + K)\subseteq V+H_x+K
}
y
\Eq{*}{
F(y) \subseteq \cl(H_y + K) \subseteq V+H_y + K.
} 
Multiplicando estas inclusiones por $t$ y $1-t$, respectivamente, sumándolas,
usando la convexidad del cono $K$ y el hecho de que $V$ es balanceado,
\Eq{J1}{
tF(x)+(1-t)F(y)&\subseteq tV + tH_x + tK + (1-t)V + (1-t)H_y +(1-t)K \\
               &\subseteq V + V + tH_x + (1-t)H_y + K.
}
Por la \prp{bddtH}, se tiene 
que tanto el conjunto $H_1:=\bigcup_{t\in[0,1]} tH_x$ como el
conjunto $H_2:=\bigcup_{t\in[0,1]} (1-t)H_y$
son acotados. Así, la inclusión \eq{J1} nos da que para todo $t\in[0,1]$,
\Eq{J2}{
tF(x)+(1-t)F(y)\subseteq V + V + H_1 + H_2 + K.
}
Por otra parte, como $F$ es localmente débil-semi-$K$-acotada superior y el segmento 
$[x,y]$ es compacto, entonces por la \prp{lem2.4}, existe un conjunto acotado $H_0$ 
tal que para todo $t\in[0,1]$,
\Eq{J3}{
 0\in \cl(F(tx+(1-t)y) + H_0 + K)\subseteq V + F(tx+(1-t)y) + H_0 + K.
} 
Ahora bien, sumando las inclusiones \eq{J2} y \eq{J3} lado a lado, se tiene que 
para todo $t$ en $[0,1]$, 
\Eq{*}{
tF(x)+(1-t)F(y)&\subseteq V + V + V+ F(tx+(1-t)y) + H_0 + H_1 + H_2 + K \\
               &\subseteq U + F(tx+(1-t)y) + H_0 + H_1 + H_2 + K.
} 
Por lo tanto,
\Eq{*}{
tF(x)+(1-t)F(y)&\subseteq \bigcap_{U\in\U}\big(U + F(tx+(1-t)y) + H_0 + H_1 + H_2 + K\big)\\
               &=\cl\big(F(tx+(1-t)y) + H_0 + H_1 + H_2 + K\big).
}
Así, la inclusión \eq{J0} se obtiene con $H:=H_0+H_1+H_2$.

Supongamos ahora que la ecuación \eq{CVn} es válida para $n$ y demostraremos que también es 
válida para $n+1$.
Supongamos que $t\in\big[0,\frac12\big]$ (el caso cuando $t\in\big[\frac12,1\big]$ es
análogo). Entonces, $d_{\Z}(t)=t$ y podemos reescribir el lado izquierdo de la inclusión 
que queremos demostrar como 
\Eq{e11}{
 tF(x)&+(1-t)F(y) + \sum_{k=0}^{n}{\dfrac{1}{2^k}A\big(2 d_{\Z}(2^k t)(x-y)\big)} \\
 &= tF(x)+(1-t)F(y)+ A\big(2t(x-y)\big) 
   + \sum_{k=1}^{n}{\dfrac{1}{2^k}A\big(2 d_{\Z}(2^kt)(x-y)\big)}.
}
Observemos que
\Eq{CVC1}{
 (1-t)F(y) = \frac{1-2t+1}{2}F(y)\subseteq \frac{1-2t}2 F(y) + \frac12 F(y),
}
por lo tanto,
\Eq{II1}{
 &tF(x)+(1-t)F(y)+ A\big(2t(x-y)\big) 
   + \sum_{k=1}^{n}{\dfrac{1}{2^k}A\big(2 d_{\Z}(2^kt)(x-y)\big)} \\
 &\subseteq \dfrac{1}{2}\bigg( 2tF(x)+(1-2t)F(y)
   + \sum_{k=0}^{n-1}{\dfrac{1}{2^k}A\big(2 d_{\Z}(2^k(2t))(x-y)\big)} \bigg) \\
&\hspace{8cm}+ \frac{1}{2}F(y) \!+\! A\big(2d_{\Z}(t)(x-y)\big).
}
Al usar la hipótesis inductiva con $2t$ en vez de $t,$ se sigue que
\Eq{II2}{
2t&F(x)+(1-2t)F(y)+ \sum_{k=0}^{n-1} \dfrac{1}{2^k}A\big(2 d_{\Z}(2^k(2t))(x-y)\big) \\
   & \subseteq \cl\bigg(F(2tx + (1-2t)y)+ \frac{1}{2^n}H + K 
   + \sum_{k=0}^{n-1} \dfrac{1}{2^k}B\big(2 d_{\Z}(2^k(2t))(x-y)\big)\bigg).
}
Combinando las inclusiones \eq{e11}, \eq{II1} y \eq{II2}, llegamos a la siguiente inclusión
\Eq{*}{
 & tF(x)+(1-t)F(y) + \sum_{k=0}^{n}{\dfrac{1}{2^k}A\big(2 d_{\Z}(2^k t)(x-y)\big)} \\
 & \subseteq \dfrac{1}{2}\cl\bigg(F(2tx + (1-2t)y)+ \frac{1}{2^n}H + K 
   + \sum_{k=0}^{n-1} \dfrac{1}{2^k}B\big(2 d_{\Z}(2^k(2t))(x-y)\big)\bigg) \\
 &\hspace{6cm} + \frac{1}{2}F(y) \!+\! A\big(2d_{\Z}(t)(x-y)\big)\\
 & \subseteq \cl\bigg(\dfrac{F(2tx + (1-2t)y)+ F(y)}{2}+ \frac{1}{2^{n+1}}H + K 
    + A\big(2t(x-y)\big) \\
&\hspace{6cm}		+ \sum_{k=0}^{n-1}\dfrac{1}{2^{k+1}}B\big(2d_{\Z}(2^k(2t))(x-y)\big)\bigg)
}
Como $F$ satisface la inclusión de tipo Jensen \eq{JCV}, entonces,
\Eq{*}{
\frac{F(2tx + (1-2t)y) + F(y)}{2} &+ A\big(2t(x-y)\big) \\
 \subseteq \cl\bigg(&F\bigg(\frac{2tx+(1-2t)y+y}{2}\bigg) + B\big(2t(x-y)\big)\bigg).
}
Por lo tanto,
\Eq{*}{
&tF(x)+(1-t)F(y) + \sum_{k=0}^{n}\dfrac{1}{2^k}A\big(2 d_{\Z}(2^kt)(x-y)\big) \\
& \subseteq \cl\bigg(\!\cl\bigg(F\bigg(\frac{2tx+(1-2t)y+y}{2}\bigg) 
     + B\big(2t(x-y)\big)\bigg) + \frac{1}{2^{n+1}}H + K \\
&\hspace{6.5cm}     + \sum_{k=0}^{n-1}\dfrac{1}{2^{k+1}}B\big(2d_{\Z}(2^k(2t))(x-y)\big)\bigg)\\
& = \cl\bigg(F\big(tx+(1-t)y\big) + \frac{1}{2^{n+1}}H + K 
  + \sum_{k=0}^{n}\dfrac{1}{2^{k}}B\big(2d_{\Z}(2^k t)(x-y)\big)\bigg).
}
Ahora podemos concluir que la inclusion \eq{CVn} es válida para todo $n\geq 0.$

Para completar la demostración del teorema, sea $t\in[0,1]$ fijo y apliquemos la \prp{L2.2} 
a las sucesiones de conjuntos y números definidas para $n\geq0$ de la siguiente manera
\Eq{*}{
 A_n &:= tF(x)+(1-t)F(y) + \sum_{k=0}^{n-1}\dfrac{1}{2^k}A(2 d_{\Z}(2^k t)(x-y)) \\
 B_n &:= F\big(tx+(1-t)y\big) + \sum_{k=0}^{n-1}\dfrac{1}{2^{k}}B\big(2d_{\Z}(2^k t)(x-y)\big) \\
 \varepsilon_n &:= \frac{1}{2^{n}}
}
Con esta notación, la inclusión \eq{CVn} es equivalente a \eq{Incn}.
Por otra parte, como $0\in A(u)\cap B(u)$ para todo $u\in (D-D)$,
entonces las sucesiones $(A_n)$ y $(B_n)$ son sucesiones no-decrecientes en $Y$.
Aplicando la \prp{L2.2}, 
\Eq{*}{
  \cl\bigg(\bigcup_{n=1}^\infty \bigg(tF(x)&+(1-t)F(y)
     + \sum_{k=0}^{n-1}\dfrac{1}{2^k}A(2 d_{\Z}(2^k t)(x-y))\bigg)\bigg)\\
  &\subseteq \cl\bigg(\bigcup_{n=1}^\infty \bigg(F\big(tx+(1-t)y\big)
     + \sum_{k=0}^{n-1}\dfrac{1}{2^{k}}B\big(2d_{\Z}(2^k t)(x-y)\big)\bigg)\bigg).
}
Ahora, aplicamos el numeral 2 del \thm{f1} en ambos lados de la inclusión anterior y se obtiene
\Eq{*}{
  \cl\bigg(tF(x)&+(1-t)F(y) + \cl\bigg(\bigcup_{n=1}^\infty
     \sum_{k=0}^{n-1}\dfrac{1}{2^k}A(2 d_{\Z}(2^k t)(x-y))\bigg)\bigg)\\
  &\subseteq \cl\bigg(F\big(tx+(1-t)y\big) +\cl\bigg(\bigcup_{n=1}^\infty 
     \sum_{k=0}^{n-1}\dfrac{1}{2^{k}}B\big(2d_{\Z}(2^k t)(x-y)\big)\bigg)\bigg),
}
lo cual es equivalente a la inclusión \eq{CV} que queríamos demostrar.
\end{proof}
%\newpage

\section[Teorema de tipo Bernstein--Doetsch para multifunciones cóncavas.]
{Teorema de tipo Bernstein--Doetsch con errores de tipo
Takagi-Hazy-P\'ales para multifunciones c\'oncavas.}
\Thm{Concave}{
Sea $D\subseteq X$ un conjunto convexo no-vacío y $A,B:(D-D)\to\P_0(Y)$ multifunciones tales que
$0\in A(x)\cap B(x)$ para todo $x\in (D-D)$. Sea $K=\overline{\rec}(B)$, la clausura del
cono recesión de $B$. Sea $F:D\to\P_0(Y)$ una multifunción que satisface la inclusión de 
concavidad tipo Jensen 
\Eq{JCC}{
F\pr{\dfrac{x+y}{2}} + A(x-y) \subseteq \cl\bigg(\dfrac{F(x) + F(y)}{2} + B(x-y)\bigg) \qquad (x,y\in D).
}
Supongamos además, que $F$ tiene las siguientes propiedades:
\begin{enumerate}[i.]
 \item $F$ es puntualmente semi-$K$-convexa, i.e., $tF(x)+(1-t)F(x)\subseteq \cl(F(x) + K)$  
para todo $x\in D$ y para todo $t\in[0,1]$;
 \item $F$ es localmente semi-$K$-acotada inferior.
\end{enumerate}
Entonces $F$ satisface la siguiente inclusión 
\Eq{CC}{
 F(tx+(1-t)y)+A^T(t,x-y)\subseteq \cl\big(tF(x)+(1-t)F(y)&+B^T(t,x-y)\big)
}
para todo $x,y\in D$ y $t\in[0,1]$.
}
\begin{proof}
Para demostrar la inclusión \eq{CC}, primero vamos a demostrar que para todo $x,y\in D$,
existe un conjunto acotado $H\subseteq Y$ tal que, para todo $n\geq 0$ y $t\in[0,1]$, 
\Eq{CCn}{
 F(tx&+(1-t)y) + \sum_{k=0}^{n-1}\dfrac{1}{2^k}A\big(2 d_{\Z}(2^k t)(x-y)\big) \\[-2mm]
 &\subseteq \cl\bigg(tF(x)+(1-t)F(y) + \dfrac{1}{2^n} H + K +
  \sum_{k=0}^{n-1}\dfrac{1}{2^k}B\big(2 d_{\Z}(2^k t)(x-y)\big)\bigg).
}

Sean $x,y\in D$ fijos. Para verificar que la inclusión \eq{CCn} es válida, 
vamos a proceder por inducción sobre $n$. Para $n=0$, se debe demostrar
que existe un conjunto acotado $H\subseteq Y$ tal que, para todo $t\in[0,1]$
\Eq{CC0}{
  F(tx+(1-t)y)\subseteq \cl\big(tF(x)+ (1-t)F(y) + H + K\big).
}
Como $F$ es semi-$K$-acotada inferior y el segmento $[x,y]$ es compacto, se tiene
que por la \prp{lem2.3}, existe un conjunto acotado $H_0\subseteq Y$ 
tal que 
\Eq{H0}{
F(tx+(1-t)y)\subseteq \cl(H_0 + K)   \qquad  (t \in[0,1]).
}
Por otra parte, como $F(x)$ y $F(y)$ son no-vacíos, podemos escoger dos elementos
$u\in F(x)$ y $v\in F(y)$, lo cual implica que
\Eq{Hxy}{
0 \in F(x)-u \qquad\mbox{y}\qquad 0 \in F(y)-v.
}    
Multiplicando ambas inclusiones en \eq{Hxy} por $t$ y $(1-t)$, respectivamente, y sumándolas
junto con la inclusión \eq{H0}, se tiene que para $t\in[0,1]$, 
\Eq{*}{
F(tx+(1-t)y) &\subseteq tF(x)+(1-t)F(y) - tu - (1-t)v + \cl(H_0 + K)\\
            &\subseteq \cl\big(tF(x)+(1-t)F(y) -[u,v] + H_0 + K\big).
}
Por lo tanto, la inclusión \eq{CC0} es válida con $H := H_0-[u,v],$ el cual es obviamente 
un conjunto acotado.

Ahora, supongamos que \eq{CCn} es válida para $n$ y veamos que también es válida para $n+1.$
Supongamos, al igual que en la demostración anterior que $t\in\big[0,\frac12\big]$ y  
observe que entonces $d_{\Z}(t)=t$. Reescribiendo el lado izquierdo de la inclusión que 
queremos demostrar se obtiene la siguiente expresión
\Eq{I1}{
 & F(tx+(1-t)y) + \sum_{k=0}^{n}{\dfrac{1}{2^k}A\big(2 d_{\Z}(2^k t)(x-y)\big)} \\
 & = F(tx+(1-t)y) + A(2d_{\Z}(t)(x-y)) + \sum_{k=1}^{n}{\dfrac{1}{2^k}A\big(2 d_{\Z}(2^k t)(x-y)\big)} \\
 & = F(tx+(1-t)y) + A\big(2t(x-y)\big)
 + \dfrac{1}{2}\sum_{k=0}^{n-1}{\dfrac{1}{2^k}A\big(2 d_{\Z}(2^k (2t))(x-y)\big)}.
}
Además, 
\Eq{*}{
tx+(1-t)y = \frac{2tx+(1-2t)y + y}{2}
}
y como $F$ satisface la inclusión de tipo Jensen dada en \eq{JCC},
\Eq{I2}{
 F\bigg(\dfrac{2tx+(1-2t)y + y}{2}\bigg) &+ A\big(2t(x-y)\big) \\
  &\subseteq \cl\bigg(\dfrac{F(2tx+(1-2t)y) + F(y)}{2} + B\big(2t(x-y)\big)\bigg).
}
Combinando \eq{I1} y \eq{I2}, se obtiene
\Eq{I3}{
 F&(tx+(1-t)y) + \sum_{k=0}^{n}\dfrac{1}{2^k}A\big(2 d_{\Z}(2^k t)(x-y)\big) \\
  &\subseteq \cl\bigg(\dfrac{F(2tx+(1-2t)y) + F(y)}{2} + B\big(2t(x-y)\big)\bigg) \\
 &\hspace{7cm}  +\dfrac{1}{2}\sum_{k=0}^{n-1}\dfrac{1}{2^k}A\big(2 d_{\Z}(2^k (2t))(x-y)\big) \\
  & \subseteq \cl\bigg(\dfrac12\bigg(F(2tx+(1-2t)y)
   +\sum_{k=0}^{n-1}\dfrac{1}{2^k}A\big(2 d_{\Z}(2^k (2t))(x-y)\big)\bigg) \\
 &\hspace{7cm}  + \dfrac12 F(y)+B\big(2d_{\Z}(t)(x-y)\big)\bigg).
}
Por la hipótesis inductiva, sabemos que  
\Eq{I4}{  
F&(2tx+(1-2t)y)+\sum_{k=0}^{n-1}\dfrac{1}{2^k}A\big(2 d_{\Z}(2^k (2t))(x-y)\big) \\ 
&\subseteq \cl\bigg(2tF(x)+(1-2t)F(y) + \dfrac{1}{2^{n}}H + K +
\sum_{k=0}^{n-1}\dfrac{1}{2^k}B\big(2 d_{\Z}(2^k (2t))(x-y)\big)\bigg).
}
Insertando \eq{I4} en \eq{I3} y usando el hecho de que 
\Eq{*}{
(1-2t)F(y)+F(y)\subseteq \cl\big((2-2t)F(y)+K\big)
}
lo cual es una consecuencia
de que $F$ es puntualmente semi-$K$-convexa, llegamos a la siguiente inclusión
\Eq{*}{
 F&(tx+(1-t)y) + \sum_{k=0}^{n}{\dfrac{1}{2^k}A\big(2 d_{\Z}(2^k t)(x-y)\big)} \\
& \subseteq \cl\bigg(\dfrac12\cl\bigg(2tF(x)+(1-2t)F(y) + 
\sum_{k=0}^{n-1}\dfrac{1}{2^k}B\big(2 d_{\Z}(2^k (2t))(x-y)\big)\bigg)\\
   &\hspace{5cm} +\dfrac{1}{2^{n}}H + K + \dfrac12 F(y)+B\big(2d_{\Z}(t)(x-y)\big)\bigg)\\
&\subseteq \cl\bigg(\dfrac{1}{2}\big(2t F(x) + (1-2t) F(y) + F(y) + K\big)  \\
&\hspace{5cm} + \dfrac{1}{2^{n+1}}H 
 +\sum_{k=0}^{n}\dfrac{1}{2^{k}}B\big(2 d_{\Z}(2^k t)(x-y)\big)\bigg) \\
&\subseteq \cl\bigg(tF(x)+(1-t)F(y)+ \dfrac{1}{2^{n+1}}H + K +
\sum_{k=0}^{n}\dfrac{1}{2^{k}}B\big(2 d_{\Z}(2^k t)(x-y)\big)\bigg). 
}
Esto completa la demostración de la inducción y así la inclusión \eq{CCn} es válida para todo $n\geq0$.

Ahora, vamos a usar la \prp{L2.2}. Para ello vamos a definir las sucesiones 
\Eq{*}{
A_n &:= F(tx+(1-t)y) + \sum_{k=0}^{n-1}{\dfrac{1}{2^k}A\big(2 d_{\Z}(2^k t)(x-y)\big)}, \\
B_n &:= tF(x)+(1-t)F(y) + \sum_{k=0}^{n-1}{\dfrac{1}{2^k}B\big(2 d_{\Z}(2^k t)(x-y)\big)}, \\
\varepsilon_n &= \frac1{2^n}.
}
para $t\in [0,1]$ fijo.

Por lo tanto, la inclusión \eq{CCn}, con las sucesiones $(A_n),(B_n)$ y $(\varepsilon_n)$ así definidas, 
es equivalente a la inclusión \eq{Incn}. Así, por la \prp{L2.2}, se sigue lo siguiente
\Eq{*}{
  \cl\bigg(\bigcup_{n=0}^{\infty}F(tx+(1-t)y) 
   &+ \sum_{k=0}^{n-1}{\dfrac{1}{2^k}A\big(2 d_{\Z}(2^kt)(x-y)\big)}\bigg)\\
  &\subseteq \cl\bigg(\bigcup_{n=0}^{\infty}tF(x)+(1-t)F(y) 
   + \sum_{k=0}^{n-1}{\dfrac{1}{2^k}B\big(2 d_{\Z}(2^k t)(x-y)\big)}\bigg).
}
De manera similar a como se hizo en la prueba del \thm{Convex}, la relación anterior
implica la inclusión que deseamos probar.
\end{proof}
%
%\Rem{R}{In each of the above theorems the closure operation can be removed from the right hand sides
%of the inclusions \eq{JCV}, \eq{CV}, \eq{JCC}, and \eq{CC} if the values of the set-valued map $F$ are 
%compact and $B$ has closed values. The closure operation can also be removed from the right hand sides
%of \eq{JCV} and \eq{CV} if $F$ has closed values and $B$ is compact valued. This observation also applies
%to the corollaries below. Another thing is which is worth mentioning is that if 
%$A(0)\subseteq\overline{\rec}(A)$ and $B(0)\subseteq \overline{\rec}(B)$, then, in view of \lem{TT},
%the inclusions \eq{CC} and \eq{CV} reduce to \eq{JCC} and \eq{JCV} for the substitution $t=\frac12$,
%respectively. Therefore, in this case, under the boundedness and convexity assumptions on $F$, 
%\eq{CC} and \eq{CV} are equivalent to \eq{JCC} and \eq{JCV}, respectively. The problem whether 
%inclusions \eq{CC} and \eq{CV} are the sharpest possible is an open problem. Results where the
%exactness of such estimates are obtained due to Boros \cite{Bor08}, Makó and Páles \cite{MakPal10b,MakPal13b}.}
%

\section{Consecuencias de los teoremas previos.}

Tomando multifunciones particulares $A,$ $B$ y usando la \prp{Tak}, vamos a establecer algunas
consecuencias importantes de los dos teoremas que acabamos de demostrar. Los siguientes 
corolarios resaltan la manera en que los resultados mencionados en la introducción están relacionados
de manera directa con nuestros resultados.

En los siguientes cuatro corolarios supondremos que $D\subseteq X$ es un conjunto convexo y no-vacío,
$K\subseteq Y$ es un cono convexo y cerrado, $S_0\subseteq Y$ es un conjunto convexo que contiene a 
$0$ y $\varphi:(D-D)\to\R_+$ es una función localmente acotada superior y no-negativa. 
Note que por la convexidad de $D$ se tiene que el conjunto 
$(D-D)$ es estrellado, por lo tanto la \prp{Tak} puede ser aplicada. 

Los primeros dos corolarios tratan sobre multifunciones fuertemente y aproximadamente
$K$-Jensen convexas respectivamente.

\Cor{Convex+1}{
Supongamos que $F:D\to\P_0(Y)$ es una multifunción puntualmente semi-$K$-acotada inferior y 
localmente débil-semi-$K$-acotada superior que satisface
\Eq{JCV+1}{
\dfrac{F(x) + F(y)}{2} \subseteq 
\cl\bigg(F\pr{\dfrac{x+y}{2}} + K + \varphi(x-y)S_0 \bigg) 
}
para todo $x,y\in D.$ Entonces
\Eq{CV+1}{
 tF(x)+(1-t)F(y) \subseteq \cl\big(F(tx+(1-t)y)+K+\varphi^T(t,x-y)S_0\big)
}
para todo $x,y\in D$ y para todo $t\in[0,1].$
}
\begin{proof}
Para cada $u\in D-D$, definamos las multifunciones $A(u)=0$ y $B(u)=K+\varphi(u)S_0$.
Ahora bien, por definición se tiene que para todo $u\in D-D$ 
\Eq{*}{
\rec(B(u))=\{y\in Y\,|\,ty+B(u)\subseteq B(u),\mbox{ para todo } t\geq0\}
}
y por lo tanto, si $y\in K$, entonces
\Eq{*}{
ty+B(u)=ty+K+\varphi(u)S_0\subseteq K+\varphi(u)S_0 =B(u).
}
Esto significa que $K\subseteq \rec(B(u))$ para todo $u\in D-D$ y 
en consecuencia $K\subseteq \overline{\rec}(B).$
Luego, $F$ es puntualmente semi-$\overline{\rec}(B)$-acotada inferior y localmente
débil-semi-$\overline{\rec}(B)$-acotada inferior. Aplicando
el \thm{Convex}, se tiene que para todo $x,y\in D$ y   
para todo $t\in[0,1]$
\Eq{*}{
 tF(x)+(1-t)F(y)\subseteq \cl\big(F(tx+(1-t)y)+B^T(t,x-y)\big).
}
Además, por la \prp{Tak} se obtiene que para $t\in[0,1]$ y $x,y\in D$ 
\Eq{*}{
B^T(t,x-y)=\cl(K+\varphi^T(t,x-y)).
}
Esto completa la demostración.
\end{proof}
\Cor{Convex+2}{
Supongamos que $F:D\to\P_0(Y)$ es una multifunción puntualmente semi-$K$-acotada inferior y 
localmente débil-semi-$K$-acotada superior que satisface
\Eq{JCV+2}{
\dfrac{F(x) + F(y)}{2} + \varphi(x-y)S_0 \subseteq \cl\bigg(F\pr{\dfrac{x+y}{2}} + K \bigg) 
}
para todo $x,y\in D.$ Entonces
\Eq{CV+2}{
 tF(x)+(1-t)F(y) + \varphi^T(t,x-y)S_0 \subseteq \cl\big(F(tx+(1-t)y) + K \big)
}
para todo $x,y\in D$ y para todo $t\in[0,1].$
}
\begin{proof}
La demostración de este resultado es completamente análoga a la demostración
anterior. Basta considerar $A(u)=\varphi(x-y)S_0$ y $B(u)=K$ para $u\in D-D$ 
y proceder de la misma manera que en la demostración anterior.
\end{proof}
Los siguientes dos corolarios tratan sobre multifunciones fuertemente y aproximadamente
$K$-Jensen cóncavas respectivamente y sus demostraciones son similares a la demostración
de los Corolarios \ref{CConvex+1} y \ref{CConvex+2} por lo cual serán omitidas.
%Usando el \thm{Convex} con las multifunciones $A(u)=0$ y $B(u)=K+\varphi(u)S_0$ 
%y aplicando la \prp{Tak}, se obtiene el resultado deseado. 
%Observe que, tenemos que $K\subseteq\overline{\rec}(B)$, 
%por lo tanto $F$ será puntualmente semi-$\overline{\rec}(B)$-acotada inferior y 
%localmente debíl-semi-$\overline{\rec}(B)$-acotada superior.
\Cor{Concave+1}{
Supongamos que $F:D\to\P_0(Y)$ es una multifunción puntualmente semi-$K$-convexa y
localmente semi-$K$-acotada inferior que satisface
\Eq{JCC+1}{
F\pr{\dfrac{x+y}{2}} \subseteq \cl\bigg(\dfrac{F(x) + F(y)}{2} + K + \varphi(x-y)S_0 \bigg)
}
para todo $x,y\in D.$ Entonces
\Eq{CC+1}{
 F(tx+(1-t)y) \subseteq \cl\big(tF(x)+(1-t)F(y) + K + \varphi^T(t,x-y)S_0 \big)
}
para todo $x,y\in D$ y para todo $t\in[0,1].$
}

\Cor{Concave+2}{
Supongamos que $F:D\to\P_0(Y)$ es una multifunción puntualmente semi-$K$-convexa y
localmente semi-$K$-acotada inferior que satisface
\Eq{JCC+2}{
F\pr{\dfrac{x+y}{2}} + \varphi(x-y)S_0 \subseteq \cl\bigg(\dfrac{F(x) + F(y)}{2} + K\bigg)
}
para todo $x,y\in D.$ Entonces
\Eq{CC+2}{
 F(tx+(1-t)y)+\varphi^T(t,x-y)S_0 \subseteq \cl\big(tF(x)+(1-t)F(y) + K \big)
}
para todo $x,y\in D$ y para todo $t\in[0,1].$
}
%
%\Rem{CC}{The results mentioned and recalled in the introduction can be derived as obvious consequences of
%the above corollaries. In the real valued setting, the Bernstein--Doetsch Theorem \cite{BerDoe15}, the 
%results of Ng--Nikodem \cite{NgNik93} and Házy--Páles \cite{HazPal04} follow if, in \cor{Convex+1}, we take 
%$Y:=\R$, $K:=\R_+$, $S_0:=[-1,0]$, $F(x):=\{f(x)\}$, and $\varphi(x):=0$, $\varphi(x):=\varepsilon$, $\varphi(x):=\varepsilon\|x\|$, respectively. Observe that, in these cases, \prp{Tak} yields 
%$\varphi^T(t,x):=0$, $\varphi^T(t,x):=2\varepsilon$, $\varphi^T(t,x):=2\varepsilon T(t)\|x\|$, respectively.
%The results of Averna, Cardinali, Nikodem, and Papalini \cite{AveCar90,CarNikPap93,Nik86,Nik87a,Nik87c,Nik89,Pap90} 
%and by Borwein \cite{Bor77} that are related to $K$-Jensen convex/concave vector valued and set-valued mappings
%can also be obtained directly. Numerous results obtained for approximate midconvexity by Makó and Páles \cite{MakPal10b,MakPal13b} and by Mure?ko, Ja. Tabor, Jó. Tabor, and ?oldak
%\cite{MurTabTab12,TabTab09b,TabTab09a,TabTabZol10b,TabTabZol10a} are generalized by Corollaries \ref{CConvex+1}--\ref{CConcave+2} to the vector valued and set-valued setting. Similarly, using the
%explicit form of the function $T_2$ described in \rem{Tak}, one can easily derive the results of Azócar, 
%Gimenez, Nikodem and Sanchez \cite{AzoGimNikSan11} and Leiva, Merentes, Nikodem, and Sanchez 
%\cite{LeiMerNikSan13} that are related to strongly $K$-Jensen convex real valued and set-valued functions 
%from \cor{Convex+2}.}

\chapter{T\'ermino de error de tipo Takagi-Tabor.}

\section{Convexidad y concavidad sobre los racionales di\'adicos.}
\setcounter{theorem}{0}

Los resultados principales de esta secci\'on est\'an contenidos
en el siguiente par de teoremas. Denotaremos por $\D$ al conjunto
de los n\'umeros racionales di\'adicos, i.e., $\D$ consiste en
los n\'umeros de la forma $\frac{k}{2^n}$, donde, $k\in\Z$ y $n\in\N$.


\Thm{ConvexTab}{
	Sea $D\subseteq X$ un subconjunto convexo no vac\'io, y sean
	$A,B:(D-D)\to\P_0(Y)$ tales que, los valores de la multifunci\'on
	$B$ son semi $K$-convexos, donde $K$ representa la clausura del
	cono recesi\'on asociado a $B$. 	
	Sea $F:D\to\P_0(Y)$ una multifunci\'on que satisface la siguiente
	inclusi\'on de convexidad tipo Jensen
	%%
	\Eq{JCV}{
		\dfrac{F(x) + F(y)}{2} + A(x-y) \subseteq 
		\cl\Big(F\Big(\dfrac{x+y}{2}\Big) + B(x-y)\Big) 
		\qquad (x,y\in D).
	}
	%%
	Entonces para $x,y\in D$ y para todo $t\in\D\cap[0,1]$ se tiene
	que $F$ satisface la siguiente inclusi\'on
	%%	
	\Eq{CV}{
		tF(x)&+(1-t)F(y)+\sum_{k=0}^{\infty} 2d_{\Z}\big(2^k t\big)A\Big(\dfrac{x-y}{2^k}\Big) \\
		&\subseteq \cl\bigg(F(tx+(1-t)y)
				  +\sum_{k=0}^{\infty} 2d_{\Z}\big(2^k t\big)B\Big(\dfrac{x-y}{2^k}\Big)\bigg)
	}
	%%
	Si adem\'as, $0\in\cl(A(u)+K)$ para cada $u\in D-D$, entonces,
	\Eq{CV+}{
		tF(x)+(1-t)F(y)+A^\perp(t,x-y) \subseteq 
		\cl\big(F(tx+(1-t)y)&+B^\perp(t,x-y)\big)
	}
	para todo $x,y\in D$ y para todo $t\in\D\cap[0,1]$.
}

\begin{proof} 
	%%
	Para la demostraci\'on de \eq{CV}, primero vamos a demostrar que
	para todo $x,y\in D$, y para todo par de enteros $n,m$ con 
	$n\geq 1$, $0\leq m\leq 2^n$, se cumple la siguiente igualdad
	%%
	\Eq{CVn}{
		\frac{m}{2^n}F(x)&+\Big(1-\frac{m}{2^n}\Big)F(y) 
		+ \sum_{k=0}^{n-1}2 d_{\Z}\Big(2^k \frac{m}{2^n}\Big)A\Big(\dfrac{x-y}{2^k}\Big) \\
		&\subseteq \cl\bigg(F\Big(\frac{m}{2^n}x+\Big(1-\frac{m}{2^n}\Big)y\Big) 
		+ \sum_{k=0}^{n-1}d_{\Z}\Big(2^k \frac{m}{2^n}\Big)B\Big(\dfrac{x-y}{2^k}\Big)\bigg). 
	}
	%%
	
Fix $x,y\in D$ arbitrarily. To verify that \eq{CVn} holds, we will proceed by induction on $n$. 
For $m=0$ or for $m=2^n$, the inclusion \eq{CVn} is obvious (because $d_\Z$ vanishes at integer
values). Thus, for $n=1$, we need to check \eq{CVn} only for $m=1$. In that case, \eq{CVn} is
equivalent to the Jensen convexity assumption \eq{JCV}.

Now, suppose that the inclusion \eq{CVn} holds for some $n$ and let us prove that it is also valid for
$n+1$. Let $0<m<2^{n+1}$ be arbitrary. If $m$ is even, i.e., $m=2\ell$ for some $0<\ell<2^n$, then
the statement follows from the inductive assumption, because $\frac{m}{2^{n+1}}=\frac{\ell}{2^n}$.
Therefore, we may assume that $m$ is odd, i.e., $m=2\ell+1$, where $0\leq \ell<2^n$.
Observe that the fraction $\frac{2\ell+1}{2^{n+1}}$ can be represented as
\Eq{am}{
   \frac{2\ell+1}{2^{n+1}}=\frac12\Big(\frac{\ell+1}{2^{n}}+\frac{\ell}{2^{n}}\Big)
}
We prove that, for $k\in\{0,\dots,n-1\}$,
\Eq{JCd}{
   d_\Z\Big(2^k\frac{2\ell+1}{2^{n+1}}\Big)
    =\frac12\Big(d_\Z\Big(2^k\frac{\ell+1}{2^{n}}\Big)+d_\Z\Big(2^k\frac{\ell}{2^{n}}\Big)\Big).
}
In view of \eq{am}, it suffices to show that $d_\Z$ is affine on the interval 
$\Big[2^k\frac{\ell}{2^{n}},2^k\frac{\ell+1}{2^{n}}\Big]$. For this, it is enough to show
that the interior of this interval does not contain an element from $\frac12\Z$. On the contrary,
assume that, for some $j\in\Z$,
\Eq{*}{
   2^k\frac{\ell}{2^{n}}<\frac{j}2<2^k\frac{\ell+1}{2^{n}}.
}
Then we have that $\ell<2^{n-k-1}j<\ell+1$, which is a contradiction because $n-1\geq k$. 

Now, we can start proving that \eq{CVn} holds for $n+1$. Using \eq{am} and \eq{JCd}, we obtain
\Eq{*}{
  \frac{m}{2^{n+1}}F(x)&+\Big(1-\frac{m}{2^{n+1}}\Big)F(y)
   + \sum_{k=0}^{n}2 d_{\Z}\Big(2^k \frac{m}{2^{n+1}}\Big)A\Big(\frac{x-y}{2^k}\Big) \\
  &= \frac{2\ell+1}{2^{n+1}}F(x)+\Big(1-\frac{2\ell+1}{2^{n+1}}\Big)F(y)
   + \sum_{k=0}^{n}2 d_{\Z}\Big(2^k \frac{2\ell+1}{2^{n+1}}\Big)A\Big(\frac{x-y}{2^k}\Big) \\
  &\subseteq \frac12\Big(\frac{\ell+1}{2^{n}}F(x)+\Big(1-\frac{\ell+1}{2^{n}}\Big)F(y)\Big) 
     +\frac12\Big(\frac{\ell}{2^{n}}F(x)+\Big(1-\frac{\ell}{2^{n}}\Big)F(y)\Big)\\
  &\qquad+\sum_{k=0}^{n-1}\Big(d_\Z\Big(2^k\frac{\ell+1}{2^{n}}\Big)+d_\Z\Big(2^k\frac{\ell}{2^{n}}\Big)\Big)
             A\Big(\frac{x-y}{2^k}\Big) + A\Big(\frac{x-y}{2^n}\Big) \\
  &= \frac12\Big(\frac{\ell+1}{2^{n}}F(x)+\Big(1-\frac{\ell+1}{2^{n}}\Big)F(y)
     +\sum_{k=0}^{n-1}2 d_\Z\Big(2^k\frac{\ell+1}{2^{n}}\Big)A\Big(\frac{x-y}{2^k}\Big)\Big)\\
  &\qquad +\frac12\Big(\frac{\ell}{2^{n}}F(x)+\Big(1-\frac{\ell}{2^{n}}\Big)F(y)
   + \sum_{k=0}^{n-1}2d_\Z\Big(2^k\frac{\ell}{2^{n}}\Big)A\Big(\frac{x-y}{2^k}\Big)\Big)
   + A\Big(\frac{x-y}{2^n}\Big). 
}
By applying the inductive assumption \eq{CVn} with $m=\ell+1$ and $m=\ell$, we obtain that
\Eq{JCVa}{
  \frac{m}{2^{n+1}}F(x)&+\Big(1-\frac{m}{2^{n+1}}\Big)F(y)
   + \sum_{k=0}^{n}2 d_{\Z}\Big(2^k \frac{m}{2^{n+1}}\Big)A\Big(\frac{x-y}{2^k}\Big) \\
 &\subseteq \frac12\cl\bigg(F\Big(\frac{\ell+1}{2^{n}}x+\Big(1-\frac{\ell+1}{2^{n}}\Big)y\Big)
     +\sum_{k=0}^{n-1} 2d_\Z\Big(2^k\frac{\ell+1}{2^{n}}\Big)B\Big(\frac{x-y}{2^k}\Big)\bigg)\\
   &\qquad+\frac12\cl\bigg(F\Big(\frac{\ell}{2^{n}}x+\Big(1-\frac{\ell}{2^{n}}\Big)y\Big)
   + \sum_{k=0}^{n-1} 2d_\Z\Big(2^k\frac{\ell}{2^{n}}\Big)B\Big(\frac{x-y}{2^k}\Big)\bigg) 
      + A\Big(\frac{x-y}{2^n}\Big)\\
 &\subseteq \cl\bigg(\frac12\bigg(F\Big(\frac{\ell+1}{2^{n}}x+\Big(1-\frac{\ell+1}{2^{n}}\Big)y\Big)
     +F\Big(\frac{\ell}{2^{n}}x+\Big(1-\frac{\ell}{2^{n}}\Big)y\Big)\bigg)
     +A\Big(\frac{x-y}{2^n}\Big) \\
   &\qquad+ \sum_{k=0}^{n-1}\bigg(d_\Z\Big(2^k\frac{\ell+1}{2^{n}}\Big)B\Big(\frac{x-y}{2^k}\Big)
      +d_\Z\Big(2^k\frac{\ell}{2^{n}}\Big) B\Big(\frac{x-y}{2^k}\Big)\bigg)\bigg). \\
}
Now, using the Jensen convexity assumption \eq{JCV}, it follows that
\Eq{*}{
  \frac12\bigg(F\Big(\frac{\ell+1}{2^{n}}x+\Big(1-\frac{\ell+1}{2^{n}}\Big)y\Big)
     &+F\Big(\frac{\ell}{2^{n}}x+\Big(1-\frac{\ell}{2^{n}}\Big)y\Big)\bigg)
     +A\Big(\frac{x-y}{2^n}\Big) \\
 &\subseteq \cl\bigg(F\Big(\frac{2\ell+1}{2^{n+1}}x+\Big(1-\frac{2\ell+1}{2^{n+1}}\Big)y\Big)
     +B\Big(\frac{x-y}{2^n}\Big)\bigg).
}
On the other hand, using that the values of the set-valued map $B$ are closedly $K$-convex and 
equation \eq{JCd}, we obtain
\Eq{*}{
 \sum_{k=0}^{n-1}&\bigg(d_\Z\Big(2^k\frac{\ell+1}{2^{n}}\Big)B\Big(\frac{x-y}{2^k}\Big)
      +d_\Z\Big(2^k\frac{\ell}{2^{n}}\Big) B\Big(\frac{x-y}{2^k}\Big)\bigg) \\
  &\!\!\!\subseteq \cl\bigg(K+\sum_{k=0}^{n-1}\bigg(d_\Z\Big(2^k\frac{\ell+1}{2^{n}}\Big)
      +d_\Z\Big(2^k\frac{\ell}{2^{n}}\Big)\bigg) B\Big(\frac{x-y}{2^k}\Big)\bigg)
  = \cl\bigg(K+\sum_{k=0}^{n-1}2d_\Z\Big(2^k\frac{2\ell+1}{2^{n+1}}\Big)B\Big(\frac{x-y}{2^k}\Big)\bigg).
}
Combining the above two inclusions with \eq{JCVa} and replacing $2\ell+1$ by $m$, we arrive at
\Eq{*}{
  \frac{m}{2^{n+1}}F(x)&+\Big(1-\frac{m}{2^{n+1}}\Big)F(y)
   + \sum_{k=0}^{n}2 d_{\Z}\Big(2^k \frac{m}{2^{n+1}}\Big)A\Big(\frac{x-y}{2^k}\Big) \\
  &\subseteq   \cl\bigg(F\Big(\frac{m}{2^{n+1}}x+\Big(1-\frac{m}{2^{n+1}}\Big)y\Big)
     +B\Big(\frac{x-y}{2^n}\Big) + K
  +\sum_{k=0}^{n-1}2d_\Z\Big(2^k\frac{m}{2^{n+1}}\Big)B\Big(\frac{x-y}{2^k}\Big)\bigg)\\
  &= \cl\bigg(F\Big(\frac{m}{2^{n+1}}x+\Big(1-\frac{m}{2^{n+1}}\Big)y\Big)
  +\sum_{k=0}^{n}2d_\Z\Big(2^k\frac{m}{2^{n+1}}\Big)B\Big(\frac{x-y}{2^k}\Big)\bigg),
}
which shows that the statement is also valid for $n+1$. This completes the proof of the induction and the first 
assertion of the theorem.

Assume now that, for all $u\in D-D$, we have $0\in\cl(A(u)+K)$. To prove \eq{CV+}, let $x,y\in D$ and 
$t\in[0,1]\cap\D$ be fixed. If $t\in\{0,1\}$, then \eq{CV+} is trivial. In the rest of the proof assume that 
$t\in]0,1[\cap\D$. Then
\Eq{*}{
 tF(x)+(1-t)F(y)&+A^\perp(t,x-y) \\
 &\subseteq \cl\bigg(tF(x)+(1-t)F(y) + \bigcup_{n=0}^{\infty} \sum_{k=0}^{n}
                 2d_{\Z}(2^kt)A\Big(\frac{x-y}{2^k}\Big)\bigg)\\
 &\subseteq \cl\bigg(tF(x)+(1-t)F(y) + \bigcup_{n=0}^{\infty} \sum_{k=0}^{n}
                 2d_{\Z}(2^kt)\cl\Big(A\Big(\frac{x-y}{2^k}\Big)+K\Big)\bigg).
}
Now, since $t$ is a diadic number, there exists an $n\in\N$ and an odd number $\ell$ such that $t=\frac{\ell}{2^{m}}$. 
Then, for $k\geq m$, $d_{\Z}(2^kt)=0$. In addition, $0\in\cl\big(A\big(\frac{x-y)}{2^k}\big)+K\big)$ for 
all $k\in\{0,\dots,m-1\}$, therefore
\Eq{*}{
  \bigcup_{n=0}^{\infty} \sum_{k=0}^{n}2d_{\Z}(2^kt)\cl\Big(A\Big(\frac{x-y}{2^k}\Big)+K\Big)
  &= \sum_{k=0}^{m-1}2d_{\Z}(2^kt)\cl\Big(A\Big(\frac{x-y}{2^k}\Big)+K\Big) \\
  &= \sum_{k=0}^{\infty}2d_{\Z}(2^kt)\cl\Big(A\Big(\frac{x-y}{2^k}\Big)+K\Big).
}
Using this formula, the previous inclusion, the first part of the theorem and $d_\Z(t)>0$ we arrive at
\Eq{AB}{
 tF(x)+(1-t)F(y)&+A^\perp(t,x-y) \\
 &\subseteq \cl\bigg(tF(x)+(1-t)F(y) 
    + \sum_{k=0}^{\infty}2d_{\Z}(2^kt)\cl\Big(A\Big(\frac{x-y}{2^k}\Big)+K\Big)\bigg)\\
 &=\cl\bigg(tF(x)+(1-t)F(y) 
    + \sum_{k=0}^{\infty}2d_{\Z}(2^kt)\Big(A\Big(\frac{x-y}{2^k}\Big)+K\Big)\bigg) \\
 &=\cl\bigg(tF(x)+(1-t)F(y) 
    + \sum_{k=0}^{\infty}2d_{\Z}(2^kt)A\Big(\frac{x-y}{2^k}\Big)+\rec(B)\bigg) \\
 &\subseteq \cl\bigg(F(tx+(1-t)y) 
    + \sum_{k=0}^{\infty}2d_{\Z}(2^kt)B\Big(\frac{x-y}{2^k}\Big)+\rec(B)\bigg) \\
 &= \cl\bigg(F(tx+(1-t)y) 
    + \sum_{k=0}^{\infty}2d_{\Z}(2^kt)B\Big(\frac{x-y}{2^k}\Big)\bigg).
}
On the other hand, we have 
\Eq{*}{
  \sum_{k=0}^{\infty}2d_{\Z}(2^kt)B\Big(\frac{x-y}{2^k}\Big)
  =\sum_{k=0}^{m-1}2d_{\Z}(2^kt)B\Big(\frac{x-y}{2^k}\Big)
  \subseteq \cl\bigg(\bigcup_{n=0}^{\infty} \sum_{k=0}^{n}2d_{\Z}(2^kt)B\Big(\frac{x-y}{2^k}\Big)\bigg)
  =B^\perp(t,x-y),
}
which combined with \eq{AB} implies \eq{CV+} and completes the proof of the theorem.
\end{proof}

To formulate the most useful corollaries of our main results we postulate the following general hypotheses.
\begin{enumerate}[(H1)]
\item $D\subseteq X$ is a nonempty convex set, $K\subseteq Y$ is a nonempty convex cone; 
\item $S_0\subseteq Y$ is a closedly $K$-convex and closedly $K$-starshaped set;
\item $\varphi:(D-D)\to\R_+$ is a nonnegative function such that \eq{phi} holds for all $x\in D-D$. 
\end{enumerate}
Note that, by the convexity of $D$, the set $(D-D)$ is starshaped, thus \prp{Tab} can be applied. 

The next two corollaries are about approximately and strongly $K$-Jensen convex set-valued maps, respectively.

\Cor{Convex+1}{Assume that (H1), (H2), and (H3) hold and $F:D\to\P_0(Y)$ is a set-valued mapping which satisfies
\Eq{*}{
\dfrac{F(x) + F(y)}{2} \subseteq \cl\Big(F\Big(\dfrac{x+y}{2}\Big) 
    + \varphi(x-y)S_0 + K \Big) \qquad (x,y\in D).
}
Then 
\Eq{*}{
 tF(x)+(1-t)F(y) \subseteq \cl\big(F(tx+(1-t)y)+\varphi^\perp(t,x-y)S_0+K\big) \qquad
 (x,y\in D,\,t\in\D\cap[0,1]).
}}

\Cor{Convex+2}{Assume that (H1), (H2), and (H3) hold and $F:D\to\P_0(Y)$ is a set-valued mapping which satisfies
\Eq{*}{
\dfrac{F(x) + F(y)}{2} + \varphi(x-y)S_0 \subseteq \cl\Big(F\Big(\dfrac{x+y}{2}\Big) 
  + K \Big) \qquad (x,y\in D).
}
Then 
\Eq{*}{
 tF(x)+(1-t)F(y) + \varphi^\perp(t,x-y)S_0 \subseteq \cl\big(F(tx+(1-t)y) + K \big)\qquad 
 (x,y\in D,\,t\in\D\cap[0,1]).
}}

\begin{proof}[Proof of the Corollaries \ref{CConvex+1} and \ref{CConvex+2}]
Using the second assertion of \thm{ConvexTab} with the set-valued maps $A(u)=\{0\}$ and $B(u)=\varphi(u)S_0+K$ (resp., 
$A(u)=\varphi(u)S_0$ and $B(u)=K$) and applying the \prp{Tab}, we obtain \cor{Convex+1} 
(resp., \cor{Convex+2}). Note that, in both cases, $0\in\cl(A(u)+K)$ for all $u\in D-D$.
\end{proof}

The next result concerns the case of concavity type inclusions.

\Thm{ConcaveTab}{Let $D\subseteq X$ be a nonempty convex set and $A,B:(D-D)\to\P_0(Y)$ be
such that the values of the set-valued map $B$ are closedly $K$-convex, where $K:=\overline{\rec}(B)$. Let 
$F:D\to\P_0(Y)$ be a set-valued mapping which satisfies the Jensen-concavity-type inclusion
\Eq{JCC}{
F\Big(\dfrac{x+y}{2}\Big) + A(x-y) \subseteq \cl\bigg(\frac{F(x) + F(y)}{2} + B(x-y)\bigg) 
   \qquad (x,y\in D),
}
and has closedly $K$-convex values, i.e, $F(x)$ is closedly $K$-convex, for all $x\in D$.
Then $F$ satisfies the convexity type inclusion
\Eq{CC}{
 F(tx&+(1-t)y)+\sum_{k=0}^{\infty} 2d_{\Z}\big(2^k t\big)A\Big(\dfrac{x-y}{2^k}\Big) \\
&\subseteq \cl\bigg(tF(x)+(1-t)F(y)+\sum_{k=0}^{\infty} 2d_{\Z}\big(2^k t\big)B\Big(\dfrac{x-y}{2^k}\Big)\bigg)
   \qquad (x,y\in D,\,t\in\D\cap[0,1]).
}
If, in addition, $0\in\cl(A(u)+K)$ for every $u\in D-D$, then
\Eq{CC+}{
 F(tx+(1-t)y)+A^\perp(t,x-y) \subseteq \cl\big(tF(x)+(1-t)F(y)&+B^\perp(t,x-y)\big)\\
   & (x,y\in D,\,t\in\D\cap[0,1]).
}}

\begin{proof}
For the proof of \eq{CC}, we are going to show that, for all $x,y\in D$,
and for all integers $n,m$ with $n\geq 1$, $0\leq m\leq 2^n$,
\Eq{CCn}{
  F\Big(\frac{m}{2^n}x+\Big(1-\frac{m}{2^n}\Big)y\Big)
  &+ \sum_{k=0}^{n-1}2 d_{\Z}\Big(2^k \frac{m}{2^n}\Big)A\Big(\dfrac{x-y}{2^k}\Big) \\
 &\subseteq \cl\bigg(\frac{m}{2^n}F(x)+\Big(1-\frac{m}{2^n}\Big)F(y) 
   + \sum_{k=0}^{n-1}2d_{\Z}\Big(2^k \frac{m}{2^n}\Big)B\Big(\dfrac{x-y}{2^k}\Big)\bigg). 
}
Fix $x,y\in D$ arbitrarily. To verify that \eq{CCn} holds, we will proceed by induction on $n$. For 
$n=1$ we have that $0\leq m\leq 2$, but if $m=0$ or $m=2$ then, equation \eq{CCn} follows immediately. 
Thus we have only to check that \eq{CCn} is valid for $m=1$, which is trivial because for $n=1$ and $m=1$
\eq{CCn} is the same as \eq{JCC}. Now suppose that \eq{CCn} holds for $n\geq1$ and $0\leq m \leq 2^n$, and 
let us prove that it is also valid for $n+1$ and for $0\leq m\leq 2^{n+1}$. As in the proof for the 
convexity type inclusion, it will be enough to consider the case when $m$ has the form $m=2\ell +1$ for some 
$\ell\in\N\cup\{0\}$. Now we can start our proof for $n+1$ using the relations \eq{am} and \eq{JCd}, to obtain 
\Eq{*}{
 F\Big(\frac{2\ell +1}{2^{n+1}}x&+\Big(1-\frac{2\ell +1}{2^{n+1}}\Big)y\Big)
  + \sum_{k=0}^{n}2 d_{\Z}\Big(2^k \frac{2\ell +1}{2^{n+1}}\Big)A\Big(\dfrac{x-y}{2^k}\Big) \\ 
&=F\Bigg(\frac12\bigg(\Big(\frac{\ell}{2^{n}} + \frac{\ell +1}{2^{n}}\Big)x
	+	\Big(2-\frac{\ell +1}{2^{n}}-\frac{\ell}{2^{n}}\Big)y\bigg)\Bigg)
  + \sum_{k=0}^{n}2 d_{\Z}\Big(2^k \frac{2\ell +1}{2^{n+1}}\Big)A\Big(\dfrac{x-y}{2^k}\Big) \\
&=F\Bigg(\frac12\bigg[\frac{\ell}{2^{n}}x + \Big(1-\frac{\ell}{2^{n}}\Big)y
	+	\frac{\ell +1}{2^{n}}x+\Big(1-\frac{\ell +1}{2^{n}}\Big)y\bigg]\Bigg) + A\Big(\frac{x-y}{2^n}\Big) \\
&\qquad+ \sum_{k=0}^{n-1}\bigg[d_\Z\Big(2^k\frac{\ell+1}{2^{n}}\Big)+d_\Z\Big(2^k\frac{\ell}{2^{n}}\Big)\bigg]
	A\Big(\dfrac{x-y}{2^k}\Big) \\
&\subseteq F\Bigg(\frac12\bigg[\frac{\ell}{2^{n}}x + \Big(1-\frac{\ell}{2^{n}}\Big)y
	+	\frac{\ell +1}{2^{n}}x+\Big(1-\frac{\ell +1}{2^{n}}\Big)y\bigg]\Bigg) + A\Big(\frac{x-y}{2^n}\Big) \\
&\qquad+ \sum_{k=0}^{n-1}d_\Z\Big(2^k\frac{\ell}{2^{n}}\Big)A\Big(\dfrac{x-y}{2^k}\Big)
	+ \sum_{k=0}^{n-1}d_\Z\Big(2^k\frac{\ell+1}{2^{n}}\Big)A\Big(\dfrac{x-y}{2^k}\Big).
}
By the Jensen concavity property of $F$, we have that 
\Eq{JCCp}{
F\Bigg(\frac12\bigg[\frac{\ell}{2^{n}}x &+ \Big(1-\frac{\ell}{2^{n}}\Big)y
	+\frac{\ell +1}{2^{n}}x+\Big(1-\frac{\ell +1}{2^{n}}\Big)y\bigg]\Bigg) + A\Big(\frac{x-y}{2^n}\Big) \\
&\subseteq \cl\Bigg(\frac12 F\Big( \frac{\ell}{2^{n}}x + \Big(1-\frac{\ell}{2^{n}}\Big)y \Big)
	+\frac12 F\Big( \frac{\ell+1}{2^{n}}x+\Big(1-\frac{\ell +1}{2^{n}}\Big)y \Big)
	+ B\Big(\frac{x-y}{2^n}\Big)\Bigg).
}
Therefore, using \eq{JCCp}, we can obtain the following inclusions
\Eq{*}{
F\Big(\frac{2\ell +1}{2^{n+1}}x&+\Big(1-\frac{2\ell +1}{2^{n+1}}\Big)y\Big)
  + \sum_{k=0}^{n}2 d_{\Z}\Big(2^k \frac{2\ell +1}{2^{n+1}}\Big)A\Big(\dfrac{x-y}{2^k}\Big) \\ 
&\subseteq \cl\Bigg(\frac12 F\Big( \frac{\ell}{2^{n}}x + \Big(1-\frac{\ell}{2^{n}}\Big)y \Big)
  +\frac12 F\Big( \frac{\ell +1}{2^{n}}x+\Big(1-\frac{\ell +1}{2^{n}}\Big)y \Big)+ B\Big(\frac{x-y}{2^n}\Big)\Bigg) \\
&\qquad+ \sum_{k=0}^{n-1}d_\Z\Big(2^k\frac{\ell}{2^{n}}\Big)A\Big(\dfrac{x-y}{2^k}\Big)
	+ \sum_{k=0}^{n-1}d_\Z\Big(2^k\frac{\ell+1}{2^{n}}\Big)A\Big(\dfrac{x-y}{2^k}\Big)	\\
&\subseteq \cl\Bigg( \frac12\bigg[ F\Big(\frac{\ell}{2^{n}}x + \Big(1-\frac{\ell}{2^{n}}\Big)y \Big)
	+\sum_{k=0}^{n-1}2d_\Z\Big(2^k\frac{\ell}{2^{n}}\Big)A\Big(\dfrac{x-y}{2^k}\Big) \\
&\qquad + F\Big( \frac{\ell +1}{2^{n}}x+\Big(1-\frac{\ell +1}{2^{n}}\Big)y \Big)
	+\sum_{k=0}^{n-1}2d_\Z\Big(2^k\frac{\ell+1}{2^{n}}\Big)A\Big(\dfrac{x-y}{2^k}\Big)\bigg]
	+B\Big(\frac{x-y}{2^n}\Big)\Bigg).
}
Our inductive assumption for $m=\ell$ and $m=\ell+1$ gives us the following relations,
\Eq{Il}{
F\Big( \frac{\ell}{2^{n}}x + \Big(1-\frac{\ell}{2^{n}}\Big)y \Big)
	&+ \sum_{k=0}^{n-1}2d_\Z\Big(2^k\frac{\ell}{2^{n}}\Big)A\Big(\dfrac{x-y}{2^k}\Big)\\
&\subseteq \cl\Bigg(\frac{\ell}{2^{n}}F(x) +  \Big(1-\frac{\ell}{2^{n}}\Big)F(y)
	+ \sum_{k=0}^{n-1}2d_\Z\Big(2^k\frac{\ell}{2^{n}}\Big)B\Big(\dfrac{x-y}{2^k}\Big)\Bigg),
}
and
\Eq{Il+}{
F\Big( \frac{\ell+1}{2^{n}}x &+ \Big(1-\frac{\ell+1}{2^{n}}\Big)y \Big)
	+ \sum_{k=0}^{n-1}2d_\Z\Big(2^k\frac{\ell+1}{2^{n}}\Big)A\Big(\dfrac{x-y}{2^k}\Big)\\
&\subseteq \cl\Bigg(\frac{\ell+1}{2^{n}}F(x) +  \Big(1-\frac{\ell+1}{2^{n}}\Big)F(y)
	+ \sum_{k=0}^{n-1}2d_\Z\Big(2^k\frac{\ell+1}{2^{n}}\Big)B\Big(\dfrac{x-y}{2^k}\Big)\Bigg).
}
Thus, by \eq{JCd}, by using that $F$ and $B$ have closedly $K$-convex values, and by  
\eq{Il} and \eq{Il+}, we arrive at
\Eq{*}{
 F&\Big(\frac{2\ell +1}{2^{n+1}}x+\Big(1-\frac{2\ell +1}{2^{n+1}}\Big)y\Big)
  + \sum_{k=0}^{n}2 d_{\Z}\Big(2^k \frac{2\ell +1}{2^{n+1}}\Big)A\Big(\dfrac{x-y}{2^k}\Big) \\
&\subseteq \cl\Bigg( \frac12\Bigg[\frac{\ell}{2^{n}}F(x) +  \Big(1-\frac{\ell}{2^{n}}\Big)F(y)
	+ \sum_{k=0}^{n-1}2d_\Z\Big(2^k\frac{\ell}{2^{n}}\Big)B\Big(\dfrac{x-y}{2^k}\Big) \\
&\qquad + \frac{\ell+1}{2^{n}}F(x) +  \Big(1-\frac{\ell+1}{2^{n}}\Big)F(y)
	+ \sum_{k=0}^{n-1}2d_\Z\Big(2^k\frac{\ell+1}{2^{n}}\Big)B\Big(\dfrac{x-y}{2^k}\Big)\Bigg]
	+ B\Big(\frac{x-y}{2^n}\Big)\Bigg) \\
&\subseteq \cl\Bigg( \frac12\Bigg[\frac{2\ell+1}{2^{n}}F(x) +  \Big(2-\frac{2\ell+1}{2^{n}}\Big)F(y)\\ 
&\qquad + \sum_{k=0}^{n-1}2\Big(d_\Z\Big(2^k\frac{\ell}{2^{n}}\Big) + d_\Z\Big(2^k\frac{\ell+1}{2^{n}}\Big)
	  \Big)B\Big(\dfrac{x-y}{2^k}\Big) \Bigg] + K + B\Big(\frac{x-y}{2^n}\Big) \Bigg) \\
&= \cl\Bigg( \frac{2\ell+1}{2^{n+1}}F(x) +  \Big(1-\frac{2\ell+1}{2^{n+1}}\Big)F(y) 
	+ \sum_{k=0}^{n}2d_\Z\Big(2^k\frac{2\ell+1}{2^{n+1}}\Big) B\Big(\dfrac{x-y}{2^k}\Big)\Bigg),
}
which shows that the statement is also valid for $n+1$. Thus the induction, and therefore, 
the proof of the theorem is complete.

The last statement of the theorem can be proved in the same manner as that of \thm{ConvexTab}.
\end{proof}

The next two corollaries are about approximately and strongly $K$-Jensen concave set-valued mapping, 
respectively.

\Cor{Concave+1}{Assume that (H1), (H2), and (H3) hold and let $F:D\to\P_0(Y)$ be a set-valued mapping with 
closedly $K$-convex values satisfying
\Eq{*}{
F\Big(\dfrac{x+y}{2}\Big) \subseteq \cl\Big(\dfrac{F(x) + F(y)}{2} 
  + \varphi(x-y)S_0 + K \Big) \qquad (x,y\in D).
}
Then 
\Eq{*}{
 F(tx+(1-t)y) \subseteq \cl\big(tF(x)+(1-t)F(y) + \varphi^\perp(t,x-y)S_0 + K \big)\qquad  
 (x,y\in D,\,t\in\D\cap[0,1]).
}}

\Cor{Concave+2}{Assume that (H1), (H2), and (H3) hold and $F:D\to\P_0(Y)$ be a set-valued mapping with 
closedly $K$-convex values satisfying
\Eq{*}{
F\Big(\dfrac{x+y}{2}\Big) + \varphi(x-y)S_0 \subseteq \cl\Big(\dfrac{F(x) + F(y)}{2} 
    + K\Big) \qquad (x,y\in D).
}
Then 
\Eq{*}{
 F(tx+(1-t)y)+\varphi^\perp(t,x-y)S_0 \subseteq \cl\big(tF(x)+(1-t)F(y) + K \big)\qquad  
 (x,y\in D,\,t\in\D\cap[0,1]).
}}

\begin{proof}[Proof of the Corollaries \ref{CConcave+1} and \ref{CConcave+2}]
Using \thm{ConcaveTab} with the set-valued maps $A(u)=\{0\}$ and $B(u)=\varphi(u)S_0+K$ (resp., 
$A(u)=\varphi(u)S_0$ and $B(u)=K$) and applying the \prp{Tab}, we obtain \cor{Concave+1} 
(resp., \cor{Concave+2}). In both settings, we have that $K\subseteq\overline{\rec}(B)$, thus
the assumption that the values of $F$ are closedly $K$-convex implies that the values are also closedly 
$\overline{\rec}(B)$-convex.
\end{proof}

%\chapter*{Conclusiones.}
\markboth{hEncabezado Izquierdoi}{Conclusiones y Recomendaciones}

Los resultados mencionados en los Capítulos 2 y 3 se pueden obtener como consecuencia directa de
los corolarios obtenidos a partir de los Teoremas \ref{TConvex} y \ref{TConcave}.
En el caso de funciones a valores reales, el Teorema de Bernstein--Doetsch \cite{BerDoe15},
puede ser obtenido a partir del \cor{Convex+1} si consideramos 
$Y:=\R$, $K:=\R_+$, $S_0:=[-1,0]$, $F(x):=\{f(x)\}$, y $\varphi(x):=0$. En este caso,
\eq{JCV+1} es equivalente a \eq{midCvxFunc} y \eq{CV+1} es equivalente a 
\eq{cvxFunc}.

En general para una función positiva arbitraria $\varphi$ se tiene la siguiente
fórmula 
$$
\varphi^T(t,u)=\sum_{n=0}^{\infty}\frac{1}{2^n}\varphi\big(2d_{\Z}(2^nt)u\big)
\qquad(t\in\R,\,x\in D),
$$
la cual coincide con la expresión que aparece 
del lado derecho en la desigualdad \eq{TabTab2}
cuando $\varphi(u)=\alpha(\|u\|)$. Esto significa que el \cor{Convex+1}
generaliza al Teorema \ref{TTabTab2} y por lo tanto generaliza
también a los Teoremas \ref{THazPal3} y \ref{TNgNik}.

Los resultados de Averna, Cardinali, Nikodem, Papalini 
\cite{AveCar90,CarNikPap93,Nik86,Nik87a,Nik87c,Nik89,Pap90} 
y por Borwein \cite{Bor77} que están relacionados a $K$-Jensen convexidad/concavidad
para multifunciones y funciones vectoriales también pueden ser obtenidos directamente. 
Numerosos resultados obtenidos para Jensen convexidad aproximada por Makó and Páles 
\cite{MakPal10b,MakPal13b} y por Mureńko, Ja. Tabor, Jó. Tabor, y Żoldak
\cite{MurTabTab12,TabTab09b,TabTab09a,TabTabZol10b,TabTabZol10a} son generalizados por los
 Corolarios \ref{CConvex+1}--\ref{CConcave+2}.
De manera similar, usando la forma explícita de la función
$T_2$ descrita en \rem{Tak} y el \cor{Convex+2}, se pueden obtener 
de manera simple los resultados de Azócar et. al. \cite{AzoGimNikSan11} y Leiva et. al.
\cite{LeiMerNikSan13} que están relacionados con midconvexidad fuerte.

\addcontentsline{toc}{chapter}{Conclusiones}
\chapter*{Conclusiones.}
\markboth{hEncabezado Izquierdoi}{Conclusiones y Recomendaciones}

Los resultados mencionados en los Capítulos 2 y 3 se pueden obtener como consecuencia directa de
los corolarios obtenidos a partir de los Teoremas \ref{TConvex} y \ref{TConcave}.
En el caso de funciones a valores reales, el Teorema de Bernstein--Doetsch \cite{BerDoe15},
puede ser obtenido a partir del \cor{Convex+1} si consideramos 
$Y:=\R$, $K:=\R_+$, $S_0:=[-1,0]$, $F(x):=\{f(x)\}$, y $\varphi(x):=0$. En este caso,
\eq{JCV+1} es equivalente a \eq{midCvxFunc} y \eq{CV+1} es equivalente a 
\eq{cvxFunc}.

En general para una función positiva arbitraria $\varphi$ se tiene la siguiente
fórmula 
$$
\varphi^T(t,u)=\sum_{n=0}^{\infty}\frac{1}{2^n}\varphi\big(2d_{\Z}(2^nt)u\big)
\qquad(t\in\R,\,x\in D),
$$
la cual coincide con la expresión que aparece 
del lado derecho en la desigualdad \eq{TabTab2}
cuando $\varphi(u)=\alpha(\|u\|)$. Esto significa que el \cor{Convex+1}
generaliza al Teorema \ref{TTabTab2} y por lo tanto generaliza
también a los Teoremas \ref{THazPal3} y \ref{TNgNik}.

Los resultados de Averna, Cardinali, Nikodem, Papalini 
\cite{AveCar90,CarNikPap93,Nik86,Nik87a,Nik87c,Nik89,Pap90} 
y por Borwein \cite{Bor77} que están relacionados a $K$-Jensen convexidad/concavidad
para multifunciones y funciones vectoriales también pueden ser obtenidos directamente. 
Numerosos resultados obtenidos para Jensen convexidad aproximada por Makó and Páles 
\cite{MakPal10b,MakPal13b} y por Mureńko, Ja. Tabor, Jó. Tabor, y Żoldak
\cite{MurTabTab12,TabTab09b,TabTab09a,TabTabZol10b,TabTabZol10a} son generalizados por los
 Corolarios \ref{CConvex+1}--\ref{CConcave+2}.
De manera similar, usando la forma explícita de la función
$T_2$ descrita en \rem{Tak} y el \cor{Convex+2}, se pueden obtener 
de manera simple los resultados de Azócar et. al. \cite{AzoGimNikSan11} y Leiva et. al.
\cite{LeiMerNikSan13} que están relacionados con midconvexidad fuerte.


\bibliography{funcequ,publ}
%\bibliography{bibliotecaFich}
\addcontentsline{toc}{chapter}{Bibliograf\'ia}
\bibliographystyle{amsplain}


\end{document}