\chapter*{Conclusiones.}
\markboth{hEncabezado Izquierdoi}{Conclusiones y Recomendaciones}

Los resultados mencionados en los Capítulos 2 y 3 se pueden obtener como consecuencia directa de
los corolarios obtenidos a partir de los Teoremas \ref{TConvex} y \ref{TConcave}.
En el caso de funciones a valores reales, el Teorema de Bernstein--Doetsch \cite{BerDoe15},
puede ser obtenido a partir del \cor{Convex+1} si consideramos 
$Y:=\R$, $K:=\R_+$, $S_0:=[-1,0]$, $F(x):=\{f(x)\}$, y $\varphi(x):=0$. En este caso,
\eq{JCV+1} es equivalente a \eq{midCvxFunc} y \eq{CV+1} es equivalente a 
\eq{cvxFunc}.

En general para una función positiva arbitraria $\varphi$ se tiene la siguiente
fórmula 
$$
\varphi^T(t,u)=\sum_{n=0}^{\infty}\frac{1}{2^n}\varphi\big(2d_{\Z}(2^nt)u\big)
\qquad(t\in\R,\,x\in D),
$$
la cual coincide con la expresión que aparece 
del lado derecho en la desigualdad \eq{TabTab2}
cuando $\varphi(u)=\alpha(\|u\|)$. Esto significa que el \cor{Convex+1}
generaliza al Teorema \ref{TTabTab2} y por lo tanto generaliza
también a los Teoremas \ref{THazPal3} y \ref{TNgNik}.

Los resultados de Averna, Cardinali, Nikodem, Papalini 
\cite{AveCar90,CarNikPap93,Nik86,Nik87a,Nik87c,Nik89,Pap90} 
y por Borwein \cite{Bor77} que están relacionados a $K$-Jensen convexidad/concavidad
para multifunciones y funciones vectoriales también pueden ser obtenidos directamente. 
Numerosos resultados obtenidos para Jensen convexidad aproximada por Makó and Páles 
\cite{MakPal10b,MakPal13b} y por Mureńko, Ja. Tabor, Jó. Tabor, y Żoldak
\cite{MurTabTab12,TabTab09b,TabTab09a,TabTabZol10b,TabTabZol10a} son generalizados por los
 Corolarios \ref{CConvex+1}--\ref{CConcave+2}.
De manera similar, usando la forma explícita de la función
$T_2$ descrita en \rem{Tak} y el \cor{Convex+2}, se pueden obtener 
de manera simple los resultados de Azócar et. al. \cite{AzoGimNikSan11} y Leiva et. al.
\cite{LeiMerNikSan13} que están relacionados con midconvexidad fuerte.
