\chapter{T\'ermino de error de tipo Takagi-Tabor.}

\section{Convexidad y concavidad sobre los racionales di\'adicos.}
\setcounter{theorem}{0}

Los resultados principales de esta secci\'on est\'an contenidos
en el siguiente par de teoremas. Denotaremos por $\D$ al conjunto
de los n\'umeros racionales di\'adicos, i.e., $\D$ consiste en
los n\'umeros de la forma $\frac{k}{2^n}$, donde, $k\in\Z$ y $n\in\N$.

\Lem{lemTec}{
	%%
	Sean $n,\ell\in\N$ dos n\'umeros naturales arbitrarios. Entonces,
	para todo $k\in\{0,\ldots,n-1\}$ se tiene que
	\Eq{dZafin}{
		%%
		d_\Z\Big(2^k\frac{2\ell+1}{2^{n+1}}\Big)
		=\frac12\Big(d_\Z\Big(2^k\frac{\ell+1}{2^{n}}\Big)+d_\Z\Big(2^k\frac{\ell}{2^{n}}\Big)\Big)
	} 
}
\begin{proof}
	%%
	Observemos que la fracci\'on $\frac{2\ell+1}{2^{n+1}}$ puede ser 
	representada como
	%%
	\Eq{*}{
		\frac{2\ell+1}{2^{n+1}}=\frac12\Big(\frac{\ell+1}{2^{n}}+\frac{\ell}{2^{n}}\Big)
	}
	%%
	Es suficiente demostrar que $d_\Z$ es af\'in
	en el intervalo $\Big[2^k\frac{\ell}{2^{n}},2^k\frac{\ell+1}{2^{n}}\Big]$.
	%%
	Para esto, veremos que el interior de este intervalo
	no contiene un elemento de $\frac12\Z$. Supongamos que para alg\'un $j\in\Z$
	%%
	\Eq{aux}{
		2^k\frac{\ell}{2^{n}}<\frac{j}2<2^k\frac{\ell+1}{2^{n}}.
	}
	Por lo tanto $\ell<2^{n-k-1}j<\ell+1$, lo que es una contradicci\'on
	ya que $n-1\geq k$ y en consecuencia, es falso suponer \eq{aux}.
		
\end{proof}

\Thm{ConvexTab}{
	%%
	Sea $D\subseteq X$ un subconjunto convexo no vac\'io, y sean
	$A,B:(D-D)\to\P_0(Y)$ tales que, los valores de la multifunci\'on
	$B$ son semi $K$-convexos, donde $K$ representa la clausura del
	cono recesi\'on asociado a $B$. 	
	Sea $F:D\to\P_0(Y)$ una multifunci\'on que satisface la siguiente
	inclusi\'on de convexidad tipo Jensen
	%%
	\Eq{JCV}{
		\dfrac{F(x) + F(y)}{2} + A(x-y) \subseteq 
		\cl\Big(F\Big(\dfrac{x+y}{2}\Big) + B(x-y)\Big) 
		\qquad (x,y\in D).
	}
	%%
	Entonces para $x,y\in D$ y para todo $t\in\D\cap[0,1]$ se tiene
	que $F$ satisface la siguiente inclusi\'on
	%%	
	\Eq{CV}{
		tF(x)&+(1-t)F(y)+\sum_{k=0}^{\infty} 2d_{\Z}\big(2^k t\big)A\Big(\dfrac{x-y}{2^k}\Big) \\
		&\subseteq \cl\bigg(F(tx+(1-t)y)
				  +\sum_{k=0}^{\infty} 2d_{\Z}\big(2^k t\big)B\Big(\dfrac{x-y}{2^k}\Big)\bigg)
	}
	%%
	Si adem\'as, $0\in\cl(A(u)+K)$ para cada $u\in D-D$, entonces,
	\Eq{CV+}{
		tF(x)+(1-t)F(y)+A^\perp(t,x-y) \subseteq 
		\cl\big(F(tx+(1-t)y)&+B^\perp(t,x-y)\big)
	}
	para todo $x,y\in D$ y para todo $t\in\D\cap[0,1]$.
}

\begin{proof} 
	%%
	Para la demostraci\'on de \eq{CV}, primero vamos a demostrar que
	para todo $x,y\in D$, y para todo par de enteros $n,m$ con 
	$n\geq 1$, $0\leq m\leq 2^n$, se cumple la siguiente igualdad
	%%
	\Eq{CVn}{
		\frac{m}{2^n}F(x)&+\Big(1-\frac{m}{2^n}\Big)F(y) 
		+ \sum_{k=0}^{n-1}2 d_{\Z}\Big(2^k \frac{m}{2^n}\Big)A\Big(\dfrac{x-y}{2^k}\Big) \\
		&\subseteq \cl\bigg(F\Big(\frac{m}{2^n}x+\Big(1-\frac{m}{2^n}\Big)y\Big) 
		+ \sum_{k=0}^{n-1}d_{\Z}\Big(2^k \frac{m}{2^n}\Big)B\Big(\dfrac{x-y}{2^k}\Big)\bigg). 
	}
	%%
	Fijemos $x,y\in D$ arbitrarios. Para verificar que \eq{CVn} se cumple, 
	vamos a proceder por inducci\'on sobre $n$. 
	%%
	Para $m=0$ o para $m=2^n$, la inclusi\'on \eq{CVn} se sigue de inmediato
	ya que la funci\'on $d_\Z$ se anula en $\Z$. As\'i, para $n=1$, solo es necesario
	verificar \eq{CVn} para $m=1$. En ese caso, \eq{CVn} es equivalente
	a la inclusi\'on de convexidad tipo Jensen en la \eq{JCV}.
	%%
	Ahora, supongamos que la inclusi\'on \eq{CVn} se cumple para alg\'un $n$
	y demostremos que tambi\'en es v\'alida para $n+1$. Sea $0<m<2^{n+1}$
	arbitrario. Si $m$ es par, i.e., $m=2\ell$ para alg\'un $0<\ell<2^n$,
	entonces, $\frac{m}{2^{n+1}}=\frac{\ell}{2^n}$ y as\'i, el resultado
	es consecuencia inmediata de la hip\'otesis inductiva. En vista de
	esta observaci\'on, podemos asumir entonces que $m$ es impar, i.e., 
	$m = 2\ell + 1$ para alg\'un $0 < \ell < 2^n$.
	
	En vista del \lem{lemTec}, se tenemos que para todo $k\in\{0,\ldots,n-1\}$ 
	\Eq{am}{
		%%
		d_\Z\Big(2^k\frac{2\ell+1}{2^{n+1}}\Big)
		=\frac12\Big(d_\Z\Big(2^k\frac{\ell+1}{2^{n}}\Big)+d_\Z\Big(2^k\frac{\ell}{2^{n}}\Big)\Big).
	} 
	%%
	Ahora bien, usando \eq{am} y \eq{JCd} obtenemos lo siguiente
	
	\Eq{*}{
	  \frac{m}{2^{n+1}}F(x)&+\Big(1-\frac{m}{2^{n+1}}\Big)F(y)
	   + \sum_{k=0}^{n}2 d_{\Z}\Big(2^k \frac{m}{2^{n+1}}\Big)A\Big(\frac{x-y}{2^k}\Big) \\
	  &= \frac{2\ell+1}{2^{n+1}}F(x)+\Big(1-\frac{2\ell+1}{2^{n+1}}\Big)F(y)
	   + \sum_{k=0}^{n}2 d_{\Z}\Big(2^k \frac{2\ell+1}{2^{n+1}}\Big)A\Big(\frac{x-y}{2^k}\Big) \\
	  &\subseteq \frac12\Big(\frac{\ell+1}{2^{n}}F(x)+\Big(1-\frac{\ell+1}{2^{n}}\Big)F(y)\Big) 
	     +\frac12\Big(\frac{\ell}{2^{n}}F(x)+\Big(1-\frac{\ell}{2^{n}}\Big)F(y)\Big)\\
	  &\qquad+\sum_{k=0}^{n-1}\Big(d_\Z\Big(2^k\frac{\ell+1}{2^{n}}\Big)+d_\Z\Big(2^k\frac{\ell}{2^{n}}\Big)\Big)
	             A\Big(\frac{x-y}{2^k}\Big) + A\Big(\frac{x-y}{2^n}\Big) \\
	  &= \frac12\Big(\frac{\ell+1}{2^{n}}F(x)+\Big(1-\frac{\ell+1}{2^{n}}\Big)F(y)
	     +\sum_{k=0}^{n-1}2 d_\Z\Big(2^k\frac{\ell+1}{2^{n}}\Big)A\Big(\frac{x-y}{2^k}\Big)\Big)\\
	  &\qquad +\frac12\Big(\frac{\ell}{2^{n}}F(x)+\Big(1-\frac{\ell}{2^{n}}\Big)F(y)
	   + \sum_{k=0}^{n-1}2d_\Z\Big(2^k\frac{\ell}{2^{n}}\Big)A\Big(\frac{x-y}{2^k}\Big)\Big) \\
	  &\qquad+ A\Big(\frac{x-y}{2^n}\Big). 
	}
	%%
	Aplicando la hip\'otesis inductiva  \eq{CVn} con $m=\ell+1$ y $m=\ell$, 
	obtenemos lo siguiente
	%%
	\Eq{JCVa}{
		%%
		\frac{m}{2^{n+1}}F(x)&+\Big(1-\frac{m}{2^{n+1}}\Big)F(y)
			+ \sum_{k=0}^{n}2 d_{\Z}\Big(2^k \frac{m}{2^{n+1}}\Big)A\Big(\frac{x-y}{2^k}\Big) \\
		%%
		&\subseteq \frac12\cl\bigg(F\Big(\frac{\ell+1}{2^{n}}x+\Big(1-\frac{\ell+1}{2^{n}}\Big)y\Big)
			+\sum_{k=0}^{n-1} 2d_\Z\Big(2^k\frac{\ell+1}{2^{n}}\Big)B\Big(\frac{x-y}{2^k}\Big)\bigg)\\
		%%
		&\qquad+\frac12\cl\bigg(F\Big(\frac{\ell}{2^{n}}x+\Big(1-\frac{\ell}{2^{n}}\Big)y\Big)
			+ \sum_{k=0}^{n-1} 2d_\Z\Big(2^k\frac{\ell}{2^{n}}\Big)B\Big(\frac{x-y}{2^k}\Big)\bigg) 
			+ A\Big(\frac{x-y}{2^n}\Big)\\
		%%
		&\subseteq \cl\bigg(\frac12\bigg(F\Big(\frac{\ell+1}{2^{n}}x+\Big(1-\frac{\ell+1}{2^{n}}\Big)y\Big)
			+F\Big(\frac{\ell}{2^{n}}x+\Big(1-\frac{\ell}{2^{n}}\Big)y\Big)\bigg)
			+A\Big(\frac{x-y}{2^n}\Big) \\
		%%
		&\qquad+ \sum_{k=0}^{n-1}\bigg(d_\Z\Big(2^k\frac{\ell+1}{2^{n}}\Big)B\Big(\frac{x-y}{2^k}\Big)
			+d_\Z\Big(2^k\frac{\ell}{2^{n}}\Big) B\Big(\frac{x-y}{2^k}\Big)\bigg)\bigg). \\
		%%
	}
Now, using the Jensen convexity assumption \eq{JCV}, it follows that
\Eq{*}{
  \frac12\bigg(F\Big(\frac{\ell+1}{2^{n}}x+\Big(1-\frac{\ell+1}{2^{n}}\Big)y\Big)
     &+F\Big(\frac{\ell}{2^{n}}x+\Big(1-\frac{\ell}{2^{n}}\Big)y\Big)\bigg)
     +A\Big(\frac{x-y}{2^n}\Big) \\
 &\subseteq \cl\bigg(F\Big(\frac{2\ell+1}{2^{n+1}}x+\Big(1-\frac{2\ell+1}{2^{n+1}}\Big)y\Big)
     +B\Big(\frac{x-y}{2^n}\Big)\bigg).
}
On the other hand, using that the values of the set-valued map $B$ are closedly $K$-convex and 
equation \eq{JCd}, we obtain
\Eq{*}{
 \sum_{k=0}^{n-1}&\bigg(d_\Z\Big(2^k\frac{\ell+1}{2^{n}}\Big)B\Big(\frac{x-y}{2^k}\Big)
      +d_\Z\Big(2^k\frac{\ell}{2^{n}}\Big) B\Big(\frac{x-y}{2^k}\Big)\bigg) \\
  &\!\!\!\subseteq \cl\bigg(K+\sum_{k=0}^{n-1}\bigg(d_\Z\Big(2^k\frac{\ell+1}{2^{n}}\Big)
      +d_\Z\Big(2^k\frac{\ell}{2^{n}}\Big)\bigg) B\Big(\frac{x-y}{2^k}\Big)\bigg)
  = \cl\bigg(K+\sum_{k=0}^{n-1}2d_\Z\Big(2^k\frac{2\ell+1}{2^{n+1}}\Big)B\Big(\frac{x-y}{2^k}\Big)\bigg).
}
Combining the above two inclusions with \eq{JCVa} and replacing $2\ell+1$ by $m$, we arrive at
\Eq{*}{
  \frac{m}{2^{n+1}}F(x)&+\Big(1-\frac{m}{2^{n+1}}\Big)F(y)
   + \sum_{k=0}^{n}2 d_{\Z}\Big(2^k \frac{m}{2^{n+1}}\Big)A\Big(\frac{x-y}{2^k}\Big) \\
  &\subseteq   \cl\bigg(F\Big(\frac{m}{2^{n+1}}x+\Big(1-\frac{m}{2^{n+1}}\Big)y\Big)
     +B\Big(\frac{x-y}{2^n}\Big) + K
  +\sum_{k=0}^{n-1}2d_\Z\Big(2^k\frac{m}{2^{n+1}}\Big)B\Big(\frac{x-y}{2^k}\Big)\bigg)\\
  &= \cl\bigg(F\Big(\frac{m}{2^{n+1}}x+\Big(1-\frac{m}{2^{n+1}}\Big)y\Big)
  +\sum_{k=0}^{n}2d_\Z\Big(2^k\frac{m}{2^{n+1}}\Big)B\Big(\frac{x-y}{2^k}\Big)\bigg),
}
which shows that the statement is also valid for $n+1$. This completes the proof of the induction and the first 
assertion of the theorem.

Assume now that, for all $u\in D-D$, we have $0\in\cl(A(u)+K)$. To prove \eq{CV+}, let $x,y\in D$ and 
$t\in[0,1]\cap\D$ be fixed. If $t\in\{0,1\}$, then \eq{CV+} is trivial. In the rest of the proof assume that 
$t\in]0,1[\cap\D$. Then
\Eq{*}{
 tF(x)+(1-t)F(y)&+A^\perp(t,x-y) \\
 &\subseteq \cl\bigg(tF(x)+(1-t)F(y) + \bigcup_{n=0}^{\infty} \sum_{k=0}^{n}
                 2d_{\Z}(2^kt)A\Big(\frac{x-y}{2^k}\Big)\bigg)\\
 &\subseteq \cl\bigg(tF(x)+(1-t)F(y) + \bigcup_{n=0}^{\infty} \sum_{k=0}^{n}
                 2d_{\Z}(2^kt)\cl\Big(A\Big(\frac{x-y}{2^k}\Big)+K\Big)\bigg).
}
Now, since $t$ is a diadic number, there exists an $n\in\N$ and an odd number $\ell$ such that $t=\frac{\ell}{2^{m}}$. 
Then, for $k\geq m$, $d_{\Z}(2^kt)=0$. In addition, $0\in\cl\big(A\big(\frac{x-y)}{2^k}\big)+K\big)$ for 
all $k\in\{0,\dots,m-1\}$, therefore
\Eq{*}{
  \bigcup_{n=0}^{\infty} \sum_{k=0}^{n}2d_{\Z}(2^kt)\cl\Big(A\Big(\frac{x-y}{2^k}\Big)+K\Big)
  &= \sum_{k=0}^{m-1}2d_{\Z}(2^kt)\cl\Big(A\Big(\frac{x-y}{2^k}\Big)+K\Big) \\
  &= \sum_{k=0}^{\infty}2d_{\Z}(2^kt)\cl\Big(A\Big(\frac{x-y}{2^k}\Big)+K\Big).
}
Using this formula, the previous inclusion, the first part of the theorem and $d_\Z(t)>0$ we arrive at
\Eq{AB}{
 tF(x)+(1-t)F(y)&+A^\perp(t,x-y) \\
 &\subseteq \cl\bigg(tF(x)+(1-t)F(y) 
    + \sum_{k=0}^{\infty}2d_{\Z}(2^kt)\cl\Big(A\Big(\frac{x-y}{2^k}\Big)+K\Big)\bigg)\\
 &=\cl\bigg(tF(x)+(1-t)F(y) 
    + \sum_{k=0}^{\infty}2d_{\Z}(2^kt)\Big(A\Big(\frac{x-y}{2^k}\Big)+K\Big)\bigg) \\
 &=\cl\bigg(tF(x)+(1-t)F(y) 
    + \sum_{k=0}^{\infty}2d_{\Z}(2^kt)A\Big(\frac{x-y}{2^k}\Big)+\rec(B)\bigg) \\
 &\subseteq \cl\bigg(F(tx+(1-t)y) 
    + \sum_{k=0}^{\infty}2d_{\Z}(2^kt)B\Big(\frac{x-y}{2^k}\Big)+\rec(B)\bigg) \\
 &= \cl\bigg(F(tx+(1-t)y) 
    + \sum_{k=0}^{\infty}2d_{\Z}(2^kt)B\Big(\frac{x-y}{2^k}\Big)\bigg).
}
On the other hand, we have 
\Eq{*}{
  \sum_{k=0}^{\infty}2d_{\Z}(2^kt)B\Big(\frac{x-y}{2^k}\Big)
  =\sum_{k=0}^{m-1}2d_{\Z}(2^kt)B\Big(\frac{x-y}{2^k}\Big)
  \subseteq \cl\bigg(\bigcup_{n=0}^{\infty} \sum_{k=0}^{n}2d_{\Z}(2^kt)B\Big(\frac{x-y}{2^k}\Big)\bigg)
  =B^\perp(t,x-y),
}
which combined with \eq{AB} implies \eq{CV+} and completes the proof of the theorem.
\end{proof}

To formulate the most useful corollaries of our main results we postulate the following general hypotheses.
\begin{enumerate}[(H1)]
\item $D\subseteq X$ is a nonempty convex set, $K\subseteq Y$ is a nonempty convex cone; 
\item $S_0\subseteq Y$ is a closedly $K$-convex and closedly $K$-starshaped set;
\item $\varphi:(D-D)\to\R_+$ is a nonnegative function such that \eq{phi} holds for all $x\in D-D$. 
\end{enumerate}
Note that, by the convexity of $D$, the set $(D-D)$ is starshaped, thus \prp{Tab} can be applied. 

The next two corollaries are about approximately and strongly $K$-Jensen convex set-valued maps, respectively.

\Cor{Convex+1}{Assume that (H1), (H2), and (H3) hold and $F:D\to\P_0(Y)$ is a set-valued mapping which satisfies
\Eq{*}{
\dfrac{F(x) + F(y)}{2} \subseteq \cl\Big(F\Big(\dfrac{x+y}{2}\Big) 
    + \varphi(x-y)S_0 + K \Big) \qquad (x,y\in D).
}
Then 
\Eq{*}{
 tF(x)+(1-t)F(y) \subseteq \cl\big(F(tx+(1-t)y)+\varphi^\perp(t,x-y)S_0+K\big) \qquad
 (x,y\in D,\,t\in\D\cap[0,1]).
}}

\Cor{Convex+2}{Assume that (H1), (H2), and (H3) hold and $F:D\to\P_0(Y)$ is a set-valued mapping which satisfies
\Eq{*}{
\dfrac{F(x) + F(y)}{2} + \varphi(x-y)S_0 \subseteq \cl\Big(F\Big(\dfrac{x+y}{2}\Big) 
  + K \Big) \qquad (x,y\in D).
}
Then 
\Eq{*}{
 tF(x)+(1-t)F(y) + \varphi^\perp(t,x-y)S_0 \subseteq \cl\big(F(tx+(1-t)y) + K \big)\qquad 
 (x,y\in D,\,t\in\D\cap[0,1]).
}}

\begin{proof}[Proof of the Corollaries \ref{CConvex+1} and \ref{CConvex+2}]
Using the second assertion of \thm{ConvexTab} with the set-valued maps $A(u)=\{0\}$ and $B(u)=\varphi(u)S_0+K$ (resp., 
$A(u)=\varphi(u)S_0$ and $B(u)=K$) and applying the \prp{Tab}, we obtain \cor{Convex+1} 
(resp., \cor{Convex+2}). Note that, in both cases, $0\in\cl(A(u)+K)$ for all $u\in D-D$.
\end{proof}

The next result concerns the case of concavity type inclusions.

\Thm{ConcaveTab}{Let $D\subseteq X$ be a nonempty convex set and $A,B:(D-D)\to\P_0(Y)$ be
such that the values of the set-valued map $B$ are closedly $K$-convex, where $K:=\overline{\rec}(B)$. Let 
$F:D\to\P_0(Y)$ be a set-valued mapping which satisfies the Jensen-concavity-type inclusion
\Eq{JCC}{
F\Big(\dfrac{x+y}{2}\Big) + A(x-y) \subseteq \cl\bigg(\frac{F(x) + F(y)}{2} + B(x-y)\bigg) 
   \qquad (x,y\in D),
}
and has closedly $K$-convex values, i.e, $F(x)$ is closedly $K$-convex, for all $x\in D$.
Then $F$ satisfies the convexity type inclusion
\Eq{CC}{
 F(tx&+(1-t)y)+\sum_{k=0}^{\infty} 2d_{\Z}\big(2^k t\big)A\Big(\dfrac{x-y}{2^k}\Big) \\
&\subseteq \cl\bigg(tF(x)+(1-t)F(y)+\sum_{k=0}^{\infty} 2d_{\Z}\big(2^k t\big)B\Big(\dfrac{x-y}{2^k}\Big)\bigg)
   \qquad (x,y\in D,\,t\in\D\cap[0,1]).
}
If, in addition, $0\in\cl(A(u)+K)$ for every $u\in D-D$, then
\Eq{CC+}{
 F(tx+(1-t)y)+A^\perp(t,x-y) \subseteq \cl\big(tF(x)+(1-t)F(y)&+B^\perp(t,x-y)\big)\\
   & (x,y\in D,\,t\in\D\cap[0,1]).
}}

\begin{proof}
For the proof of \eq{CC}, we are going to show that, for all $x,y\in D$,
and for all integers $n,m$ with $n\geq 1$, $0\leq m\leq 2^n$,
\Eq{CCn}{
  F\Big(\frac{m}{2^n}x+\Big(1-\frac{m}{2^n}\Big)y\Big)
  &+ \sum_{k=0}^{n-1}2 d_{\Z}\Big(2^k \frac{m}{2^n}\Big)A\Big(\dfrac{x-y}{2^k}\Big) \\
 &\subseteq \cl\bigg(\frac{m}{2^n}F(x)+\Big(1-\frac{m}{2^n}\Big)F(y) 
   + \sum_{k=0}^{n-1}2d_{\Z}\Big(2^k \frac{m}{2^n}\Big)B\Big(\dfrac{x-y}{2^k}\Big)\bigg). 
}
Fix $x,y\in D$ arbitrarily. To verify that \eq{CCn} holds, we will proceed by induction on $n$. For 
$n=1$ we have that $0\leq m\leq 2$, but if $m=0$ or $m=2$ then, equation \eq{CCn} follows immediately. 
Thus we have only to check that \eq{CCn} is valid for $m=1$, which is trivial because for $n=1$ and $m=1$
\eq{CCn} is the same as \eq{JCC}. Now suppose that \eq{CCn} holds for $n\geq1$ and $0\leq m \leq 2^n$, and 
let us prove that it is also valid for $n+1$ and for $0\leq m\leq 2^{n+1}$. As in the proof for the 
convexity type inclusion, it will be enough to consider the case when $m$ has the form $m=2\ell +1$ for some 
$\ell\in\N\cup\{0\}$. Now we can start our proof for $n+1$ using the relations \eq{am} and \eq{JCd}, to obtain 
\Eq{*}{
 F\Big(\frac{2\ell +1}{2^{n+1}}x&+\Big(1-\frac{2\ell +1}{2^{n+1}}\Big)y\Big)
  + \sum_{k=0}^{n}2 d_{\Z}\Big(2^k \frac{2\ell +1}{2^{n+1}}\Big)A\Big(\dfrac{x-y}{2^k}\Big) \\ 
&=F\Bigg(\frac12\bigg(\Big(\frac{\ell}{2^{n}} + \frac{\ell +1}{2^{n}}\Big)x
	+	\Big(2-\frac{\ell +1}{2^{n}}-\frac{\ell}{2^{n}}\Big)y\bigg)\Bigg)
  + \sum_{k=0}^{n}2 d_{\Z}\Big(2^k \frac{2\ell +1}{2^{n+1}}\Big)A\Big(\dfrac{x-y}{2^k}\Big) \\
&=F\Bigg(\frac12\bigg[\frac{\ell}{2^{n}}x + \Big(1-\frac{\ell}{2^{n}}\Big)y
	+	\frac{\ell +1}{2^{n}}x+\Big(1-\frac{\ell +1}{2^{n}}\Big)y\bigg]\Bigg) + A\Big(\frac{x-y}{2^n}\Big) \\
&\qquad+ \sum_{k=0}^{n-1}\bigg[d_\Z\Big(2^k\frac{\ell+1}{2^{n}}\Big)+d_\Z\Big(2^k\frac{\ell}{2^{n}}\Big)\bigg]
	A\Big(\dfrac{x-y}{2^k}\Big) \\
&\subseteq F\Bigg(\frac12\bigg[\frac{\ell}{2^{n}}x + \Big(1-\frac{\ell}{2^{n}}\Big)y
	+	\frac{\ell +1}{2^{n}}x+\Big(1-\frac{\ell +1}{2^{n}}\Big)y\bigg]\Bigg) + A\Big(\frac{x-y}{2^n}\Big) \\
&\qquad+ \sum_{k=0}^{n-1}d_\Z\Big(2^k\frac{\ell}{2^{n}}\Big)A\Big(\dfrac{x-y}{2^k}\Big)
	+ \sum_{k=0}^{n-1}d_\Z\Big(2^k\frac{\ell+1}{2^{n}}\Big)A\Big(\dfrac{x-y}{2^k}\Big).
}
By the Jensen concavity property of $F$, we have that 
\Eq{JCCp}{
F\Bigg(\frac12\bigg[\frac{\ell}{2^{n}}x &+ \Big(1-\frac{\ell}{2^{n}}\Big)y
	+\frac{\ell +1}{2^{n}}x+\Big(1-\frac{\ell +1}{2^{n}}\Big)y\bigg]\Bigg) + A\Big(\frac{x-y}{2^n}\Big) \\
&\subseteq \cl\Bigg(\frac12 F\Big( \frac{\ell}{2^{n}}x + \Big(1-\frac{\ell}{2^{n}}\Big)y \Big)
	+\frac12 F\Big( \frac{\ell+1}{2^{n}}x+\Big(1-\frac{\ell +1}{2^{n}}\Big)y \Big)
	+ B\Big(\frac{x-y}{2^n}\Big)\Bigg).
}
Therefore, using \eq{JCCp}, we can obtain the following inclusions
\Eq{*}{
F\Big(\frac{2\ell +1}{2^{n+1}}x&+\Big(1-\frac{2\ell +1}{2^{n+1}}\Big)y\Big)
  + \sum_{k=0}^{n}2 d_{\Z}\Big(2^k \frac{2\ell +1}{2^{n+1}}\Big)A\Big(\dfrac{x-y}{2^k}\Big) \\ 
&\subseteq \cl\Bigg(\frac12 F\Big( \frac{\ell}{2^{n}}x + \Big(1-\frac{\ell}{2^{n}}\Big)y \Big)
  +\frac12 F\Big( \frac{\ell +1}{2^{n}}x+\Big(1-\frac{\ell +1}{2^{n}}\Big)y \Big)+ B\Big(\frac{x-y}{2^n}\Big)\Bigg) \\
&\qquad+ \sum_{k=0}^{n-1}d_\Z\Big(2^k\frac{\ell}{2^{n}}\Big)A\Big(\dfrac{x-y}{2^k}\Big)
	+ \sum_{k=0}^{n-1}d_\Z\Big(2^k\frac{\ell+1}{2^{n}}\Big)A\Big(\dfrac{x-y}{2^k}\Big)	\\
&\subseteq \cl\Bigg( \frac12\bigg[ F\Big(\frac{\ell}{2^{n}}x + \Big(1-\frac{\ell}{2^{n}}\Big)y \Big)
	+\sum_{k=0}^{n-1}2d_\Z\Big(2^k\frac{\ell}{2^{n}}\Big)A\Big(\dfrac{x-y}{2^k}\Big) \\
&\qquad + F\Big( \frac{\ell +1}{2^{n}}x+\Big(1-\frac{\ell +1}{2^{n}}\Big)y \Big)
	+\sum_{k=0}^{n-1}2d_\Z\Big(2^k\frac{\ell+1}{2^{n}}\Big)A\Big(\dfrac{x-y}{2^k}\Big)\bigg]
	+B\Big(\frac{x-y}{2^n}\Big)\Bigg).
}
Our inductive assumption for $m=\ell$ and $m=\ell+1$ gives us the following relations,
\Eq{Il}{
F\Big( \frac{\ell}{2^{n}}x + \Big(1-\frac{\ell}{2^{n}}\Big)y \Big)
	&+ \sum_{k=0}^{n-1}2d_\Z\Big(2^k\frac{\ell}{2^{n}}\Big)A\Big(\dfrac{x-y}{2^k}\Big)\\
&\subseteq \cl\Bigg(\frac{\ell}{2^{n}}F(x) +  \Big(1-\frac{\ell}{2^{n}}\Big)F(y)
	+ \sum_{k=0}^{n-1}2d_\Z\Big(2^k\frac{\ell}{2^{n}}\Big)B\Big(\dfrac{x-y}{2^k}\Big)\Bigg),
}
and
\Eq{Il+}{
F\Big( \frac{\ell+1}{2^{n}}x &+ \Big(1-\frac{\ell+1}{2^{n}}\Big)y \Big)
	+ \sum_{k=0}^{n-1}2d_\Z\Big(2^k\frac{\ell+1}{2^{n}}\Big)A\Big(\dfrac{x-y}{2^k}\Big)\\
&\subseteq \cl\Bigg(\frac{\ell+1}{2^{n}}F(x) +  \Big(1-\frac{\ell+1}{2^{n}}\Big)F(y)
	+ \sum_{k=0}^{n-1}2d_\Z\Big(2^k\frac{\ell+1}{2^{n}}\Big)B\Big(\dfrac{x-y}{2^k}\Big)\Bigg).
}
Thus, by \eq{JCd}, by using that $F$ and $B$ have closedly $K$-convex values, and by  
\eq{Il} and \eq{Il+}, we arrive at
\Eq{*}{
 F&\Big(\frac{2\ell +1}{2^{n+1}}x+\Big(1-\frac{2\ell +1}{2^{n+1}}\Big)y\Big)
  + \sum_{k=0}^{n}2 d_{\Z}\Big(2^k \frac{2\ell +1}{2^{n+1}}\Big)A\Big(\dfrac{x-y}{2^k}\Big) \\
&\subseteq \cl\Bigg( \frac12\Bigg[\frac{\ell}{2^{n}}F(x) +  \Big(1-\frac{\ell}{2^{n}}\Big)F(y)
	+ \sum_{k=0}^{n-1}2d_\Z\Big(2^k\frac{\ell}{2^{n}}\Big)B\Big(\dfrac{x-y}{2^k}\Big) \\
&\qquad + \frac{\ell+1}{2^{n}}F(x) +  \Big(1-\frac{\ell+1}{2^{n}}\Big)F(y)
	+ \sum_{k=0}^{n-1}2d_\Z\Big(2^k\frac{\ell+1}{2^{n}}\Big)B\Big(\dfrac{x-y}{2^k}\Big)\Bigg]
	+ B\Big(\frac{x-y}{2^n}\Big)\Bigg) \\
&\subseteq \cl\Bigg( \frac12\Bigg[\frac{2\ell+1}{2^{n}}F(x) +  \Big(2-\frac{2\ell+1}{2^{n}}\Big)F(y)\\ 
&\qquad + \sum_{k=0}^{n-1}2\Big(d_\Z\Big(2^k\frac{\ell}{2^{n}}\Big) + d_\Z\Big(2^k\frac{\ell+1}{2^{n}}\Big)
	  \Big)B\Big(\dfrac{x-y}{2^k}\Big) \Bigg] + K + B\Big(\frac{x-y}{2^n}\Big) \Bigg) \\
&= \cl\Bigg( \frac{2\ell+1}{2^{n+1}}F(x) +  \Big(1-\frac{2\ell+1}{2^{n+1}}\Big)F(y) 
	+ \sum_{k=0}^{n}2d_\Z\Big(2^k\frac{2\ell+1}{2^{n+1}}\Big) B\Big(\dfrac{x-y}{2^k}\Big)\Bigg),
}
which shows that the statement is also valid for $n+1$. Thus the induction, and therefore, 
the proof of the theorem is complete.

The last statement of the theorem can be proved in the same manner as that of \thm{ConvexTab}.
\end{proof}

The next two corollaries are about approximately and strongly $K$-Jensen concave set-valued mapping, 
respectively.

\Cor{Concave+1}{Assume that (H1), (H2), and (H3) hold and let $F:D\to\P_0(Y)$ be a set-valued mapping with 
closedly $K$-convex values satisfying
\Eq{*}{
F\Big(\dfrac{x+y}{2}\Big) \subseteq \cl\Big(\dfrac{F(x) + F(y)}{2} 
  + \varphi(x-y)S_0 + K \Big) \qquad (x,y\in D).
}
Then 
\Eq{*}{
 F(tx+(1-t)y) \subseteq \cl\big(tF(x)+(1-t)F(y) + \varphi^\perp(t,x-y)S_0 + K \big)\qquad  
 (x,y\in D,\,t\in\D\cap[0,1]).
}}

\Cor{Concave+2}{Assume that (H1), (H2), and (H3) hold and $F:D\to\P_0(Y)$ be a set-valued mapping with 
closedly $K$-convex values satisfying
\Eq{*}{
F\Big(\dfrac{x+y}{2}\Big) + \varphi(x-y)S_0 \subseteq \cl\Big(\dfrac{F(x) + F(y)}{2} 
    + K\Big) \qquad (x,y\in D).
}
Then 
\Eq{*}{
 F(tx+(1-t)y)+\varphi^\perp(t,x-y)S_0 \subseteq \cl\big(tF(x)+(1-t)F(y) + K \big)\qquad  
 (x,y\in D,\,t\in\D\cap[0,1]).
}}

\begin{proof}[Proof of the Corollaries \ref{CConcave+1} and \ref{CConcave+2}]
Using \thm{ConcaveTab} with the set-valued maps $A(u)=\{0\}$ and $B(u)=\varphi(u)S_0+K$ (resp., 
$A(u)=\varphi(u)S_0$ and $B(u)=K$) and applying the \prp{Tab}, we obtain \cor{Concave+1} 
(resp., \cor{Concave+2}). In both settings, we have that $K\subseteq\overline{\rec}(B)$, thus
the assumption that the values of $F$ are closedly $K$-convex implies that the values are also closedly 
$\overline{\rec}(B)$-convex.
\end{proof}
