\chapter{T\'ermino de error de tipo Takagi-Tabor.}

\setcounter{theorem}{0}

En el cap\'itulo 2 de esta monograf\'ia, mencionamos algunos de los 
resultados obtenidos por distintos autores que est\'an relacionados
con la teor\'ia de convexidad aproximada. En este cap\'itulo, se pretende
construir las bases para obtener un resultado an\'alogo al
\thm{TabTab3} en el contexto de las multifunciones.
En consecuencia, vamos a introducir una nueva transformaci\'on:
La transformaci\'on de Takagi-Tabor para una multifunci\'on.

\section{Tranformaci\'on de Takagi-Tabor para una multifunci\'on}


Supongamos que $D\subseteq X$ es un conjunto estrellado
y consideremos una multifunci\'on $S$ definida en $D$ a valores
en $\P_0(Y)$. 

\Defi{TakTabTrans}{
	La \textit{transformaci\'on de Takagi-Tabor} de $S$,
	es la multifunci\'on $$S^\perp:\R\times D\to \P_0(Y)$$
	definida por 
	\Eq{Tab0}{
		S^\perp(t,x):=\cl\bigg(\bigcup_{n=0}^{\infty} \sum_{k=0}^{n}
		2d_{\Z}(2^kt)S\Big(\frac{x}{2^k}\Big)\bigg)\qquad((t,x)\in\R\times D),
	}
	donde, $d_{\Z}:\R\to\R$ es la funci\'on definida en \eq{dz}.
}

En vista de que la funci\'on $d_\Z$ se anula en $\Z$, es inmediato
ver que $S^\perp(t,x)=\{0\}$ para todo $(t,x)\in\Z\times D$.

En el siguiente lemma, estableceremos la relaci\'on entre una multifunci\'on
y su tranformaci\'on de Takagi-Tabor, as\'i como, la relaci\'on entre los
conos recesi\'on asociados.

\Lem{TT}{
	Sean $D\subseteq X$ un conjunto estrellado y $S:D\to\P_0(Y)$.
	Entonces,
	%%
	\Eq{TT1}{
		S^\perp\big(\tfrac12,x)\big)=\cl(S(x)) \qquad (x\in D)
	}
	y
	\Eq{TT2}{
		\overline{\rec}\, S\subseteq \rec \big(S^\perp(t,x)\big) \qquad((t,x)\in(\R\setminus\Z)\times D).
	}
}
\begin{proof} 
	Observe que $d_{\Z}\big(\tfrac12\big)=\tfrac12$ y $d_{\Z}\big(2^k\cdot\tfrac12\big)=0$
	para $k\in\N$. As\'i
	\Eq{*}{
		S^\perp\big(\tfrac12,x\big)
		=\cl\bigg(\bigcup_{n=0}^{\infty} \sum_{k=0}^{n}2d_{\Z}\Big(\frac{2^k}2\Big)S\Big(\frac{x}{2^k}\Big)\bigg)
		=\cl(S(x))
	}
	para todo $x\in D$.
	
	Para la demostraci\'on de \eq{TT2}, denotemos por $K:=\rec S$. 
	Entonces, para todo $x\in D$, tenemos lo siguiente
	%%
	\Eq{*}{
		S(x)+K\subseteq S(x).
	}
	%%
	Sea $(t,x)\in(\R\setminus\Z)\times D$ arbitrario. Usando la inclusi\'on
	anterior y $d_\Z(t)>0$, tenemos que
	%%
	\Eq{*}{
		\bigg(\sum_{k=0}^{n}2d_{\Z}(2^kt)S\Big(\frac{x}{2^k}\Big)\bigg)+K
		&=\sum_{k=0}^{n}2d_{\Z}(2^kt)\Big(S\Big(\frac{x}{2^k}\Big)+K\Big) \\
		&\subseteq \sum_{k=0}^{n}2d_{\Z}(2^kt)S\Big(\frac{x}{2^k}\Big)\subseteq S^\perp(t,x).
	}
	Por lo tanto,
	\Eq{*}{
		\bigcup_{n=0}^\infty \bigg(\sum_{k=0}^{n}2d_{\Z}(2^kt)S\Big(\frac{x}{2^k}\Big)\bigg)+K
		\subseteq S^\perp(t,x),
	}
	lo cual implica que 
	\Eq{*}{
		S^\perp(t,x)+\cl(K)\subseteq S^\perp(t,x).
	}
	Esto completa la demostraci\'on de \eq{TT2}.
\end{proof}

El siguiente resultado establece una f\'ormula para la transformaci\'on
de Takagi-Tabor de una multifunci\'on que se contruye como el producto
de una funci\'on escalar no-negativa y un subconjunto de $Y$.

\Prp{Tab}{
	Sean $D\subseteq X$ un conjunto estrellado, $K\subseteq Y$ un cono convexo
	y $S_0\subseteq Y$ un conjunto semi-$K$-convexo y semi-$K$-estrellado.
	Sea $\varphi:D\to \R_+$ una funci\'on no-negativa tal que 
	\Eq{phi}{
		\sum_{k=0}^\infty \varphi\Big(\frac{x}{2^n}\Big)<\infty \qquad(x\in D).
	}
	Definamos $S:D\to\P_0(Y)$ por $S(x):=\varphi(x)S_0+K$. Entonces
	\Eq{Tab1}{
		S^\perp(t,x)=\cl\big(\varphi^\perp(t,x) S_0+K\big) \qquad ((t,x)\in(\R\setminus\Z)\times D), 
	}
	donde
	\Eq{Tab2}{
		\varphi^\perp(t,x):=\sum_{n=0}^{\infty}2d_{\Z}(2^nt)\varphi\Big(\frac{x}{2^n}\Big)
		\qquad((t,x)\in\R\times D).
	}
}
	
\begin{proof} 
	Usando la condici\'on de convergencia \eq{phi} y la estimaci\'on 
	$0\leq 2d_\Z\leq1$, se sigue que la serie en \eq{Tab2} converge 
	uniformemente en $t$ y en consecuencia, para todo $x\in D$,
	la aplicaci\'on $t\mapsto\varphi^\perp(t,x)$ es continua en $\R$.
	
	Para demostrar \eq{Tab1}, fijemos $(t,x)\in(\R\setminus\Z)\times D$.
	Entonces, $d_\Z(t)>0$. Si, para un elemento $x\in D$, tenemos 
	$\varphi^\perp(t,x)=0$, entonces, (como $\varphi$ es no-negativa)
	se sigue que $d_{\Z}(2^nt)\varphi\big(\frac{x}{2^n}\big)=0$
	para todo $n\geq0$. Luego,
	%%
	\Eq{*}{
		S^\perp(t,x)
		&=\cl\bigg(\bigcup_{n=0}^{\infty} \sum_{k=0}^{n}2d_{\Z}(2^kt)\Big(\varphi\Big(\frac{x}{2^k}\Big)S_0+K\Big)\bigg)\\
		&=\cl\bigg(\bigcup_{n=0}^{\infty} 
		\Big(\Big(\sum_{k=0}^{n}2d_{\Z}(2^kt)\varphi\Big(\frac{x}{2^k}\Big)S_0\Big)+K\Big)\bigg)\\
		&=\cl(K)=\cl\big(\varphi^\perp(t,x) S_0+K\big),
	}
	lo que muestra la validez de \eq{Tab1} en este caso.
	
	Por lo tanto, de aqui en adelante, tambi\'en podemos asumir que 
	$x\in D$ se elige de tal manera que $\varphi^\perp(t,x)\neq0$.
	
	Usando que $S_0$ es semi-$K$-convexo y semi-$K$-estrellado,
	tenemos, para todo $t_0,\dots,t_n\geq0$ y $t_0+\cdots+t_n\leq t$, 
	que
	%%
	\Eq{ttt}{
		t_0S_0+\cdots+t_nS_0\subseteq \cl(tS_0+K).
	}
	De hecho, si $t_1=\cdots=t_n=t=0$, entonces, \eq{ttt} es equivalente
	a la inclusi\'on trivial $0\in\cl(K)$. Si $t_1=\cdots=t_n=0<t$, entonces
	\eq{ttt} se reduce a $0\in t\cl(S_0+K)$, la cual es 
	una consecuencia de la propiedad de $S_0$ de ser semi-$K$-estrellado.
	Finalmente, si $t_0+\cdots+t_n>0$, entonces usando que $S_0$ is semi-$K$-convexo
	y semi-$K$-estrellado, obtenemos
	\Eq{*}{
		t_0S_0+\cdots+t_nS_0
		&=(t_0+\cdots+t_n)\Big(\frac{t_0}{t_0+\cdots+t_n}S_0+\cdots+\frac{t_n}{t_0+\cdots+t_n}S_0\Big)\\
		&\subseteq(t_0+\cdots+t_n)\cl(S_0+K)=\cl((t_0+\cdots+t_n)S_0+K)\\
		&=\cl\Big(t\frac{t_0+\cdots+t_n}{t}S_0+K\Big)\subseteq \cl(t\cl(S_0+K)+K)=\cl(tS_0+K).
	}
	Para completar la demostraci\'on de \eq{Tab1} en el caso general,
	verificaremos primero la inclusi\'on $\subseteq$.
	Aplicando \eq{ttt}, para todo $n\geq0$ obtenemos
	%%
	\Eq{*}{
		\sum_{k=0}^{n}2d_{\Z}(2^kt)\varphi\Big(\frac{x}{2^k}\Big)S_0
		\subseteq\cl\Bigg(\bigg(\sum_{k=0}^{\infty}2d_{\Z}(2^kt)\varphi\Big(\frac{x}{2^k}\Big)\bigg)S_0+K\Bigg)
		=\cl\big(\varphi^\perp(t,x)S_0+K\big).
	}
	Por lo tanto (usando tambi\'en $d_\Z(t)>0$),
	\Eq{*}{
		\sum_{k=0}^{n}2d_{\Z}(2^kt)\Big(\varphi\Big(\frac{x}{2^k}\Big)S_0+K\Big)
		&\subseteq \bigg(\sum_{k=0}^{n}2d_{\Z}(2^kt)\varphi\Big(\frac{x}{2^k}\Big)S_0\bigg)+K \\
		&\subseteq \cl\big(\varphi^\perp(t,x)S_0+K\big)+K=\cl\big(\varphi^\perp(t,x)S_0+K\big).
	}
	Como esta \'ultima inclusi\'on es v\'alida para todo $n\geq0$,
	la inclusi\'on deseada se sigue de inmediato.

	Para la inclusi\'on opuesta, en vista de que $\varphi^\perp(t,x)\neq0$,
	podemos escoger $n_0$ tal que $d_{\Z}(2^{n_0}t)\varphi\big(\frac{x}{2^{n_0}}\big)>0$.
	Ahora observemos que, para todo $n\geq n_0$,
	\Eq{*}{
		\bigg(\sum_{k=0}^n & 2d_\Z(2^kt)\varphi\Big(\frac{x}{2^k}\Big)\bigg)S_0 + K
		=\bigg(\sum_{k=0}^n2d_\Z(2^kt)\varphi\Big(\frac{x}{2^k}\Big)\bigg)(S_0+K) \\
		&\subseteq\sum_{k=0}^n2d_\Z(2^kt)\bigg(\varphi\Big(\frac{x}{2^k}\Big)S_0+\varphi\Big(\frac{x}{2^k}\Big)K\bigg)
		\subseteq\sum_{k=0}^n2d_\Z(2^kt)\bigg(\varphi\Big(\frac{x}{2^k}\Big)S_0+\cl(K)\bigg)\\
		&\subseteq\cl\bigg(\sum_{k=0}^n2d_\Z(2^kt)\bigg(\varphi\Big(\frac{x}{2^k}\Big)S_0+K\bigg)\bigg)
		\subseteq\cl\bigg(\bigcup_{\ell=0}^\infty\sum_{k=0}^\ell 2d_\Z(2^kt)S\Big(\frac{x}{2^k}\Big)\bigg)
		= S^\perp(t,x).
	}
	Sea $u\in K$ y $v\in S_0$. Entonces, la inclusi\'on anterior nos lleva a que para
	todo $n\geq n_0$,
	%%
	\Eq{*}{
		y_n:=\sum_{k=0}^n2d_\Z(2^kt)\varphi\Big(\frac{x}{2^k}\Big)v+u\in S^\perp(t,x).
	}
	El lado derecho es un conjunto cerrado, y en consecuencia, el l\'imite de la 
	suceci\'on $(y_n)$ tambi\'en estar\'a en $S^\perp(t,x)$. Por lo tanto, 
	para todo $u\in K$ y $v\in S_0$,
	%%
	\Eq{*}{
		\varphi^\perp(t,x)v+u
		=\sum_{k=0}^\infty 2d_\Z(2^kt)\varphi\Big(\frac{x}{2^k}\Big)v+u
		=\lim_{n\to\infty} y_n
		\in S^\perp(t,x).
	}
	lo que implica que 
	\Eq{*}{
		\varphi^\perp(t,x)S_0+K\subseteq S^\perp(t,x).
	}
	Esto completa la demostraci\'on de la inclusi\'on $\supseteq$ en \eq{Tab1}.
\end{proof}
	
\Cor{Tak}{
	Sean $X$ un espacio normado, $D\subseteq X$ un conjunto estrellado,
	$K\subseteq Y$ un cono convexo, $S_0\subseteq Y$ un conjunto
	semi-$K$-convexo que contiene al $0\in Y$ y $\alpha>0$.
	Definamos, $S:D\to\P_0(Y)$ mediante $S(x):=\|x\|^\alpha S_0+K$.
	Entonces,
	\Eq{Tab1+}{
		S^\perp(t,x)
		=\cl\big(\tau_\alpha(t)\|x\|^\alpha S_0+K\big) \qquad((t,x)\in(\R\setminus\Z)\times D), \\
	}               
	donde $\tau_\alpha:\R\to\R$ esta definida por
	\Eq{Tab2+}{
		\tau_\alpha(t):=\sum_{n=0}^{\infty}2^{1-\alpha n} d_{\Z}(2^nt)
		\qquad(t\in\R).
	}
}
\begin{proof} 
	Para demostrar nuestra afirmaci\'on, podemos aplicar la \prp{Tab}
	con la funci\'on $\varphi$ definida por $\varphi(x):=\|x\|^\alpha$.
	Observe que
	\Eq{*}{
		\sum_{n=0}^{\infty}\varphi\Big(\frac{x}{2^n}\Big)
		=\sum_{n=0}^{\infty}\frac{\|x\|^\alpha}{2^{\alpha n}}
		=\frac{2^\alpha}{2^\alpha-1}\|x\|^\alpha < \infty
		\qquad(x\in D)
	}
	y
	\Eq{*}{
		\varphi^\perp(t,x)&=\sum_{n=0}^{\infty}2d_{\Z}(2^nt)\varphi\Big(\frac{x}{2^n}\Big)\\
		&=\sum_{n=0}^{\infty}2d_{\Z}(2^nt)\frac{\|x\|^\alpha}{2^{\alpha n}}
		=\tau_\alpha(t)\|x\|^\alpha
		\qquad((t,x)\in\R\times D).
	}
	Por lo tanto, \eq{Tab1+} es una consecuencia de la f\'ormula \eq{Tab1} en la \prp{Tab}.
\end{proof}
		
\Rem{Tak}{
	Un caso part\'icular importante es cuando $\alpha=1$,
	ya que, $\tau_1(t)=2T$ donde $T$ es la funci\'on de Takagi que
	definimos en la introducci\'on mediante la f\'ormula
	\Eq{*}{
		T(t):=\sum_{n=0}^\infty\frac{d_\Z(2^nt)}{2^n} \qquad(t\in\R)
	} 
	Observe que $\tau_\alpha$ (para cualquier $\alpha>0$) satisface
	la ecuaci\'on funcional
	\Eq{TT}{
		\tau_\alpha(t)=2d_{\Z}(t)+\frac1{2^\alpha}\tau_\alpha(2t) \qquad(t\in\R).
	}
	Aplicando el teorema del punto fijo de Banach, esta ecuaci\'on
	funcional tiene una \'unica soluci\'on en el espacio de Banach de las 
	funciones a valores reales y acotadas definidas sobre la recta real
	(equipado con la norma del supremo).
}
El siguiente lema sera \'util para demostrar los resultados principales 
del pr\'oximo cap\'itulo relacionados con inclusiones de tipo
convexidad y concavidad.

Para su formulaci\'on, vamos a introducir una propiedad de convergencia 
para una sucesi\'on de conjuntos en $Y$. Dado un cono convexo $K\subseteq Y$	
y una sucesi\'on $(S_k)\subseteq Y$, tenemos la siguiente 

\Defi{serKCauch}{
	La sucesi\'on $(S_k)\subseteq Y$, es \textit{$K$-Cauchy en serie} si,
	para todo $U\in\U(Y)$, existe $m\in\N$ tal que, para todo
	$n\geq m$, 
	\Eq{iii}{
		\sum_{k=m}^nS_k \subseteq U+K.
	}
}

\Lem{CchyConv}{
	Sea $K\subseteq Y$ un cono convexo y sea $(S_k)$ una sucesi\'on
	de subconjuntos no vac\'ios de $Y$ tales que 
	\begin{enumerate}[(i)]
		\item Para todo $k\geq 0$, el conjunto $S_k$ es semi-$K$-estrellado, y
		semi-$K$-acotado inferiormente;
		\item La sucesi\'on $(S_k)$ es $K$-Cauchy en serie. 
	\end{enumerate}
	Entonces, para todo  $U\in\U(Y)$, existe un n\'umero positivo
	$\delta$ tal que, para todo $t,s\in\R$ con $|t-s|<\delta$,
	\Eq{ts0}{
		\cl\bigg(\bigcup_{n=0}^\infty\sum_{k=0}^n d_\Z(2^kt)S_k\bigg) 
		\subseteq \cl\bigg(\bigcup_{n=0}^\infty\sum_{k=0}^n d_\Z(2^ks)S_k\bigg) + U + K.
	}
}			
\begin{proof}
	Sea $U\in\U(Y)$. Entonces, existe $V\in\U(Y)$ tal que $[6]V\subseteq U$.
	
	Primero, vamos a demostrar que existe un n\'umero positivo $\delta$			
	tal que, para todo $n\geq0$ y para todo  $t,s\in\R$ con $|t-s|<\delta$,
	tenemos
	\Eq{ts}{
		\sum_{k=0}^n d_\Z(2^kt)S_k \subseteq \sum_{k=0}^n d_\Z(2^ks)S_k + [5]V + K.
	}
	
	Por la hip\'otesis (ii) existe un entero positivo $m$ tal que, para $n\geq m$, 
	la ecuaci\'on \eq{iii} es v\'alida. Los miembros de la suceci\'on  $(S_k)$
	son semi-$K$-acotados inferiormente, y la familia de conjuntos semi-$K$-acotados
	inferiormente es cerrada bajo uniones finitas, bajo sumas algebraicas y bajo
	multiplicaci\'on por escalares, por lo tanto, existe un conjunto acotado $H\subseteq Y$
	tal que 
	\Eq{*}{
		\bigcup_{\ell=0}^{m-1}\sum_{k=0}^{\ell} 2^kS_k \subseteq \cl(H+K)\subseteq H+V+K.
	} 
	Para el conjunto acotado $H$ existe un n\'umero positivo $\delta\leq 1$
	tal que $\delta H\subseteq V$. As\'i,
	\Eq{dd}{
		\delta\bigg(\bigcup_{\ell=0}^{m-1}\sum_{k=0}^{\ell} 2^kS_k\bigg) \subseteq \delta(H+V+K) \subseteq [2]V+K.
	}
	En vista de la propiedad de Lipschitz que satisface la funci\'on $d_\Z$, y por 
	la desigualdad $0\leq d_Z\leq \frac12$, para todo $p,q\in\R$, tenemos que 
	\Eq{pq}{
		d_\Z(p) \leq d_\Z(q)+|d_\Z(p)-d_\Z(q)| \leq d_\Z(q) +\min\big\{1,|p-q|\big\}.
	}
	Para un conjunto semi-$K$-estrellado $S$, para $0\leq c\leq d$ y para todo 
	$W\in\U(Y)$, tenemos
	\Eq{*}{
		cS\subseteq dS+W+K.
	}
	De hecho, esta inclusi\'on es evidente para $c=d$. Si $c<d$, entonces, 
	\Eq{*}{
		cS=d\Big(\frac{c}{d}S\Big)\subseteq d\cl(S+K) \subseteq d\Big(S+\frac1d W+K\Big)=dS+W+K.
	}
	Sea $n\in N$ y $0\leq k\leq n$. Escojamos $W\in\U(Y)$ tal que $[n+1]W\subseteq V$.
	Usando la desigualdad \eq{pq} y que el conjunto $S_k$ es semi-$K$-estrellado, 
	para todo $K\geq0$ y para todo $s,t\in\R$, obtenemos
	\Eq{*}{
		d_\Z(2^kt)S_k &\subseteq \big(d_\Z(2^ks) +\min\big\{1,2^k|t-s|\}\big)S_k + W+K \\
		&\subseteq d_\Z(2^ks)S_k +\min\big\{1,2^k|t-s|\big\}S_k +W+K.
	}
	As\'i, 
	\Eq{split}{
		\sum_{k=0}^n d_\Z(2^kt)S_k  
		\subseteq \sum_{k=0}^n d_\Z(2^ks)S_k  + \sum_{k=0}^n\min\big\{1,2^k|t-s|\big\}S_k + V+K.
	}
	Para completar la demostraci\'on de \eq{ts}, es suficiente mostrar, para todo 
	$n\geq0$,
	\Eq{fs}{
		\sum_{k=0}^n\min\big\{1,2^k|t-s|\big\}S_k\subseteq [4]V+K
	}
	siempre y cuando $|t-s|<\delta$. 
	
	Sean $t,s\in\R$ tales que $|t-s|\leq\delta$. Tenemos que 
	$\min\big\{1,2^k|t-s|\big\}\leq 1$
	y adem\'as $\min\big\{1,2^k|t-s|\big\}\leq 2^{k}\delta$ para todo $k\geq0$.
	De nuevo, usando que $S_k$ es semi-$K$-estrellado y las estimaciones 
	\eq{iii} y \eq{dd}, obtenemos 
	\Eq{*}{
		\sum_{k=0}^n\min\big\{1,2^k|t-s|\big\}S_k  
		&\subseteq\sum_{k=0}^{\min\{n,m-1\}}\big(2^{k}\delta S_k+W+K\big)  
		+ \sum_{k=\min(n+1,m)}^n\big(S_k+W+K\big) \\
		&\subseteq \delta\sum_{k=0}^{\min\{n,m-1\}} 2^kS_k +[2]V+K \\
		&\subseteq \delta\bigg(\bigcup_{\ell=0}^{m-1}\sum_{k=0}^{\ell} 2^kS_k\bigg) +[2]V+K\subseteq [4]V+K.
	} 
	Esto completa la demostraci\'on de \eq{fs} y por lo tanto \eq{ts} es v\'alida 
	para todo $n\geq0$. Aplicando \eq{ts}, para todo $n\geq 0$, se sigue que 
	\Eq{*}{
		\sum_{k=0}^n d_\Z(2^kt)S_k 
		\subseteq \cl\bigg(\bigcup_{\ell=0}^\infty\sum_{k=0}^\ell d_\Z(2^ks)S_k\bigg) + [5]V + K.
	}
	Por lo tanto,
	\Eq{*}{
		\cl\bigg(\bigcup_{n=0}^\infty\sum_{k=0}^n d_\Z(2^kt)S_k\bigg)
		\subseteq \bigcup_{n=0}^\infty\sum_{k=0}^n d_\Z(2^kt)S_k + V
		\subseteq \cl\bigg(\bigcup_{\ell=0}^\infty\sum_{k=0}^\ell d_\Z(2^ks)S_k\bigg) + U + K.
	}
	As\'i, la demostraci\'on de la inclusi\'on \eq{ts0} esta completa.
\end{proof}
			
En el siguiente lema vamos a describir un caso importante en el que la propiedad
de $K$-Cauchy en serie de usa sucesi\'on puede ser establecida.
\Lem{KC}{
	Supongamos que $Y$ es un espacio vectorial topol\'ogico localmente convexo.
	Sea $K\subseteq Y$ un cono convexo y $S\subseteq Y$ un conjunto semi-$K$-acotado
	inferiormente. Sea $(\lambda_n)$  una sucesi\'on de n\'umeros no-negativos, tales
	que la serie $\sum \lambda_n$ es convergente. Definamos, $S_n:=\lambda_n S+K$
	para $n\geq 0$. Entonces, la sucesi\'on $(S_n)$ es $K$-Cauchy en serie.
}
			
\begin{proof} 
	Como $S$ es semi-$K$-acotado inferiormente, existe un conjunto acotado $H\subseteq Y$
	tal que $S\subseteq\cl(H+K)$.
	
	Para demostrar que $(S_n)$ es una suceci\'on de $K$-Cauchy en serie,
	vamos a fijar un entorno abierto $U\in\U(Y)$ arbitrario.
	Ahora, escojamos un abierto convexo $V\in\U(Y)$ tal que $[2]V\subseteq U$
	y un n\'umero positivo $t>0$ tal que $tH\subseteq V$.
	
	Por la convergencia de la serie $\sum\lambda_n$, existe $m\in\N$ tal que,
	para todo $n\geq m$,
	\Eq{*}{
		t_n:=\sum_{k=m}^n \lambda_k<t.
	}
	Entonces, para todo $n\geq m$,
	\Eq{*}{
		\sum_{k=m}^n S_k
		&=\sum_{k=m}^n (\lambda_k S) + K
		\subseteq \sum_{k=m}^n (\lambda_k \cl(H+K)) + K
		\subseteq \cl\bigg(\sum_{k=m}^n (\lambda_k H) + K\bigg)\\
		&\subseteq \sum_{k=m}^n (\lambda_k H) + V + K
		\subseteq \sum_{k=m}^n \Big(\frac{\lambda_k}{t}V\Big) + \frac{t-t_n}{t}\{0\} + V + K
		\subseteq [2]V+K\subseteq U+K.
	}
	Esto demuestra que \eq{iii} se cumple y en consecuencia $(S_k)$ es una sucesi\'on
	$K$-Cauchy en serie.
\end{proof}
				
				

\section{Convexidad y concavidad sobre los racionales di\'adicos.}
\setcounter{theorem}{0}

Los resultados principales de esta secci\'on est\'an contenidos
en el siguiente par de teoremas. Denotaremos por $\D$ al conjunto
de los n\'umeros racionales di\'adicos, i.e., $\D$ consiste en
los n\'umeros de la forma $\frac{k}{2^n}$, donde, $k\in\Z$ y $n\in\N$.

\Lem{lemTec}{
	%%
	Sean $n,\ell\in\N$ dos n\'umeros naturales arbitrarios. Entonces,
	para todo $k\in\{0,\ldots,n-1\}$ se tiene que
	\Eq{dZafin}{
		%%
		d_\Z\Big(2^k\frac{2\ell+1}{2^{n+1}}\Big)
		=\frac12\Big(d_\Z\Big(2^k\frac{\ell+1}{2^{n}}\Big)+d_\Z\Big(2^k\frac{\ell}{2^{n}}\Big)\Big)
	} 
}
\begin{proof}
	%%
	Observemos que la fracci\'on $\frac{2\ell+1}{2^{n+1}}$ puede ser 
	representada como
	%%
	\Eq{*}{
		\frac{2\ell+1}{2^{n+1}}=\frac12\Big(\frac{\ell+1}{2^{n}}+\frac{\ell}{2^{n}}\Big)
	}
	%%
	Es suficiente demostrar que $d_\Z$ es af\'in
	en el intervalo $\Big[2^k\frac{\ell}{2^{n}},2^k\frac{\ell+1}{2^{n}}\Big]$.
	%%
	Para esto, veremos que el interior de este intervalo
	no contiene un elemento de $\frac12\Z$. Por el contrario, supongamos que para alg\'un $j\in\Z$
	%%
	\Eq{aux}{
		2^k\frac{\ell}{2^{n}}<\frac{j}2<2^k\frac{\ell+1}{2^{n}}.
	}
	Por lo tanto $\ell<2^{n-k-1}j<\ell+1$, lo que es una contradicci\'on
	ya que $n-1\geq k$ y en consecuencia, es falso suponer \eq{aux}.
		
\end{proof}

\Thm{ConvexTab}{
	%%
	Sea $D\subseteq X$ un subconjunto convexo no vac\'io, y consideremos
	las multifunciones $A,B:(D-D)\to\P_0(Y)$ tales que, los valores 
	de $B$ son semi-$K$-convexos, donde $K$ representa la clausura del
	cono recesi\'on asociado a $B$. 	
	Sea $F:D\to\P_0(Y)$ una multifunci\'on que satisface la siguiente
	inclusi\'on de convexidad tipo Jensen
	%%
	\Eq{JCV}{
		\dfrac{F(x) + F(y)}{2} + A(x-y) \subseteq 
		\cl\Big(F\Big(\dfrac{x+y}{2}\Big) + B(x-y)\Big) 
		\qquad (x,y\in D).
	}
	%%
	Entonces para $x,y\in D$ y para todo $t\in\D\cap[0,1]$ se tiene
	que $F$ satisface la siguiente inclusi\'on
	%%	
	\Eq{CV}{
		tF(x)&+(1-t)F(y)+\sum_{k=0}^{\infty} 2d_{\Z}\big(2^k t\big)A\Big(\dfrac{x-y}{2^k}\Big) \\
		&\subseteq \cl\bigg(F(tx+(1-t)y)
				  +\sum_{k=0}^{\infty} 2d_{\Z}\big(2^k t\big)B\Big(\dfrac{x-y}{2^k}\Big)\bigg)
	}
	%%
	Si adem\'as, $0\in\cl(A(u)+K)$ para cada $u\in D-D$, entonces,
	\Eq{CV+}{
		tF(x)+(1-t)F(y)+A^\perp(t,x-y) \subseteq 
		\cl\big(F(tx+(1-t)y)&+B^\perp(t,x-y)\big)
	}
	para todo $x,y\in D$ y para todo $t\in\D\cap[0,1]$.
}

\begin{proof} 
	%%
	Para la demostraci\'on de \eq{CV}, primero vamos a demostrar que
	para todo $x,y\in D$, y para todo par de enteros $n,m$ con 
	$n\geq 1$, $0\leq m\leq 2^n$, se cumple la siguiente igualdad
	%%
	\Eq{CVn}{
		\frac{m}{2^n}F(x)&+\Big(1-\frac{m}{2^n}\Big)F(y) 
		+ \sum_{k=0}^{n-1}2 d_{\Z}\Big(2^k \frac{m}{2^n}\Big)A\Big(\dfrac{x-y}{2^k}\Big) \\
		&\subseteq \cl\bigg(F\Big(\frac{m}{2^n}x+\Big(1-\frac{m}{2^n}\Big)y\Big) 
		+ \sum_{k=0}^{n-1}d_{\Z}\Big(2^k \frac{m}{2^n}\Big)B\Big(\dfrac{x-y}{2^k}\Big)\bigg). 
	}
	%%
	Fijemos $x,y\in D$ arbitrarios. Para verificar que \eq{CVn} se cumple, 
	vamos a proceder por inducci\'on sobre $n$. 
	%%
	Para $m=0$ o para $m=2^n$, la inclusi\'on \eq{CVn} se sigue de inmediato
	ya que la funci\'on $d_\Z$ se anula en $\Z$. As\'i, para $n=1$, solo es necesario
	verificar \eq{CVn} para $m=1$. En ese caso, \eq{CVn} es equivalente
	a la inclusi\'on de convexidad tipo Jensen en la \eq{JCV}.
	%%
	Ahora, supongamos que la inclusi\'on \eq{CVn} se cumple para alg\'un $n$
	y demostremos que tambi\'en es v\'alida para $n+1$. Sea $0<m<2^{n+1}$
	arbitrario. Si $m$ es par, i.e., $m=2\ell$ para alg\'un $0<\ell<2^n$,
	entonces, $\frac{m}{2^{n+1}}=\frac{\ell}{2^n}$ y as\'i, el resultado
	es consecuencia inmediata de la hip\'otesis inductiva. En vista de
	esta observaci\'on, podemos asumir entonces que $m$ es impar, i.e., 
	$m = 2\ell + 1$ para alg\'un $0 < \ell < 2^n$.
	
	En vista del \lem{lemTec}, se tenemos que para todo $k\in\{0,\ldots,n-1\}$ 
	%%
	\Eq{am}{
		%%
		d_\Z\Big(2^k\frac{2\ell+1}{2^{n+1}}\Big)
		=\frac12\Big(d_\Z\Big(2^k\frac{\ell+1}{2^{n}}\Big)+d_\Z\Big(2^k\frac{\ell}{2^{n}}\Big)\Big).
	} 
	%%
	Ahora bien, usando \eq{am} y \eq{dZafin} obtenemos lo siguiente
	%%
	\Eq{*}{
	  \frac{m}{2^{n+1}}F(x)&+\Big(1-\frac{m}{2^{n+1}}\Big)F(y)
	   + \sum_{k=0}^{n}2 d_{\Z}\Big(2^k \frac{m}{2^{n+1}}\Big)A\Big(\frac{x-y}{2^k}\Big) \\
	  &= \frac{2\ell+1}{2^{n+1}}F(x)+\Big(1-\frac{2\ell+1}{2^{n+1}}\Big)F(y)
	   + \sum_{k=0}^{n}2 d_{\Z}\Big(2^k \frac{2\ell+1}{2^{n+1}}\Big)A\Big(\frac{x-y}{2^k}\Big) \\
	  &\subseteq \frac12\Big(\frac{\ell+1}{2^{n}}F(x)+\Big(1-\frac{\ell+1}{2^{n}}\Big)F(y)\Big) 
	     +\frac12\Big(\frac{\ell}{2^{n}}F(x)+\Big(1-\frac{\ell}{2^{n}}\Big)F(y)\Big)\\
	  &\qquad+\sum_{k=0}^{n-1}\Big(d_\Z\Big(2^k\frac{\ell+1}{2^{n}}\Big)+d_\Z\Big(2^k\frac{\ell}{2^{n}}\Big)\Big)
	             A\Big(\frac{x-y}{2^k}\Big) + A\Big(\frac{x-y}{2^n}\Big) \\
	  &= \frac12\Big(\frac{\ell+1}{2^{n}}F(x)+\Big(1-\frac{\ell+1}{2^{n}}\Big)F(y)
	     +\sum_{k=0}^{n-1}2 d_\Z\Big(2^k\frac{\ell+1}{2^{n}}\Big)A\Big(\frac{x-y}{2^k}\Big)\Big)\\
	  &\qquad +\frac12\Big(\frac{\ell}{2^{n}}F(x)+\Big(1-\frac{\ell}{2^{n}}\Big)F(y)
	   + \sum_{k=0}^{n-1}2d_\Z\Big(2^k\frac{\ell}{2^{n}}\Big)A\Big(\frac{x-y}{2^k}\Big)\Big) \\
	  &\qquad+ A\Big(\frac{x-y}{2^n}\Big). 
	}
	%%
	Aplicando la hip\'otesis inductiva  \eq{CVn} con $m=\ell+1$ y $m=\ell$, 
	obtenemos lo siguiente
	%%
	\Eq{JCVa}{
		%%
		\frac{m}{2^{n+1}}F(x)&+\Big(1-\frac{m}{2^{n+1}}\Big)F(y)
			+ \sum_{k=0}^{n}2 d_{\Z}\Big(2^k \frac{m}{2^{n+1}}\Big)A\Big(\frac{x-y}{2^k}\Big) \\
		%%
		&\subseteq \frac12\cl\bigg(F\Big(\frac{\ell+1}{2^{n}}x+\Big(1-\frac{\ell+1}{2^{n}}\Big)y\Big)
			+\sum_{k=0}^{n-1} 2d_\Z\Big(2^k\frac{\ell+1}{2^{n}}\Big)B\Big(\frac{x-y}{2^k}\Big)\bigg)\\
		%%
		&\qquad+\frac12\cl\bigg(F\Big(\frac{\ell}{2^{n}}x+\Big(1-\frac{\ell}{2^{n}}\Big)y\Big)
			+ \sum_{k=0}^{n-1} 2d_\Z\Big(2^k\frac{\ell}{2^{n}}\Big)B\Big(\frac{x-y}{2^k}\Big)\bigg) \\
		&\qquad+A\Big(\frac{x-y}{2^n}\Big)\\
		%%
		&\subseteq \cl\bigg(\frac12\bigg(F\Big(\frac{\ell+1}{2^{n}}x+\Big(1-\frac{\ell+1}{2^{n}}\Big)y\Big)
			+F\Big(\frac{\ell}{2^{n}}x+\Big(1-\frac{\ell}{2^{n}}\Big)y\Big)\bigg) \\			
		%%
		&\qquad+ \sum_{k=0}^{n-1}\bigg(d_\Z\Big(2^k\frac{\ell+1}{2^{n}}\Big)B\Big(\frac{x-y}{2^k}\Big)
			+d_\Z\Big(2^k\frac{\ell}{2^{n}}\Big) B\Big(\frac{x-y}{2^k}\Big)\bigg) \\
		&\qquad+A\Big(\frac{x-y}{2^n}\Big)\bigg) \\
		%%
	}
	%%
	Ahora bien, usando la desigualdad de tipo Jensen \eq{JCV}, se sigue que
	\Eq{*}{
		%%
		\frac12\bigg(F\Big(\frac{\ell+1}{2^{n}}x+\Big(1-\frac{\ell+1}{2^{n}}\Big)y\Big)
	     %%
	     &+F\Big(\frac{\ell}{2^{n}}x+\Big(1-\frac{\ell}{2^{n}}\Big)y\Big)\bigg)
	     %%
	     +A\Big(\frac{x-y}{2^n}\Big) \\
	     %%%%
		 &\subseteq \cl\bigg(F\Big(\frac{2\ell+1}{2^{n+1}}x+\Big(1-\frac{2\ell+1}{2^{n+1}}\Big)y\Big)
	     %%
	     +B\Big(\frac{x-y}{2^n}\Big)\bigg).
	}
	Por otra parte, usando el hecho que los valores de la multifunci\'on $B$ son semi-$K$-convexos
	y la ecuaci\'on \eq{dZafin}, obtenemos
	\Eq{*}{
		\sum_{k=0}^{n-1}&\bigg(d_\Z\Big(2^k\frac{\ell+1}{2^{n}}\Big)B\Big(\frac{x-y}{2^k}\Big)
	      +d_\Z\Big(2^k\frac{\ell}{2^{n}}\Big) B\Big(\frac{x-y}{2^k}\Big)\bigg) \\
	      &\subseteq \cl\bigg(K+\sum_{k=0}^{n-1}\bigg(d_\Z\Big(2^k\frac{\ell+1}{2^{n}}\Big)
	      +d_\Z\Big(2^k\frac{\ell}{2^{n}}\Big)\bigg) B\Big(\frac{x-y}{2^k}\Big)\bigg)\\
		  &=\cl\bigg(K+\sum_{k=0}^{n-1}2d_\Z\Big(2^k\frac{2\ell+1}{2^{n+1}}\Big)
		    B\Big(\frac{x-y}{2^k}\Big)\bigg).
	}
	Combinando las inclusiones anteriores con \eq{JCVa} y reemplazando $2\ell+1$ por $m$,
	llegamos a
	\Eq{*}{
	  \frac{m}{2^{n+1}}F(x)&+\Big(1-\frac{m}{2^{n+1}}\Big)F(y)
	   + \sum_{k=0}^{n}2 d_{\Z}\Big(2^k \frac{m}{2^{n+1}}\Big)A\Big(\frac{x-y}{2^k}\Big) \\
	  &\subseteq   \cl\bigg(F\Big(\frac{m}{2^{n+1}}x+\Big(1-\frac{m}{2^{n+1}}\Big)y\Big)
	     +B\Big(\frac{x-y}{2^n}\Big) + K \\
	  &\qquad+\sum_{k=0}^{n-1}2d_\Z\Big(2^k\frac{m}{2^{n+1}}\Big)B\Big(\frac{x-y}{2^k}\Big)\bigg)\\
	  &= \cl\bigg(F\Big(\frac{m}{2^{n+1}}x+\Big(1-\frac{m}{2^{n+1}}\Big)y\Big)
	  +\sum_{k=0}^{n}2d_\Z\Big(2^k\frac{m}{2^{n+1}}\Big)B\Big(\frac{x-y}{2^k}\Big)\bigg),
	}
	lo que muestra que la tesis tambi\'en es valida para $n+1$. Esto completa la demostraci\'on 
	de la inducci\'on y la primera parte del teorema.
	
	Supongamos ahora, que para todo $u\in D-D$, tenemos $0\in\cl(A(u) + K)$.
	Para demostrar \eq{CV+}, sean $x,y\in D$ y $t\in[0,1]\cap\D$ fijos.
	Si $t=0$ \'o $t=1$, entonces \eq{CV+} se obtiene de inmediato.
	En el resto de la demostraci\'on asumiremos que $t\in(0,1)\cap\D$.
	Luego,
	\Eq{*}{
	 tF(x)&+(1-t)F(y)+A^\perp(t,x-y) \\
	 &\subseteq \cl\bigg(tF(x)+(1-t)F(y) + \bigcup_{n=0}^{\infty} \sum_{k=0}^{n}
	                 2d_{\Z}(2^kt)A\Big(\frac{x-y}{2^k}\Big)\bigg)\\
	 &\subseteq \cl\bigg(tF(x)+(1-t)F(y) + \bigcup_{n=0}^{\infty} \sum_{k=0}^{n}
	                 2d_{\Z}(2^kt)\cl\Big(A\Big(\frac{x-y}{2^k}\Big)+K\Big)\bigg).
	}
	Ahora, como $t$ es un n\'umero di\'adico, existe un n\'umero natural $n\in\N$
	y un numero impar $\ell$ tal que $t=\frac{\ell}{2^m}$. Entonces, para $k\geq  m$,
	$d_{\Z}(2^kt)=0$. Adem\'as, $0\in\cl\big(A\big(\frac{x-y)}{2^k}\big)+K\big)$
	para todo $k\in\{0,\dots,m-1\}$, por lo tanto
	\Eq{*}{
	  \bigcup_{n=0}^{\infty} \sum_{k=0}^{n}2d_{\Z}(2^kt)\cl\Big(A\Big(\frac{x-y}{2^k}\Big)+K\Big)
	  &= \sum_{k=0}^{m-1}2d_{\Z}(2^kt)\cl\Big(A\Big(\frac{x-y}{2^k}\Big)+K\Big) \\
	  &= \sum_{k=0}^{\infty}2d_{\Z}(2^kt)\cl\Big(A\Big(\frac{x-y}{2^k}\Big)+K\Big).
	}
	Usando esta f\'ormula, la inclusi\'on previa, la primera parte del teorema y
	 $d_\Z(t)>0$, obtenemos lo siguiente
	\Eq{AB}{
	 tF(x)&+(1-t)F(y)+A^\perp(t,x-y) \\
	 &\subseteq \cl\bigg(tF(x)+(1-t)F(y) 
	    + \sum_{k=0}^{\infty}2d_{\Z}(2^kt)\cl\Big(A\Big(\frac{x-y}{2^k}\Big)+K\Big)\bigg)\\
	 &=\cl\bigg(tF(x)+(1-t)F(y) 
	    + \sum_{k=0}^{\infty}2d_{\Z}(2^kt)\Big(A\Big(\frac{x-y}{2^k}\Big)+K\Big)\bigg) \\
	 &=\cl\bigg(tF(x)+(1-t)F(y) 
	    + \sum_{k=0}^{\infty}2d_{\Z}(2^kt)A\Big(\frac{x-y}{2^k}\Big)+\rec(B)\bigg) \\
	 &\subseteq \cl\bigg(F(tx+(1-t)y) 
	    + \sum_{k=0}^{\infty}2d_{\Z}(2^kt)B\Big(\frac{x-y}{2^k}\Big)+\rec(B)\bigg) \\
	 &= \cl\bigg(F(tx+(1-t)y) 
	    + \sum_{k=0}^{\infty}2d_{\Z}(2^kt)B\Big(\frac{x-y}{2^k}\Big)\bigg).
	}
	Por otra parte, tenemos
	\Eq{*}{
	  \sum_{k=0}^{\infty}2d_{\Z}(2^kt)B\Big(\frac{x-y}{2^k}\Big)
	  &=\sum_{k=0}^{m-1}2d_{\Z}(2^kt)B\Big(\frac{x-y}{2^k}\Big) \\
	  &\subseteq \cl\bigg(\bigcup_{n=0}^{\infty} \sum_{k=0}^{n}2d_{\Z}(2^kt)B\Big(\frac{x-y}{2^k}\Big)\bigg)
	  =B^\perp(t,x-y),
	}
	que en combinaci\'on con \eq{AB} implica \eq{CV+} y esto completa
	la demostraci\'on del teorema.
\end{proof}

Para formular los corolarios del teorema previo, postularemos las siguiente
hip\'otesis generales.
\begin{enumerate}[(H1)]
\item $D\subseteq X$ es un conjunto convexo no-vac\'io, 
 $K\subseteq Y$ es un cono convexo, no-vac\'io;
\item $S_0\subseteq Y$ es un conjunto semi-$K$-convexo y semi-$K$-estrellado;
\item $\varphi:(D-D)\to\R_+$ es una funci\'on no-negativa tal que  \eq{phi} 
se cumple para todo $x\in D-D$. 
\end{enumerate}
Note que, por la convexidad de $D$, el conjunto $(D-D)$ es estrellado,
as\'i, podemos aplicar la \prp{Tab}. 

Los siguientes corolarios son sobre multifunciones $K$-Jensen aproximadamente y fuertemente convexas
respectivamente.

\Cor{Convex+1}{
	Asumamos que (H1), (H2) y (H3) se cumplen y $F:D\to\P_0(Y)$ es una multifunci\'on
	que satisface lo siguiente
	\Eq{*}{
	\dfrac{F(x) + F(y)}{2} \subseteq \cl\Big(F\Big(\dfrac{x+y}{2}\Big) 
	    + \varphi(x-y)S_0 + K \Big) \qquad (x,y\in D).
	}
	Entonces, para todo $x,y\in D,\,t\in\D\cap[0,1]$, se tiene que
	\Eq{*}{
	 tF(x)+(1-t)F(y) \subseteq \cl\big(F(tx+(1-t)y)+\varphi^\perp(t,x-y)S_0+K\big)
	}
}

\Cor{Convex+2}{
	Asumamos que (H1), (H2) y (H3) se cumplen y $F:D\to\P_0(Y)$ es una multifunci\'on
	que satisface lo siguiente
	\Eq{*}{
	\dfrac{F(x) + F(y)}{2} + \varphi(x-y)S_0 \subseteq \cl\Big(F\Big(\dfrac{x+y}{2}\Big) 
	  + K \Big) \qquad (x,y\in D).
	}
	Entonces, para todo $x,y\in D,\,t\in\D\cap[0,1]$, se tiene que 
	\Eq{*}{
	 tF(x)+(1-t)F(y) + \varphi^\perp(t,x-y)S_0 \subseteq \cl\big(F(tx+(1-t)y) + K \big)
	}
}

\begin{proof}[Demostraci\'on de los corolarios \ref{CConvex+1} y \ref{CConvex+2}]
	Usando la segunda parte del \thm{ConvexTab} con las multifunciones 
	$A(u) = \{0\}$ y $B(u)=\varphi(u)S_0+K$ (resp., $A(u)=\varphi(u)S_0$ y $B(u)=K$)
	y aplicando la \prp{Tab}, obtenemos  \cor{Convex+1} (resp., \cor{Convex+2}).
	Note que, en ambos casos, $0\in\cl(A(u)+K)$ para todo $u\in D-D$.
\end{proof}

El siguiente resultado es concerniente al caso de inclusiones de tipo concavidad.

\Thm{ConcaveTab}{
	%%
	Sean $D\subseteq X$ un subconjunto no vac\'io y $A,B:(D-D)\to\P_0(Y)$
	tales que los valores de la multifunci\'on $B$ son semi-$K$-convexos,
	donde $K:=\overline{\rec}(B)$. Sea $F$ una multifunci\'on definida en $D$
	a valores en $\P_0(Y)$ que satisface la inclusi\'on de tipo Jensen para concavidad
	%%
	\Eq{JCC}{
		F\Big(\dfrac{x+y}{2}\Big) + A(x-y) \subseteq \cl\bigg(\frac{F(x) + F(y)}{2} + B(x-y)\bigg) 
		   \qquad (x,y\in D),
	}
	%%
	y que tiene valores semi-$K$ convexos, i.e., $F(x)$ es semi-$K$-convexo,
	para todo $x\in D$.
	Entonces, $F$ satisface la inclusi\'on de tipo convexidad, para todo 
	$x,y\in D,\,t\in\D\cap[0,1]$
	%%
	\Eq{CC}{
	 F(tx&+(1-t)y)+\sum_{k=0}^{\infty} 2d_{\Z}\big(2^k t\big)A\Big(\dfrac{x-y}{2^k}\Big) \\
	&\subseteq \cl\bigg(tF(x)+(1-t)F(y)+\sum_{k=0}^{\infty} 2d_{\Z}\big(2^k t\big)B\Big(\dfrac{x-y}{2^k}\Big)\bigg)
	}
	%%
	Si adicionalmente, $0\in\cl(A(u)+K)$ para todo $u\in D-D$, entonces
	para todo $x,y\in D,\,t\in\D\cap[0,1]$
	%%
	\Eq{CC+}{
	 F(tx+(1-t)y)+A^\perp(t,x-y) \subseteq \cl\big(tF(x)+(1-t)F(y)+B^\perp(t,x-y)\big)
	}}
	%%
\begin{proof}
	Para la demostraci\'on de \eq{CC}, vamos a demostrar que, para todo $x,y\in D$,
	y para todo par de enteros $n,m$ con $n\geq 1$, $0\leq m\leq 2^n$,
	%%
	\Eq{CCn}{
		F\Big(\frac{m}{2^n}x&+\Big(1-\frac{m}{2^n}\Big)y\Big)
		  + \sum_{k=0}^{n-1}2 d_{\Z}\Big(2^k \frac{m}{2^n}\Big)A\Big(\dfrac{x-y}{2^k}\Big) \\
		%%
		&\subseteq \cl\bigg(\frac{m}{2^n}F(x)+\Big(1-\frac{m}{2^n}\Big)F(y) 
		   + \sum_{k=0}^{n-1}2d_{\Z}\Big(2^k \frac{m}{2^n}\Big)B\Big(\dfrac{x-y}{2^k}\Big)\bigg). 
	}
	%%
	Fijemos $x,y\in D$ arbitrarios. Para verificar que \eq{CCn} se cumple,
	vamos a proceder por inducci\'on sobre $n$. 
	
	Para $n=1$, tenemos que $0\leq m\leq 2$, pero si $m=0$ o $m=2$ entonces, 
	la ecuaci\'on \eq{CCn} se sigue de inmediato. As\'i, solo debemos verificar
	que \eq{CCn} es v\'alida para $m=1$, lo cual es sencillo ya que para 
	$n=m=1$, \eq{CCn} es identica a \eq{JCC}. 
	
	Ahora, supongamos que \eq{CCn} es v\'alida para $n\geq1$ y $0\leq m \leq 2^n$,
	y demostremos que tambi\'en es v\'alida para $n+1$ y $0\leq m\leq 2^{n+1}$.
	Al igual que procedimos en la demostraci\'on para la inclusi\'on de tipo 
	convexidad, ser\'a suficiente considerar el caso cuando $m$ tiene la forma
	$m = 2\ell + 1$, para alg\'un $n\in\N\cup\{0\}$. Con estas definiciones, 
	podemos comenzar nuestra demostraci\'on para $n+1$ usando las relaciones 
	\eq{am} y \eq{dZafin}, para obtener
	
	\Eq{*}{
	 F\Big(\frac{2\ell +1}{2^{n+1}}x&+\Big(1-\frac{2\ell +1}{2^{n+1}}\Big)y\Big)
	  + \sum_{k=0}^{n}2 d_{\Z}\Big(2^k \frac{2\ell +1}{2^{n+1}}\Big)A\Big(\dfrac{x-y}{2^k}\Big) \\ 
	&=F\Bigg(\frac12\bigg(\Big(\frac{\ell}{2^{n}} + \frac{\ell +1}{2^{n}}\Big)x 
	  +	\Big(2-\frac{\ell +1}{2^{n}}-\frac{\ell}{2^{n}}\Big)y\bigg)\Bigg) \\
	  &\qquad+ \sum_{k=0}^{n}2 d_{\Z}\Big(2^k \frac{2\ell +1}{2^{n+1}}\Big)A\Big(\dfrac{x-y}{2^k}\Big) \\
	&=F\Bigg(\frac12\bigg[\frac{\ell}{2^{n}}x + \Big(1-\frac{\ell}{2^{n}}\Big)y
		+	\frac{\ell +1}{2^{n}}x+\Big(1-\frac{\ell +1}{2^{n}}\Big)y\bigg]\Bigg) + A\Big(\frac{x-y}{2^n}\Big) \\
	&\qquad+ \sum_{k=0}^{n-1}\bigg[d_\Z\Big(2^k\frac{\ell+1}{2^{n}}\Big)+d_\Z\Big(2^k\frac{\ell}{2^{n}}\Big)\bigg]
		A\Big(\dfrac{x-y}{2^k}\Big) \\
	&\subseteq F\Bigg(\frac12\bigg[\frac{\ell}{2^{n}}x + \Big(1-\frac{\ell}{2^{n}}\Big)y
		+	\frac{\ell +1}{2^{n}}x+\Big(1-\frac{\ell +1}{2^{n}}\Big)y\bigg]\Bigg) + A\Big(\frac{x-y}{2^n}\Big) \\
	&\qquad+ \sum_{k=0}^{n-1}d_\Z\Big(2^k\frac{\ell}{2^{n}}\Big)A\Big(\dfrac{x-y}{2^k}\Big)
		+ \sum_{k=0}^{n-1}d_\Z\Big(2^k\frac{\ell+1}{2^{n}}\Big)A\Big(\dfrac{x-y}{2^k}\Big).
	}
	
	Por la Jensen concavidad de $F$, tenemos que
	
	\Eq{*}{
	F\Bigg(&\frac12\bigg[\frac{\ell}{2^{n}}x + \Big(1-\frac{\ell}{2^{n}}\Big)y
		+\frac{\ell +1}{2^{n}}x+\Big(1-\frac{\ell +1}{2^{n}}\Big)y\bigg]\Bigg) + A\Big(\frac{x-y}{2^n}\Big) \\
	&\subseteq \cl\Bigg(\frac12 F\Big( \frac{\ell}{2^{n}}x + \Big(1-\frac{\ell}{2^{n}}\Big)y \Big)
		+\frac12 F\Big( \frac{\ell+1}{2^{n}}x+\Big(1-\frac{\ell +1}{2^{n}}\Big)y \Big)
		+ B\Big(\frac{x-y}{2^n}\Big)\Bigg).
	}
	
	Por lo tanto, usando la inclusi\'on anterior, obtenemos las siguientes inclusiones

	\Eq{*}{
	F\Big(&\frac{2\ell +1}{2^{n+1}}x+\Big(1-\frac{2\ell +1}{2^{n+1}}\Big)y\Big)
	  + \sum_{k=0}^{n}2 d_{\Z}\Big(2^k \frac{2\ell +1}{2^{n+1}}\Big)A\Big(\dfrac{x-y}{2^k}\Big) \\ 
	&\subseteq \cl\Bigg(\frac12 F\Big( \frac{\ell}{2^{n}}x + \Big(1-\frac{\ell}{2^{n}}\Big)y \Big)
	  +\frac12 F\Big( \frac{\ell +1}{2^{n}}x+\Big(1-\frac{\ell +1}{2^{n}}\Big)y \Big)+ B\Big(\frac{x-y}{2^n}\Big)\Bigg) \\
	&\qquad+ \sum_{k=0}^{n-1}d_\Z\Big(2^k\frac{\ell}{2^{n}}\Big)A\Big(\dfrac{x-y}{2^k}\Big)
		+ \sum_{k=0}^{n-1}d_\Z\Big(2^k\frac{\ell+1}{2^{n}}\Big)A\Big(\dfrac{x-y}{2^k}\Big)	\\
	&\subseteq \cl\Bigg( \frac12\bigg[ F\Big(\frac{\ell}{2^{n}}x + \Big(1-\frac{\ell}{2^{n}}\Big)y \Big)
		+\sum_{k=0}^{n-1}2d_\Z\Big(2^k\frac{\ell}{2^{n}}\Big)A\Big(\dfrac{x-y}{2^k}\Big) \\
	&\qquad + F\Big( \frac{\ell +1}{2^{n}}x+\Big(1-\frac{\ell +1}{2^{n}}\Big)y \Big)
		+\sum_{k=0}^{n-1}2d_\Z\Big(2^k\frac{\ell+1}{2^{n}}\Big)A\Big(\dfrac{x-y}{2^k}\Big)\bigg]
		+B\Big(\frac{x-y}{2^n}\Big)\Bigg).
	}	
	Nuestra hip\'otesis inductiva para $m=\ell$ y $m=\ell +1$ nos permiten
	escribir las siguientes relaciones
	\Eq{Il}{
	F\Big( \frac{\ell}{2^{n}}x &+ \Big(1-\frac{\ell}{2^{n}}\Big)y \Big)
		+ \sum_{k=0}^{n-1}2d_\Z\Big(2^k\frac{\ell}{2^{n}}\Big)A\Big(\dfrac{x-y}{2^k}\Big)\\
	&\subseteq \cl\Bigg(\frac{\ell}{2^{n}}F(x) +  \Big(1-\frac{\ell}{2^{n}}\Big)F(y)
		+ \sum_{k=0}^{n-1}2d_\Z\Big(2^k\frac{\ell}{2^{n}}\Big)B\Big(\dfrac{x-y}{2^k}\Big)\Bigg),
	}
	y
	\Eq{Il+}{
	F\Big( &\frac{\ell+1}{2^{n}}x + \Big(1-\frac{\ell+1}{2^{n}}\Big)y \Big)
		+ \sum_{k=0}^{n-1}2d_\Z\Big(2^k\frac{\ell+1}{2^{n}}\Big)A\Big(\dfrac{x-y}{2^k}\Big)\\
	&\subseteq \cl\Bigg(\frac{\ell+1}{2^{n}}F(x) +  \Big(1-\frac{\ell+1}{2^{n}}\Big)F(y)
		+ \sum_{k=0}^{n-1}2d_\Z\Big(2^k\frac{\ell+1}{2^{n}}\Big)B\Big(\dfrac{x-y}{2^k}\Big)\Bigg).
	}
	
	As\'i, por \eq{dZafin}, y usando que $F$ y $B$ tienen valores semi-$K$-convexos y 
	usando \eq{Il} y \eq{Il+}, llegamos a lo siguiente

	\Eq{*}{
	 F&\Big(\frac{2\ell +1}{2^{n+1}}x+\Big(1-\frac{2\ell +1}{2^{n+1}}\Big)y\Big)
	  + \sum_{k=0}^{n}2 d_{\Z}\Big(2^k \frac{2\ell +1}{2^{n+1}}\Big)A\Big(\dfrac{x-y}{2^k}\Big) \\
	&\subseteq \cl\Bigg( \frac12\Bigg[\frac{\ell}{2^{n}}F(x) +  \Big(1-\frac{\ell}{2^{n}}\Big)F(y)
		+ \sum_{k=0}^{n-1}2d_\Z\Big(2^k\frac{\ell}{2^{n}}\Big)B\Big(\dfrac{x-y}{2^k}\Big) \\
	&\qquad + \frac{\ell+1}{2^{n}}F(x) +  \Big(1-\frac{\ell+1}{2^{n}}\Big)F(y)
		+ \sum_{k=0}^{n-1}2d_\Z\Big(2^k\frac{\ell+1}{2^{n}}\Big)B\Big(\dfrac{x-y}{2^k}\Big)\Bigg]
		+ B\Big(\frac{x-y}{2^n}\Big)\Bigg) \\
	&\subseteq \cl\Bigg( \frac12\Bigg[\frac{2\ell+1}{2^{n}}F(x) +  \Big(2-\frac{2\ell+1}{2^{n}}\Big)F(y)\\ 
	&\qquad + \sum_{k=0}^{n-1}2\Big(d_\Z\Big(2^k\frac{\ell}{2^{n}}\Big) + d_\Z\Big(2^k\frac{\ell+1}{2^{n}}\Big)
		  \Big)B\Big(\dfrac{x-y}{2^k}\Big) \Bigg] + K + B\Big(\frac{x-y}{2^n}\Big) \Bigg) \\
	&= \cl\Bigg( \frac{2\ell+1}{2^{n+1}}F(x) +  \Big(1-\frac{2\ell+1}{2^{n+1}}\Big)F(y) 
		+ \sum_{k=0}^{n}2d_\Z\Big(2^k\frac{2\ell+1}{2^{n+1}}\Big) B\Big(\dfrac{x-y}{2^k}\Big)\Bigg),
	}
	
	Lo cual muestra que nuestra afirmaci\'on tambi\'en es v\'alida para $n+1$.
	As\'i la inducci\'on, y en consecuencia, la demostraci\'on del teorema
	estan completas.
	

\end{proof}

Los siguientes dos corolarios son acerca de $K$-Jensen concavidad. El primero, con
respecto a concavidad aproximada y el segundo con respecto a concavidad fuerte.

\Cor{Concave+1}{
	Supongamos que (H1), (H2) y (H3) se cumplen y sea $F:D\to\P_0(Y)$ una multifunci\'on
	con valores semi-$K$-convexos que satisface
	\Eq{*}{
		F\Big(\dfrac{x+y}{2}\Big) \subseteq \cl\Big(\dfrac{F(x) + F(y)}{2} 
		  + \varphi(x-y)S_0 + K \Big) \qquad (x,y\in D).
	}
	Entonces, para todo $x,y\in D,\,t\in\D\cap[0,1]$, se tiene que
	\Eq{*}{
		 F(tx+(1-t)y) \subseteq \cl\big(tF(x)+(1-t)F(y) + \varphi^\perp(t,x-y)S_0 + K \big)
	}
}

\Cor{Concave+2}{
	Supongamos que (H1), (H2) y (H3) se cumplen y sea $F:D\to\P_0(Y)$ una multifunci\'on
	con valores semi-$K$-convexos que satisface
	\Eq{*}{
		F\Big(\dfrac{x+y}{2}\Big) + \varphi(x-y)S_0 \subseteq \cl\Big(\dfrac{F(x) + F(y)}{2} 
		    + K\Big) \qquad (x,y\in D).
	}
	Entonces, para todo $x,y\in D,\,t\in\D\cap[0,1]$, se tiene que
	\Eq{*}{
		 F(tx+(1-t)y)+\varphi^\perp(t,x-y)S_0 \subseteq \cl\big(tF(x)+(1-t)F(y) + K \big)
	}
}

\begin{proof}[Demostraciones de los corolarios \ref{CConcave+1} y \ref{CConcave+2}]

	Usando el \thm{ConcaveTab} con las multifunciones $A(u) = \{0\}$ y
	$B(u) = \varphi(u)S_0 + K$, (resp., $A(u)=\varphi(u)S_0$ y $B(u)=K$)
	y aplicando la \prp{Tab}, obtenemos \cor{Concave+1} (resp., \cor{Concave+2}).
	
	En ambos casos, tenemos que $K\subseteq\overline{\rec}(B)$, y en consecuencia
	el hecho de que los valores de $F$ son semi-$K$-convexos, implica que estos
	valores tambi\'en son semi-$\overline{\rec}(B)$-convexos.
\end{proof}
