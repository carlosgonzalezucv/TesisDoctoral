\chapter{Multifunciones.}
\label{ChapMultifunciones}
\setcounter{theorem}{0}
En el siguiente capítulo, se establece el concepto
de multifunción o función conjunto-valuada junto con
algunas propiedades importantes. Las definiciones aquí establecidas
han sido tomadas del libro de análisis conjunto-valuado 
de Aubin et. al. \cite{AubPieFra09}.

\section{Definiciones Básicas.}
A menos que se especifique otra cosa, $X$ y $Y$ denotarán
espacios topológicos lineales.
\Defi{SVM}{
Si a cada $x\in X$ le corresponde un subconjunto
$F(x)\in\P(Y)$, se dice que $F$ es una multifunción
de $X$ en $Y$ y simplemente la denotaremos como 
$F:X\to\P(Y).$
}
\Defi{domSVM}{
El dominio de la multifunción $F:X\to\P(Y)$ es el conjunto
$$
\Dom(F):= \{x\in X \quad|\quad F(x)\neq\emptyset\}.
$$
}
\Exa{svm1}{
El primer ejemplo de una multifunción, surge naturalmente
a partir de una función dada $f:X\to Y$. Defínase 
$F:Y\to\P(X)$, tal que 
$$F(y) = f^{-1}(y)= \{x\in X \quad|\quad y=f(x)\}.$$
Note que $F(y)\neq\emptyset$ si y sólo si, $y\in f(X)$,
por lo tanto, $\Dom(F) = f(X)$.
}
A menos que se especifique otra cosa,
$F$ denotará a una multifunción de $X$ en $Y$ y al dominio
de $F$ lo denotaremos por $D$.
\Defi{grSVM}{
El gráfico de una multifunción lo denotaremos por $\Gr(F)$
y se define como
\Eq{*}{
\Gr(F):=\{(x,y)\in X\times Y \quad|\quad y\in F(x)\}.
}
}
El concepto de convexidad para una multifunción $F$
está relacionado con su gráfico, en este sentido, se tiene
la siguiente definición.
\Defi{cvxSVM}{
Se dice que la multifunción $F$ es convexa, si $\Gr(F)$ es 
un subconjunto convexo de $X\times Y$.
}
\Prp{cvxSVM1}{
La multifunción $F$ es convexa si y sólo si para todo $x_1,x_2\in X$
y para todo $t\in[0,1]$, se tiene que 
\Eq{cvxInc}{
tF(x_1) + (1-t)F(x_2)\subseteq F(tx_1+(1-t)x_2).
}
}
\begin{proof}
Supongamos que $F$ es convexa, ie, $\Gr(F)$ es un conjunto convexo de 
$X\times Y.$ Sean $x_1,x_2\in X$ y consideremos $z\in tF(x_1) + (1-t)F(x_2)$.
Por lo tanto, existen $y_1\in F(x_1)$, $y_2\in F(x_2)$ tales que, 
$
z=ty_1+(1-t)y_2.
$
Como $\Gr(F)$ es convexo, entonces, para todo $t\in[0,1]$
\Eq{*}{
t(x_1,y_1)+(1-t)(x_2,y_2)=(tx_1+(1-t)x_2,ty_1+(1-t)y_2)\in\Gr(F),
}
es decir,
$z=ty_1+(1-t)y_2\in F(tx_1+(1-t)x_2)$, para todo $t\in[0,1]$. Lo que
demuestra que la inclusión \eq{cvxInc} es válida. El recíproco, se demuestra
de manera análoga.
\end{proof}

\Prp{HcvxS}{
Si $H\subseteq X$ es un subconjunto estrellado de $X$ con respecto al
origen, entonces, para $0<a<b$ se tiene que $aH\subseteq bH$.
}
\begin{proof}
Como el conjunto $H$ es estrellado con respecto a $0\in H$, se tiene que para 
todo $\alpha\in[0,1]$ y para todo $h\in H$, $\alpha h + (1-\alpha)0\in H$.
Por lo tanto, para $\alpha\in[0,1]$ se tiene que $\alpha H\subseteq H$.
Ahora bien, como $0<a<b$ entonces $0<\frac{a}{b}<1$ y por lo tanto
$aH=b\bigg(\dfrac{a}{b}H\bigg)\subseteq bH,$ lo que completa la demostración. 
\end{proof}
\Rem{HcvxS}{
La proposición anterior es válida si $a<b<0$, pero no necesariamente
lo es cuando $a<0<b$. Note que si $a<b<0$ entonces, $0<\frac{a}{b}<1$
y por lo tanto se puede aplicar el mismo m\'etodo que se aplic\'o en
la demostraci\'on anterior. Para el caso $a<0<b$, basta considerar
$X=\R$, $H=[0,1]$, $a=-1$ y $b=1$.  
}

\Exa{B1}{
Sea $H\subseteq \R^3$ un conjunto convexo y no vacío que posee al origen de $\R^3.$
Sea $G:\R\to\P(\R^3)$ la multifunción definida por $G(x):= -x^2H$. 
Siendo $g(x) = -x^2$ una función cóncava, se tiene que para todo $t\in[0,1]$
y para todo $x,y\in\R$
$$
-(tx^2+(1-t)y^2)\leq -(tx+(1-t)y)^2.
$$ 
Como el conjunto $H$ es convexo y posee al origen, podemos aplicar la
\prp{HcvxS} para llegar a la siguiente inclusión
$$
-(tx^2+(1-t)y^2)H\subseteq -(tx+(1-t)y)^2H=G(tx+(1-t)y).
$$ 
De nuevo, usando la convexidad del conjunto $H$ se obtiene que 
$$
-(tx^2+(1-t)y^2)H=-tx^2H-(1-t)y^2H=tG(x)+(1-t)G(y),
$$
y por lo tanto
$$
tG(x)+(1-t)G(y)\subseteq G(tx+(1-t)y).
$$
}
\Defi{midCvxSVM}{
La multifunción $F$ es midconvexa o Jensen-convexa si satisface
la inclusión \eq{cvxInc} para $t=1/2$, es decir
\Eq{midCvxSVM}{
\frac{F(x)+F(y)}{2}\subseteq F\bigg(\frac{x+y}{2}\bigg).
}
}
\Exa{cvxSVM1}{
Sean $f,g:[a,b]\subseteq\R\to\R$ dos funciones tales que $f(x)<g(x)$
para todo $x\in[a,b]$ y que además $f$ y $-g$ son funciones convexas,
es decir que para todo $x,y\in[a,b]$ y $t\in[0,1]$ se satisface lo siguiente
\Eq{cvxF}{
f(tx+(1-t)y)&\leq tf(x) + (1-t)f(y) \\
tg(x) + (1-t)g(y)&\leq g(tx+(1-t)y).
}
Entonces, mediante un cálculo elemental, se puede ver que
la multifunción $F:[a,b]\to\mathcal{\R}$ definida por la fórmula 
$F(x):=[f(x),g(x)]$, para $x\in [a,b]$ es convexa. 
}
Así como en el caso de funciones a valores reales, también es posible generalizar
el concepto de concavidad para una multifunción. 
\Defi{ccvSVM}{
Se dice que la multifuncion $F$ es cóncava en $D$ si para todo $x,y\in D$
y para todo $t\in[0,1]$ se tiene que 
\Eq{ccvSVM}{
F(tx+(1-t)y)\subseteq tF(x)+(1-t)F(y)
}
}
Recordemos en el caso de las funciones, se tiene que una función $f$ es 
cóncava si y sólo si, la función $-f$ es convexa y esta caracterización
permite extender los resultados obtenidos para funciones convexas a funciones
cóncavas sin mayores dificultades. Lamentablemente, para el caso de multifunciones
esta caracterización no existe y el siguiente ejemplo lo ilustra de manera sencilla.

\Exa{Fcvx-Fcvx}{
Para $r>0$, se denotará por $D(r)$ al disco cerrado de radio $r$ centrado
en el origen, esto es 
$$
D(r):=\{(x,y)\in\R^2\quad|\quad x^2+y^2\leq r^2\}.
$$
Sea $F:[0,\infty)\to\P(\R^2)$ la multifunción definida por
$F(r):= D(r)$ para $r\geq0$. El gráfico de la multifunción $F$
es el conjunto
\Eq{*}{
\Gr(F)&=\{(r,(x,y))\in[0,\infty)\times\R^2\quad|\quad (x,y)\in F(r)\} \\
      &=\{(r,(x,y))\in[0,\infty)\times\R^2\quad|\quad x^2+y^2\leq r^2\}.
}
Es claro que al conjunto $\Gr(F)$ 
lo podemos representar como un cono en $\R^3$ con
vértice en el origen. Además, la multifunción $G=-F$ por
simetría posee el mismo gráfico de $F$ y esto significa
que sigue siendo una multifunción convexa.
}

Es decir que, si $F$ es convexa no necesariamente se tiene que
$-F$ es cóncava. Esto trae como consecuencia, que los resultados
obtenidos para multifunciones convexas no puedan extenderse directamente 
a resultados para multifunciones cóncavas y por lo tanto, ambas direcciones
deben tratarse por separado.

\Exa{cvxValuedNoCvx}{
Si las imagenes de una multifunción $F$ son conjuntos convexos,
no necesariamente ella es convexa. Si definimos
$F(r):=D(|r|)$ para $r\in\R$, es claro que $F(r)$
es un conjunto convexo para todo $r\in\R$, sin embargo,
el gráfico de dicha multifunción es un bicono en $\R^3$ con
vértice en el origen el cual no es un conjunto convexo.
}

\Rem{uniCase}{
Si $f:D\to Y$ es una función a valores en $Y$ y $F:D\to\P(Y)$
se define como $F(x):=\{f(x)\}$ para $x\in D$, entonces, la inclusión
\eq{cvxInc} se transforma en
$$
t\{f(x)\}+(1-t)\{f(y)\}\subseteq \{f(tx+(1-t)y)\}.
$$
Pero esto se cumple si y sólo si $f(tx+(1-t)y)=tf(x)+(1-t)f(y)$
para todo $t\in[0,1]$ y para todo $x,y\in D$. Note que 
las funciones lineales, tienen dicha propiedad. Aquí se evidencia
que la definición de convexidad dada por la inclusión 
\eq{cvxInc} no generaliza la \defi{cvxFunc}.
}

El siguiente Teorema es una generalización de un teorema 
clásico en la teoría de funciones convexas. Para su demostración
vamos a necesitar el siguiente lema, también conocido como
ley de cancelación de R\aa dstrom.

\begin{lemma}[\cite{Rad52}, Lema 1.]
\label{LRad52}
Sean $A,B$ y $C$ conjuntos dados de un espacio lineal topológico.
Suponga que $B$ es cerrado y convexo, $C$ es acotado y que además
$A+C\subseteq B+C$. Entonces, $A\subseteq B$.
\end{lemma}

\Thm{loc1}{
Sea $F:D\to \P_0(Y)$ una multifunción midconvexa tal que para todo
$x\in D$, el conjunto $F(x)$ es cerrado, convexo y acotado. Entonces,
para cualquier colección de puntos $x_1,\ldots,x_n$ en $D$
con $n\in\N$, se tiene que 
\Eq{loc1}{
\frac1n(F(x_1)+\cdots+F(x_n))\subseteq F\bigg(\frac{x_1+\cdots x_n}{n}\bigg).
}
}
\begin{proof}
Se puede demostrar fácilmente por inducción que para $p\in\N$, 
la multifunción $F$ satisface la inclusión
\Eq{loc11}{
\frac1{2^p}(F(x_1)+\cdots+F(x_{2^p}))\subseteq F\bigg(\frac{x_1+\cdots x_{2^p}}{2^p}\bigg).
}
Ahora bien, para $n\in\N$ fijo, consideremos $p\in\N$ tal que $n\leq2^p$.
Sean $x_1,\ldots,x_n$ puntos arbitrarios en $D$ y definamos
$x_k:=\bar{x}:=(x_1+\cdots+x_n)/n$ para $k=n+1,\ldots,2^p$.

Observe que con esta notación se tiene  
$\bar{x}=(x_1+\cdots+x_{2^p})/2^p$ y por lo tanto la inclusión \eq{loc11}
se convierte en
\Eq{loc12}{
\frac1{2^p}(F(x_1)+\cdots+F(x_n)+F(x_{n+1})+\cdots+F(x_{2^p}))\subseteq F(\bar{x}).
}
Como $x_k:=\bar{x}$ para $k=n+1,\ldots,2^p$ entonces, 
$F(x_{n+1})=\cdots=F(x_{2^p})=F(\bar{x})$ por lo tanto, 
la convexidad de las imágenes de la multifunción $F$ 
nos da la siguiente fórmula
\Eq{*}{
F(x_{n+1})+\cdots+F(x_{2^p})=(2^p-n)F(\bar{x}).
}
Luego, la inclusión \eq{loc12} equivale a
\Eq{loc13}{
\frac1{2^p}\big(F(x_1)+\cdots+F(x_n)+(2^p-n)F(\bar{x})\big)\subseteq F(\bar{x}).
}
Multiplicando \eq{loc13} por $2^p$ y usando de nuevo la convexidad de las imágenes de la 
multifunción $F$ se tiene lo siguiente
\Eq{*}{
F(x_1)+\cdots+F(x_n)+(2^p-n)F(\bar{x})\subseteq (2^p-n)F(\bar{x})+nF(\bar{x}).
}
Aplicando el \lem{Rad52} a la inclusión anterior se obtiene \eq{loc1} y esto
completa la demostración.
\end{proof}

\Cor{loc}{
Bajo las hipótesis del \thm{loc1} se cumple que para
todo $x,y\in D$ y para todo $t\in[0,1]\cap\Q$ 
\Eq{*}{
tF(x)+(1-t)F(y)\subseteq F(tx+(1-t)y).
}
}

\section{$K$-convexidad y $K$-concavidad de multifunciones.}

Sea $K\subseteq Y$ un cono convexo, las siguientes definiciones generalizan
las definiciones de convexidad y concavidad dadas en la sección anterior.
\Defi{K-cvxSVM}{
Se dice que la multifunción $F$ es $K$-convexa en $D$ si para todo $x,y\in D$ 
y para todo $t\in[0,1]$
\Eq{Kcvxsvm}{
tF(x)+(1-t)F(y)\subseteq F(tx+(1-t)y)+K
}
}
\Defi{K-ccvSVM}{
Se dice que la multifunción $F$ es $K$-cóncava en $D$ si para todo $x,y\in D$ 
y para todo $t\in[0,1]$
\Eq{Kccvsvm}{
F(tx+(1-t)y)\subseteq tF(x)+(1-t)F(y)+K
}
}

\Prp{nikHabil1}{
Si $K\subseteq Y$ es un cono que posee al origen, entonces,
una multifunción $F:X\to\P_0(Y)$ es $K$-convexa (resp. $K$-cóncava) si y sólo si
la multifunción $F+K:X\to \P_0(Y)$ definida por $(F+K)(x):=F(x)+K$
es convexa (resp. cóncava).
}
\begin{proof}
Supongamos que la multifunción $F$ es $K$-convexa, es decir, que para todo
$x,y\in D$ y para todo $t\in[0,1]$ se satisface la siguiente inclusión
\Eq{*}{
tF(x)+(1-t)F(y)\subseteq F(tx+(1-t)y)+K.
}
Ahora bien, para $x,y\in D$ y para $t\in[0,1]$ 
\Eq{*}{
t(F+K)(x)+(1-t)(F+K)(y) &= t(F(x)+K)+(1-t)(F(y)+K)\\
&=tF(x)+tK+(1-t)F(y)+(1-t)K
}
pero como $K$ es un cono convexo entonces, 
$tK+(1-t)K\subseteq K$ para todo $t\in[0,1]$ por lo tanto,
\Eq{*}{
t(F+K)(x)+(1-t)(F+K)(y)\subseteq tF(x)+(1-t)F(y)+K.
}
Finalmente, la $K$-convexidad de $F$ implica que 
\Eq{*}{
tF(x)+(1-t)F(y)+K&\subseteq F(tx+(1-t)y)+K+K \\
&\subseteq F(tx+(1-t)y)+K=(F+K)(tx+(1-t)y),
}
y así
\Eq{*}{
t(F+K)(x)+(1-t)(F+K)(y)\subseteq (F+K)(tx+(1-t)y),
}
lo que demuestra que la multifunción $(F+K)$ es convexa.

Recíprocamente, supongamos que la multifunción $(F+K)$ es
convexa. Como $0\in K$, entonces, para $t\in[0,1]$ y para 
$x,y\in D$,
\Eq{*}{
tF(x)+(1-t)F(y)\subseteq tF(x)+(1-t)F(y)+K.
}
Por otra parte, usando la convexidad del cono $K$, se sigue
que $K=tK+(1-t)K$ y así
\Eq{*}{
tF(x)+(1-t)F(y)+K &= t(F(x)+K)+(1-t)(F(y)+K) \\ 
&= t(F+K)(x)+(1-t)(F+K)(y).
}
Ahora bien, como $(F+K)$ es convexa entonces
\Eq{*}{
t(F+K)(x)+(1-t)(F+K)(y) \subseteq (F+K)(tx+(1-t)y).
}
Luego, para todo $t\in[0,1]$ y para todo $x,y\in D$
\Eq{*}{
tF(x)+(1-t)F(y)\subseteq F(tx+(1-t)y)+K,
}
lo que finaliza la demostración.
\end{proof}

\Defi{K-midCvxSVM}{
Se dice que la multifunción $F$ es $K$-midconvexa en $D$ si para todo $x,y\in D$ 
$F$ satisface la inclusión \eq{Kcvxsvm} para $t=\frac12$, i.e,
\Eq{Kmcvxsvm}{
\frac{F(x)+F(y)}2\subseteq F\bigg(\frac{x+y}2\bigg)+K.
}
}

\Defi{K-midCcvSVM}{
Se dice que la multifunción $F$ es $K$-midcóncava en $D$ si para todo $x,y\in D$ 
$F$ satisface la inclusión \eq{Kccvsvm} para $t=\frac12$, i.e,
\Eq{Kmccvsvm}{
F\bigg(\frac{x+y}2\bigg)\subseteq \frac{F(x)+F(y)}2+K.
}
}
\Rem{cvxSV}{
Dado un cono convexo $K\subseteq Y,$ definamos la relación $\leq_K$ en $Y$
de la siguiente manera
$$
x\leq_K y \iff y-x\in K.
$$
Si $F(x)=\{f(x)\}$, donde $f:X\to Y$ es una función cualquiera, entonces las 
inclusiones \eq{Kcvxsvm} y \eq{Kccvsvm} equivalen a 
\Eq{*}{
f(tx+(1-t)y)\leq_K tf(x)+(1-t)f(y)
\quad\mbox{y}\quad
tf(x)+(1-t)f(y)\leq_K f(tx+(1-t)y)
}
respectivamente. De hecho si $Y=\R$ y $K=\R_+$, entonces estas definiciones
coinciden con las definiciones de convexidad y concavidad de funciones 
respectivamente introducidas en el Capítulo 2.
}
%Muchos resultados importantes relacionados con multifunciones $K$-convexas 
%y $K$-cóncavas pueden ser encontrados en la tesis doctoral del
%profesor Kazimierz Nikodem \cite{Nik89}.
%
%En los resultados principales las siguientes definiciones serán importantes

Como es de esperarse, no toda multifunción $K$-midconvexa es convexa. Sin
embargo, K. Nikodem en \cite{Nik86} demuestra el siguiente resultado

\begin{theorem}
\label{TNik860}
Sea $F:D\to \P_0(Y)$ es una multifunción $K$-midconvexa, entonces
$F$ satisface la inclusión \eq{Kcvxsvm} para todo $x,y\in D$
y para todo $t\in\D\cap[0,1]$.
\end{theorem}

\Defi{Klbddsvm}{
Se dice que la multifunción $F$ es localmente $K$-acotada inferior en $x_0\in D$, si existe
un abierto $U\in\U(X)$ y un conjunto acotado $H\subseteq Y$ tales que 
\Eq{*}{
F(u)\subseteq H+K,\qquad\mbox{para todo }u\in (x_0+U)\cap D.
}
}
\Defi{Kubddsvm}{
Se dice que la multifunción $F$ es localmente $K$-acotada superior en $x_0\in D$, 
si $F$ es localmente $(-K)$-acotada inferior en dicho punto.
}
\Exa{Klwbdd}{
Definamos $F(x):=(\sin(x),\infty),$ $x\in\R$. Evidentemente, para todo $x\in\R$ 
el conjunto $F(x)$ no es acotado. Sin embargo, al considerar 
$K:=(0,\infty)$ se tiene que $F(x)\subseteq \{-1\}+K$ para todo $x\in \R$ por lo tanto,
esta multifunción así definida es $K$-acotada inferiormente en todo $\R$. 
}
\Defi{clKlbddsvm}{
Se dice que la multifunción $F$ es localmente semi-$K$-acotada inferior en $x_0\in D$, 
si existe un abierto $U\in\U(X)$ y un conjunto acotado $H\subseteq Y$ tales que 
\Eq{*}{
F(u)\subseteq \cl(H+K),\qquad\mbox{para todo }u\in (x_0+U)\cap D.
}
}
\Rem{equiv}{
Si $Y$ es un espacio localmente acotado, i.e., existe un abierto $V\in\U(Y)$ que es acotado,
entonces la \defi{Klbddsvm} y la \defi{clKlbddsvm} son equivalentes.
Basta ver que si $F(u)\subseteq\cl(H+K)$ entonces,
$F(u)\subseteq H+V+K$. Siendo $H+V$ un conjunto acotado,
se tiene que $F$ es localmente $K$-acotada inferior.
}
\Defi{weakKlbddsvm}{
Se dice que la multifunción $F$ es localmente débil-$K$-acotada superior en $x_0\in D$, si existe
un abierto $U\in\U(X)$ y un conjunto acotado $H\subseteq Y$ tales que 
\Eq{*}{
0\in F(u)+H+K,\qquad\mbox{para todo }u\in (x_0+U)\cap D.
}
}
\Defi{weakclKlbddsvm}{
Se dice que la multifunción $F$ es localmente débil-semi-$K$-acotada superior en $x_0\in D$, 
si existe un abierto $U\in\U(X)$ y un conjunto acotado $H\subseteq Y$ tales que 
\Eq{*}{
0\in\cl(F(u)+H+K),\qquad\mbox{para todo }u\in (x_0+U)\cap D.
}
}
\Prp{lem2.3}{
Supongamos que $F:D\to\P_0(Y)$ es una multifunción localmente semi-$K$-acotada inferior. 
Entonces, para cada subconjunto compacto $C\subseteq D$, existe un conjunto acotado 
$H\subseteq Y$ tal que, $F(x)\subseteq\cl(H+K),$ para todo $x\in C$.
}
\begin{proof}
Sea $C\subseteq D$ un conjunto compacto no-vacío.
Como la multifunción $F$ es localmente semi-$K$-acotada inferior, entonces, para cada
$x\in C$ existe un abierto $U_x\in \U(X)$ y un conjunto acotado $H_x\subseteq Y$ tales que
para todo $u\in (x+U_x)\cap D$, $F(u)\subseteq\cl(H_x+K)$. Ahora bien, es claro que 
$C\subseteq \bigcup_{x\in C}(x+U_x)$, es decir que la familia $\{x+U_x:x\in C\}$ es un 
cubrimiento por abiertos del compacto $C$, por lo tanto, existen $x_1,\ldots,x_n\in C$
tales que $C\subseteq\bigcup_{i=1}^n(x_i+U_{x_i})$. Sea $H:=H_{x_1}\cup\cdots\cup H_{x_n}$,
es claro que $H$ es un conjunto acotado, pues es la unión finita de conjuntos acotados. Además,
si $z\in C$ es un elemento arbitrario de $C$ entonces $z\in x_i+U_{x_i}$ para algún $i=0,\ldots,n$
y por lo tanto $F(z)\subseteq\cl(H_{x_i}+K)\subseteq\cl(H+K)$. Lo cual, finaliza la demostración. 
\end{proof}

\Prp{lem2.4}{
Supongamos que $F:D\to\P_0(Y)$ es una multifunción localmente débil-semi-$K$-acotada superior. 
Entonces, para cada subconjunto compacto $C\subseteq D$, existe un conjunto acotado 
$H\subseteq Y$ tal que, $0\in\cl(F(x)+H+K),$ para todo $x\in C$.
}
\begin{proof}
Sea $C\subseteq D$ un conjunto compacto no-vacío.
Como la multifunción $F$ es localmente débil-semi-$K$-acotada superior, entonces para cada
$x\in C$ existe un abierto $U_x\in \U(X)$ y un conjunto acotado $H_x\subseteq Y$ tales que
para todo $u\in (x+U_x)\cap D$, $0\in\cl(F(u)+H_x+K)$. Ahora bien, es claro que 
$C\subseteq \bigcup_{x\in C}(x+U_x)$, es decir que la familia $\{x+U_x:x\in C\}$ es un 
cubrimiento por abiertos del compacto $C$, por lo tanto, existen $x_1,\ldots,x_n\in C$
tales que $C\subseteq\bigcup_{i=1}^n(x_i+U_{x_i})$. Sea $H:=H_{x_1}\cup\cdots\cup H_{x_n}$,
es claro que $H$ es un conjunto acotado, pues es la unión finita de conjuntos acotados. Además,
si $z\in C$ es un elemento arbitrario de $C$ entonces $z\in x_i+U_{x_i}$ para algún $i=0,\ldots,n$
y por lo tanto, $0\in\cl(F(z)+H_{x_i}+K)\subseteq\cl(F(z)+H+K)$. Lo cual, finaliza la demostración. 
\end{proof}

\section{El teorema de Bernstein--Doetsch para multifunciones.}

Con el fin de establecer algunos resultados importantes relacionados con
el Teorema de Bernstein--Doetsch en el contexto de multifunciones convexas,
es necesario establecer las siguientes definiciones.

\Defi{KupSemConSVM}{
Sea $F:D\to\P_0(Y)$ una multifunción. Se dice que $F$ es $K$-semicontinua superior si
para todo abierto $V\in \U(Y)$, existe un abierto $U\in\U(X)$ tal que
\Eq{KupSemConSVM}{
F(x)\subseteq F(x_0)+V+K\qquad(x\in x_0+U).
}
}
\Defi{KloSemConSVM}{
Sea $F:D\to\P_0(Y)$ una multifunción. Se dice que $F$ es $K$-semicontinua inferior si
para todo abierto $V\in \U(Y)$, existe un abierto $U\in\U(X)$ tal que
\Eq{KloSemConSVM}{
F(x_0)\subseteq F(x)+V+K\qquad(x\in x_0+U).
}
}
\Defi{KConSVM}{
Sea $F:D\to\P_0(Y)$ una multifunción. Se dice que $F$ es $K$-continua si
$F$ es $K$-semicontinua superior e inferior al mismo tiempo.
}
\Defi{DirConSup}{ %continuidad direccional superior
	Sea $F:D\to\P_0(Y)$ una multifunci\'on. Decimos que $F$ es
	\emph{direccionalmente $K$-semicontinua superior en el punto $p\in D$} 
	si, para cualquier direcci\'on $h\in X$ 
	y para todo entorno abierto $U\in\U(Y)$, existe un n\'umero positivo
	$\delta$ tal que
	\Eq{*}{
		F(p+th)\subseteq F(p)+U+K
	}
	para todo $0<t<\delta$ con $p+th\in D$. En el caso particular cuando
	$K=\{0\}$, nos referiremos a esta definici\'on como 
	\textit{semicontinuidad direccional superior en $p$}. 
}
Claramente, toda multifunci\'on direccionalmente semicontinua superior
es direccionalmente $K$-semicontinua superior para cualquier
cono $K$. An\'alogamente, tenemos la siguiente 
\Defi{DirConInf} {
	Sea $F:D\to\P_0(Y)$ una multifunci\'on. Decimos que $F$ es
	\emph{direccionalmente $K$-semicontinua inferior en el punto $p\in D$} 
	si, para cualquier direcci\'on $h\in X$ 
	y para todo entorno abierto $U\in\U(Y)$, existe un n\'umero positivo
	$\delta$ tal que
	\Eq{*}{
		F(p)\subseteq F(p+th)+U+K
	}
	para todo $0<t<\delta$ con $p+th\in D$. En el caso particular cuando
	$K=\{0\}$, nos referiremos a esta definici\'on como 
	\emph{semicontinuidad direccional inferior en $p$}.
}
\Defi{ContDir} { % Continuidad direccional
	Si $F$ es al mismo tiempo direccionalmente $K$-semicontiua superior e inferior
	en $p$, entonces diremos que $F$ es \emph{direccionalmente $K$-continua en $p$.}	
}
	
\Lem{dusc}{
	Sea $K\subseteq Y$ un cono convexo y $S,T\subseteq Y$ conjuntos no-vac\'ios 
	y semi-$K$-acotados inferiormente que son semi-$K$-estrellados con respecto a algunos 
	elementos de $Y$. Entonces, la multifunci\'on $t\mapsto tS+(1-t)T$ 
	es direccionalmente $K$-continua en $[0,1]$.
}
\begin{proof}
	Sea $U\in\U(X)$. Es suficiente demostrar que existe un n\'umero real positivo
	$\delta$ tal que, para $s,t\in[0,1]$ con $|t-s|<\delta$,  
	\Eq{st}{
		tS+(1-t)T\subseteq sS+(1-s)T + U+K.
	}
	Primero, escojamos $V\in\U(Y)$ tal que $[3]V\subseteq U$.
	Obviamente, \eq{st} es v\'alida para $s=t$. Sin p\'erdida de generalidad,
	podemos asumir que $0\leq s<t\leq 1$. Supongamos que  $T$ es semi-$K$-estrellado con 
	respecto a $v\in Y$. Entonces, tenemos
	\Eq{est}{
		tS+(1-t)T
		&\subseteq sS + (t-s)S + (1-t)T + (t-s)v+(s-t)v \\
		&= sS + (t-s)S + (1-s)\bigg(\frac{1-t}{1-s}T+ \frac{t-s}{1-s}v\bigg)+(s-t)v \\
		&\subseteq sS + (t-s)(S-v) + (1-s)\cl(T+K) \\
		&\subseteq sS + (1-s)T + (t-s)(S-v) +V+K.
	}
	Como el conjunto $(S-v)$ es semi-$K$-acotado inferiormente, existe un subconjunto 
	acotado $H\subseteq Y$ tal que $S-v\subseteq\cl(H+K)$. Como el conjunto $H$
	es acotado, existe un n\'umero positivo $\delta\leq 1$ tal que $\delta H\subseteq V$.
	Por lo tanto, si $t-s<\delta$, entonces 	
	\Eq{*}{
		(t-s)(S-v)\subseteq (t-s)\cl(H+K)\subseteq (t-s)(H+V+K) \subseteq (t-s)H+V+K \subseteq [2]V+K.
	}
	Combinando estas estimaciones con \eq{est}, obtenemos que \eq{st} se cumple.
	La demostraci\'on para el caso $t<s$ es completamente an\'aloga y usa que $S$
	es $K$ estrellado y $T$ es semi-$K$-acotado inferiormente.
\end{proof}
	

\Rem{semCon}{
Cuando $K=\{0\}$, la $K$-continuidad de una multifunción equivale a la continuidad con 
respecto a la topología de Hausdorff, ver \cite{DeBPia83}. 

Por otra parte, si $F(x)={f(x)}$, para alguna función $f:D\to \R$
y $K=[0,\infty)$, entonces $K$-semicontinuidad superior e inferior
equivalen a semicontinuidad superior e inferior respectivamente.
}

\Exa{Kcon1}{
Si $A\subseteq Y$ es un conjunto acotado y $K\subseteq Y$ es un cono convexo. Entonces,
la multifunción $F:\R\to\P_0(Y)$ definida mediante la fórmula
$F(t):= tA+K$ es $K$-continua con respecto a la topología de Hausdorff.

Sea $t_0\in\R$ y sea $V\in\U(Y)$ un entorno abierto del origen en $Y$. 
Observemos lo siguiente
\Eq{*}{
F(t_0)=t_0A+K=(t_0-t+t)A+K\subseteq(t_0-t)A+tA+K=(t_0-t)A+F(t).
}
Ahora bien, como el conjunto $A$ es acotado, existe $\delta>0$ tal que 
si $0<s<\delta$ entonces $sA\subseteq V.$ Luego, si $0<t-t_0<\delta$
entonces $(t-t_0)A\subseteq V$ y por lo tanto,
\Eq{*}{
F(t_0)\subseteq F(t)+W.
}
Esto demuestra que la multifunción $F$ así definida es
semicontinua inferior con respecto a la topología de Hausdorff.
Para demostrar la semicontinuidad superior de $F$ se procede de manera
similar.
}

Es bien conocido que toda función midconvexa y continua es convexa.
A continuación, veremos un resultado análogo para la clase de multifunciones
midconvexas. Para un conjunto $A$ y $n\in\N$ se denotará 
$$
[n]A:=\{x_1+\cdots x_n | x_1,\ldots,x_n\in A\}.
$$ 
\Thm{midConSVM}{
Sean $X,Y$ espacios topológicos lineales.
Sea $F:D\to \P_0(Y)$ una multifunción $K$-semicontinua superior
tal que para todo $x\in D$ el conjunto
$F(x)\subseteq Y$  es compacto. Si $F$ es $K$-midconvexa y $K$-semicontinua superior
en $D$, entonces, $F$ es $K$-convexa.
}
\begin{proof}
Sean $x,y\in D$ y $t\in[0,1]$ fijos. Sea $(q_n)_n\subseteq\D$ una sucesión de números
diádicos racionales que converge a $t$. Sea $V\in \U(Y)$ un abierto 
arbitrario y consideremos $W\in\U(Y)$ tal que $[3]\,W\subseteq V.$ 
Por el \thm{Nik860} se tiene que para
todo $n\in\N$, la multifunción $F$ satisface la inclusión
\Eq{*}{
q_nF(x)+(1-q_n)F(y)\subseteq F(q_nx+(1-q_n)y)+K.
}
Como las imágenes de $F$ son conjuntos acotados, entonces existen
números naturales $n_1, n_2$ tales que para $n\geq n_1$, 
$$tF(x)\subseteq q_nF(x)+W$$
 y para $n\geq n_2$
$$(1-t)F(y)\subseteq(1-q_n)F(y)+W.$$ 
Por otra parte, la $K$-semicontinuidad superior de $F$ en 
el punto $tx+(1-t)y$ da como resultado que para $n\geq n_3$,
$n_3\in\N$ 
\Eq{*}{
F(q_nx+(1-q_n)y)\subseteq F(tx+(1-t)y)+W+K.
}
Ahora bien, si $n\geq\max\{n_1,n_2,n_3\}$ entonces
\Eq{*}{
tF(x)+(1-t)F(y)&\subseteq q_nF(x)+(1-q_n)F(y)+[2]W\\
&\subseteq F(q_nx+(1-q_n)y)+[2]W+K
 \subseteq F(tx+(1-t)y)+[3]W+K \\
&\subseteq F(tx+(1-t)y)+V+K.
}
Como el abierto $V$ es arbitrario, entonces
\Eq{*}{
tF(x)+(1-t)F(y)\subseteq \bigcap_{V\in\U(Y)}(F(tx+(1-t)y)+K+V)
=\cl(F(tx+(1-t)y)+K).
}
Como $K$ es un cono cerrado y $F(tx+(1-t)y)$ es un conjunto
compacto, entonces, por la \prp{sumSets} se tiene que
la suma de ellos es un conjunto cerrado y por lo tanto
$\cl(F(tx+(1-t)y)+K)=F(tx+(1-t)y)+K$, lo que completa la demostración.
\end{proof}


En este contexto, K. Nikodem en \cite{Nik86} establece el siguiente resultado, que 
generaliza el \thm{BD15} y además engloba parte de los resultados 
obtenidos por Trudzik en \cite{Tru84}.

\begin{theorem}[\cite{Nik86}, Teorema 1]
\label{Nik861}
Sean $X,Y$ espacios topológicos lineales, y sea $K\subseteq Y$ un cono convexo
tal que $0\in K$.
Sea $F:D\to \P_0(Y)$ una multifunción tal que para todo $x\in D$ el conjunto
$F(x)\subseteq Y$ es acotado. Si $F$ es $K$-midconvexa y $K$-acotada superior
en un subconjunto $H\subseteq D$ con interior no vacío, entonces,
$F$ es $K$-continua en $D$.
\end{theorem}

Cuando $K=\{0\}$, entonces, $K$-convexidad es simplemente convexidad
y K. Nikodem en 1987 obtuvo los siguientes resultados, 
\begin{theorem}[\cite{Nik87c}, Teorema 1]
\label{Nik87c1}
Sean $X,Y$ espacios topológicos lineales.
Sea $F:D\to \P_0(Y)$ una multifunción tal que para todo $x\in D$ el conjunto
$F(x)\subseteq Y$ es acotado. Si $F$ es midconvexa y acotada superiormente
en un subconjunto $H\subseteq D$ con interior no vacío, entonces,
$F$ es continua en $D$ con respecto a la topología de Hausdorff.
\end{theorem}

\begin{theorem}[\cite{Nik87a}, Teorema 1]
\label{Nik87a1}
Sean $X,Y$ espacios topológicos lineales.
Sea $F:D\to \P_0(Y)$ una multifunción tal que para todo $x\in D$ el conjunto
$F(x)\subseteq Y$ es acotado y convexo. Si $F$ es midcóncava y acotada 
en un subconjunto $H\subseteq D$ con interior no vacío, entonces,
$F$ es continua en $D$ con respecto a la topología de Hausdorff.
\end{theorem}

\section{Convexidad fuerte.}
Supongamos ahora que $(X,\|\cdot\|)$ y $(Y,\|\cdot\|)$
son espacios vectoriales normados
y sea $B_Y\subseteq Y$ el interior 
de la bola unitaria en $Y$. Huang en su artículo 
del año 2010 \cite{Hua10}, establece la siguiente 
definición, generalizando el concepto de convexidad fuerte
para funciones, introducido por Polyak en \cite{Pol66}. 

\Defi{StrgCvxSVM}{
Sea $F:D\to\P_0(Y)$ una multifunción y sea $c>0$. Se dice que 
$F$ es fuertemente convexa con módulo $c$
si 
\Eq{StrgCvxSVM}{
tF(x)+(1-t)F(y)+ct(1-t)\|x-y\|^2\overline{B_Y} \subseteq 
F(tx+(1-t)y)
}
para todo $x,y\in D$ y para todo $t\in[0,1]$.
}
En base a esta definición surge de forma natural la
definición de midconvexidad fuerte.
\Defi{StrgMidCvxSVM}{
Sea $F:D\to\P_0(Y)$ una multifunción y sea $c>0$. Se dice que 
$F$ es fuertemente midconvexa con módulo $c$ si 
\Eq{StrgMidCvxSVM}{
\frac{F(x)+F(y)}2+\frac{c}4\|x-y\|^2\overline{B_Y} \subseteq 
F\bigg(\frac{x+y}2\bigg).
}
}
El siguiente Teorema caracteriza a la familia de multifunciones
fuertemente convexas con m\'odulo $c$.
\begin{theorem}[\cite{LeiMerNikSan13}, Teorema 12]
Sean $(X,\|\cdot\|)$ un espacio con producto interno, $t$
un n\'umero fijo en $(0,1)$ y $D$ un subconjunto
convexo de $X$. Una multifunci\'on $F:D\to\P_0(Y)$ tal que $F(x)$
es un conjunto convexo y cerrado para todo $x\in D,$ es 
fuertemente convexa con m\'odulo $c$ si y s\'olo si
la multifunci\'on $G$ definida por $G(x):=F(x)+\|x\|^2\overline{B_Y}$
para $x\in D$ es convexa.
\end{theorem}

Es evidente que toda multifunción fuertemente convexa con módulo $c$
es convexa. Ahora bien, para esta nueva clase de multifunciones,
H. Leiva et. al. en \cite{LeiMerNikSan13} obtienen el siguiente
resultado de tipo Bernstein--Doetsch.

\begin{theorem}[\cite{LeiMerNikSan13}, Teorema 4]
Sea $F:D\to\P_0(Y)$ una multifunci\'on fuertemente midconvexa con m\'odulo
$c$ tal que para todo $x\in D$ el conjunto $F(x)$ es cerrado y acotado. Si
$F$ es semicontinua superior en $D$, entonces esta es fuertemente
convexa con m\'odulo $c$.
\end{theorem}   

Como consecuencia del teorema anterior surge el siguiente

\begin{corollary}
Supongamos que $D\subseteq X$ es un conjunto abierto y convexo. Si 
una multifunci\'on $F:D\to\P_0(Y)$ es fuertemente convexa con m\'odulo
$c$ y semicontinua inferior en un punto de $D$ entonces, $F$ es continua
y fuertemente convexa con m\'odulo $c$.
\end{corollary}

Hasta ahora hemos presentado la mayoría de los resultados de tipo Bernstein--Doetsch
que han sido obtenidos a lo largo del tiempo tanto para funciones como para
multifunciones. En el cap\'itulo 4 como se mencion\'o en la introducci\'on 
presentaremos dos Teoremas que pretenden englobar a gran parte de los resultados
de tipo Bernstein--Doetsch obtenidos hasta el momento.


\section{Transformación de Takagi.}

En el Capítulo \ref{chapPrevio} definimos la función de Takagi
$T$ para cada $x\in\R$, mediante la fórmula
\Eq{*}{
T(x):=\sum_{k=0}^\infty\frac{d_\Z(2^kt)}{2^k},
}
donde 
\Eq{dz}{
	d_\Z(x) := \dist(\Z,x):=\inf\{|z-x| : z\in\Z\},\quad x\in\R.	
}

Supongamos ahora que $D$ es un conjunto estrellado con respecto al origen, 
y consideremos una multifunción $S:D\to\P_0(Y)$ con la propiedad de 
que $0\in S(x)$ para todo $x\in D$.
Para dicha multifunción, se define $S^T:\R\times D\to Y$ de la siguiente manera
\Eq{TakTrS}{
 S^T(t,x):=\cl\bigg(\bigcup_{n=0}^{\infty} \sum_{k=0}^{n} 
                 \frac{1}{2^k}S\big(2d_{\Z}(2^kt)x\big)\bigg)\qquad(t\in\R,\,x\in D).
}
\Rem{TakTrans}{
	El hecho de que $0\in S(x)$ para todo $x\in D$ es crucial, ya que esto trae como consecuencia
	que la sucesión de conjuntos 
	\Eq{*}{
		\Bigg(\sum_{k=0}^{n}\frac{1}{2^k}S\big(2d_{\Z}(2^kt)x\big)\Bigg)_{n\in\N}
	}
	sea creciente. Por lo tanto, $S^T$ no es más que el límite inferior
	de dicha sucesión.
}
La multifunción $S^T$ será llamada transformación de Takagi de $S$ a lo largo del trabajo. 
\Defi{Recsvm}{
El cono recesión de la multifunción $S:D\to\P_0(Y)$ es el conjunto
\Eq{*}{
\rec(S):= \bigcap_{x\in D}\rec S(x)
}
}
De la definición se puede observar que para todo $x\in D$ se tiene que 
$$\rec(S)+S(x)\subseteq S(x).$$ 
A continuación, se establecerá la relación entre
una multifunción y su transformación de Takagi.
\Prp{SST}{
Sea $D\subseteq X$ un conjunto estrellado y sea $S:D\to\P_0(Y)$ una multifunción
tal que $0\in S(x)$ para todo $x\in D$. Entonces
\Eq{TT1}{
   \cl(S(x))\subseteq S^T\big(\tfrac12,x\big) \qquad (x\in D).
}
Si además, $S(0)\subseteq \overline\rec(S)$, entonces
\Eq{TT2}{
   \cl(S(x))= S^T\big(\tfrac12,x\big) \qquad (x\in D).
}
}
\begin{proof} 
Observe que $d_{\Z}\big(\tfrac12\big)=\tfrac12$ y $d_{\Z}\big(2^k\cdot\tfrac12\big)=0$ para $k\in\N$.
Por lo tanto,
\Eq{*}{
  S^T\big(\tfrac12,x\big)
   =\cl\bigg(\bigcup_{n=0}^{\infty} \sum_{k=0}^{n}\frac{1}{2^k}S\big(2d_{\Z}(2^k\cdot\tfrac12)x\big)\bigg)
   =\cl\bigg(S(x)+\bigcup_{n=0}^{\infty} \sum_{k=1}^{n}\frac{1}{2^k}S(0)\bigg).
}
Como $0\in S(0)$, entonces la inclusión \eq{TT1} se sigue inmediatamente. Para demostrar
\eq{TT2}, asuma que $S(0)\subseteq \overline\rec(S)$. Entonces, 
$S(0)\subseteq \overline\rec(S(x))\subseteq\rec(\overline{S(x)})$.
Como $\rec(\overline{S(x)})$ es un cono convexo, se tiene que este conjunto es cerrado
bajo la suma y bajo la multiplicación por escalares positivos.
Así, para todo $n\in\N$,
\Eq{*}{
  \sum_{k=1}^{n}\frac{1}{2^k}S(0)
  \subseteq\sum_{k=1}^{n}\frac{1}{2^k}\rec(\overline{S(x)})
  \subseteq\rec(\overline{S(x)}).
}
En consecuencia, 
\Eq{*}{
\bigcup_{n=0}^{\infty} \sum_{k=1}^{n}\frac{1}{2^k}S(0)\subseteq\rec(\overline{S(x)}).
}
Luego,
\Eq{*}{
  S^T\big(\tfrac12,x\big)
   =\cl\bigg(S(x)+\bigcup_{n=0}^{\infty}\sum_{k=1}^{n}\frac{1}{2^k}S(0)\bigg)
   \subseteq\cl\bigg(\overline{S(x)}+\rec(\overline{S(x)})\bigg)
   \subseteq \cl(S(x)),
}
lo cual completa la demostración de \eq{TT2}. 
\end{proof}

\Prp{Tak}{
Sea $D\subseteq X$ un conjunto estrellado, $S_0\subseteq Y$ un conjunto convexo  
que contiene a $0\in Y$ y $K\subseteq Y$ un cono convexo. Sea $\varphi:D\to \R_+$ 
una función localmente acotada superior y no negativa.
Definamos $S:D\to\P_0(Y)$ por $S(x):=K+\varphi(x)S_0$. Entonces
\Eq{Tak1}{
  S^T(t,x)=\cl\big(K+\varphi^T(t,x) S_0\big)
  \qquad(t\in\R,\,x\in D),
}
donde
\Eq{Tak2}{  
  \varphi^T(t,x)=\sum_{n=0}^{\infty}\frac{1}{2^n}\varphi\big(2d_{\Z}(2^nt)x\big)
  \qquad(t\in\R,\,x\in D).
}
Si además, $\varphi(0)=0$, entonces
\Eq{Tak+}{
  \varphi^T\big(\tfrac12,x\big)=\varphi(x) \qquad\mbox{y}\qquad 
  S^T\big(\tfrac12,x\big)=\cl(K+\varphi(x)S_0)=\cl(S(x))\qquad(x\in D).
}}

\begin{proof} 
Para $t\in\R$ y $n\geq 0$, se tiene que $0\leq 2d_{\Z}(2^nt)\leq 1$,
por lo tanto $2d_{\Z}(2^nt)x\in[0,x]$. Como la función $\varphi$ es localmente acotada 
superior en $D$, entonces es acotada superior en el segmento $[0,x]$ por alguna constante
$M(x)$. Luego
\Eq{*}{
  \varphi^T(t,x)=\sum_{n=0}^{\infty}\frac{1}{2^n}\varphi\big(2d_{\Z}(2^nt)x\big) 
    \leq \sum_{n=0}^{\infty}\frac{1}{2^n}M(x)=2M(x) \qquad(t\in\R).
}

Para probar \eq{Tak1}, fijemos $(t,x)\in\R\times D$. 

Para demostrar la inclusión $\subseteq$ en \eq{Tak1}, primero se demostrará que
\Eq{Tak3}{
\bigcup_{n=0}^{\infty} \sum_{k=0}^{n}\frac{1}{2^k}S\big(2d_{\Z}(2^kt)x\big)
\subseteq K+\varphi^T(t,x) S_0.
}
Usando la definición de $S$, la convexidad del conjunto $S_0$ y el hecho 
de que la función $\varphi$ es no-negativa, se tiene que para $n\geq 0$
\Eq{*}{
\sum_{k=0}^{n} \frac{1}{2^k}S\big(2d_{\Z}(2^kt)x\big) 
      &= \sum_{k=0}^{n} \bigg(K + \frac{1}{2^k}\varphi\big(2d_{\Z}(2^kt)x\big)S_0\bigg) 
      = K + \bigg(\sum_{k=0}^{n} \frac{1}{2^k}\varphi\big(2d_{\Z}(2^kt)x\big)\bigg)S_0 \\
      &\subseteq K + \bigg(\sum_{k=0}^{\infty} \frac{1}{2^k}\varphi\big(2d_{\Z}(2^kt)x\big)\bigg)S_0
      = K+\varphi^T(t,x) S_0.
}
Así, queda demostrada la inclusión \eq{Tak3}. Tomando la clausura en ambos lados, 
la inclusión $\subseteq$ en \eq{Tak1} se sigue de inmediato. 

Para la prueba de la inclusión $\supseteq$ en \eq{Tak1}, 
es suficiente mostrar que
\Eq{*}{
  K+\varphi^T(t,x)S_0\subseteq S^T(t,x).
}
Observe que para $n\geq0$ se tiene lo siguiente
\Eq{*}{
K&+\Bigg(\sum_{k=0}^{n} \frac{1}{2^k}\varphi\big(2d_{\Z}(2^kt)x\big)\Bigg)S_0
  \subseteq \Bigg(\sum_{k=0}^{n} \frac{1}{2^k}\varphi\big(2d_{\Z}(2^kt)x\big)\Bigg)(S_0+K)\\
	 &\subseteq \sum_{k=0}^{n} \frac{1}{2^k}\varphi\big(2d_{\Z}(2^kt)x\big)(S_0+K) 
	\subseteq \sum_{k=0}^{n} \frac{1}{2^k}\big(\varphi\big(2d_{\Z}(2^kt)x\big)S_0+K\big)\\
	&=\sum_{k=0}^{n} \frac{1}{2^k}S(2d_{\Z}(2^kt)x) 
	\subseteq \bigcup_{\ell=0}^{\infty} \sum_{k=0}^{\ell}\frac{1}{2^k}S\big(2d_{\Z}(2^kt)x\big).
}
Por lo tanto,
\Eq{*}{
K+\varphi^T(t,x)S_0 
	&=K+\Bigg(\sum_{k=0}^{\infty} \frac{1}{2^k}\varphi\big(2d_{\Z}(2^kt)x\big)\Bigg)S_0 \\
  &\subseteq \cl\Bigg(\bigcup_{\ell=0}^{\infty} \sum_{k=0}^{\ell}
								\frac{1}{2^k}S\big(2d_{\Z}(2^kt)x\big)\Bigg)
	= S^T(t,x).
}
Con lo cual se termina la demostración.

En el caso de que $\varphi(0)=0$, la primera igualdad en \eq{Tak+} es inmediata, la segunda
igualdad es consecuencia de \eq{Tak1}. 
\end{proof}

\Cor{Tak}{
Sea $X$ un espacio normado, $D\subseteq X$ un conjunto estrellado, $S_0\subseteq Y$ un 
conjunto convexo que contiene a $0\in Y$, $K\subseteq Y$ un cono convexo, 
y $\alpha>0$. Definamos $S:D\to\P_0(Y)$ por
$S(x):=K+\|x\|^\alpha S_0$. Entonces
\Eq{Tak1+}{
  S^T(t,x)=\cl\big(K+T_\alpha(t)\|x\|^\alpha S_0\big)
  \qquad(t\in\R,\,x\in D),
}
donde $T_\alpha:\R\to\R$ es la función de Takagi de orden $\alpha$
y es definida por
\Eq{Tak2+}{
  T_\alpha(t):=\sum_{n=0}^{\infty}2^{\alpha-n}(d_{\Z}(2^nt))^\alpha
  \qquad(t\in\R).
}}

\begin{proof}
Aplicando la \prp{Tak} con la función $\varphi$
definida mediante la f\'ormula
 $\varphi(x):=\|x\|^\alpha$, se observa que para
todo $t\in\R$ y para todo $x\in D$
\Eq{*}{
  \varphi^T(t,x)=\sum_{n=0}^{\infty}\frac{1}{2^n}\varphi\big(2d_{\Z}(2^nt)x\big)
  =\sum_{n=0}^{\infty}2^{\alpha-n} \big(d_{\Z}(2^nt)\big)^\alpha\|x\|^\alpha
  = T_\alpha(t)\|x\|^\alpha.
}
Por lo tanto, \eq{Tak1+} es consecuencia de \eq{Tak1}. 
\end{proof}

\Rem{Tak}{
Un caso particular importante es cuando $\alpha=1$, entonces $T_1=2T$, donde $T$ es la función de Takagi 
definida por \eq{Tak} en el Capítulo \ref{chapPrevio}. 
En el caso $\alpha=2$ un argumento interesante nos da una forma cerrada para $T_2$. 
Observe que $T_\alpha$ (para cualquier $\alpha>0$) satisface la ecuación funcional
\Eq{TT}{
  T_\alpha(t)=2^{\alpha}\big(d_{\Z}(t)\big)^\alpha+\frac12 T_\alpha(2t) \qquad(t\in\R).
}
Por el teorema de punto fijo de Banach, esta ecuación funcional tiene solución única en el espacio 
de Banach de las funciones reales, acotadas sobre la recta real (equipado con la norma
del supremo). Asi $T_\alpha$ es la única solución a \eq{TT}. Por otro lado, para $\alpha=2$ 
se puede verificar que la función periódica $T_2^*$ definida en $[0,1]$ por 
$T_2^*(t)=4t(1-t)$ también es solución de \eq{TT}, así, debe ocurrir que 
$T_2(t)=4t(1-t)$ para $t\in [0,1]$. Para mayores detalles, puede revisar \cite{MakPal13b}.}

\Cor{Tk}{
Sea $X$ un espacio normado, $D\subseteq X$ un conjunto estrellado, $S_0\subseteq Y$ un 
conjunto convexo que contiene a $0\in Y$, $K\subseteq Y$ un cono convexo. 
Definamos $S:D\to\P_0(Y)$ por $S(x):=K+S_0$. Entonces
\Eq{Tk}{
  S^T(t,x)=\cl\big(K+2S_0\big)  \qquad(t\in\R,\,x\in D).
}
}
\begin{proof} 
Aplicaremos la \prp{Tak} a la función constante $\varphi\equiv 1$.
Por lo tanto \eq{Tak2} nos dá $\varphi^T\equiv 2$, luego \eq{Tak1} es equivalente a lo
que queremos demostrar.
\end{proof}
