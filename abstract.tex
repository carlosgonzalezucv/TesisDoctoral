\chapter*{Abstract}
\markright{ABSTRACT}
In general, every midconvex function is not necessarly convex. However in the class of 
continuous functions, it is well know that a function is midconvex and  continuous 
if and only if it is convex \cite{Kuc09}. The Bernstein-Doetsch Theorem published 100 years ago, gives some regularity 
conditions weaker than continuity, for a real valued  midconvex function defined on a 
convex subset of the real line to be convex, and hence continuous. 
This theorem was stablished  in 1915, and it is one of the most important and classical 
results obtained in convexity theory. Since then, it has been generalized in many 
different ways and by many authors. 

The main theorem of this paper generalizes some of the results  obtained by Ng, Nikodem,
Averna, Cardinali, Papalini, and others, related to strongly and approximately convex
functions, see for instance \cite{MakPal10b,MakPal13b,Nik89} and the references therein. 
Here, we consider a set valued map $F(\cdot)$ defined on a convex subset $D$ of a 
topological Hausdorff space X, which takes his values at the class of nonempty subsets 
of another topological Hausdorff space Y and satisfies the following Jensen type inclusion:
$$
\frac{F(x)+F(y)}2 + A(x-y) \subseteq \overline{\bigg(F\bigg(\frac{x+y}2\bigg)+B(x-y)\bigg)},
\qquad
(x,y\in D)
$$
where, $A(\cdot)$ and $B(\cdot)$ are set valued maps defined on $D-D$ and for all $x\in D-D$,
$0\in A(x)\cap B(x)$. Now, under some regularity assumptions  
one can prove that the set valued map $F(\cdot)$ satisfies for $x,y\in D$ and $t\in[0,1]$
the following convexity type inclusion:
$$
tF(x)+(1-t)F(y) + A^T(t,x-y) \subseteq \overline{\bigg(F(tx+(1-t)y)+B^T(t,x-y)\bigg)}.
$$
$A^T$ denotes the Takagi transformation associated to the set valued map $A$, 
and it is defined for $t\in[0,1]$ and $x\in D- D$ by the following formula:
$$
A^T(t,x) = \overline{\bigcup_{n=0}^{\infty}\sum_{k=0}^n{\frac1{2^k}A(2\mbox{dist}(2^kt,\Z)x)}}.
$$


{\bf Key words:} K-Jensen convexity/concavity, set-valued map, Takagi transformation, approximate convexity,
strong convexity.
