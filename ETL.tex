\chapter{Espacios Topológicos Lineales.}
\setcounter{theorem}{0}
En este capítulo se hará un breve desarrollo de algunos conceptos 
de topología que serán usados a lo largo de todo el trabajo.
Los resultados que mencionaremos en este capítulo se encuentran
desarrollados en el libro del profesor W. Rudin\cite{Rud91}.

\section{Espacios topológicos.}
Sea $X$ un conjunto, se denota por $\P(X)$ al conjunto potencia de $X$. 
\Defi{espTop}{
Un \textbf{espacio topológico} consiste en una dupla $(X,\Tau )$,
donde $X$ es un conjunto y $\Tau $ es un subconjunto de 
$\P(X)$ que cumple con las siguientes propiedades:
\begin{enumerate}[i.]
\item $\emptyset \in \Tau $ y $X\in\Tau $.
\item Si $\{U_\alpha\}_{\alpha\in\Delta}\subseteq\Tau $
es una colección de elementos de $\Tau $, entonces:
\Eq{*}{
\bigcup_{\alpha\in\Delta} U_\alpha \in \Tau.
}
\item Si 
$\{U_k\}_{k=1}^n\subseteq\Tau $
es una colección finita de elementos de $\Tau $, entonces:
\Eq{*}{
\bigcap_{k=1}^n U_k \in \Tau.
}
\end{enumerate}
}
Cuando $(X,\Tau)$ es un espacio topológico, los elementos de $\Tau $ 
se llaman \textbf{conjuntos abiertos}. En lo que sigue a continuación,
el par $(X,\Tau )$ denotará un espacio topológico. Cuando no haya lugar
a confusión se omitirá la topología.

\Defi{Haus}{
Decimos que el espacio topológico $(X,\Tau)$, es un espacio
de Hausdorff, si para cada par de puntos diferentes, $x,y\in X$
existen abiertos disjuntos $U_x, U_y\in\Tau$ tales que
$x\in U_x$ y $y\in U_y$.
}

\Defi{basetop}{
Dado un espacio topológico $(X,\Tau  )$, y un subconjunto 
$\mathcal{B}\subseteq\Tau  $, se dice que $\mathcal{B}$ es una base 
para la topología $\Tau  $ si todo conjunto abierto $U \in\Tau $ es la 
unión de elementos de $\mathcal{B}$. De manera equivalente,
$\mathcal{B}$ es una base para $\Tau$ si y sólo si, para todo punto
$p$ perteneciente a cualquier abierto $O\in\Tau$ existe 
$B\in\mathcal{B}$, tal que $p\in B\subseteq O$.
}

\Defi{baseLocal}{
Dado un espacio topológico $(X,\Tau)$ y un punto $p\in X$.
Una base local en $p$, consiste en una colección de abiertos
$\mathcal{B}_p$, tal que, cualquier entorno abierto 
$U\subseteq X$ de $p$, contiene al menos un miembro de 
$\mathcal{B}_p$.
}

\Defi{acumPoints}{
Dado un conjunto $A\subseteq X$ y un punto $p\in X$, se dice que $p$
es un \textbf{punto de acumulación} del conjunto $A$, si para todo abierto $U\in\Tau $ que contiene a $p$, se tiene que 
\Eq{acumP}{
A \cap (U\setminus \{p\}) \neq \emptyset
}
}
\Defi{Closed}{
Un conjunto $V\subseteq X$ es \textbf{cerrado}, si el conjunto $X\setminus V$ es
abierto.
}
Los conjuntos cerrados también se pueden caracterizar de la siguiente
manera:
\Prp{acumP}{
Un subconjunto $V\subseteq X$ es cerrado, si y sólo si $V$ contiene 
todos sus puntos de acumulación.
}

\begin{proof}
Sea $V\subseteq X$ un conjunto cerrado y sea $p\in X$ un punto de acumulación
de $V$. Supongamos que $p\notin V$. Como $V$ es cerrado, entonces por definición
su complemento $U=X\setminus V$ es abierto y como $p\in U$, existe
un elemento básico $B\in\mathcal{B}$, tal que
$p\in B\subseteq U$. En consecuencia, $B\cap V=\emptyset$,
pero esto contradice el hecho de que $p$ es un punto de acumulación de $V$.
Luego, es falso suponer que $p\notin V$ y así, queda establecido que el
conjunto $V$ posee a todos sus puntos de acumulación.

Supongamos ahora que el conjunto $V$ posee a todos sus puntos de acumulación y
sea $U=X\setminus V$. Veamos que $U$ es abierto. Sea $p\in U$, por definición
se tiene que $p$ no es punto de acumulación de $V$, en consecuencia,
existe un entorno abierto $B\subseteq X$ de $p$ tal que $V\cap (B\setminus\{p\})=\emptyset$.
Por lo tanto, $p\in B\subseteq U$ y así, $U$ es abierto.
\end{proof}

Es posible definir una topología en términos de conjuntos cerrados.
\Thm{topClosed}{
Sea $X$ un espacio topológico, la clase de conjuntos cerrados en $X$,
posee las siguientes propiedades:
\begin{enumerate}[i.]
	\item $X$ y $\emptyset$ son conjuntos cerrados.
	\item La intersección arbitraria de conjuntos cerrados, es cerrado.
	\item La unión finita de conjuntos cerrados, es cerrado.
\end{enumerate}
}
\Defi{cl}{
Sea $A\subseteq X$. La \textbf{clausura} de $A$, se denota por
$\cl(A)$ y se define como la intersección de todos los conjuntos 
cerrados que contienen al conjunto $A$. Esto es,
\Eq{*}{
\cl(A) := \bigcap\{V\subseteq X: A\subseteq V \quad\mbox{y}
\quad V\quad\mbox{es cerrado}\}.
}
}
Se puede ver que $\cl(A)$ es un conjunto cerrado ya que es la 
intersección de conjuntos cerrados. Además, $\cl(A)$
 es el conjunto cerrado más pequeño que contiene al conjunto $A$. 
Es decir, si $V$ es un conjunto cerrado que contiene al conjunto $A$, 
entonces:
\Eq{*}{
A\subseteq\cl(A) \subseteq V.
}
Además, $V\subseteq X$ es un conjunto cerrado si y sólo si $V=\cl(V)$.
Todo esto se resume en la siguiente proposición.
\Prp{clcl}{
Sea $A\subseteq X$. Entonces,
\begin{enumerate}[i.]
	\item $\cl(A)$ es cerrado y  $A\subseteq\cl(A)$ .
	\item Si $V$ es un conjunto cerrado que contiene al conjunto $A$, 
	entonces $\cl(A)\subseteq V.$
	\item $A$ es cerrado si y sólo si $A=\cl(A)$.
\end{enumerate}
}

\begin{proof} Sea $A\subseteq X$.
\begin{enumerate}[i.]
	\item Por definición, el conjunto $\cl(A)$ es la intersección arbitraria
	de conjuntos cerrados y por el \thm{topClosed} se tiene que es un conjunto cerrado.
	Además, por definición $A\subseteq\cl(A)$.
	\item Es consecuencia directa de la definición de clausura.
	\item Supongamos que $A$ es cerrado. Entonces, por el item II 
	se tiene que $\cl(A)\subseteq A$ y en consecuencia $\cl(A)=A$.
	Recíprocamente, si $A=\cl(A)$, entonces, evidentemente  $A$ es cerrado, 
	lo que completa la demostración.
\end{enumerate}
\end{proof}

\Prp{cl2}{
Sea $A\subseteq X$. La clausura de $A$ es la unión de $A$ con 
su conjunto de puntos de acumulación $A'$. Esto es,
$\cl(A) = A \cup A'.$
}

\begin{proof}
Veamos que $\cl(A)\subseteq A\cup A'$. Sea $p\in\cl(A)$, y supongamos
que $p\notin A\cup A'$, esto es $p\notin A$ y $p\notin A'$.
Como $p\notin A'$, por definición existe un entorno abierto $B\subseteq X$
de $p$, tal que $A\cap(B\setminus\{p\})=\emptyset$, por lo tanto 
$B\setminus\{p\}\subseteq X\setminus A$ y en consecuencia $B\subseteq X\setminus A$.
Pero esto último equivale a $A\subseteq X\setminus B,$ y siendo $B$
un conjunto abierto, se tiene que $X\setminus B$ es cerrado. Como además,
este conjunto contiene al conjunto $A$, entonces por la definición de clausura
se tiene que $\cl(A)\subseteq X\setminus B$ y por lo tanto
$B\subseteq X\setminus \cl(A).$ Como $p\in B$, debe ocurrir entonces
que $p\notin\cl(A)$, pero esto es una contradicción. Luego, es falso 
suponer que $p\notin A\cup A'$ y así $\cl(A)\subseteq A\cup A'$.

Veamos ahora que $A\cup A'\subseteq\cl(A)$. Sea $p\in A\cup A'$. 
Si $p\in A$, entonces $p\in\cl(A)$ y en este caso la demostración es 
directa. Si por el contrario, $p\in A'$, entonces supongamos que 
$p\notin\cl(A)$. Como $\cl(A)$ es un conjunto cerrado entonces su 
complemento es abierto y por lo tanto existe un entorno abierto $B$
de $p$ tal que $p\in B\subseteq X\setminus\cl(A)$. En consecuencia,
$B\cap\cl(A)=\emptyset$ y como $A\subseteq\cl(A)$ entonces 
$B\cap A=\emptyset$, pero esto es una contradicción ya que $p$ es punto de
acumulación del conjunto $A$. Luego, es falso suponer que $p\notin\cl(A)$
y finalmente se tiene que $A\cup A'\subseteq\cl(A)$. Esto completa la demostración.
\end{proof}

\Prp{Kura}{Sean $A,B\subseteq X$, el operador $\cl(\cdot)$ cumple
las siguientes propiedades
\begin{multicols}{2}
\begin{enumerate}[i.]
	%\item $\cl(\emptyset) = \emptyset.$
	%\item $A\subseteq\cl(A).$
	\item $\cl(A\cup B) = \cl(A)\cup\cl(B).$
	\item $\cl(\cl(A)) = \cl(A).$
\end{enumerate}
\end{multicols}
}
\begin{proof}
La demostración de esta proposición es consecuencia directa
de la \prp{clcl} y de la \prp{cl2}.
\end{proof}
%%%%%%%%%%%%%%%%%%%%%%%%%%%%%%%%%%%%%%%%%%%%%%%%%%%%%%%%%
%%%%%%%%%%%%%%%%%%%%%%%%%%%%%%%%%%%%%%%%%%%%%%%%%%%%%%%%%%%
\section{Espacios vectoriales.}
\setcounter{theorem}{0}
Denótese por $\R$ al cuerpo de los números reales y por 
$\C$ al cuerpo de los números complejos. Por ahora,
se usará $\Phi$ para denotar tanto a $\R$ como a $\C$.
Un escalar, es un elemento del cuerpo $\Phi$.
\Defi{VecSp}{
Un \textit{espacio vectorial} sobre $\Phi$ es un conjunto
$X$, junto con dos operaciones, suma y producto por un escalar
las cuales satisfacen las siguientes propiedades:
\begin{enumerate}[i.]
	\item A cada par de vectores $x,y\in X$ les corresponde 
	un vector $x+y\in X$, de manera que 
	\Eq{*}{
	x+y=y+x\qquad\mbox{y}\qquad x+(y+z) = (x+y)+z.
	}
	\item $X$ contiene un único vector $0$ el cual se llamará 
	con frecuencia el origen de $X$, tal que $x+0=x,$ para 
	todo $x\in X$. Además a cada vector $x\in X$ le corresponde
	un vector $-x\in X$ tal que $x+(-x)=0$.
	\item A cada par $(\alpha,x)$ con $\alpha\in\Phi$, y 
	$x\in X$ le corresponde un vector $\alpha x\in X,$
	de manera que
	\Eq{*}{
	1x=x,\qquad \alpha(\beta x) = (\alpha\beta)x,
	}
	y además se cumplen las siguientes leyes distributivas
	\Eq{*}{
	\alpha(x+y) = \alpha x+\alpha y, \qquad
	(\alpha + \beta) x = \alpha x + \beta x.
	}
\end{enumerate}
}
\Rem{vect1}{
Un espacio vectorial real, es un espacio vectorial donde 
$\Phi = \R,$ y un espacio vectorial complejo, es un espacio
vectorial donde $\Phi=\C$. Cualquier afirmación sobre
un espacio vectorial, donde no se especifique explícitamente
el cuerpo, será válida para ambos casos.
}
\Defi{sumSet}{
Sea $X$ un espacio vectorial y sean $A,B\subseteq X$, $x\in X$ y
$\lambda\in\Phi$, entonces
\Eq{*}{
x+A &= \{x+a : a\in A\}, \\
A+B &= \{a+b : a\in A, b\in B\}, \\
\lambda A &= \{\lambda a: a\in A\}\\
x-A &= \{x-a : a\in A\}.
} 
}
\Rem{A+A}{
Con estas operaciones así definidas, puede ocurrir que 
$A+A\neq 2A.$
}
\Defi{subsp}{
Un subconjunto $W\subseteq X$ es llamado subespacio de $X$
si el mismo es un espacio vectorial (con respecto a las mismas
operaciones). 
}
\Prp{subsp}{
Sea $X$ un espacio vectorial y $W\subseteq X$. $W$ es un 
subespacio de $X$ si y sólo si, $0\in W$ y además,
para $\alpha,\beta\in\Phi$ se tiene que
\Eq{*}{
\alpha W + \beta W \subseteq W.
}
}
\Defi{cvxSet}{
Un conjunto $D\subseteq X$ es convexo, si para todo $t\in[0,1]$
se tiene que 
\Eq{*}{
tD + (1-t)D\subseteq D.
}
En otras palabras, el conjunto $D$ es convexo si para 
cualesquiera par de puntos $x,y\in D$ se tiene que
\Eq{*}{
[x,y]:= \{ty+(1-t)x : t\in [0,1]\}\subseteq D.
} 
}


\Defi{starSet}{
Sea $H\subseteq X$ y sea $p\in H$. 
Se dice que el conjunto $H$ es estrellado con respecto al punto $p$
si para todo $h\in H$ se tiene que el segmento $[h,p]\subseteq H$.
}
Existe una conexión entre los conjuntos estrellados y los conjuntos 
convexos. La siguiente proposición muestra este hecho
\Prp{CvxStr}{
El conjunto $H\subseteq X$ es convexo si y sólo si $H$ es 
estrellado con respecto a cada uno de sus puntos.
}
\Prp{D-DStr}{
Sea $H\subseteq X$ un conjunto convexo no-vacío. 
Entonces, el conjunto $H-H$ es estrellado con respecto al origen.
}
\begin{proof}
Sea $p\in H-H$. Por definición existen $h_1,h_2\in H$ tales que 
$p=h_1-h_2$. Luego, si $t\in[0,1]$, y $h\in H$ es un elemento cualquiera
de $H$ entonces 
$$
tp = th_1-th_2= th_1-th_2+(1-t)(h-h)=th_1+(1-t)h-(th_2+(1-t)h).
$$ 
Como $H$ es un conjunto
convexo, se tiene que $tp\in H-H$ para todo $t\in[0,1]$. Por lo tanto
el conjunto $H-H$ es estrellado con respecto al origen.
\end{proof}

\Defi{BlSet}{
Un conjunto $D\subseteq X$ se dice que es balanceado, si para todo $\alpha\in\Phi,$
 con $|\alpha|\leq 1$, se tiene que $\alpha D\subseteq D,$
}
%\Defi{dimX}{
%El espacio vectorial $X,$ tiene dimensión $n$ $(\dim X=n),$
%si $X$ posee una base $\B=\{u_1,\ldots,u_n\}$, de 
%$n$ vectores linealmente independientes. Esto quiere
%decir que todo $x\in X$, tiene una representación única
%de la forma
%\Eq{*}{
%x=\alpha_1u_1+\cdots+\alpha_nu_n, \quad \alpha_i\in\Phi,
%\quad i=1,\ldots,n.
%} 
%}
%\Rem{dim}{
%Si $\dim X = n$ para algún número natural $n$, se dice que 
%$X$ es de dimensión finita. Si $X=\{0\}$, entonces
%$\dim X = 0$.
%}


%%%%%%%%%%%%%%%%%%%%%%%%%%%%%%%%%%%%%%%%%%%%%%%%%%%%%%%%%%%
%%%%%%%%%%%%%%%%%%%%%%%%%%%%%%%%%%%%%%%%%%%%%%%%%%%%%%%%%%%

\section{Espacios topológicos lineales.}
\setcounter{theorem}{0}

Supongamos que $\Tau$ es una topología para un espacio vectorial $X$
que cumple con lo siguiente:
\begin{enumerate}[i.]
	\item Todo punto de $X$ es un conjunto cerrado.
	\item Las operaciones del espacio vectorial, son continuas 
	con respecto a $\Tau$.
\end{enumerate}

Bajo estas condiciones, se dice que $\Tau$ es una topología vectorial
en $X$ y $X$ es un \textit{espacio topológico lineal}.


\Defi{TaMl}{
Sea $X$ un espacio topológico lineal. Para cada $a\in X$
y para cada $\lambda\neq0$, se definen los operadores de 
traslación y multiplicación, $T_a(x)$ y $M_\lambda(x)$
respectivamente, de la siguiente manera
\Eq{TaMl}{
T_a(x) = x + a \qquad \mbox{y} \qquad
M_\lambda(x) = \lambda x, \quad (x\in X).
}
}

\Prp{hom}{
$T_a$ y $M_\lambda$ son homeomorfismos de $X$ en $X$.
}
\begin{proof}
Los axiomas de espacio vectorial, implican que estos
operadores son biyectivos, y además que sus inversas 
son $T_{-a}$ y $M_{1/\lambda}$, respectivamente. 
El hecho de que las operaciones, suma y producto por un 
escalar sean continuas, implica que tanto $T_a$ como 
$M_\lambda$, así como sus inversas sean continuas.
\end{proof}

Como consecuencia de esta proposición, se tiene que 
toda topología vectorial $\Tau$ es invariante. Es decir, 
un conjunto $E\subseteq X$ es abierto, si y sólo si, para todo $a\in X,$
el conjunto $a+E$ es abierto. Así, $\Tau$ está completamente
determinada por cualquier base local.

En este contexto, el término base local, se refiere a la base 
local en $0\in X$.
Se denotará por $\U(X)$ a dicha base.

\Defi{bddSet}{
Un subconjunto $H\subseteq X$ es acotado, si para todo entorno 
abierto $U$ de $0\in X,$ existe un número $s>0$ tal que 
$H\subseteq tU$ para todo $t>s$.
}
\Prp{bddtH}{
Sea $H\subseteq X$ un conjunto acotado. Para cada $\alpha\in[0,1]$
el conjunto $\alpha H$ es acotado y además $\bigcap_{\alpha\in[0,1]} \alpha H$
es acotado.
}
\begin{proof}
Sea $U\subseteq X$ un conjunto abierto tal que $0\in U$ y
sea $\alpha\in[0,1]$. 
Como $H$ es acotado, existe $s_\alpha>0$ tal que si $t>s_\alpha$, entonces
$H\subseteq t\big(\frac1\alpha U\big)$. En consecuencia, 
$\alpha H\subseteq \alpha t\big(\frac1\alpha U\big)=tU$ siempre y cuando
$t>s_\alpha$. Como $U$ es arbitrario se tiene que el conjunto
$\alpha H$ es acotado. 

Para ver que la intersección de los conjuntos de la forma $\alpha H$
con $\alpha\in[0,1]$ es un conjunto acotado, consideremos un abierto balanceado arbitrario
$U\subseteq X$ tal que $0\in U$. Como el conjunto $H$ es acotado, existe $s>0$
tal que si $t>s$ entonces, $H\subseteq tU$. Ahora bien, para $\alpha\in[0,1]$
se tiene que $\alpha H\subseteq \alpha tU$ y como 
$U$ es balanceado entonces $tU$ también lo es, luego,
como $\alpha\leq1$, entonces $\alpha tU\subseteq tU$. 
Por lo tanto, para $t>s$ se tiene lo siguiente
$$
\bigcap_{\alpha\in[0,1]}\alpha H\subseteq\bigcap_{\alpha\in[0,1]}\alpha tU
\subseteq\bigcap_{\alpha\in[0,1]}tU=tU.
$$
Como el abierto $U$ es arbitrario, entonces queda demostrado que el conjunto 
$\bigcap_{\alpha\in[0,1]} \alpha H$ es acotado.

\end{proof}

\Defi{symm}{
Sea $(X,\Tau)$ un espacio topológico lineal. Se dice 
que el abierto $U\subseteq X$, es simétrico 
si y sólo si $U=-U.$ 
}
\Prp{symm}{
Si $W\subseteq X$ es un entorno abierto de $0\in X$, entonces, 
existe un entorno abierto y simétrico $U\subseteq X$, de 
$0$ tal que $U+U\subseteq W.$
}
\begin{proof}
Sea $W\subseteq X$ un entorno abierto de $0$. Como la operación 
suma es continua y $0=0+0$, entonces, existen entornos abiertos 
$V_1,V_2$ de $0$, tales que
$V_1+V_2\subseteq W.$ 
Definamos $U:= V_1\cap V_2\cap (-V_1)\cap (-V_2).$ Es claro que 
$U$ es un abierto simétrico, y que además
$U+U\subseteq W.$
\end{proof}
\Prp{bddPrp}{
La familia de subconjuntos acotados es cerrada bajo la suma
y el producto por escalares positivos.
}
\begin{proof}
Sean $H_1, H_2\subseteq X$ conjuntos acotados. Veamos que la suma
$H:=H_1+H_2$ es un conjunto acotado. Sea $V\subseteq X$ un entorno abierto de $0\in X$
 y escojamos $U\in\Tau$ tal que $0\in U$ y $U+U\subseteq V$.
Como $H_1$ y $H_2$ son acotados, existen escalares positivos
$t_1$ y $t_2$ tales que
\Eq{*}{
H_1 &\subseteq s_1 U \qquad\mbox{si}\quad s_1>t_1, \\
H_2 &\subseteq s_2 U \qquad\mbox{si}\quad s_2>t_2.
}
Ahora bien, si $s>\max(t_1,t_2)$ entonces,
\Eq{*}{
H=H_1+H_2\subseteq sU + sU = s(U+U) \subseteq sV.
}
Por lo tanto, el conjunto $H$ es acotado. De manera similar
se demuestra que $tH_1$ es acotado para cualquier escalar
positivo $t$.
\end{proof}
\Thm{sep1}{
Suponga que $K$ y $C$ son subconjuntos de un espacio topológico
lineal $X$, $K$ es compacto, $C$ es cerrado y que 
$K\cap C=\emptyset$. Entonces, existe un entorno $V$ de $0\in X$ 
tal que 
\Eq{*}{
(K+V)\cap(C+V)=\emptyset.
}
}
\begin{proof}
Si $K=\emptyset$, entonces $K+V=\emptyset$ y la conclusión del 
teorema es directa. Por lo tanto, asuma que $K\neq\emptyset$.
Sea $x\in K$. Como $K$ y $C$ son disjuntos, se tiene que $x$
no está en $C$, por la \prp{symm}, existe un abierto simétrico 
$V_x\subseteq X$ tal que 
\Eq{*}{
(x+V_x+V_x+V_x)\cap C = \emptyset.
}
Como $V_x$ es simétrico, la condición anterior equivale a
\Eq{cond0}{
(x+V_x+V_x)\cap(C+V_x) = \emptyset.
}
Por otra parte, $K$ es compacto, es decir que admite un 
cubrimiento finito
\Eq{*}{
K\subseteq \bigcup_{i=1}^n(x_i + V_{x_i}).
}
Ahora bien, al considerar $V:= V_{x_1}\cap\cdots\cap V_{x_n},$
se tiene que 
\Eq{*}{
K+V \subseteq \bigcup_{i=1}^n(x_i + V_{x_i}+V)
    \subseteq \bigcup_{i=1}^n(x_i + V_{x_i}+V_{x_i}),
}
y por \eq{cond0}, ningún término en esta última unión
intersecta a $C+V$. Finalmente,
\Eq{*}{
(K+V)\bigcap(C+V) =\emptyset.
}
\end{proof}
Como consecuencia del resultado anterior, se tienen los siguientes teoremas
\Thm{clbase}{
Para todo elemento $B_1\in\U(X)$, existe otro elemento
$B_2\in\U(X)$, tal que
\Eq{*}{
\cl(B_2)\subseteq B_1.
}
}
\begin{proof}
Sea $B_1\in\U(X)$, definamos los conjuntos 
$C:=X\setminus B_1$ y $K:=\{0\}$. Es claro que $C$ es
cerrado, pues, es el complemento de un elemento en $\U(X)$,
y que además, $K$ es compacto. Por otra parte, 
$0\notin C$, es decir, $K\cap C=\emptyset$. Por el 
\thm{sep1}, existe un abierto $W\in\U(X)$, tal que
$W \cap (W+C)=\emptyset$. Ahora bien, esta última condición
implica que $W\cap C=\emptyset$, lo que se reduce a
$W\subseteq B_1$.

Aplicando la \prp{symm} al abierto $W$, se tiene que existe
un abierto simétrico, $B_2\in\U(X)$, tal que
$B_2+B_2\subseteq W$, por lo tanto, se tiene la siguiente
cadena de inclusiones
\Eq{*}{
\cl(B_2)\subseteq B_2+B_2 \subseteq W\subseteq B_1,
} 
la cual, finaliza la demostración.
\end{proof}
\Thm{HausTop}{
Todo espacio topológico lineal, es un espacio de Hausdorff.
}
\begin{proof}
Basta considerar el hecho de que los conjuntos unipuntuales
en un espacio topológico lineal, son cerrados y aplicar
el \thm{sep1}.
\end{proof}
\Thm{f1}{
Sea $X$ un espacio topológico lineal.
\begin{enumerate}[i.]
	\item Si $A\subseteq X$ entonces, 
	$\cl(A)=\bigcap_{V\in\U(X)}(A+V)$, 
	\item Si $A\subseteq X$ y $B\subseteq X$, entonces, 
	$\cl(A) +\cl(B) \subseteq \cl(A+B) = \cl(\cl(A)+\cl(B)).$
	\item Si $A\subseteq X$ es un conjunto acotado, entonces
	$\cl(A)$ también lo será.
\end{enumerate}
}
\begin{proof} Consideremos el espacio topológico lineal
$X$.
\begin{enumerate}[i.]
	\item Sea $A\subseteq X$, veamos que $\cl(A)=\bigcap(A+V)$, 
	donde $V$ recorre todos los entornos abiertos de $0$.
	Ahora bien, $x\in \cl(A)$ si y sólo si, 
	$(x+V)\cap A \neq\emptyset$ para todo entorno abierto 
	$V\in\U(X)$. 
	Pero esta condición se cumple si y sólo si 
	$x\in A+(-V)$, para todo
	abierto $V\in\U(X).$ Además, $V$ es un 
	entorno abierto de $0$, si y sólo si, $-V$ lo es, 
	por lo tanto, 
	\Eq{*}{
	\cl(A) = \bigcap_{V\in\U(X)}(A+V).
	}
	\item Sean $U,V\in\U(X)$ tales que $V+V\subseteq U$. 
	Sean $a\in\cl(A)$ y $b\in\cl(B)$, evidentemente,
	$a+b\in\cl(A)+\cl(B)$, y además,
	\Eq{*}{
	a+b\in\cl(A)+\cl(B)\subseteq A+V+B+V\subseteq A+B+U.
	}
	Como $U$ es arbitrario, se tiene lo siguiente
	\Eq{*}{
	a+b\in\bigcap_{U\in\U(X)}(A+B+U)=\cl(A+B).
	}
	Esto es,
	\Eq{*}{
	\cl(A)+\cl(B)\subseteq\cl(A+B).
	}
	\item Sea $A\subseteq X$ un conjunto acotado. Veamos que
	$\cl(A)$ también, es un conjunto acotado. 
	Considere $U\in \U(X)$, por el \thm{clbase} existe un 
	abierto $W\in\U(X)$ tal que, $\cl(W)\subseteq U$. Como
	$A$ es acotado, existe 
	un número real $s>0$, tal que $A\subseteq tW$,
	para todo número real $t>s$. Por lo tanto
	\Eq{*}{
	\cl(A)\subseteq t\cl(W)\subseteq tU,\qquad (t>s),
	}
	es decir, que $\cl(A)$ es acotado.
\end{enumerate}
\end{proof}

\Prp{sumSets}{
Si $\cl(A)$ es un conjunto compacto entonces
$\cl(A+B)=\cl(A)+\cl(B).$ Esto equivale
a decir que la suma de un conjunto compacto 
con un conjunto cerrado resulta ser un conjunto
cerrado.
}

%%%%%%%%%%%%%%%%%%%%%%%%%%%%%%%%%%%%%%%%%%%%%%%%%%%%%%%%
%%%%%%%%%%%%%%%%%%%%%%%%%%%%%%%%%%%%%%%%%%%%%%%%%%%%%%%%
\section{Conos convexos.}
\setcounter{theorem}{0}
A menos que se especifique otra cosa, $X$ denotará
un espacio topológico lineal.
\Defi{cono}{
El conjunto $K\subseteq X$
es un cono convexo, si 
$K+K\subseteq K, $ y $tK\subseteq K,$ para todo escalar 
$t>0$.
}
%\Defi{K<}{
%Dado un cono 
%$K\subseteq X$, se define la relación $\leq_K$,
%para $x,y\in X$, de la siguiente manera
%\Eq{<K}{
%x\leq_K y \iff y-x\in K\iff y\in x+K.
%}
%}
%De aquí en adelante, $K\subseteq X$ representa
%un cono convexo a menos que se especifique otra cosa.
%\Prp{<trans}{
%La relación $\leq_K$ es antisimétrica y transitiva.
%}
%\begin{proof}
%Veamos que $\leq_K$ es antisimétrica. Sean $x,y\in X$, 
%tales que
%$x\leq_K y$. Por definición esto significa que,
%$y-x\in K$. Pero, esto es equivalente a
%$-x-(-y) \in K$, por lo tanto, $-y\leq_K -x$.
%Veamos ahora que $\leq_K$ es transitiva, para ello,
%sean $a,b,c\in X$ tales que $a\leq_K b$ y $b\leq_K c$.
%Por definición, se tiene que $b-a\in K$ y que $c-b\in K$.
%Ahora bien, usando el hecho de que $K$ es cerrado bajo la 
%suma se sigue que 
%\Eq{*}{
%c-a = (c-b)+(b-a)\in K+K\subseteq K.
%}
%Esto es, precisamente $a\leq_K c$.
%\end{proof}
%\Prp{relOrd}{
%Si $0\in K$, entonces, la relación $\leq_K$,
%es reflexiva.
%}
%\begin{proof}
%La demostración es directa, pues $x-x=0\in K$ para todo
%$x\in X$. 
%\end{proof}
\Defi{Klbdd}{
Se dice que un conjunto $S\subseteq X$ es
$K$-acotado inferiormente, si existe un conjunto acotado
$H,$ tal que, $S\subseteq H+K$.
}
\Prp{KlbUn}{
La unión finita de conjuntos $K$-acotados inferiormente,
es de nuevo un conjunto $K$-acotado inferiormente.
}
\begin{proof}
Basta con probar que la unión de dos conjuntos
$K$-acotados inferiormente es de nuevo un conjunto 
$K$-acotado inferiormente. Sean $S_1,S_2\subseteq X$,
dos conjuntos tales que existen conjuntos acotados 
$H_1,H_2\subseteq X$ que satisfacen 
\Eq{*}{
S_1&\subseteq H_1+K \qquad\mbox{y}\\
S_2&\subseteq H_2+K.
}
Ahora bien, el resultado es consecuencia de la siguiente
cadena de inclusiones
\Eq{*}{
S_1\cup S_2 &\subseteq (H_1+K)\cup(H_2+K) 
								= \bigcup_{h\in H_1}(h+K)\cup
								  \bigcup_{h\in H_2}(h+K) \\
								&= \bigcup_{h\in H_1\cup H_2}(h+K)
								= (H_1\cup H_2) + K.									
}
\end{proof}
\Prp{Klbbop}{
La familia de subconjuntos $K$-acotados inferiormente 
es cerrada bajo la suma y el producto por escalares
positivos.
}
\begin{proof}
Sean $S_1, S_2\subseteq X$, dos conjuntos, $K$-acotados 
inferiormente.
Por definición, existen conjuntos acotados 
$H_1,H_2\subseteq X$ tales que 
\Eq{I0}{
S_1&\subseteq H_1+K \qquad\mbox{y}\qquad
S_2&\subseteq H_2+K. 
}
Por lo tanto
\Eq{*}{
S=S_1+S_2 \subseteq H_1+H_2+K+K \subseteq H_1+H_2+K,
}
pero por la \prp{bddPrp}, el conjunto $H:=H_1+H_2$
es acotado. Luego, $S\subseteq H+K$, es decir, que 
el conjunto $S$ es $K$-acotado inferiormente.

Para ver que $tS_1$ es $K$-acotado inferiormente, basta
con multiplicar la primera inclusión en \eq{I0}
por el escalar $t$ y aplicar de nuevo la \prp{bddPrp}.
\end{proof}
\Defi{sKlbdd}{
Se dice que un conjunto $S\subseteq X$ es
 semi-$K$-acotado inferiormente si existe un conjunto 
acotado $H,$ tal que, $S\subseteq\cl(H+K)$.
}
\Rem{KimpclK}{
De la definición se obtiene de inmediato que todo conjunto 
$S,$ $K$-acotado inferiormente, automáticamente es
semi-$K$-acotado inferiormente.
}
\Prp{sKlbddOp}{
La familia de conjuntos semi-$K$-acotados inferiormente
es cerrada bajo la suma y el producto por escalares 
positivos.
}
\begin{proof}
Sean $S_1,S_2\subseteq X$ conjuntos semi-$K$-acotados
inferiormente. Consideremos un abierto arbitrario 
$V\in\U(X)$. En vista de la \prp{symm} existe un abierto 
$U\in\U(X)$ tal que $U+U\subseteq V$ y
por definición, existen conjuntos acotados 
$H_1,H_2\subseteq X$ tales que 
\Eq{*}{
S_1&\subseteq\cl(H_1+K)\subseteq H_1+K+U,\qquad\mbox{y}\\
S_2&\subseteq\cl(H_2+K)\subseteq H_2+K+U.
}
Luego, usando la convexidad del cono $K$, y el hecho de que
$U+U\subseteq V$, se tiene que para todo abierto $V\in\U(X)$
\Eq{*}{
S_1+S_2\subseteq H_1+H_2+K+K+U+U
			 \subseteq H_1+H_2+K+V=H+K+V,
}
donde $H:=H_1+H_2$. Como $V$ es arbitrario, entonces
\Eq{*}{
S_1+S_2\subseteq\bigcap_{V\in\U(X)}(H+K+V)=\cl(H+K).
}
Finalmente, concluimos que $S_1+S_2$ es semi-$K$-acotado
inferiormente.

Veamos ahora que $tS_1$ también es semi-$K$-acotado
inferiormente, para cualquier escalar $t>0$. 
Sea $V\in\U(X)$ un abierto arbitrario. Por continuidad,
existe un abierto $U\subseteq X$ tal que $tU\subset V$.
Ahora bien, por definición $S_1\subseteq\cl(H_1+K)$, y 
por lo tanto
\Eq{*}{
tS_1\subseteq t\cl(H_1+K)\subseteq tH_1 + tK + tU
		\subseteq tH_1+K+V,
}
pero $H:=tH_1$ es acotado y $V$ es arbitrario, 
en consecuencia
\Eq{*}{
tS_1\subseteq\bigcap_{V\in\U(X)}(H+K+V) = \cl(H+K).
}
Es decir, $tS_1$ es semi-$K$-acotado inferiormente.
\end{proof}
\Prp{SKl=Kl-bdd}{
Si el espacio $X$ es localmente acotado, es decir, si 
existe un abierto $U$ que es acotado,
entonces la familia de conjuntos $K$-acotados inferiormente
y semi-$K$-acotados inferiormente coinciden.
}
\begin{proof}
Basta probar que si $X$ es localmente acotado, entonces
todo subconjunto $S$ de $X$ semi-$K$-acotado 
inferiormente es $K$-acotado inferiormente.
Sea $S\subseteq X$ un conjunto semi-$K$-acotado 
inferiormente y sea $U\in\U(X)$ un conjunto 
abierto y acotado. Por definición existe un conjunto
acotado $H_1\subseteq X$ tal que $S\subseteq \cl(H_1+K)$.
Sea $H:= H_1+U$. Es claro que el conjunto $H$ es acotado
y además,
\Eq{*}{
S\subseteq\cl(H_1+K)\subseteq H_1+K+U = H+K,
}  
en conclusión, el conjunto $S$ es $K$-acotado inferiormente.
\end{proof}
%%%%%%%%%%%%%%%%%%%%%%%%%%%%%%%%%%%%%%%%%%%%%%%%%%%%%%%%%%%
%%%%%%%%%%%%%%%%%%%%%%%%%%%%%%%%%%%%%%%%%%%%%%%%%%%%%%%%%%%

\section{Conjuntos $K$-convexos.}
\setcounter{theorem}{0}
\Defi{KCvx}{
Sea $D\subseteq X$. Se dice que $D$ es $K$-convexo si 
para todo $x,y\in X$ se tiene que $[x,y]\subseteq D+K.$
}
\Rem{0Cvx}{
Si $K=\{0\}$ la definición anterior se reduce
a la definición estandar de convexidad.
}
\Rem{1Cvx}{
Si $0\in K$, entonces $D\subseteq D+K$ para cualquier 
subconjunto $D\subseteq X$. 
}
Como consecuencia inmediata de la observación anterior,
se tiene la siguiente proposición.

\Prp{KCvx-1}{
Sea $K\subseteq X$ un cono convexo tal que $0\in K$. Si
el conjunto $D\subseteq X$ es convexo, entonces,
también es $K$-convexo.
}
En general, un conjunto $K$-convexo, no necesariamente
tiene que ser convexo. 
\Exa{Kcvx1}{
Consideremos $X=\R^2$, $K=[0,\infty)\times \{0\}$ y 
al subconjunto de $X$ dado en coordenadas polares por
 $D:=\{(r,\theta) : 0\leq r\leq1
\quad\mbox{y}\quad    \pi/4\leq\theta\leq7\pi/4\}$. 
El conjunto $D$ es $K$-convexo, pero sin embargo, 
no es convexo.
}
\Exa{Kcvx2}{
Si $0\notin K$ y $D$ es un conjunto convexo, entonces 
no necesariamente, $D$ es $K$-convexo. Para ver esto,
consideremos $X=\R^2,$ $K=(0,\infty)\times\{0\}$ y sea
$D=\{p\}$ con $p=(x_0,y_0)\in\R^2$. Es obvio que el conjunto 
$D$ es convexo, pero $D+K=(x_0,\infty)\times\{y_0\}$
y por lo tanto $D\not\subseteq D+K$, en consecuencia
el conjunto $D$ no es $K$-convexo.
}
\Prp{Kcvx-cvx}{
Supongamos que $0\in K$.
Un conjunto $D\subseteq X$ es $K$-convexo si y sólo si
el conjunto $D+K\subseteq X$ es convexo.
}
\begin{proof}
Sea $t\in[0,1]$. 
Supongamos que $D\subseteq X$ es $K$-convexo, entonces
\Eq{*}{
t(D+K)+(1-t)(D+K) \subseteq tD+(1-t)D + K 
\subseteq (D+K)+K \subseteq D+K,
} 
es decir, que el conjunto $D+K$ es convexo.
Supongamos ahora que el conjunto $D+K$ es convexo,
por lo tanto,
\Eq{*}{
tD+(1-t)D\subseteq tD+(1-t)D+K=t(D+K)+(1-t)(D+K)
\subseteq D+K.
}
Es decir, el conjunto $D$ es $K$-convexo.
\end{proof}
No necesariamente un cono convexo $K$ ha de tener al cero
como uno de sus elementos, sin embargo, 
\Prp{0inclK}{
$0\in\cl(K),$ para cualquier cono convexo $K$.
}
\begin{proof}
Si $0\in K$ la demostración es trivial. Supongamos que
$0\notin K$ y 
sean $V\in\U(X)$ un abierto simétrico y $k\in K$.
Consideremos la sucesión de números reales $(1/2^n)$,
cuyo límite es cero. Por lo tanto, existe un número
natural $N$, tal que si $n\geq N$ entonces
$k/2^n\in V.$ Lo cual es equivalente a
\Eq{*}{
0\in \frac{k}{2^n} + V \subseteq K + V.
}
Como $V$ es arbitrario, se tiene que 
$0\in\bigcap_{V\in\U(X)}(K+V) = \cl(K),$ lo que
completa la prueba. 
\end{proof}
\Defi{recH}{
Dado un subconjunto no vacío $H\subseteq X$, se define el cono
recesión de $H$, $(\rec(H))$ de la siguiente manera
\Eq{recH}{
\rec(H) = \{x\in X : tx+H\subseteq H, \mbox{ para todo }
	t\geq0\}
}
}
La siguiente proposición ilustra algunas propiedades 
del cono recesión asociado al conjunto $H$
\Prp{recH}{
Sea $H\subseteq X$ un conjunto no vacío. Entonces
\begin{enumerate}[i.]
	\item $0\in\rec(H)$ y $\rec(H)$ es un cono convexo.
	\item $K=\rec(H)$ es el cono convexo más grande 
	tal que $K+H\subseteq H$.
	\item $\cl(\rec(H)) \subseteq \rec(\cl(H)).$
	\item Para todo $x\in X$, $t>0$, $\rec(x+tH) = \rec(H)$.
	\item Para cualesquiera conjuntos no vacíos 
	$H_1,H_2\subseteq X$, 
	$$\rec(H_1)+\rec(H_2)\subseteq\rec(H_1+H_2).$$
\end{enumerate}
}
\begin{proof}$\quad$\vspace{-.2cm}
\begin{enumerate}[i.]
	\item En primer lugar, es evidente que $0\in\rec(H)$ pues, 
	$0t+H=H$. Para ver que $\rec(H)$ es convexo, sean
	$x,y\in\rec(H)$ y sea $s\in[0,1]$. Ahora bien, 
	como $x$ e $y$ están en $\rec(H)$, entonces
	$t(1-s)y + H \subseteq H$ y $tsx+H\subseteq H$, para
	cualquier número no-negativo $t$. Por lo tanto,
	\Eq{*}{
	t(sx+(1-s)y) + H = tsx + t(1-s)y + H 
	\subseteq tsx + H \subseteq H.
	}
	Es decir, que el segmento $[x,y]$ está contenido en 
	$\rec(H)$ para cualesquiera $x,y\in H$. De esta manera, 
	se ha demostrado que en efecto, $\rec(H)$ es un conjunto 
	convexo.
	
	\item Supongamos que $K\subseteq X$ es un cono convexo, 
	con la propiedad $K+H\subseteq H$, por lo tanto, 
	para cualquier $t\geq0$ se tiene que
	\Eq{*}{
	tK+H\subseteq K+H \subseteq H,
	}
	lo cual equivale a $K\subseteq\rec(H)$.
	
	\item En vista del \thm{f1}, se tiene que 
	$\cl(\rec(H))+\cl(H)\subseteq\cl(\rec(H)+H)$. 
	Además, de la definición se sigue el hecho de que 
	$\rec(H)+H\subseteq H$. Por lo tanto,
	\Eq{*}{
	\cl(\rec(H))+\cl(H)\subseteq\cl(\rec(H)+H)
	\subseteq\cl(H).
	}
	Como $\cl(\rec(H))$ es un cono convexo, pues 
	$\rec(H)$ lo es, entonces usando el numeral 2 de
	esta proposición, se tiene que 
	\Eq{*}{
	\cl(\rec(H)) \subseteq \rec(\cl(H)).
	}
	
	\item La demostración es directa de la definición.
	
	\item Es consecuencia directa del numeral 2.
\end{enumerate}
\end{proof}

\Prp{L2.2}{
Sean $(A_n)_{n\in\N},(B_n)_{n\in\N}$ dos sucesiones no-decrecientes
de subconjuntos de $X$, sea $H\subseteq X$ un conjunto
acotado, sea $K\subseteq\bigcap_{n\in\N}\cl(\rec(B_n))$ y sea
$(\epsilon_n)_{n\in\N}$ una sucesión de números reales que converge a 
cero. Asumamos que, para todo $n\geq0,$
\Eq{hip2.2}{
A_n\subseteq\cl(\epsilon_nH+K+B_n).
}
Entonces,
\Eq{Incn}{
\cl\Bigg(\bigcup_{n=0}^\infty A_n\Bigg) \subseteq
\cl\Bigg(\bigcup_{n=0}^\infty B_n\Bigg)
}
}
\begin{proof}
Sean $U\in\U(X)$ arbirario y sea $V\in\U(X)$ un abierto
balanceado tal que $V+V+V\subseteq U$. Como $H$ es acotado
y $(\epsilon_n)$ converge a cero, existe $N\in\N$
tal que si $n\geq N$ entonces, $\epsilon_nH\subseteq V$.
Además, como $K\subseteq\bigcap_{n\in\N}\cl(\rec(B_n))$,
se tiene que $K\subseteq\rec(B_n) + V$ para todo $n\in\N$.
De esta manera, para $n\geq N$
\Eq{*}{
A_n&\subseteq\cl(\epsilon_nH+K+B_n)
\subseteq\epsilon_nH+K+B_n+V \\
&\subseteq \rec(B_n)+B_n+V+V+V
\subseteq B_n + U \subseteq U+\bigcup_{n=0}^\infty B_n,
}
por lo tanto,
\Eq{*}{
\bigcup_{n\geq N}A_n \subseteq U+\bigcup_{n=0}^\infty B_n.
}
Como $U$ es arbitrario, y $(A_n)$ es no-decreciente
entonces
\Eq{*}{
\bigcup_{n=0}^\infty A_n\subseteq
\bigcap_{U\in\U(X)}\Bigg(U+\bigcup_{n=0}^\infty B_n\Bigg)
= \cl\Bigg(\bigcup_{n=0}^\infty B_n\Bigg).
}
El resultado se sigue de inmediato ya que el lado derecho de esta 
última inclusión es un conjunto cerrado.
\end{proof}
%%%%%%%%%%%%%%%%%%%%%%%%%%%%%%%%%%%%%%%%%%%%%%%%%%%%%%%%%%%%%%%%%%%%%%%%%%
%%%%%%%%%%%%%%%%%%%%%%%%%%%%%%%%%%%%%%%%%%%%%%%%%%%%%%%%%%%%%%%%%%%%%%%%%%%%%%%%%%%%%%%CHAPTER
%%%%%%%%%%%%%%%%%%%%%%%%%%%%%%%%%%%%%%%%%%%%%%%%%%%%%%%%%%%%%%%%%%%%%%%%%%
