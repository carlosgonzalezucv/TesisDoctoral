
\chapter{T\'ermino de error de tipo Takagi-Hazy-P\'ales.}
\setcounter{theorem}{0}
%Los resultados principales de esta investigación están concentrados en los siguientes
%dos teoremas. 
A lo largo de este capítulo, se asume que $X$ y $Y$ son espacios topológicos
lineales.

%El siguiente teorema teorema dá condiciones suficientes sobre una multifunción
%que satisface una inclusión de convexidad tipo Jensen, para que esta satisfaga
%una inclusión de tipo convexidad más general.
\section[Teorema de tipo Bernstein--Doetsch para multifunciones convexas]
{Teorema de tipo Bernstein--Doetsch con errores de tipo
THP para multifunciones convexas.}
\Thm{Convex}{
Sea $D\subseteq X$ un conjunto convexo no-vacío y $A,B:(D-D)\to\P_0(Y)$ multifunciones tales que
$0\in A(x)\cap B(x)$ para todo $x\in (D-D)$. Sea $K=\overline{\rec}(B)$, la clausura del
cono recesión de $B$. Sea $F:D\to\P_0(Y)$ una multifunción que satisface la inclusión de 
convexidad tipo Jensen 
\Eq{JCV}{
\dfrac{F(x) + F(y)}{2} + A(x-y) \subseteq \cl\bigg(F\pr{\dfrac{x+y}{2}} + B(x-y)\bigg) \qquad (x,y\in D).
}
Supongamos además, que $F$ tiene las siguientes propiedades:
\begin{enumerate}[i.]
 \item $F$ es puntualmente semi-$K$-acotada inferior.
 \item $F$ es localmente debíl-semi-$K$-acotada superior en $D$.
\end{enumerate}
Entonces $F$ satisface la siguiente inclusión 
\Eq{CV}{
 tF(x)+(1-t)F(y)+A^T(t,x-y)\subseteq \cl\big(F(tx+(1-t)y)+B^T(t,x-y)\big)
}
para todo $x,y\in D$ y $t\in[0,1].$
}

\begin{proof}
El primer paso en la demostración de \eq{CV}, será mostrar que, para todo $x,y\in D$
existe un conjunto acotado $H\subseteq Y$ tal que, para todo $n\geq 0$, $t\in[0,1]$, 
\Eq{CVn}{
 tF(x)&+(1-t)F(y) + \sum_{k=0}^{n-1}{\dfrac{1}{2^k}A\big(2 d_{\Z}(2^k t)(x-y)\big)} \\
 &\subseteq \cl\bigg(F(tx+(1-t)y) + \dfrac{1}{2^n} H + K +
  \sum_{k=0}^{n-1}{\dfrac{1}{2^k}B\big(2 d_{\Z}(2^k t)(x-y)\big)}\bigg). 
}

Fijemos $x,y\in D$ arbitrarios. Para verificar que \eq{CVn} se cumple, vamos a proceder aplicando
inducción sobre $n.$ 
Para el caso $n=0$, debemos demostrar que existe un conjunto acotado $H\subseteq Y$ tal que para todo
$t\in[0,1]$, 
\Eq{J0}{
tF(x)+(1-t)F(y)\subseteq \cl\big(F(tx+(1-t)y) + H + K\big).
}
Sea $U\in\U(Y)$ y escojamos un abierto balanceado $V\in\U(Y)$ tal que $V+V+V\subseteq U$. 
Como $F$ es puntualmente semi-$K$-acotada inferior, existen conjuntos acotados $H_x,H_y\subseteq Y$
tales que 
\Eq{*}{
F(x) \subseteq \cl(H_x + K)\subseteq V+H_x+K
}
y
\Eq{*}{
F(y) \subseteq \cl(H_y + K) \subseteq V+H_y + K.
} 
Multiplicando estas inclusiones por $t$ y $1-t$, respectivamente, sumándolas,
usando la convexidad del cono $K$ y el hecho de que $V$ es balanceado,
\Eq{J1}{
tF(x)+(1-t)F(y)&\subseteq tV + tH_x + tK + (1-t)V + (1-t)H_y +(1-t)K \\
               &\subseteq V + V + tH_x + (1-t)H_y + K.
}
Por la \prp{bddtH}, se tiene 
que tanto el conjunto $H_1:=\bigcup_{t\in[0,1]} tH_x$ como el
conjunto $H_2:=\bigcup_{t\in[0,1]} (1-t)H_y$
son acotados. Así, la inclusión \eq{J1} nos da que para todo $t\in[0,1]$,
\Eq{J2}{
tF(x)+(1-t)F(y)\subseteq V + V + H_1 + H_2 + K.
}
Por otra parte, como $F$ es localmente débil-semi-$K$-acotada superior y el segmento 
$[x,y]$ es compacto, entonces por la \prp{lem2.4}, existe un conjunto acotado $H_0$ 
tal que para todo $t\in[0,1]$,
\Eq{J3}{
 0\in \cl(F(tx+(1-t)y) + H_0 + K)\subseteq V + F(tx+(1-t)y) + H_0 + K.
} 
Ahora bien, sumando las inclusiones \eq{J2} y \eq{J3} lado a lado, se tiene que 
para todo $t$ en $[0,1]$, 
\Eq{*}{
tF(x)+(1-t)F(y)&\subseteq V + V + V+ F(tx+(1-t)y) + H_0 + H_1 + H_2 + K \\
               &\subseteq U + F(tx+(1-t)y) + H_0 + H_1 + H_2 + K.
} 
Por lo tanto,
\Eq{*}{
tF(x)+(1-t)F(y)&\subseteq \bigcap_{U\in\U}\big(U + F(tx+(1-t)y) + H_0 + H_1 + H_2 + K\big)\\
               &=\cl\big(F(tx+(1-t)y) + H_0 + H_1 + H_2 + K\big).
}
Así, la inclusión \eq{J0} se obtiene con $H:=H_0+H_1+H_2$.

Supongamos ahora que la ecuación \eq{CVn} es válida para $n$ y demostraremos que también es 
válida para $n+1$.
Supongamos que $t\in\big[0,\frac12\big]$ (el caso cuando $t\in\big[\frac12,1\big]$ es
análogo). Entonces, $d_{\Z}(t)=t$ y podemos reescribir el lado izquierdo de la inclusión 
que queremos demostrar como 
\Eq{e11}{
 tF(x)&+(1-t)F(y) + \sum_{k=0}^{n}{\dfrac{1}{2^k}A\big(2 d_{\Z}(2^k t)(x-y)\big)} \\
 &= tF(x)+(1-t)F(y)+ A\big(2t(x-y)\big) 
   + \sum_{k=1}^{n}{\dfrac{1}{2^k}A\big(2 d_{\Z}(2^kt)(x-y)\big)}.
}
Observemos que
\Eq{CVC1}{
 (1-t)F(y) = \frac{1-2t+1}{2}F(y)\subseteq \frac{1-2t}2 F(y) + \frac12 F(y),
}
por lo tanto,
\Eq{II1}{
 &tF(x)+(1-t)F(y)+ A\big(2t(x-y)\big) 
   + \sum_{k=1}^{n}{\dfrac{1}{2^k}A\big(2 d_{\Z}(2^kt)(x-y)\big)} \\
 &\subseteq \dfrac{1}{2}\bigg( 2tF(x)+(1-2t)F(y)
   + \sum_{k=0}^{n-1}{\dfrac{1}{2^k}A\big(2 d_{\Z}(2^k(2t))(x-y)\big)} \bigg) \\
&\hspace{8cm}+ \frac{1}{2}F(y) \!+\! A\big(2d_{\Z}(t)(x-y)\big).
}
Al usar la hipótesis inductiva con $2t$ en vez de $t,$ se sigue que
\Eq{II2}{
2t&F(x)+(1-2t)F(y)+ \sum_{k=0}^{n-1} \dfrac{1}{2^k}A\big(2 d_{\Z}(2^k(2t))(x-y)\big) \\
   & \subseteq \cl\bigg(F(2tx + (1-2t)y)+ \frac{1}{2^n}H + K 
   + \sum_{k=0}^{n-1} \dfrac{1}{2^k}B\big(2 d_{\Z}(2^k(2t))(x-y)\big)\bigg).
}
Combinando las inclusiones \eq{e11}, \eq{II1} y \eq{II2}, llegamos a la siguiente inclusión
\Eq{*}{
 & tF(x)+(1-t)F(y) + \sum_{k=0}^{n}{\dfrac{1}{2^k}A\big(2 d_{\Z}(2^k t)(x-y)\big)} \\
 & \subseteq \dfrac{1}{2}\cl\bigg(F(2tx + (1-2t)y)+ \frac{1}{2^n}H + K 
   + \sum_{k=0}^{n-1} \dfrac{1}{2^k}B\big(2 d_{\Z}(2^k(2t))(x-y)\big)\bigg) \\
 &\hspace{6cm} + \frac{1}{2}F(y) \!+\! A\big(2d_{\Z}(t)(x-y)\big)\\
 & \subseteq \cl\bigg(\dfrac{F(2tx + (1-2t)y)+ F(y)}{2}+ \frac{1}{2^{n+1}}H + K 
    + A\big(2t(x-y)\big) \\
&\hspace{6cm}		+ \sum_{k=0}^{n-1}\dfrac{1}{2^{k+1}}B\big(2d_{\Z}(2^k(2t))(x-y)\big)\bigg)
}
Como $F$ satisface la inclusión de tipo Jensen \eq{JCV}, entonces,
\Eq{*}{
\frac{F(2tx + (1-2t)y) + F(y)}{2} &+ A\big(2t(x-y)\big) \\
 \subseteq \cl\bigg(&F\bigg(\frac{2tx+(1-2t)y+y}{2}\bigg) + B\big(2t(x-y)\big)\bigg).
}
Por lo tanto,
\Eq{*}{
&tF(x)+(1-t)F(y) + \sum_{k=0}^{n}\dfrac{1}{2^k}A\big(2 d_{\Z}(2^kt)(x-y)\big) \\
& \subseteq \cl\bigg(\!\cl\bigg(F\bigg(\frac{2tx+(1-2t)y+y}{2}\bigg) 
     + B\big(2t(x-y)\big)\bigg) + \frac{1}{2^{n+1}}H + K \\
&\hspace{6.5cm}     + \sum_{k=0}^{n-1}\dfrac{1}{2^{k+1}}B\big(2d_{\Z}(2^k(2t))(x-y)\big)\bigg)\\
& = \cl\bigg(F\big(tx+(1-t)y\big) + \frac{1}{2^{n+1}}H + K 
  + \sum_{k=0}^{n}\dfrac{1}{2^{k}}B\big(2d_{\Z}(2^k t)(x-y)\big)\bigg).
}
Ahora podemos concluir que la inclusion \eq{CVn} es válida para todo $n\geq 0.$

Para completar la demostración del teorema, sea $t\in[0,1]$ fijo y apliquemos la \prp{L2.2} 
a las sucesiones de conjuntos y números definidas para $n\geq0$ de la siguiente manera
\Eq{*}{
 A_n &:= tF(x)+(1-t)F(y) + \sum_{k=0}^{n-1}\dfrac{1}{2^k}A(2 d_{\Z}(2^k t)(x-y)) \\
 B_n &:= F\big(tx+(1-t)y\big) + \sum_{k=0}^{n-1}\dfrac{1}{2^{k}}B\big(2d_{\Z}(2^k t)(x-y)\big) \\
 \varepsilon_n &:= \frac{1}{2^{n}}
}
Con esta notación, la inclusión \eq{CVn} es equivalente a \eq{Incn}.
Por otra parte, como $0\in A(u)\cap B(u)$ para todo $u\in (D-D)$,
entonces las sucesiones $(A_n)$ y $(B_n)$ son sucesiones no-decrecientes en $Y$.
Aplicando la \prp{L2.2}, 
\Eq{*}{
  \cl\bigg(\bigcup_{n=1}^\infty \bigg(tF(x)&+(1-t)F(y)
     + \sum_{k=0}^{n-1}\dfrac{1}{2^k}A(2 d_{\Z}(2^k t)(x-y))\bigg)\bigg)\\
  &\subseteq \cl\bigg(\bigcup_{n=1}^\infty \bigg(F\big(tx+(1-t)y\big)
     + \sum_{k=0}^{n-1}\dfrac{1}{2^{k}}B\big(2d_{\Z}(2^k t)(x-y)\big)\bigg)\bigg).
}
Ahora, aplicamos el numeral 2 del \thm{f1} en ambos lados de la inclusión anterior y se obtiene
\Eq{*}{
  \cl\bigg(tF(x)&+(1-t)F(y) + \cl\bigg(\bigcup_{n=1}^\infty
     \sum_{k=0}^{n-1}\dfrac{1}{2^k}A(2 d_{\Z}(2^k t)(x-y))\bigg)\bigg)\\
  &\subseteq \cl\bigg(F\big(tx+(1-t)y\big) +\cl\bigg(\bigcup_{n=1}^\infty 
     \sum_{k=0}^{n-1}\dfrac{1}{2^{k}}B\big(2d_{\Z}(2^k t)(x-y)\big)\bigg)\bigg),
}
lo cual es equivalente a la inclusión \eq{CV} que queríamos demostrar.
\end{proof}
%\newpage

\section[Teorema de tipo Bernstein--Doetsch para multifunciones cóncavas.]
{Teorema de tipo Bernstein--Doetsch con errores de tipo
Takagi-Hazy-P\'ales para multifunciones c\'oncavas.}
\Thm{Concave}{
Sea $D\subseteq X$ un conjunto convexo no-vacío y $A,B:(D-D)\to\P_0(Y)$ multifunciones tales que
$0\in A(x)\cap B(x)$ para todo $x\in (D-D)$. Sea $K=\overline{\rec}(B)$, la clausura del
cono recesión de $B$. Sea $F:D\to\P_0(Y)$ una multifunción que satisface la inclusión de 
concavidad tipo Jensen 
\Eq{JCC}{
F\pr{\dfrac{x+y}{2}} + A(x-y) \subseteq \cl\bigg(\dfrac{F(x) + F(y)}{2} + B(x-y)\bigg) \qquad (x,y\in D).
}
Supongamos además, que $F$ tiene las siguientes propiedades:
\begin{enumerate}[i.]
 \item $F$ es puntualmente semi-$K$-convexa, i.e., $tF(x)+(1-t)F(x)\subseteq \cl(F(x) + K)$  
para todo $x\in D$ y para todo $t\in[0,1]$;
 \item $F$ es localmente semi-$K$-acotada inferior.
\end{enumerate}
Entonces $F$ satisface la siguiente inclusión 
\Eq{CC}{
 F(tx+(1-t)y)+A^T(t,x-y)\subseteq \cl\big(tF(x)+(1-t)F(y)&+B^T(t,x-y)\big)
}
para todo $x,y\in D$ y $t\in[0,1]$.
}
\begin{proof}
Para demostrar la inclusión \eq{CC}, primero vamos a demostrar que para todo $x,y\in D$,
existe un conjunto acotado $H\subseteq Y$ tal que, para todo $n\geq 0$ y $t\in[0,1]$, 
\Eq{CCn}{
 F(tx&+(1-t)y) + \sum_{k=0}^{n-1}\dfrac{1}{2^k}A\big(2 d_{\Z}(2^k t)(x-y)\big) \\[-2mm]
 &\subseteq \cl\bigg(tF(x)+(1-t)F(y) + \dfrac{1}{2^n} H + K +
  \sum_{k=0}^{n-1}\dfrac{1}{2^k}B\big(2 d_{\Z}(2^k t)(x-y)\big)\bigg).
}

Sean $x,y\in D$ fijos. Para verificar que la inclusión \eq{CCn} es válida, 
vamos a proceder por inducción sobre $n$. Para $n=0$, se debe demostrar
que existe un conjunto acotado $H\subseteq Y$ tal que, para todo $t\in[0,1]$
\Eq{CC0}{
  F(tx+(1-t)y)\subseteq \cl\big(tF(x)+ (1-t)F(y) + H + K\big).
}
Como $F$ es semi-$K$-acotada inferior y el segmento $[x,y]$ es compacto, se tiene
que por la \prp{lem2.3}, existe un conjunto acotado $H_0\subseteq Y$ 
tal que 
\Eq{H0}{
F(tx+(1-t)y)\subseteq \cl(H_0 + K)   \qquad  (t \in[0,1]).
}
Por otra parte, como $F(x)$ y $F(y)$ son no-vacíos, podemos escoger dos elementos
$u\in F(x)$ y $v\in F(y)$, lo cual implica que
\Eq{Hxy}{
0 \in F(x)-u \qquad\mbox{y}\qquad 0 \in F(y)-v.
}    
Multiplicando ambas inclusiones en \eq{Hxy} por $t$ y $(1-t)$, respectivamente, y sumándolas
junto con la inclusión \eq{H0}, se tiene que para $t\in[0,1]$, 
\Eq{*}{
F(tx+(1-t)y) &\subseteq tF(x)+(1-t)F(y) - tu - (1-t)v + \cl(H_0 + K)\\
            &\subseteq \cl\big(tF(x)+(1-t)F(y) -[u,v] + H_0 + K\big).
}
Por lo tanto, la inclusión \eq{CC0} es válida con $H := H_0-[u,v],$ el cual es obviamente 
un conjunto acotado.

Ahora, supongamos que \eq{CCn} es válida para $n$ y veamos que también es válida para $n+1.$
Supongamos, al igual que en la demostración anterior que $t\in\big[0,\frac12\big]$ y  
observe que entonces $d_{\Z}(t)=t$. Reescribiendo el lado izquierdo de la inclusión que 
queremos demostrar se obtiene la siguiente expresión
\Eq{I1}{
 & F(tx+(1-t)y) + \sum_{k=0}^{n}{\dfrac{1}{2^k}A\big(2 d_{\Z}(2^k t)(x-y)\big)} \\
 & = F(tx+(1-t)y) + A(2d_{\Z}(t)(x-y)) + \sum_{k=1}^{n}{\dfrac{1}{2^k}A\big(2 d_{\Z}(2^k t)(x-y)\big)} \\
 & = F(tx+(1-t)y) + A\big(2t(x-y)\big)
 + \dfrac{1}{2}\sum_{k=0}^{n-1}{\dfrac{1}{2^k}A\big(2 d_{\Z}(2^k (2t))(x-y)\big)}.
}
Además, 
\Eq{*}{
tx+(1-t)y = \frac{2tx+(1-2t)y + y}{2}
}
y como $F$ satisface la inclusión de tipo Jensen dada en \eq{JCC},
\Eq{I2}{
 F\bigg(\dfrac{2tx+(1-2t)y + y}{2}\bigg) &+ A\big(2t(x-y)\big) \\
  &\subseteq \cl\bigg(\dfrac{F(2tx+(1-2t)y) + F(y)}{2} + B\big(2t(x-y)\big)\bigg).
}
Combinando \eq{I1} y \eq{I2}, se obtiene
\Eq{I3}{
 F&(tx+(1-t)y) + \sum_{k=0}^{n}\dfrac{1}{2^k}A\big(2 d_{\Z}(2^k t)(x-y)\big) \\
  &\subseteq \cl\bigg(\dfrac{F(2tx+(1-2t)y) + F(y)}{2} + B\big(2t(x-y)\big)\bigg) \\
 &\hspace{7cm}  +\dfrac{1}{2}\sum_{k=0}^{n-1}\dfrac{1}{2^k}A\big(2 d_{\Z}(2^k (2t))(x-y)\big) \\
  & \subseteq \cl\bigg(\dfrac12\bigg(F(2tx+(1-2t)y)
   +\sum_{k=0}^{n-1}\dfrac{1}{2^k}A\big(2 d_{\Z}(2^k (2t))(x-y)\big)\bigg) \\
 &\hspace{7cm}  + \dfrac12 F(y)+B\big(2d_{\Z}(t)(x-y)\big)\bigg).
}
Por la hipótesis inductiva, sabemos que  
\Eq{I4}{  
F&(2tx+(1-2t)y)+\sum_{k=0}^{n-1}\dfrac{1}{2^k}A\big(2 d_{\Z}(2^k (2t))(x-y)\big) \\ 
&\subseteq \cl\bigg(2tF(x)+(1-2t)F(y) + \dfrac{1}{2^{n}}H + K +
\sum_{k=0}^{n-1}\dfrac{1}{2^k}B\big(2 d_{\Z}(2^k (2t))(x-y)\big)\bigg).
}
Insertando \eq{I4} en \eq{I3} y usando el hecho de que 
\Eq{*}{
(1-2t)F(y)+F(y)\subseteq \cl\big((2-2t)F(y)+K\big)
}
lo cual es una consecuencia
de que $F$ es puntualmente semi-$K$-convexa, llegamos a la siguiente inclusión
\Eq{*}{
 F&(tx+(1-t)y) + \sum_{k=0}^{n}{\dfrac{1}{2^k}A\big(2 d_{\Z}(2^k t)(x-y)\big)} \\
& \subseteq \cl\bigg(\dfrac12\cl\bigg(2tF(x)+(1-2t)F(y) + 
\sum_{k=0}^{n-1}\dfrac{1}{2^k}B\big(2 d_{\Z}(2^k (2t))(x-y)\big)\bigg)\\
   &\hspace{5cm} +\dfrac{1}{2^{n}}H + K + \dfrac12 F(y)+B\big(2d_{\Z}(t)(x-y)\big)\bigg)\\
&\subseteq \cl\bigg(\dfrac{1}{2}\big(2t F(x) + (1-2t) F(y) + F(y) + K\big)  \\
&\hspace{5cm} + \dfrac{1}{2^{n+1}}H 
 +\sum_{k=0}^{n}\dfrac{1}{2^{k}}B\big(2 d_{\Z}(2^k t)(x-y)\big)\bigg) \\
&\subseteq \cl\bigg(tF(x)+(1-t)F(y)+ \dfrac{1}{2^{n+1}}H + K +
\sum_{k=0}^{n}\dfrac{1}{2^{k}}B\big(2 d_{\Z}(2^k t)(x-y)\big)\bigg). 
}
Esto completa la demostración de la inducción y así la inclusión \eq{CCn} es válida para todo $n\geq0$.

Ahora, vamos a usar la \prp{L2.2}. Para ello vamos a definir las sucesiones 
\Eq{*}{
A_n &:= F(tx+(1-t)y) + \sum_{k=0}^{n-1}{\dfrac{1}{2^k}A\big(2 d_{\Z}(2^k t)(x-y)\big)}, \\
B_n &:= tF(x)+(1-t)F(y) + \sum_{k=0}^{n-1}{\dfrac{1}{2^k}B\big(2 d_{\Z}(2^k t)(x-y)\big)}, \\
\varepsilon_n &= \frac1{2^n}.
}
para $t\in [0,1]$ fijo.

Por lo tanto, la inclusión \eq{CCn}, con las sucesiones $(A_n),(B_n)$ y $(\varepsilon_n)$ así definidas, 
es equivalente a la inclusión \eq{Incn}. Así, por la \prp{L2.2}, se sigue lo siguiente
\Eq{*}{
  \cl\bigg(\bigcup_{n=0}^{\infty}F(tx+(1-t)y) 
   &+ \sum_{k=0}^{n-1}{\dfrac{1}{2^k}A\big(2 d_{\Z}(2^kt)(x-y)\big)}\bigg)\\
  &\subseteq \cl\bigg(\bigcup_{n=0}^{\infty}tF(x)+(1-t)F(y) 
   + \sum_{k=0}^{n-1}{\dfrac{1}{2^k}B\big(2 d_{\Z}(2^k t)(x-y)\big)}\bigg).
}
De manera similar a como se hizo en la prueba del \thm{Convex}, la relación anterior
implica la inclusión que deseamos probar.
\end{proof}
%
%\Rem{R}{In each of the above theorems the closure operation can be removed from the right hand sides
%of the inclusions \eq{JCV}, \eq{CV}, \eq{JCC}, and \eq{CC} if the values of the set-valued map $F$ are 
%compact and $B$ has closed values. The closure operation can also be removed from the right hand sides
%of \eq{JCV} and \eq{CV} if $F$ has closed values and $B$ is compact valued. This observation also applies
%to the corollaries below. Another thing is which is worth mentioning is that if 
%$A(0)\subseteq\overline{\rec}(A)$ and $B(0)\subseteq \overline{\rec}(B)$, then, in view of \lem{TT},
%the inclusions \eq{CC} and \eq{CV} reduce to \eq{JCC} and \eq{JCV} for the substitution $t=\frac12$,
%respectively. Therefore, in this case, under the boundedness and convexity assumptions on $F$, 
%\eq{CC} and \eq{CV} are equivalent to \eq{JCC} and \eq{JCV}, respectively. The problem whether 
%inclusions \eq{CC} and \eq{CV} are the sharpest possible is an open problem. Results where the
%exactness of such estimates are obtained due to Boros \cite{Bor08}, Makó and Páles \cite{MakPal10b,MakPal13b}.}
%

\section{Consecuencias de los teoremas previos.}

Tomando multifunciones particulares $A,$ $B$ y usando la \prp{Tak}, vamos a establecer algunas
consecuencias importantes de los dos teoremas que acabamos de demostrar. Los siguientes 
corolarios resaltan la manera en que los resultados mencionados en la introducción están relacionados
de manera directa con nuestros resultados.

En los siguientes cuatro corolarios supondremos que $D\subseteq X$ es un conjunto convexo y no-vacío,
$K\subseteq Y$ es un cono convexo y cerrado, $S_0\subseteq Y$ es un conjunto convexo que contiene a 
$0$ y $\varphi:(D-D)\to\R_+$ es una función localmente acotada superior y no-negativa. 
Note que por la convexidad de $D$ se tiene que el conjunto 
$(D-D)$ es estrellado, por lo tanto la \prp{Tak} puede ser aplicada. 

Los primeros dos corolarios tratan sobre multifunciones fuertemente y aproximadamente
$K$-Jensen convexas respectivamente.

\Cor{Convex+1}{
Supongamos que $F:D\to\P_0(Y)$ es una multifunción puntualmente semi-$K$-acotada inferior y 
localmente débil-semi-$K$-acotada superior que satisface
\Eq{JCV+1}{
\dfrac{F(x) + F(y)}{2} \subseteq 
\cl\bigg(F\pr{\dfrac{x+y}{2}} + K + \varphi(x-y)S_0 \bigg) 
}
para todo $x,y\in D.$ Entonces
\Eq{CV+1}{
 tF(x)+(1-t)F(y) \subseteq \cl\big(F(tx+(1-t)y)+K+\varphi^T(t,x-y)S_0\big)
}
para todo $x,y\in D$ y para todo $t\in[0,1].$
}
\begin{proof}
Para cada $u\in D-D$, definamos las multifunciones $A(u)=0$ y $B(u)=K+\varphi(u)S_0$.
Ahora bien, por definición se tiene que para todo $u\in D-D$ 
\Eq{*}{
\rec(B(u))=\{y\in Y\,|\,ty+B(u)\subseteq B(u),\mbox{ para todo } t\geq0\}
}
y por lo tanto, si $y\in K$, entonces
\Eq{*}{
ty+B(u)=ty+K+\varphi(u)S_0\subseteq K+\varphi(u)S_0 =B(u).
}
Esto significa que $K\subseteq \rec(B(u))$ para todo $u\in D-D$ y 
en consecuencia $K\subseteq \overline{\rec}(B).$
Luego, $F$ es puntualmente semi-$\overline{\rec}(B)$-acotada inferior y localmente
débil-semi-$\overline{\rec}(B)$-acotada inferior. Aplicando
el \thm{Convex}, se tiene que para todo $x,y\in D$ y   
para todo $t\in[0,1]$
\Eq{*}{
 tF(x)+(1-t)F(y)\subseteq \cl\big(F(tx+(1-t)y)+B^T(t,x-y)\big).
}
Además, por la \prp{Tak} se obtiene que para $t\in[0,1]$ y $x,y\in D$ 
\Eq{*}{
B^T(t,x-y)=\cl(K+\varphi^T(t,x-y)).
}
Esto completa la demostración.
\end{proof}
\Cor{Convex+2}{
Supongamos que $F:D\to\P_0(Y)$ es una multifunción puntualmente semi-$K$-acotada inferior y 
localmente débil-semi-$K$-acotada superior que satisface
\Eq{JCV+2}{
\dfrac{F(x) + F(y)}{2} + \varphi(x-y)S_0 \subseteq \cl\bigg(F\pr{\dfrac{x+y}{2}} + K \bigg) 
}
para todo $x,y\in D.$ Entonces
\Eq{CV+2}{
 tF(x)+(1-t)F(y) + \varphi^T(t,x-y)S_0 \subseteq \cl\big(F(tx+(1-t)y) + K \big)
}
para todo $x,y\in D$ y para todo $t\in[0,1].$
}
\begin{proof}
La demostración de este resultado es completamente análoga a la demostración
anterior. Basta considerar $A(u)=\varphi(x-y)S_0$ y $B(u)=K$ para $u\in D-D$ 
y proceder de la misma manera que en la demostración anterior.
\end{proof}
Los siguientes dos corolarios tratan sobre multifunciones fuertemente y aproximadamente
$K$-Jensen cóncavas respectivamente y sus demostraciones son similares a la demostración
de los Corolarios \ref{CConvex+1} y \ref{CConvex+2} por lo cual serán omitidas.
%Usando el \thm{Convex} con las multifunciones $A(u)=0$ y $B(u)=K+\varphi(u)S_0$ 
%y aplicando la \prp{Tak}, se obtiene el resultado deseado. 
%Observe que, tenemos que $K\subseteq\overline{\rec}(B)$, 
%por lo tanto $F$ será puntualmente semi-$\overline{\rec}(B)$-acotada inferior y 
%localmente debíl-semi-$\overline{\rec}(B)$-acotada superior.
\Cor{Concave+1}{
Supongamos que $F:D\to\P_0(Y)$ es una multifunción puntualmente semi-$K$-convexa y
localmente semi-$K$-acotada inferior que satisface
\Eq{JCC+1}{
F\pr{\dfrac{x+y}{2}} \subseteq \cl\bigg(\dfrac{F(x) + F(y)}{2} + K + \varphi(x-y)S_0 \bigg)
}
para todo $x,y\in D.$ Entonces
\Eq{CC+1}{
 F(tx+(1-t)y) \subseteq \cl\big(tF(x)+(1-t)F(y) + K + \varphi^T(t,x-y)S_0 \big)
}
para todo $x,y\in D$ y para todo $t\in[0,1].$
}

\Cor{Concave+2}{
Supongamos que $F:D\to\P_0(Y)$ es una multifunción puntualmente semi-$K$-convexa y
localmente semi-$K$-acotada inferior que satisface
\Eq{JCC+2}{
F\pr{\dfrac{x+y}{2}} + \varphi(x-y)S_0 \subseteq \cl\bigg(\dfrac{F(x) + F(y)}{2} + K\bigg)
}
para todo $x,y\in D.$ Entonces
\Eq{CC+2}{
 F(tx+(1-t)y)+\varphi^T(t,x-y)S_0 \subseteq \cl\big(tF(x)+(1-t)F(y) + K \big)
}
para todo $x,y\in D$ y para todo $t\in[0,1].$
}
%
%\Rem{CC}{The results mentioned and recalled in the introduction can be derived as obvious consequences of
%the above corollaries. In the real valued setting, the Bernstein--Doetsch Theorem \cite{BerDoe15}, the 
%results of Ng--Nikodem \cite{NgNik93} and Házy--Páles \cite{HazPal04} follow if, in \cor{Convex+1}, we take 
%$Y:=\R$, $K:=\R_+$, $S_0:=[-1,0]$, $F(x):=\{f(x)\}$, and $\varphi(x):=0$, $\varphi(x):=\varepsilon$, $\varphi(x):=\varepsilon\|x\|$, respectively. Observe that, in these cases, \prp{Tak} yields 
%$\varphi^T(t,x):=0$, $\varphi^T(t,x):=2\varepsilon$, $\varphi^T(t,x):=2\varepsilon T(t)\|x\|$, respectively.
%The results of Averna, Cardinali, Nikodem, and Papalini \cite{AveCar90,CarNikPap93,Nik86,Nik87a,Nik87c,Nik89,Pap90} 
%and by Borwein \cite{Bor77} that are related to $K$-Jensen convex/concave vector valued and set-valued mappings
%can also be obtained directly. Numerous results obtained for approximate midconvexity by Makó and Páles \cite{MakPal10b,MakPal13b} and by Mure?ko, Ja. Tabor, Jó. Tabor, and ?oldak
%\cite{MurTabTab12,TabTab09b,TabTab09a,TabTabZol10b,TabTabZol10a} are generalized by Corollaries \ref{CConvex+1}--\ref{CConcave+2} to the vector valued and set-valued setting. Similarly, using the
%explicit form of the function $T_2$ described in \rem{Tak}, one can easily derive the results of Azócar, 
%Gimenez, Nikodem and Sanchez \cite{AzoGimNikSan11} and Leiva, Merentes, Nikodem, and Sanchez 
%\cite{LeiMerNikSan13} that are related to strongly $K$-Jensen convex real valued and set-valued functions 
%from \cor{Convex+2}.}
