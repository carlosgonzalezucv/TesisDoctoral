\documentclass[notheorems,envcountsect,serif,mathserif,professionalfonts,spanish,10pt]{beamer}
%envcountsect,serif,mathserif,professionalfont
%\setbeamertemplate{theorems}[numbered]

\setbeamertemplate{bibliography item}[text]

\usepackage[english]{babel}
\usepackage[utf8]{inputenc}

\usepackage{times,ifthen,lmodern,pxfonts,eulervm}
\usepackage[T1]{fontenc}
\usepackage{mathrsfs} 

\newcommand{\phii}{\varphi}
\newcommand{\R}{\mathbb{R}}
\newcommand{\Q}{\mathbb{Q}}
\newcommand{\N}{\mathbb{N}}
\newcommand{\Z}{\mathbb{Z}}
\newcommand{\D}{\mathbb{D}}

\newcommand{\T}{\mathcal{T}}
\newcommand{\MS}{\mathcal{S}}

\renewcommand{\P}{\mathscr{P}}

\newcommand{\ds}{\displaystyle}
\newcommand{\implica}{\Rightarrow}
\newcommand{\suma}[2]{\ds\sum_{k = #1}^{#2} }
\newcommand{\pr}[1]{\left( #1\right) }
\newcommand{\ch}[1]{\left[ #1\right] }

\newcommand{\dist}{\mathop{\hbox{\rm dist}}\nolimits}
\newcommand{\conv}{\mathop{\hbox{\rm conv}}\nolimits}
\newcommand{\rec}{\mathop{\hbox{\rm rec}}\nolimits}
\newcommand{\rsg}{\mathop{\hbox{\rm rsg}}\nolimits} 
\newcommand{\cl}{\mathop{\hbox{\rm cl}}\nolimits}
\newcommand{\dx}{\mathop{\hbox{\rm dx}}\nolimits}
\newcommand{\Dom}{\mathop{\hbox{\rm Dom}}\nolimits}


\newtheorem{theorem}{Teorema}
\renewcommand{\thetheorem}{\textrm{\thesection.\arabic{theorem}}}
\newtheorem{theorem*}{Teorema}
\def\Thm#1#2#3{\ifthenelse{\equal{#1}{*}}{\begin{theorem*}[#3]#2\end{theorem*}}
  {\begin{theorem}[#3]\label{T#1}#2\end{theorem}}}
\newtheorem{Atheorem}{Theorem}
\renewcommand{\theAtheorem}{\Alph{Atheorem}}
\def\THM#1#2{\begin{Atheorem}\label{T#1}#2\end{Atheorem}}
\def\thm#1{Teorema~\ref{T#1}}
\newtheorem{proposition}{Proposición}
\newtheorem{proposition*}{Proposición}
\def\Prp#1#2{\ifthenelse{\equal{#1}{*}}{\begin{proposition*}#2\end{proposition*}}
             {\begin{proposition}\label{P#1}#2\end{proposition}}}
\def\prp#1{Proposición~\ref{P#1}}
\newtheorem{corollary}[theorem]{Corolario}
\newtheorem*{corollary*}{Corolario}
\def\Cor#1#2#3{\ifthenelse{\equal{#1}{*}}{\begin{corollary*}[#3]#2\end{corollary*}}
             {\begin{corollary}[#3]\label{C#1}#2\end{corollary}}}
\def\cor#1{Corolario~\ref{C#1}}
\newtheorem{lemma}{Lema}
\newtheorem{lemma*}{Lema}
\def\Lem#1#2{\ifthenelse{\equal{#1}{*}}{\begin{lemma*}#2\end{lemma*}}
             {\begin{lemma}\label{L#1}#2\end{lemma}}}
\def\lem#1{Lema~\ref{L#1}}
\newtheorem{example}{Ejemplo}
\newtheorem{example*}{Ejemplo}
\long\def\Exa#1#2{\ifthenelse{\equal{#1}{*}}{\begin{example*}\rm #2\end{example*}}
            {\begin{example}\label{Ex#1}\rm #2\end{example}}}
\def\exa#1{Example~\ref{E#1}}
\newtheorem{problem}[subsection]{Problem}
\def\Prob#1#2{\begin{problem}\label{Prob#1}\rm #2\end{problem}}
\def\prob#1{Problem~\ref{Prob#1}}
\theoremstyle{definition}
\newtheorem{definition}{Definición}
\newtheorem*{definition*}{Definición}
\def\Defi#1#2{\ifthenelse{\equal{#1}{*}}{\begin{definition*}#2\end{definition*}}
      {\begin{definition}\label{D#1}#2\end{definition}}}
\def\defi#1{Definición~\ref{D#1}}
\newtheorem{remark}{Observación}
\newtheorem{remark*}{Observación}
\long\def\Rem#1#2{\ifthenelse{\equal{#1}{*}}{\begin{remark*}#2\end{remark*}}
             {\begin{remark}\label{R#1}#2\end{remark}}}
\def\rem#1{Observación~\ref{R#1}}
\def\eq#1{{\rm(\ref{E#1})}}
\renewcommand{\theequation}{\thesection.\arabic{equation}}
\def\Eq#1#2{\ifthenelse{\equal{#1}{*}}
  {\begin{equation*}\begin{aligned}[]#2\end{aligned}\end{equation*}}
  {\begin{equation}\begin{aligned}[]\label{E#1}#2\end{aligned}\end{equation}}}

\mode<presentation>
{
  %\usetheme{Warsaw}
	\usetheme{Darmstadt}
	%\usecolortheme{dove}
  % or ...

  %\setbeamercovered{transparent}
  % or whatever (possibly just delete it)
}


% Or whatever. Note that the encoding and the font should match. If T1
% does not look nice, try deleting the line with the fontenc.


\title% (optional, use only with long paper titles)
{Teoremas de tipo Bernstein-Doetsch con errores de tipo Tabor para multifunciones
aproximadamente y fuertemente midconvexas.}


\author[C. Gonz\'alez, K. Nikodem, Z. P\'ales, G. Roa.]
{
	Attila Gilányi\inst{1} \and
\underline{Carlos~Gonz\'alez}\inst{2} \and
  Kazimierz~Nikodem\inst{3} \and
  Zsolt~P\'ales\inst{1}  
}
\institute[Central University of Venezuela, University of Bielsko-Biala, 
University of Debrecen, Central Bank of Venezuela.]
{
	\inst{1}%
	Universidad de Debrecen, Hungria.
  \inst{2}%
  Universidad Central de Venezuela.
  \inst{3}%
  Universidad de Bielsko-Biala, Polonia.
  }
\date[XXVIII-JVM] % (optional, should be abbreviation of conference name)
{XXVIII Jornadas Venezolanas de Matemática, 2015}

\subject{Mathematical analysis, Inequalities.}
% This is only inserted into the PDF information catalog. Can be left
% out. 

\pgfdeclareimage[height=.5cm]{university-logo}{logo_ucv}
\logo{\pgfuseimage{university-logo}}

% If you wish to uncover everything in a step-wise fashion, uncomment
% the following command: 

%\beamerdefaultoverlayspecification{<+->}


\begin{document}

\begin{frame}
  \titlepage
\end{frame}

\begin{frame}{Contenido}
  \tableofcontents
  % You might wish to add the option [pausesections]
\end{frame}

\section{Introducción}

\subsection{Funciones convexas y cóncavas a valores reales.}
\begin{frame}{Funciones a valores reales.}
Sean $X$ un espacio normado real, $D\subseteq X$ un subconjunto abierto y convexo y 
$f:D\to\R$ una función.
\Defi{Cvx}{
Se dice que la función $f$ es \textbf{convexa} en $D,$ si para todo $x,y\in D$:
\Eq{Cvx}{
f(tx+(1-t)y)\leq tf(x)+(1-t)f(y), \quad t\in[0,1].
}
}
\Defi{Ccv}{
Se dice que la función $f$ es \textbf{cóncava} en $D,$ si para todo $x,y\in D$:
\Eq{Ccv}{
tf(x)+(1-t)f(y) \leq f(tx+(1-t)y),\quad t\in[0,1].
}
}
\begin{block}{Obervación.}
Es evidente que $f$ es cóncava si y sólo si $-f$ es convexa.
\end{block}
\end{frame}

\begin{frame}{Funciones Jensen-convexas a valores reales.}
\Defi{MidCvx}{
Se dice que la función $f$ es \textbf{Jensen-convexa} en $D,$ si para todo
$x,y\in D$:
\Eq{MidCvx}{
f\bigg(\frac{x+y}2\bigg)\leq\frac{f(x)+f(y)}2.
} 
}
\Thm{*}{
Sea $D\subseteq X$ un subconjunto abierto y convexo, y sea $f:D\to\R$ una
función Jensen-convexa. Entonces $f$ satisface la siguiente desigualdad para todo
$x,y\in D$ y para todo $t\in[0,1]\cap\Q$:
\Eq{cvx}{
f(tx+(1-t)y)\leq tf(x)+(1-t)f(y).
}
}{\cite{Kuc85}, Teorema 5.3.5.}
\end{frame}

\subsection{El Teorema de Bernstein--Doetsch.}

\begin{frame}{El Teorema de Bernstein--Doetsch.}
\Thm{*}{
Toda función Jensen-convexa $f:D\to\R$ en $D$, localmente acotada superior en un punto 
$x_0\in D$ is continua y por lo tanto convexa en $D$.
}{\cite{Kuc85}, Teorema 6.4.2}
Este teorema teorema fue formulado por F. Bernstein and G. Doetsch en 1915 
\cite{BerDoe15}, y desde entonces ha sido muy importante en la teoría de convexidad,
razón por la cual ha sido generalizado de muchas maneras diferentes y por varios autores.
Como consecuencia directa se tiene el siguiente
\Cor{*}{
Una función $f:D\to\R$ es convexa si y sólo si es 
continua y Jensen-convexa.
}{\cite{Kuc85}, Teorema 7.1.1}
\end{frame}

\subsection{Convexidad aproximada}

\begin{frame}{Funciones aproximadamente convexas.}
Sea $\alpha:\R^+\to\R^+$, una función no-decreciente.
\Defi{e-Cvx}{
Se dice que la función, $f:D\to\R$ es \textbf{$\alpha$-Jensen-convexa}
en $D,$ si para todo $x,y \in D$
\Eq{e-Cvx}{
f\bigg(\frac{x+y}2\bigg) \leq \frac{f(x)+f(y)}2 + \alpha(|x-y|).
}
}
\Rem{*}{
Cuando $\alpha(x)=\epsilon>0,$ entonces, $\alpha$-Jensen convexidad es simplemente,
$\epsilon$-Jensen convexidad \cite{HyeUla52}. Resultados de tipo B-D para este tipo de convexidad
fueron obtenidos por Ng y Nikodem en 1993 \cite{NgNik93}.
}
\end{frame}

\begin{frame}
\Thm{*}{
Sea $D\subseteq \R$ un subconjunto abierto y convexo de la recta real.
Si $f:D\to\R$ es una función localmente acotada superior en un punto y 
$\alpha$-Jensen-convexa en $D$, entonces, para todo $x,y\in D$ y para todo
$t\in[0,1]$ 
$$
f(tx+(1-t)y)\leq tf(x)+(1-t)f(y)+\T_\alpha(t,|x-y|),
$$
donde 
$$
\T_\alpha(t,u)=\sum_{n=0}^{\infty}\frac{1}{2^n}\alpha(2\dist_\Z(2^nt)u),\quad
t\in[0,1], u\in D-D.
$$
}{\cite{HazPal04,MakPal10b}}
\Rem{*}{
Si $\alpha(u)=\epsilon |u|^p + \delta,$ con $\epsilon,\delta,p >0$ y $u\in\R$. Entonces 
$$
\T_\alpha(t,u)=\epsilon\Bigg(\sum_{n=0}^{\infty}2^{p-n}\dist_\Z^p(2^nt)\Bigg)|u|^p
+2\delta = \epsilon T_p(t)|u|^p + 2\delta.
$$
}

\end{frame}

\begin{frame}
\Thm{*}{
Sea $D\subseteq \R$ un subconjunto abierto y convexo de la recta real.
Si $\sum_{n=0}^\infty\alpha(2^{-n})<\infty$ y
$f:D\to\R$ es una función localmente acotada superior en un punto y 
$\alpha$-Jensen-convexa en $D$, entonces, para todo $x,y\in D$ y para todo
$t\in[0,1]$ 
$$
f(tx+(1-t)y)\leq tf(x)+(1-t)f(y)+\MS_\alpha(t,\|x-y\|),
$$
donde 
$$
\MS_\alpha(t,u)=\sum_{n=0}^{\infty}2\dist_\Z(2^nt)\alpha\Big(\frac{u}{2^{n+1}}\Big),\quad
t\in[0,1], u\in D-D.
$$
}{\cite{TabTab09b}}
\Rem{*}{
Si $\alpha(u)=\epsilon |u|^p,$ con $\epsilon,p >0$ y $u\in\R$. Entonces 
$$
\MS_\alpha(t,u)=\epsilon\Bigg(\sum_{n=0}^{\infty}
													\frac{\dist_\Z(2^nt)}{2^{np+p+1}} \Bigg)|u|^p 
							 = \epsilon S_p(t)|u|^p.
$$
}
\end{frame}

\begin{frame}{Funciones fuertemente convexas.}
Sea $c$ un número real positivo. Siguiendo a Polyak, \cite{Pol66}
\Defi{StrCvx}{
Una función $f:D\to \R$ es \textbf{fuertemente convexa}
con módulo $c$ si para todo $x,y\in D$ y para todo $t\in[0,1]$
\Eq{StrCvx}{
f(tx+(1-t)y)\leq tf(x) + (1-t)f(y) - ct(1-t)|x-y|^2 
}
}
\Thm{*}{
Si $f:D\to\R$ es fuertemente Jensen-convexa con módulo $c$,
y localmente acotada superior en un punto de $D$ entonces 
$f$ es continua y fuertemente convexa con módulo $c$.
}{\cite{AzoGimNikSan11}, Teorema 2.3}
\end{frame}

\section{Multifunciones.}
\subsection{Terminología básica.}

\begin{frame}{Multifunciones $K$-Convexas.}
Sean $X,Y$ espacios topológicos lineales, $K\subseteq Y$
un cono convexo cerrado y $D\subseteq X$ un conjunto convexo
y abierto.
Denote por $\P(Y)$ a la clase de subconjuntos no-vacios de $Y$.
\Defi{KCvxSvm}{
Una multifunción $F:D\to \P(Y)$ 
es \textbf{$K$-convexa} en $D$, si para todo $x,y\in D$ y todo 
$t\in [0,1]$
\Eq{KCvxSvm}{
tF(x) + (1-t)F(y) \subseteq F(tx+(1-t)y) + K.
}
}
\end{frame}

\begin{frame}{Cono Recesión.}
\Defi{RecCone}{
Sea $H\subseteq X$ un conjunto no vacío. El \textbf{cono recesión} de $H$ 
denotado por $\rec(H)$ es el conjunto
\Eq{recH}{
rec(H):=\{x\in X \,|\, tx + H\subseteq H,\mbox{ for all } t\geq0\}.
}
}
\begin{block}{Propiedades.}
\begin{enumerate}
	\item $\rec(H)$ es un cono convexo que contiene al origen;
	\item $K=\rec(H)$ es el cono más grande con la propiedad $K+H\subseteq H$;
	\item $\overline{\rec}(H)\subseteq\rec(\overline{H});$
	\item para todo $x\in X$, $t>0,$ $\rec(tx+H) = \rec(H);$
	\item %Para todo par de conjuntos $H_1,H_2\subseteq X$,
	$\rec(H_1)+\rec(H_2)\subseteq\rec(H_1+H_2),$ para todo $H_1,H_2\subseteq X$.
\end{enumerate}
\end{block}
\end{frame}

%\begin{frame}{Cono Recesión de una multifunción.}
%\Defi{recSvm}{
%Given a set valued map $S:D\to\P(Y)$, \textbf{the recession cone} of $S$,
%denoted by $\rec(S)$, is the set:
%\Eq{recSvm}{
%\rec(S) = \bigcap_{x\in D}\rec(S(x)).
%}
%}
%\begin{block}{Properties}
%\begin{enumerate}
	%\item $\rec(S)\neq\emptyset$.
	%\item If $S(x)$ is bounded for some $x\in D$, then $\rec(S)=\{0\}$.
	%\item $\rec(S)+S(x) \subseteq S(x)$ for all $x\in D$.
%\end{enumerate}
%\end{block}
%\end{frame}

\begin{frame}{Multifunciones acotadas.}
\Defi{bdd1}{
Sea $S:D\to\P(Y)$ una multifunción. Se dice que $S$ es 
\textbf{localmente semi-$K$-acotada inferior} si para todo $x\in D$ existe
un entorno abierto $U\subseteq X$ de $x$ y un conjunto acotado $H\subseteq X$,
tal que 
\Eq{*}{
S(u)\subseteq\cl(H+K), \qquad (u\in U\cap D). 
}
}
\Defi{bdd2}{
Sea $S:D\to\P(Y)$ una multifunción. Se dice que $S$ es 
\textbf{localmente débil-semi-$K$-acotada superior} si para todo $x\in D$ existe
un entorno abierto $U\subseteq X$ de $x$ y un conjunto acotado $H\subseteq X$,
tal que
\Eq{*}{
0\in\cl(S(u)+H+K), \qquad (u\in U\cap D). 
}
}
\end{frame}

\begin{frame}{$K$-continuidad direccional.}
%Sea $F:D\to\P(Y)$ una multifunción. 
\Defi{*}{
Decimos que $F$ es direccionalmente $K$-semicontinua superior en un punto 
$p\in D$, si para toda dirección $h\in X$ y para todo entorno abierto 
$U$ de $0\in Y$, existe un nùmero positivo $\delta$ tal que
$$
F(p+th)\subseteq F(p) + U +K,
$$
para todo $t\in(0,\delta)$ tal que $p+th\in D.$
}
\Defi{*}{
Decimos que $F$ es direccionalmente $K$-semicontinua inferior en un punto 
$p\in D$, si para toda dirección $h\in X$ y para todo entorno abierto 
$U$ de $0\in Y$, existe un nùmero positivo $\delta$ tal que
$$
F(p)\subseteq F(p+th) + U +K,
$$
para todo $t\in(0,\delta)$ tal que $p+th\in D.$
}
\end{frame}

\subsection{Transformación de Takagi-Tabor.}
\begin{frame}
Asumamos que $D\subseteq X$ es un conjunto estrellado.
\Defi{TakTabT}{
Para una multifunción $S:D\to\P(Y)$, tal que $0\in S(x)$
para todo $x\in D$, definimos la 
\textbf{transformación de Takagi-Tabor} de $S$, como la multifunción
$S^\perp:\R\times D\to\P(Y)$ tal que
\Eq{TakagiT}{
S^\perp(t,x):= \cl\Bigg(
\bigcup_{n=0}^\infty\sum_{k=0}^n2\dist(2^kt,\Z)S\bigg(\frac{x}{2^{n+1}}\bigg)
\Bigg).
}
}
\begin{block}{Relación entre $S$ y $S^\perp$.}
Sea $S:D\to\P(Y)$ una multifunción tal que $0\in S(x)$
para todo $x\in D$. Entonces
\Eq{S-St}{
S^\perp\big(\tfrac12,x\big)=\cl(S(x))\qquad (x\in D).
} 
\end{block}
\end{frame}

\section{Resultados principales}

\subsection{Teoremas.}

\begin{frame}
\Thm{*}{
Sean $D\subseteq X$ un subconjunto convexo no-vacío, y $A,B:(D-D)\to\P_0(Y)$ 
tales que los valores de la multifunción $B$ son semi-$K$-convexos, donde 
$K:=\overline{\rec}(B)$. 
Sea $F:D\to\P_0(Y)$ una multifunción que satisface la siguiente inclusión de 
tipo Jensen para $x,y\in D$
\Eq{JCV}{
\dfrac{F(x) + F(y)}{2} + A(x-y) \subseteq \cl\bigg(F\bigg(\dfrac{x+y}{2}\bigg) + B(x-y)\bigg).
}
Entonces, $F$ satisface
\Eq{CV}{
 tF(x)&+(1-t)F(y)+\sum_{k=0}^{\infty} 2d_{\Z}\big(2^k t\big)A\Big(\dfrac{x-y}{2^k}\Big) \\
&\subseteq \cl\bigg(F(tx+(1-t)y)+\sum_{k=0}^{\infty} 2d_{\Z}\big(2^k t\big)B\Big(\dfrac{x-y}{2^k}\Big)\bigg),   
}
para todo $x,y\in D$ y $t\in\D\cap[0,1].$
}{}
\end{frame}

\begin{frame}
	
\Lem{*}{
Sean $K\subseteq Y$ un cono convexo y $S,T\subseteq Y$ subconjuntos no vacíos 
tales que 
\begin{enumerate}[(i)]
	\item $S$ y $T$ son conjuntos semi-$K$-acotados inferiormente,
	\item $S$ y $T$ son semi-$K$-estrellados con respecto a algún elemento de $Y$, i.e, 
	existen $u,v\in Y$ tal que
	\Eq{*}{
	tu + (1-t)S &\subseteq S\qquad (t\in[0,1]), \\
	tv + (1-t)T &\subseteq T\qquad (t\in[0,1]). 
  }
\end{enumerate} 
 Entonces, la multifunción 
$t\mapsto tS+(1-t)T$ es direccionalmente $K$-continua en $[0,1]$.
}{ }
	
\end{frame}

\begin{frame}
	
	\Lem{*}{
Sea $K\subseteq Y$ un cono convexo y sea $(S_k)$ una sucesión de subconjuntos
no vacíos de $Y$ tales que 
\begin{enumerate}[(i)]
  \item Para todo $k\geq 0$, el conjunto $S_k$ es semi-$K$-estrellado y 
	semi-$K$-acotado inferior.
  \item La sucesión $(S_k)$ es $K$-Cauchy, i.e., para todo abierto $V\subseteq Y$,
	entorno de $0\in Y$, existe $m\in\N$ tal que, para todo $n\geq m$,
\Eq{iii}{
 \sum_{k=m}^nS_k \subseteq V+K.
}
\end{enumerate}
Entonces, para todo $U\in U(Y)$, existe un número positivo $\delta$ tal que,
para todo $t,s\in\R$ con $|t-s|<\delta$, 
\Eq{ts0}{
\cl\bigg(\bigcup_{n=0}^\infty\sum_{k=0}^n d_\Z(2^kt)S_k\bigg) 
  \subseteq \cl\bigg(\bigcup_{n=0}^\infty\sum_{k=0}^n d_\Z(2^ks)S_k\bigg) + U + K.
}
}{}
	
\end{frame}

\begin{frame}
	
\Thm{*}{
Sean $D\subseteq X$ un subconjunto convexo no-vacío, y $A,B:(D-D)\to\P_0(Y)$.
Sea $K:=\overline{\rec}(B)$ y sea $F:D\to\P_0(Y)$ una multifunción que satisface la inclusión de tipo Jensen \eq{JCV}.
Supongamos que además
\begin{enumerate}[(i)]
	\item Para todo $x\in D$, $F(x)$ es semi-$K$-estrellado con respecto a algún 
	 elemento de $Y$ y también semi-$K$-acotado inferior.
	\item $F$ es direccionalmente $K$-semicontinua en $D$.
	\item Para todo $u\in D-D$, los conjuntos $A(u)$ y $B(u)$ son semi-$K$-acotados 		
	inferiores, además $A(u)$ es semi-$K$-estrellado y $B(u)$ es semi-$K$-convexo.
	\item Para todo $u\in D-D$, las sucesiones $\Big(A\Big(\dfrac{u}{2^k}\Big)\Big)$ y 
	$\Big(B\Big(\dfrac{u}{2^k}\Big)\Big)$ son $K$-Cauchy.
\end{enumerate}
Entonces, para todo $t\in[0,1]$ y para todo $x,y\in D,$ la multifunción
$F$ satisface la siguiente inclusión 
\Eq{*}{
tF(x) &+ (1-t)F(y)+ A^\perp(t,x-y) \\
&\subseteq\cl\big(F(tx+(1-t)y)+B^\perp(t,x-y)+K\big).
} 
}{ }
	
\end{frame}

\begin{frame}[allowframebreaks]
  \frametitle{References}
   
	\scriptsize{\bibliographystyle{amsalpha}}
	\nocite{GilGonNikPal15}
  \bibliography{funcequ,publ}
	
  \beamertemplatearticlebibitems


\end{frame}



\end{document}

