\documentclass[notheorems,envcountsect,serif,mathserif,professionalfonts,spanish,10pt]{beamer}
%envcountsect,serif,mathserif,professionalfont
%\setbeamertemplate{theorems}[numbered]

\setbeamertemplate{bibliography item}[text]
\setbeamertemplate{frametitle continuation}{}
\usepackage[english]{babel}
\usepackage[utf8]{inputenc}

\usepackage{times,ifthen,lmodern,pxfonts,cancel}
\usepackage[T1]{fontenc}
\usepackage{mathrsfs} 
\usepackage{animate,media9,xcolor}
%\usepackage{chronosys} 
%\usepackage[sorting=ynt]{biblatex}
\newcommand{\phii}{\varphi}
\newcommand{\R}{\mathbb{R}}
\newcommand{\Q}{\mathbb{Q}}
\newcommand{\N}{\mathbb{N}}
\newcommand{\Z}{\mathbb{Z}}
\newcommand{\D}{\mathbb{D}}

\renewcommand{\P}{\mathcal{P}}
\newcommand{\U}{\mathcal{U}}
\newcommand{\T}{\mathcal{T}}
\newcommand{\MS}{\mathcal{S}}

\newcommand{\ds}{\displaystyle}
\newcommand{\implica}{\Rightarrow}
\newcommand{\suma}[2]{\ds\sum_{k = #1}^{#2} }
\newcommand{\pr}[1]{\left( #1\right) }
\newcommand{\ch}[1]{\left[ #1\right] }
\newcommand{\ccvx}[2]{t#1 + (1-t)#2 } %combinacion convexa
\newcommand{\cmcvx}[2]{ \frac{#1+#2}2} %combinacion midconvexa


\newcommand{\dist}{\mathop{\hbox{\rm dist}}\nolimits}
\newcommand{\conv}{\mathop{\hbox{\rm conv}}\nolimits}
\newcommand{\rec}{\mathop{\hbox{\rm rec}}\nolimits}
\newcommand{\rsg}{\mathop{\hbox{\rm rsg}}\nolimits} 
\newcommand{\cl}{\mathop{\hbox{\rm cl}}\nolimits}
\newcommand{\dx}{\mathop{\hbox{\rm dx}}\nolimits}
\newcommand{\Dom}{\mathop{\hbox{\rm Dom}}\nolimits}


\newtheorem{theorem}{Teorema}
\renewcommand{\thetheorem}{\textrm{\thesection.\arabic{theorem}}}
\newtheorem{theorem*}{Teorema}
\def\Thm#1#2#3{\ifthenelse{\equal{#1}{*}}{\begin{theorem*}[#3]#2\end{theorem*}}
  {\begin{theorem}[#3]\label{T#1}#2\end{theorem}}}
\newtheorem{Atheorem}{Theorem}
\renewcommand{\theAtheorem}{\Alph{Atheorem}}
\def\THM#1#2{\begin{Atheorem}\label{T#1}#2\end{Atheorem}}
\def\thm#1{Teorema~\ref{T#1}}
\newtheorem{proposition}{Proposición}
\newtheorem{proposition*}{Proposición}
\def\Prp#1#2{\ifthenelse{\equal{#1}{*}}{\begin{proposition*}#2\end{proposition*}}
             {\begin{proposition}\label{P#1}#2\end{proposition}}}
\def\prp#1{Proposición~\ref{P#1}}
\newtheorem{corollary}[theorem]{Corolario}
\newtheorem*{corollary*}{Corolario}
\def\Cor#1#2#3{\ifthenelse{\equal{#1}{*}}{\begin{corollary*}[#3]#2\end{corollary*}}
             {\begin{corollary}[#3]\label{C#1}#2\end{corollary}}}
\def\cor#1{Corolario~\ref{C#1}}
\newtheorem{lemma}{Lema}
\newtheorem{lemma*}{Lema}
\def\Lem#1#2{\ifthenelse{\equal{#1}{*}}{\begin{lemma*}#2\end{lemma*}}
             {\begin{lemma}\label{L#1}#2\end{lemma}}}
\def\lem#1{Lema~\ref{L#1}}
\newtheorem{example}{Ejemplo}
\newtheorem{example*}{Ejemplo}
\long\def\Exa#1#2{\ifthenelse{\equal{#1}{*}}{\begin{example*}\rm #2\end{example*}}
            {\begin{example}\label{Ex#1}\rm #2\end{example}}}
\def\exa#1{Example~\ref{E#1}}
\newtheorem{problem}[subsection]{Problem}
\def\Prob#1#2{\begin{problem}\label{Prob#1}\rm #2\end{problem}}
\def\prob#1{Problem~\ref{Prob#1}}
\theoremstyle{definition}
\newtheorem{definition}{Definición}
\newtheorem*{definition*}{Definición}
\def\Defi#1#2{\ifthenelse{\equal{#1}{*}}{\begin{definition*}#2\end{definition*}}
      {\begin{definition}\label{D#1}#2\end{definition}}}
\def\defi#1{Definición~\ref{D#1}}
\def\DefiOp#1#2#3{\ifthenelse{\equal{#1}{*}}{\begin{definition*}[#3]#2\end{definition*}}
  {\begin{definition}[#3]\label{T#1}#2\end{definition}}}


\newtheorem{remark}{Observación}
\newtheorem{remark*}{Observación}
\long\def\Rem#1#2{\ifthenelse{\equal{#1}{*}}{\begin{remark*}#2\end{remark*}}
             {\begin{remark}\label{R#1}#2\end{remark}}}
\def\rem#1{Observación~\ref{R#1}}
\def\eq#1{{\rm(\ref{E#1})}}
\renewcommand{\theequation}{\thesection.\arabic{equation}}
\def\Eq#1#2{\ifthenelse{\equal{#1}{*}}
  {\begin{equation*}\begin{aligned}[]#2\end{aligned}\end{equation*}}
  {\begin{equation}\begin{aligned}[]\label{E#1}#2\end{aligned}\end{equation}}}

\mode<presentation>
{
  \usetheme{Frankfurt}
	%\usetheme{progressbar}%Darmstadt}
%	\usecolortheme{crane}
  % or ...

  %\setbeamercovered{transparent}
  % or whatever (possibly just delete it)
}

%\PassOptionsToPackage{demo}{graphicx}
%
%\makeatletter
%\newcommand\titlegraphicii[1]{\def\inserttitlegraphicii{#1}}
%\titlegraphicii{}
%
%\setbeamertemplate{title page}
%{
  %\vbox{}
   %{\usebeamercolor[fg]{titlegraphic}\inserttitlegraphic\hfill\inserttitlegraphicii\par}
  %\begin{centering}
    %\begin{beamercolorbox}[sep=8pt,center]{institute}
      %\usebeamerfont{institute}\insertinstitute
    %\end{beamercolorbox}
    %\begin{beamercolorbox}[sep=8pt,center]{title}
      %\usebeamerfont{title}\inserttitle\par%
      %\ifx\insertsubtitle\@empty%
      %\else%
        %\vskip0.25em%
        %{\usebeamerfont{subtitle}\usebeamercolor[fg]{subtitle}\insertsubtitle\par}%
      %\fi%     
    %\end{beamercolorbox}%
    %\vskip1em\par
    %\begin{beamercolorbox}[sep=8pt,center]{date}
      %\usebeamerfont{date}\insertdate
    %\end{beamercolorbox}%\vskip0.5em
    %\begin{beamercolorbox}[sep=8pt,center]{author}
      %\usebeamerfont{author}\insertauthor
    %\end{beamercolorbox}
  %\end{centering}
  %%\vfill
%}
%% Or whatever. Note that the encoding and the font should match. If T1
%% does not look nice, try deleting the line with the fontenc.
%\makeatother
%\titlegraphic{\includegraphics[height=1cm,width=2cm]{logo1}}
%\titlegraphicii{\includegraphics[height=1cm,width=2cm]{logo2}}


\title[Teoremas de tipo Bernstein--Doetsch.]% (optional, use only with long paper titles)
{Teoremas de tipo Bernstein-Doetsch para multifunciones 
convexas y c\'oncavas.}


\author[C. Gonz\'alez.]
{
Autor: Carlos~Gonz\'alez.\\\vspace{10pt} 
\small{Tutor: Dr. Nelson Merentes. \\Co-Tutor: Dr. Zsolt Páles.}
%Tutor: \\
%Co-Tutor: Dr. Zsolt~Páles.
}

\institute[Central University of Venezuela]
{
  Universidad Central de Venezuela.
}
\date[\today] % (optional, should be abbreviation of conference name)
{Presentaci\'on para optar al t\'itulo de Magister Scientiarium en Matem\'atica.}

\subject{Mathematical analysis, Inequalities.}
% This is only inserted into the PDF information catalog. Can be left
% out. 

\pgfdeclareimage[height=.5cm]{university-logo}{logo_ucv}
\logo{\pgfuseimage{university-logo}}

% If you wish to uncover everything in a step-wise fashion, uncomment
% the following command: 

%\beamerdefaultoverlayspecification{<+->}


\begin{document}

\begin{frame}
  \titlepage
\end{frame}

\begin{frame}{Contenido}
  \tableofcontents
  % You might wish to add the option [pausesections]
\end{frame}

\section{Introducción.}
\subsection{Funciones a valores reales.}
\begin{frame}{Funciones a valores reales.}
Sea $X$ un espacio normado real, $D\subseteq X$ un conjunto abierto y convexo y 
$f:D\to\R$ una función.
\Defi{Cvx}{
La función $f$ es \textbf{convexa} en $D$ si para todo $x,y\in D$:
\Eq{Cvx}{
f(tx+(1-t)y)\leq tf(x)+(1-t)f(y), \quad t\in(0,1).
}
}
\Defi{*}{
La función $f$ es \textbf{cóncava} en $D$ si para todo 
$x,y\in D$:
\Eq{Ccv>}{
tf(x)+(1-t)f(y) \leq f(tx+(1-t)y),\quad t\in(0,1).
}
}
\begin{block}{Observación.}
Como consecuencia directa se tiene que $f$ is cóncava si y sólo si, $-f$ es convexa.
\end{block}
\end{frame}

\begin{frame}{Funciones a valores reales.}
\Defi{*}{
La función $f$ es \textbf{midconvexa} en $D$ si para todo
$x,y\in D$:
\Eq{*}{
f\bigg(\frac{x+y}2\bigg)\leq\frac{f(x)+f(y)}2.
} 
}
\Thm{*}{
Sea $D\subseteq X$ un conjunto convexo y $f:D\to\R$ una
función midconvexa. Entonces, $f$ satisface la siguiente desigualdad para todo
$x,y\in D$ y para todo $t\in[0,1]\cap\Q$:
\Eq{cvx}{
f(tx+(1-t)y)\leq tf(x)+(1-t)f(y).
}
}{\cite{Kuc85}, Theorem 5.3.5.}
\end{frame}

\subsection{Multifunciones.}

\begin{frame}{Multifunciones.}
	Sean $X$ y $Y$ espacios topol\'ogicos lineales y sea $D\subseteq X$
	un subconjunto abierto y convexo. Denotemos por $\P_0(Y)$ a la clase
	de subconjuntos no-vac\'ios de $Y$.
	\Defi{*}{
	Una multifunci\'on $F:D\to\P_0(Y)$ es midconvexa en $D,$ si para todo
	$x,y\in D$ se satisface 
	\Eq{*}{
	\cmcvx{F(x)}{F(y)} \subseteq F\bigg(\cmcvx{x}{y}\bigg).
	}
	}
	\Defi{*}{
	Una multifunci\'on $F:D\to\P_0(Y)$ es midc\'oncava en $D,$ si para todo
	$x,y\in D$ se satisface 
	\Eq{*}{
	F\bigg(\cmcvx{x}{y}\bigg)\subseteq\cmcvx{F(x)}{F(y)}.
	}
	}
\end{frame}
	
\begin{frame}{Multifunciones.}
	\Exa{*}{
	Sea $H$ un subconjunto no-vac\'io de $Y$. Definamos a la multifunci\'on $F$ mediante 
	la f\'ormula $F(t) = tH$, $t\in \R_+$. Para $t_1,t_2\in\R_+$ tenemos lo siguiente
	\Eq{*}{
	F\bigg(\frac{t_1+t_2}2\bigg) &= \frac{t_1+t_2}2H = \frac12(t_1+t_2)H \\ 
	&\subseteq \frac12(t_1H + t_2H) = \frac{t_1H + t_2H}2
	=\frac{F(t_1)+F(t_2)}2.
	}
	Por lo tanto, $F$ es una multifunci\'on c\'oncava. Sin embargo, la multifunción, 
	$G(t):=-F(t) = t(-H)$ no es midconvexa.
	}
	\Rem{*}{
	\begin{center}
	$F$ es midconvexa $\xcancel{\iff}$ $-F$ es midc\'oncava.
	\end{center}
	%Este ejemplo evidencia que no existe hay una analog\'ia directa entre 
	%las multifunciones mid-convexas y las multifunciones mid-c\'oncavas. En 
	%consecuencia, los resultados obtenidos con respecto a estas clases de 
	%multifunciones deben tratarse por separado.
	}
\end{frame}

\begin{frame}{Multifunciones.}
	\Defi{*}{
	Sea $F:D\to\P_0(Y)$ una multifunción. Se dice que $F$ es continua
	con respecto a la topología de Hausdorff si 
	para todo abierto $V\in \U(Y)$, existe un abierto $U\in\U(X)$ tal que
	\Eq{KupSemConSVM}{
	F(x)\subseteq F(x_0)+V\quad\mbox{y}\quad F(x_0)\subseteq F(x)+V,
	}
	para todo $x\in x_0+U.$
	}

	\Thm{*}{
	Si $F:D\to \mbox{B}(Y)$ es midconvexa y continua con respecto a la topolog\'ia 
	de Hausdorff, entonces, $F$ es convexa.
	}{\cite{Nik87c}}
			
	\Thm{*}{
	Si $F:D\to \mbox{B}(Y)$ es midc\'oncava y continua con respecto a la topolog\'ia 
	de Hausdorff, entonces, $F$ es c\'oncava.
	}{\cite{Nik87a}}
	
\end{frame}

\section{Antecedentes.}

\subsection{El Teorema de Bernstein--Doetsch.}
%PLAN B A LA HORA DEL TÉ
%\begin{frame}{Bernstein--Doetsch Theorem.}
%\Thm{*}{
%Let $D\subseteq X$ be an open convex set.	Every Jensen convex function $f:D\to\R$	
%on $D$, locally upper bounded at a point $x_0\in D$ is continuous and
%hence convex on $D$.
%}{\cite{Kuc85}, Theorem 6.4.2}
%
%This theorem was formulated by Bernstein and Doetsch in 1915 
%\cite{BerDoe15}, and it has been very important in convexity theory, 
%althought it has been generalized in many different ways and by many
%authors. As a consecuence we have the following
%\Cor{*}{
%A function $f:D\to\R$ is convex if and only if it is 
%continuous and Jensen convex.
%}{\cite{Kuc85}, Theorem 7.1.1}
%\end{frame}

\begin{frame}{El Teorema de Bernstein--Doetsch.}
	%El teorema de Bernstein--Doetsch proporciona una condici\'on 
	%suficiente y m\'as d\'ebil que la continuidad
	%para que una funci\'on midconvexa sea continua y por lo tanto
	%convexa. 
	\Thm{*}{
	Sea $D$ un subconjunto abierto y convexo de $\R^n$ y 
	sea $f:D\to\R$ una funci\'on midconvexa. Si $f$ es acotada 
	superiormente en un subconjunto abierto y no vac\'io de $D$
	entonces $f$ es continua.
	}{\cite{BerDoe15}}
	
%	\vspace{.5cm}
%	\startchronology[startyear=1984,stopyear=2015,color=blue!50!white,arrow=false,
%	dateselevation=5pt,height=5pt]
%	
%	\chronoevent[date=false,markdepth=60pt]{2014}{\tiny{\cite{GonNikPalRoa14}}}
%	\chronoevent[date=false,markdepth=40pt]{2013}{\tiny{\cite{LeiMerNikSan13}}}
%	\chronoevent[date=false,markdepth=25pt,ifcolorbox=false]{2012}{\tiny{\cite{MakPal12a}}}
%	\chronoevent[date=false,markdepth=32pt,mark=false,ifcolorbox=false]{2012}{\tiny{\cite{Mur12DM}}}
%	\chronoevent[date=false,markdepth=60pt]{2011}{\tiny{\cite{AzoGimNikSan11}}}
%	\chronoevent[date=false,markdepth=68pt,mark=false,ifcolorbox=false]{2011}{\tiny{\cite{Haz11}}}
%	\chronoevent[date=false,markdepth=73pt,mark=false,ifcolorbox=false]{2011}{\tiny{\cite{GilTro11}}}
%	
%	\chronoevent[date=false,ifcolorbox=false,markdepth=40pt]{2010}{\tiny{\cite{HazBur10}}}
%	\chronoevent[date=false,ifcolorbox=false,markdepth=45pt]{2010}{\tiny{\cite{Haz10}}}
%	\chronoevent[date=false,ifcolorbox=false]{2009}{\tiny{\cite{TabTab09a}}}
%	\chronoevent[date=false,markdepth=18pt,mark=false,ifcolorbox=false]{2009}{\tiny{\cite{BurHazJuh09}}}
%	\chronoevent[date=false,markdepth=25pt,mark=false,ifcolorbox=false]{2009}{\tiny{\cite{Haz09a}}}
%	\chronoevent[date=false]{2005}{\tiny{\cite{HazPal05}}}
%	\chronoevent[date=false,markdepth=40pt]{2004}{\tiny{\cite{HazPal04}}}
%	\chronoevent[date=false,markdepth=50pt,mark=false]{2004}{\tiny{\cite{GilNikPal04}}}
%	\chronoevent[date=false,markdepth=22pt,ifcolorbox=false]{2003}{\tiny{\cite{Nik03}}}
%	\chronoevent[date=false,markdepth=30pt]{2001}{\tiny{\cite{Nik01}}}
%	\chronoevent[date=false,markdepth=40pt,mark=false]{2001}{\tiny{\cite{NikPal01}}}
%	\chronoevent[date=false,ifcolorbox=false]{2000}{\tiny{\cite{Pal00b}}}
%	
%	\chronoevent[date=false,markdepth=30pt]{1999}{\tiny{\cite{ChadMir99JST}}}
%	\chronoevent[date=false]{1993}{\tiny{\cite{NgNik93}}}
%	\chronoevent[date=false,markdepth=40pt]{1991}{\tiny{\cite{KomKuc91b}}}
%	
%	\chronoevent[date=false,markdepth=20pt,mark=false]{1989}{\tiny{\cite{Nik89}}}
%	\chronoevent[date=false]{1989}{\tiny{[KomKuc89]\nocite{KomKuc89b}}}
%	\chronoevent[date=false,markdepth=40pt]{1987}{\tiny{\cite{Nik87a}}}
%	\chronoevent[date=false,ifcolorbox=false,markdepth=50pt,mark=false]{1987}{\tiny{\cite{Nik87c}}}
%	\chronoevent[date=false,ifcolorbox=false]{1986}{\tiny{\cite{Nik86}}}
%	\chronoevent[date=false,ifcolorbox=false]{1984}{\tiny{\cite{Tru84}}}
%	
%	\stopchronology
	
\end{frame}



\begin{frame}{El Teorema de Bernstein--Doetsch.}
	
	\Thm{*}{
	Si $F:D\to \mbox{B}(Y)$ es midconvexa y acotada en un subconjunto $A\subseteq D$
	con interior no-vac\'io entonces, $F$ es continua con respecto a la topolog\'ia 
	de Hausdorff.
	}{\cite{Nik87c}}

	\Thm{*}{
	Si $F:D\to \mbox{B}(Y)$ es midc\'oncava y acotada en un subconjunto $A\subseteq D$
	con interior no-vac\'io entonces, $F$ es continua con respecto a la topolog\'ia 
	de Hausdorff.
	}{\cite{Nik87a}}
\end{frame}

\subsection{Convexidad aproximada.}

\begin{frame}{Convexidad Aproximada.}
	A menos que se especifique otra cosa, $X$ denotar\'a un espacio 
	normado real y $D\subseteq X$ es un conjunto abierto y convexo. 
	\DefiOp{*}{
	Sea $\varepsilon$ una constante positiva. Se dice que 
	la funci\'on $f:D\to \R$ es $\varepsilon$-midconvexa
	si para todo $x,y\in X$ 
	\Eq{*}{
	f\bigg(\cmcvx{x}{y}\bigg)\leq \cmcvx{f(x)}{f(y)} +\varepsilon.
	}
	}{\cite{HyeUla52}}
	\Thm{*}{
	\label{TNgNik}
	Si $f:D\to\R$ es una función localmente acotada superior en un punto de $D$, y
	$\varepsilon$-midconvexa	entonces, $f$ es $2\varepsilon$-convexa, i.e,
	para todo $x,y\in X$ y para todo $t\in[0,1]$
	\Eq{*}{
	f(tx+(1-t)y)\leq tf(x)+(1-t)f(y)+2\varepsilon.
	}
	}{\cite{NgNik93,Lac99}} 	
\end{frame}

\begin{frame}{Función de Takagi}
	\Defi{*}{
	La función $T:\R\to[0,1]$ definida mediante la fórmula 
	\Eq{Tak}{
	T(t) = \sum_{n=0}^{\infty}\frac1{2^n}\dist(2^nt,\Z),\qquad(t\in\R),
	}
	se conoce como la función de Takagi.
	}
	\Rem{*}{
	\Eq{*}{
	T(t) = \ds\lim_{n\to\infty}T_n(t),\quad t\in\R
	}
	donde
	\Eq{*}{
	T_n(t) = T_{n-1}(t) +\textcolor{blue}{\frac1{2^n}\dist(2^nt,\Z)}, \quad n\in\N,t\in\R,
	}
	y además $T_0(t) = \dist(t,\Z).$
	}
\end{frame}

\begin{frame}{Gráfico de la función de Takagi.}
	\begin{figure}[h]
		
			
			\animategraphics[autoplay,palindrome,controls,height=6cm]{2}{prueba1-}{1}{20}
			
			\caption{Función de Takagi.}
	\end{figure}
\end{frame}

\begin{frame}
	\DefiOp{*}{
	Una función $f$ definida en un subconjunto abierto
	y convexo $D$ de un espacio normado real $X,$ es $(\delta,\varepsilon)$-midconvexa
	si satisface
	%\Eq{*}{
	%f(tx+(1-t)y)\leq tf(x)+(1-t)f(y)+\delta t(1-t)\|x-y\|+\varepsilon
	%}
	\Eq{*}{
	f\bigg(\cmcvx{x}{y}\bigg)\leq \cmcvx{f(x)}{f(y)}+\delta\|x-y\|+\varepsilon
	}
	para todo $x,y\in D$.
	}{\cite{HazPal04}}
	
	\Thm{*}{
	%\label{THazPal3}
	Sean $\delta$ y $\varepsilon$ dos números no negativos. Si $f:D\to\R$ es una función
	$(\delta,\varepsilon)$-midconvexa acotada superiormente en un punto de $D$ entonces,
	para todo $x,y\in D$ y para todo $t\in(0,1)$
	\Eq{HazPal33}{
	f(tx+(1-t)y) \leq tf(x)+(1-t)f(y)+2\delta T(t)\|x-y\|+2\varepsilon
	}
	donde $T$ es la función de Takagi definida en \eq{Tak}.
	}{\cite{HazPal04}, Teorema 4}
\end{frame}

\begin{frame}
	\Thm{*}{
	La función de Takagi es $\big(\frac12,0\big)$-midconvexa, i.e.,
	\Eq{*}{
	T\bigg(\cmcvx{x}{y}\bigg)\leq \cmcvx{T(x)}{T(y)}+\Big|\frac{x-y}2\Big|,
	}
	para todo $x,y\in\R$.
	}{\cite{Bor08}}
	\Rem{*}{
	Supongamos que el teorema anterior sigue siendo válido si se reemplaza
	$T$ por una función $\phi$ arbitraria.
	Aplicando el teorema anterior a la función de Takagi, obtenemos
	que para todo $x,y\in\R$ y todo $\lambda\in[0,1]$
	\Eq{*}{
	T(\lambda x+(1-\lambda)y) \leq \lambda T(x)+(1-\lambda)T(y)+\phi(\lambda )|x-y|.
	}
	En particular, para $x=1$ y $y=0$ la ecuación anterior se convierte en
	\Eq{*}{
	T(\lambda) \leq \phi(\lambda).
	}
	}
\end{frame}

\begin{frame}
	\DefiOp{*}{
	Dada una función 
	$\alpha:[0,\infty)\to[0,\infty)$ 
	no decreciente, se dice que una función $f:D\to\R$ es 
	$\alpha(\cdot)$-midconvexa si
	\Eq{*}{
	f\bigg(\frac{x+y}{2}\bigg)\leq\frac{f(x)+f(y)}2+\alpha(\|x-y\|)
	}
	para todo $x,y\in D$.
	}{\cite{TabTab09b}}
	
	\Thm{*}{
	Sea $f:D\to\R$ una función $\alpha(\cdot)$-midconvexa y
	localmente acotada superior en un punto. Entonces $f$ es localmente
	acotada en cada punto de $\mbox{int}D$. Si además, 
	$
	\ds\lim_{r\to0^+}\alpha(r) = 0,
	$
	entonces $f$ es continua en $D$.
	}{\cite{TabTab09b}}	
\end{frame}

\begin{frame}
	\Thm{*}{
	Sea $f:D\to\R$ una función $\alpha(\cdot)$-midconvexa. Entonces,
	\Eq{TabTab2}{
	f(tx+(1-t)y)\leq tf(x)&+(1-t)f(y) \\
	&+\sum_{n=0}^\infty\frac{1}{2^n}\,\alpha\big(\dist(2^nt,\Z)\|x-y\|\big)
	}
	para todo $x,y\in D$, $t\in[0,1]\cap\Q$. Más aún, si $f$ es localmente 
	acotada superior en un punto de $D$, entonces, la desigualdad
	\eq{TabTab2} es válida para todo $t\in[0,1]$.
	}{\cite{TabTab09b,MakPal12a}}
\end{frame}

\begin{frame}
	\Thm{*}{
	Sea $f:D\to\R$ una función $\alpha(\cdot)$-midconvexa. Entonces,
	\Eq{TabTab3}{
	f(tx+(1-t)y)\leq tf(x)&+(1-t)f(y) \\
	&+\sum_{n=0}^\infty\dist(2^nt,\Z)\alpha\bigg(\frac{\|x-y\|}{2^n}\bigg)
	}
	para todo $x,y\in D$, $t\in[0,1]\cap\D$, donde $\D$ es el conjunto
	de los racionales diádicos. Más aún, si $f$ es localmente 
	acotada superior en un punto de $D$ y 
	$$
	\sum_{n=0}^\infty\alpha(1/2^n)<\infty
	$$
	entonces, $f$ es continua en $[0,1]$ y la desigualdad
	\eq{TabTab3} es válida para todo $t\in[0,1]$.
	}{\cite{TabTab09b}}
\end{frame}

\subsection{Convexidad fuerte.}

\begin{frame}{Convexidad fuerte.}
	\DefiOp{*}{
	Sea $c>0$. Se dice que una función $f:D\to\R$ 
	es fuertemente midconvexa con módulo $c$, si 	
	para todo $x,y\in D$ y $t\in[0,1]$
	\Eq{strgMidPol}{
	f\bigg(\frac{x+y}{2}\bigg)\leq \frac{f(x)+f(y)}2-\frac{c}4\|x-y\|^2,
	}
	}{\cite{Pol66}}
	\Thm{*}{
	Sea $c>0$. Si $f:D\to\R$ es una función fuertemente midconvexa 
	con módulo $c$ y acotada superiormente en un subconjunto de $D$ 
	con interior no vacío, entonces, $f$ es una función continua
	y además fuertemente convexa con módulo $c$, i.e.,
	\Eq{strgPol}{
	f(tx+(1-t)y)\leq tf(x)+(1-t)f(y)-ct(1-t)\|x-y\|^2,
	}
	para todo $x,y\in D,$ $t\in[0,1]$.
	}{\cite{AzoGimNikSan11}}
\end{frame}

\begin{frame}{$K$-convexidad}
	Sean $X,Y$ espacios topológicos lineales, $D\subseteq X$
	un subconjunto abierto y convexo y sea $K\subseteq Y$
	un cono convexo. Denotemos por $\U(X)$ y $\U(Y)$
	a las bases locales de $X$ y $Y$ respectivamente.
	\Defi{*}{
	Se dice que la multifunción $F$ es $K$-convexa en $D$ si para todo $x,y\in D$ 
	y para todo $t\in[0,1]$ se tiene que 
	\Eq{Kcvxsvm}{
	tF(x)+(1-t)F(y)\subseteq F(tx+(1-t)y)+K
	}
	}
	\Defi{*}{
	Se dice que la multifunción $F$ es $K$-cóncava en $D$ si para todo $x,y\in D$ 
	y para todo $t\in[0,1]$ se tiene que 
	\Eq{Kccvsvm}{
	F(tx+(1-t)y)\subseteq tF(x)+(1-t)F(y)+K
	}
	}
\end{frame}

\begin{frame}{$K$-convexidad.}
\Rem{cvxSV}{
Dado un cono convexo $K\subseteq Y,$ definamos la relación $\leq_K$ en $Y$
de la siguiente manera
$$
x\leq_K y \iff y-x\in K.
$$
Si $F(x)=\{f(x)\}$, donde $f:X\to Y$ es una función cualquiera, entonces las 
inclusiones \eq{Kcvxsvm} y \eq{Kccvsvm} equivalen a 
\Eq{*}{
f(tx+(1-t)y)&\leq_K tf(x)+(1-t)f(y)
\quad\mbox{y}\\
tf(x)+(1-t)f(y)&\leq_K f(tx+(1-t)y)
}
respectivamente. De hecho si $Y=\R$ y $K=\R_+$, entonces estas definiciones
coinciden con las definiciones de convexidad y concavidad de funciones 
respectivamente introducidas previamente.
}
\end{frame}

\begin{frame}{$K$-continuidad de multifunciones.}
	\Defi{*}{
	Sea $F:D\to\P_0(Y)$ una multifunción. Se dice que $F$ es $K$-semicontinua superior si
	para todo abierto $V\in \U(Y)$, existe un abierto $U\in\U(X)$ tal que
	\Eq{KupSemConSVM}{
	F(x)\subseteq F(x_0)+V+K\qquad(x\in x_0+U)
	}
	}
	\Defi{*}{
	Sea $F:D\to\P_0(Y)$ una multifunción. Se dice que $F$ es $K$-semicontinua inferior si
	para todo abierto $V\in \U(Y)$, existe un abierto $U\in\U(X)$ tal que
	\Eq{KloSemConSVM}{
	F(x_0)\subseteq F(x)+V+K\qquad(x\in x_0+U)
	}
	}
	\Defi{*}{
	Sea $F:D\to\P_0(Y)$ una multifunción. Se dice que $F$ es $K$-continua si
	$F$ es $K$-semicontinua superior e inferior al mismo tiempo.
	}
\end{frame}

\begin{frame}
	\Thm{*}{
	Supongamos que $0\in K$ y sea $F:D\to\P_0(Y)$ una multifunción tal que
	para todo $x\in D$ el conjunto $F(x)\subseteq Y$ es acotado. Si $F$ es 
	$K$-midconvexa y $K$-acotada superior en un subconjunto $H\subseteq D$ con
	interior no vacío, entonces, $F$ es $K$-continua y $K$-convexa en $D$.
	}{\cite{Nik86}}
	
	\Thm{*}{
	Supongamos que $0\in K$ y sea $F:D\to\P_0(Y)$ una multifunción tal que
	para todo $x\in D$ el conjunto $F(x)\subseteq Y$ es cerrado y acotado. 
	Si $F$ es $K$-midcóncava y $K$-acotada inferior en un subconjunto $H\subseteq D$ con
	interior no vacío, entonces, $F$ es $K$-continua y $K$-cóncava en $D$.
	}{\cite{Nik86}}
	
\end{frame}

\begin{frame}{Convexidad Fuerte.}
	Cuando $(X,\|\cdot\|)$ y $(Y,\|\cdot\|)$
	son espacios vectoriales normados
	y  $B_Y\subseteq Y$ es el interior 
	de la bola unitaria en $Y$. 
	\DefiOp{*}{
	Sea $F:D\to\P_0(Y)$ una multifunción y sea $c>0$. Se dice que 
	$F$ es fuertemente convexa con módulo $c$
	si 
	\Eq{StrgCvxSVM}{
	tF(x)+(1-t)F(y)+ct(1-t)\|x-y\|^2\overline{B_Y} \subseteq 
	F(tx+(1-t)y)
	}
	para todo $x,y\in D$ y para todo $t\in[0,1]$.
	}{\cite{Hua10}}
	\Defi{*}{
	Sea $F:D\to\P_0(Y)$ una multifunción y sea $c>0$. Se dice que 
	$F$ es fuertemente midconvexa con módulo $c$ si 
	\Eq{StrgMidCvxSVM}{
	\frac{F(x)+F(y)}2+\frac{c}4\|x-y\|^2\overline{B_Y} \subseteq 
	F\bigg(\frac{x+y}2\bigg).
	}
	}
\end{frame}

\begin{frame}{Convexidad Fuerte.}
	\Thm{*}{
	Sea $F:D\to\P_0(Y)$ una multifunci\'on fuertemente midconvexa con m\'odulo
	$c$ tal que para todo $x\in D$ el conjunto $F(x)$ es cerrado y acotado. Si
	$F$ acotada en un subconjunto de $D$ con interior no-vacío, entonces 
	$F$ es fuertemente convexa con m\'odulo $c$.
	}{\cite{LeiMerNikSan13}}
\end{frame}

\section{Un problema general.}

\subsection{Planteamiento.}
\begin{frame}
	Sean $X,Y$ espacios topológicos lineales, $D\subseteq X$
	un subconjunto abierto y convexo. Sean $A,B:D-D\to\P_0(Y)$
	y $F:D\to\P_0(Y)$.
	\begin{block}{Problema 1}
		Hallar condiciones de regularidad sobre las multifunciones
		$A,B$ y $F$ para que la siguiente inclusión 
		\Eq{cvxHyp}{
		\cmcvx{F(x)}{F(y)}+A(x-y)
		\subseteq F\bigg(\cmcvx{x}{y}\bigg)+B(x-y),\quad(x,y\in D),
		}
		implique una inclusión de tipo
		\Eq{cvxTh}{
		\ccvx{F(x)}{F(y)}+\Phi_A(t,x-y)
		&\subseteq F\big(\ccvx{x}{y}\big)+\Phi_B(t,x-y),\\
		(x,y\in D,&\quad t\in[0,1]),
		}
		donde $\Phi_A$ y $\Phi_B$ son multifunciones definidas en $[0,1]\times(D-D)$
		obtenidas a partir de $A$ y $B$ respectivamente mediante alguna transformación
		especial.
	\end{block}
\end{frame}

\begin{frame}
	Sean $X,Y$ espacios topológicos lineales, $D\subseteq X$
	un subconjunto abierto y convexo. Sean $A,B:D-D\to\P_0(Y)$
	y $F:D\to\P_0(Y)$.
	\begin{block}{Problema 2}
		Hallar condiciones de regularidad sobre las multifunciones
		$A,B$ y $F$ para que la siguiente inclusión 
		\Eq{cvxHyp}{
		F\bigg(\cmcvx{x}{y}\bigg)+A(x-y)
		\subseteq \cmcvx{F(x)}{F(y)}+B(x-y),\quad(x,y\in D),
		}
		implique una inclusión de tipo
		\Eq{cvxTh}{
		F\big(\ccvx{x}{y}\big)+\Phi_A(t,x-y)
		&\subseteq \ccvx{F(x)}{F(y)}+\Phi_B(t,x-y),\\
		(x,y\in D,&\quad t\in[0,1]),
		}
		donde $\Phi_A$ y $\Phi_B$ son multifunciones definidas en $[0,1]\times(D-D)$
		obtenidas a partir de $A$ y $B$ respectivamente mediante alguna transformación
		especial.
	\end{block}
\end{frame}

\subsection{Definiciones y resultados auxiliares.}

\begin{frame}{Condiciones de regularidad.}
\Defi{bdd1}{
Sea $S:D\to\P(Y)$ una multifunción. Decimos que $S$ es \textbf{localmente
semi $K$-acotada inferior} si, para todo $x\in D$ existe un entorno abierto 
$U$ de $x$ y un conjunto acotado $H\subseteq Y$ tal que
\Eq{*}{
S(u)\subseteq\cl(H+K), \qquad (u\in U\cap D). 
}
}
\Defi{bdd2}{
Sea $S:D\to\P(Y)$ una multifunción. Decimos que $S$ es \textbf{localmente
débil semi $K$-acotada superior} si, para todo $x\in D$ existe un entorno abierto 
$U$ de $x$ y un conjunto acotado $H\subseteq Y$ tal que 
\Eq{*}{
0\in\cl(S(u)+H+K), \qquad (u\in U\cap D). 
}
}
\end{frame}

\begin{frame}{Condiciones de regularidad.}
	\Thm{*}{
	Supongamos que $F:D\to\P_0(Y)$ es una multifunción localmente semi-$K$-acotada inferior. 
	Entonces, para cada subconjunto compacto $C\subseteq D$, existe un conjunto acotado 
	$H\subseteq Y$ tal que, $F(x)\subseteq\cl(H+K),$ para todo $x\in C$.
	}{}
	\Thm{*}{
	Supongamos que $F:D\to\P_0(Y)$ es una multifunción localmente débil-semi-$K$-acotada superior. 
	Entonces, para cada subconjunto compacto $C\subseteq D$, existe un conjunto acotado 
	$H\subseteq Y$ tal que, $0\in\cl(F(x)+H+K),$ para todo $x\in C$.
	}{}
\end{frame}
\begin{frame}{Cono Recesión.}
\Defi{RecCone}{
Sea $H\subseteq X$ un subconjunto no-vacío. El \textbf{cono recesión} de $H$ 
denotado por $\rec(H)$ es el conjunto:
\Eq{recH}{
rec(H):=\{x\in X \,|\, tx + H\subseteq H,\mbox{ para todo } t\geq0\}.
}
}
\begin{block}{Propiedades.}
\begin{enumerate}
	\item $\rec(H)$ es un cono convexo que posee al origen;
	\item $K=\rec(H)$ cono más grande tal que $K+H\subseteq H$.
	\item $\overline{\rec}(H)\subseteq\rec(\overline{H});$
	\item para todo $x\in X$, $t>0,$ $\rec(tx+H) = \rec(H);$
	\item para cualesquiera conjuntos no-vacíos $H_1,H_2\subseteq X$,
	$\rec(H_1)+\rec(H_2)\subseteq\rec(H_1+H_2).$
\end{enumerate}
\end{block}
\end{frame}

\begin{frame}{Cono recesión de una multifunción.}
\Defi{recSvm}{
Dada la multifunción $S:D\to\P(Y)$, \textbf{el cono recesión} de $S$,
denotado por $\rec(S)$, es el conjunto:
\Eq{recSvm}{
\rec(S) = \bigcap_{x\in D}\rec(S(x)).
}
}
\begin{block}{Propiedades.}
\begin{enumerate}
	\item $\rec(S)\neq\emptyset$.
	\item Si $S(x)$ es acotada para algún $x\in D$, entonces $\rec(S)=\{0\}$.
	\item $\rec(S)+S(x) \subseteq S(x)$ para todo $x\in D$.
\end{enumerate}
\end{block}
\end{frame}

\subsection{Transformación de Takagi de una multifunción.}
\begin{frame}{Transformación de Takagi de una multifunción.}
Supongamos que $D\subseteq X$ es un conjunto estrellado.
\Defi{TakagiT}{
Para una multifunción $S:D\to\P(Y)$, tal que $0\in S(x)$
para todo $x\in D$, definimos la  
\textbf{transformación de Takagi} de $S$, como la multifunción
$S^T:\R\times D\to\P(Y)$ tal que:
\Eq{TakagiT}{
S^T(t,x):= \cl\Bigg(
\bigcup_{n=0}^\infty\sum_{k=0}^n\frac1{2^n}S\big(2\dist(2^kt,\Z)x\big)
\Bigg).
}
}
\begin{block}{Relación entre $S$ y $S^T$.}
Sea $S:D\to\P(Y)$ una multifunción tal que $0\in S(x)$
para todo $x\in D$. Entonces,
\Eq{S-St}{
\cl(S(x))\subseteq S^T\big(\tfrac12,x\big)\qquad (x\in D).
} 
Además si $S(0)\subseteq\overline{\rec}(S)$ entonces la inclusión \eq{S-St},
se convierte en una igualdad.
\end{block}
\end{frame}

\begin{frame}
	Observe que $d_{\Z}\big(\tfrac12\big)=\tfrac12$ y $d_{\Z}\big(2^k\cdot\tfrac12\big)=0$ para $k\in\N$.
	Por lo tanto,
	\Eq{*}{
		S^T\big(\tfrac12,x\big)
		=\cl\Bigg(\bigcup_{n=0}^{\infty} 
		\sum_{k=0}^{n}\frac{1}{2^k}S\Big(2d_{\Z}\big(2^k\cdot\tfrac12\big)x\Big)\Bigg)
		=\cl\Bigg(S(x)+\bigcup_{n=0}^{\infty} \sum_{k=1}^{n}\frac{1}{2^k}S(0)\Bigg).
	}
	Como $0\in S(0)$, entonces
	\Eq{*}{
	S(x)\subseteq S(x)+\bigcup_{n=0}^{\infty} \sum_{k=1}^{n}\frac{1}{2^k}S(0).
	}
	Luego si $S(0)\subseteq\overline{\rec}(S)\subseteq\rec(\overline{S(x)})$, entonces
	\Eq{*}{
	S^T\big(\tfrac12,x\big)
	\subseteq S(x)+\bigcup_{n=0}^{\infty} \sum_{k=1}^{n}\frac{1}{2^k}\overline{\rec}(S)
	\subseteq \cl\Big(\overline{S(x)}+\rec\big(\overline{S(x)}\big)\Big)\subseteq \cl\big(S(x)\big).
	} 
\end{frame}

\begin{frame}
\begin{block}{Un ejemplo.}
Sea $S_0\subseteq Y$ un conjunto convexo que posee al origen y $\varphi$
una función localmente acotada superior y no-negativa.
\begin{enumerate}
\item $S(x) = K + \varphi(x)S_0 \stackrel{T.T}{\longrightarrow} 
	       S^T(t,x) = \cl\Big(K+\varphi^T(t,x)S_0\Big),$ donde
\Eq{*}{
\varphi^T(t,x) =\sum_{n=0}^\infty\frac1{2^n}\varphi(2\dist(2^nt,\Z)x).
}	
\item Si $X$ es un espacio normado y $\varphi(x) = \|x\|^\alpha$, entonces:
$%\Eq{*}{
\varphi^T(t,x)=T_\alpha(t)\|x\|^\alpha,
$%}
donde la función $T_\alpha(\cdot):\R\to\R$ es la 
función de takagi de orden $\alpha$ y se define por
\Eq{a-tak}{
T_\alpha(t):=\sum_{n=0}^\infty2^{\alpha-n}(\dist(2^nt,\Z))^\alpha
}
\end{enumerate}
\end{block}
\end{frame}

\begin{frame}
	Observemos lo siguiente
	\Eq{a-tak}{
	T_\alpha(t):&=\sum_{n=0}^\infty2^{\alpha-n}(\dist(2^nt,\Z))^\alpha \\
	            &=2^{\alpha}(\dist(t,\Z))^\alpha+\sum_{n=1}^\infty2^{\alpha-n}(\dist(2^nt,\Z))^\alpha\\
							&=2^{\alpha}(\dist(t,\Z))^\alpha+\sum_{n=0}^\infty2^{\alpha-(n+1)}(\dist(2^{n+1}t,\Z))^\alpha\\
							&=2^{\alpha}(\dist(t,\Z))^\alpha+\frac12\sum_{n=0}^\infty2^{\alpha-n}(\dist(2^n(2t),\Z))^\alpha
	}
	Por lo tanto $T_\alpha$ es la \textbf{única} solución de la ecuación funcional
	\begin{block}{}
	\Eq{*}{
	\phi(t) = 2^{\alpha}(\dist(t,\Z))^\alpha+\frac12\phi(2t).
	}
	\end{block}
\end{frame}

\begin{frame}{Caso $\alpha=2$.}
\vspace{-.6cm}
\Eq{*}{
\textcolor{blue}{f(t)=4(t-\lfloor t\rfloor)(1-(t-\lfloor t\rfloor))},\quad
\textcolor{black}{g(t)=4(\dist(t,\Z))^2},\quad
\textcolor{red}{h(t)=\frac12f(2t)}.
}
$$f(t) = 4(\dist(t,\Z))^2+\frac12f(2t).$$

\vspace{-.5cm}
	\begin{figure}[h]
		
			\animategraphics[autoplay,height=5.5cm,controls]{5}{./dibujos2/SolTak2-}{53}{181}
			
			%\caption{$\phi(t) = 2^{\alpha}(\dist(t,\Z))^\alpha+\frac12\phi(2t).$}
	\end{figure}
	%$$T_2(t)=4t(1-t)$$
\end{frame}

\begin{frame}
\Rem{*}{
	%Cuando $\alpha=2$, se puede verificar sin dificultad que la función de periodo 1
	%definida para $t\in[0,1]$ mediante la fórmula $h(t)=4t(1-t)$ también satisface dicha
	%ecuación funcional y en consecuencia 
	%%\begin{block}{}
	%\Eq{*}{
	%T_2(t) = 4t(1-t), \qquad(t\in[0,1]).
	%}
	%%\end{block}
	Luego, si $S(x) = K + \frac{c}{4}\|x\|^2S_0,$ entonces,
	%\begin{block}{}
	\Eq{*}{
	S^T(t,x) = \cl\big(K+ct(1-t)\|x\|^2\,S_0\big)
	}
	%\end{block}
}

	\begin{block}{Lema auxiliar}
	Sean $(A_n)_{n\in\N},(B_n)_{n\in\N}$ dos sucesiones no-decrecientes
	de subconjuntos de $X$, sea $H\subseteq X$ un conjunto
	acotado, sea $K\subseteq\bigcap_{n\in\N}\cl(\rec(B_n))$ y sea
	$(\epsilon_n)_{n\in\N}$ una sucesión de números reales que converge a 
	cero. Asumamos que, para todo $n\geq0,$
	\Eq{*}{
	A_n\subseteq\cl(\epsilon_nH+K+B_n).
	}
	Entonces,
	\Eq{*}{
	\cl\Bigg(\bigcup_{n=0}^\infty A_n\Bigg) \subseteq
	\cl\Bigg(\bigcup_{n=0}^\infty B_n\Bigg)
	}
	\end{block}
\end{frame}

\section{Resultados principales.}
\subsection{Teoremas.}
\begin{frame}
\Thm{*}{
Sea $D\subseteq X$ un conjunto convexo no-vacío y $A,B:(D-D)\to\P(Y)$, tales 
que $0\in A(x)\cap B(x)$ para todo $x\in (D-D).$ Consideremos $K=\overline{\rec}(B)$
y sea $F:D\to\P(Y)$ una multifunción que satisface 
para $x,y\in D$ la inclusión de tipo Jensen
\Eq{JCvx}{
\frac{F(x)+F(y)}2 + A(x-y)\subseteq
			\cl\bigg(F\Big(\frac{x+y}2\Big)+B(x-y)\bigg).
}
Supongamos que además $F$ es puntualmente semi $K$-acotada inferior
y localmente débil semi-$K$-acotada superior en $D$. Entonces, $F$ satisface 
para $x,y\in D$ y $t\in[0,1]$, la siguiente inclusión:
\Eq{Cvx}{
tF(x)+(1-t)F(y) &+ A^T(t,x-y) \\
&\subseteq \cl\big( F(tx+(1-t)y) + B^T(t,x-y)\big). 
}
}{\cite{GonNikPalRoa14}, Theorem 4.1}
\end{frame}

\begin{frame}[allowframebreaks]{Resumen de la demostración.}
	
	\begin{block}{}	
	Vamos a demostrar por inducción que para todo $x,y\in D$
	existe un conjunto acotado $H\subseteq Y$ tal que, para todo $n\geq 0$, $t\in[0,1]$, 
	\Eq{*}{
	tF(x)&+(1-t)F(y) + \sum_{k=0}^{n-1}{\dfrac{1}{2^k}A\big(2 d_{\Z}(2^k t)(x-y)\big)} \\
	&\subseteq \cl\bigg(F(tx+(1-t)y) + \dfrac{1}{2^n} H + K +
		\sum_{k=0}^{n-1}{\dfrac{1}{2^k}B\big(2 d_{\Z}(2^k t)(x-y)\big)}\bigg). 
	}
	\end{block}
	\begin{block}{}
	Fijemos $x,y\in D$ arbitrarios.
	Para $n=0$ basta encontrar un conjunto acotado $H\subseteq Y$ tal que para todo
	$t\in[0,1]$, 
	\Eq{*}{
		tF(x)+(1-t)F(y)\subseteq \cl\big(F(tx+(1-t)y) + H + K\big).
	}
	\end{block}
	
	\begin{block}{}
	Sea $U\in\U(Y)$ y escojamos un abierto balanceado $V\in\U(Y)$ tal que $V+V+V\subseteq U$.
	Como $F$ es puntualmente semi-$K$-acotada inferior, existen conjuntos acotados $H_x,H_y\subseteq Y$
	tales que 
	\Eq{*}{
	F(x) \subseteq \cl(H_x + K)\subseteq V+H_x+K
	}
	y
	\Eq{*}{
	F(y) \subseteq \cl(H_y + K) \subseteq V+H_y + K.
	} 
	Por lo tanto
	\Eq{*}{
	tF(x)+(1-t)F(y)\subseteq V + V + H_1 + H_2 + K.
	}
	donde $H_1:=\bigcup_{t\in[0,1]} tH_x$ y $H_2:=\bigcup_{t\in[0,1]} (1-t)H_y.$	
	Por otra parte, 
	%como $F$ es localmente débil-semi-$K$-acotada superior y el segmento 
	%$[x,y]$ es compacto, entonces, 
	existe un conjunto acotado $H_0\subseteq Y$ 
	tal que para todo $t\in[0,1]$
	\Eq{*}{
	0\in \cl(F(tx+(1-t)y) + H_0 + K)\subseteq V + F(tx+(1-t)y) + H_0 + K.
	}
	\end{block}
	
	\begin{block}{}
	Luego, para todo $t$ en $[0,1]$, 
	\Eq{*}{
	tF(x)+(1-t)F(y)\subseteq U + F(tx+(1-t)y) + H_0 + H_1 + H_2 + K.
	} 
	Por lo tanto,
	\Eq{*}{
	tF(x)+(1-t)F(y)&\subseteq \bigcap_{U\in\U(Y)}\big(U + F(tx+(1-t)y) + H_0 + H_1 + H_2 + K\big)\\
									&=\cl\big(F(tx+(1-t)y) + H + K\big),
	}
	donde $H:=H_0 + H_1 + H_2.$
	
	Supongamos ahora que la hipotesis inductiva es cierta para $n$ y demostraremos
	que también lo es para $n+1$.
	\end{block}
	
	\begin{block}{}
	Veamos que para $t\in[0,1/2]$ se tiene lo siguiente\vspace{-.3cm}
	\Eq{*}{
	tF(x)&+(1-t)F(y) + \sum_{k=0}^{n}{\dfrac{1}{2^k}A\big(2 d_{\Z}(2^k t)(x-y)\big)} \\
	&\subseteq \cl\bigg(F(tx+(1-t)y) + \dfrac{1}{2^{n+1}} H + K +
		\sum_{k=0}^{n}{\dfrac{1}{2^k}B\big(2 d_{\Z}(2^k t)(x-y)\big)}\bigg). 
	}
	\end{block}
	
	\begin{block}{}
	\Eq{*}{
		&tF(x)+\frac{2-2t}2F(y)+ A\big(2t(x-y)\big) 
		+ \sum_{k=1}^{n}{\dfrac{1}{2^k}A\big(2 d_{\Z}(2^kt)(x-y)\big)} \\
		&\subseteq \dfrac{1}{2}\bigg( 2tF(x)+(1-2t)F(y)
		+ \sum_{k=0}^{n-1}{\dfrac{1}{2^k}A\big(2 d_{\Z}(2^k(2t))(x-y)\big)} \bigg) \\
		&\hspace{5cm}+ \frac{1}{2}F(y) \!+\! A\big(2t(x-y)\big).
	}
	\end{block}
	
	\begin{block}{}

	\Eq{*}{
 & tF(x)+(1-t)F(y) + \sum_{k=0}^{n}{\dfrac{1}{2^k}A\big(2 d_{\Z}(2^k t)(x-y)\big)} \\
 & \subseteq \dfrac{1}{2}\cl\bigg(F(2tx + (1-2t)y)+ \frac{1}{2^n}H + K 
   + \sum_{k=0}^{n-1} \dfrac{1}{2^k}B\big(2 d_{\Z}(2^k(2t))(x-y)\big)\bigg) \\
 &\hspace{5cm} + \frac{1}{2}F(y) \!+\! A\big(2t(x-y)\big)\\
 & \subseteq \cl\bigg(\dfrac{F(2tx + (1-2t)y)+ F(y)}{2}+ A\big(2t(x-y)\big) 
     + \frac{1}{2^{n+1}}H + K\\
&\hspace{5cm}		+ \sum_{k=0}^{n-1}\dfrac{1}{2^{k+1}}B\big(2d_{\Z}(2^k(2t))(x-y)\big)\bigg)\\
& \subseteq  \cl\bigg(F\big(tx+(1-t)y\big) + \frac{1}{2^{n+1}}H + K 
  + \sum_{k=0}^{n}\dfrac{1}{2^{k}}B\big(2d_{\Z}(2^k t)(x-y)\big)\bigg).
	}
	\end{block}
	
	\begin{block}{}
	Aplicando el lema auxiliar obtenemos lo siguiente
	\Eq{*}{
  \cl\bigg(tF(x)&+(1-t)F(y) + \cl\bigg(\bigcup_{n=1}^\infty
     \sum_{k=0}^{n-1}\dfrac{1}{2^k}A(2 d_{\Z}(2^k t)(x-y))\bigg)\bigg)\\
  &\subseteq \cl\bigg(F\big(tx+(1-t)y\big) +\cl\bigg(\bigcup_{n=1}^\infty 
     \sum_{k=0}^{n-1}\dfrac{1}{2^{k}}B\big(2d_{\Z}(2^k t)(x-y)\big)\bigg)\bigg).
	}
	\end{block}
	
	\begin{block}{}
	Finalmente,
	\Eq{*}{
	tF(x)+(1-t)F(y) &+ A^T(t,x-y) \\
	&\subseteq \cl\big( F(tx+(1-t)y) + B^T(t,x-y)\big). 
	}
	\end{block}
	
	%\Eq{*}{
		%tF(x)&+(1-t)F(y) + \sum_{k=0}^{n}{\dfrac{1}{2^k}A\big(2 d_{\Z}(2^k t)(x-y)\big)} \\
		%&= tF(x)+(1-t)F(y)+ A\big(2t(x-y)\big) 
			%+ \sum_{k=1}^{n}{\dfrac{1}{2^k}A\big(2 d_{\Z}(2^kt)(x-y)\big)} \\
		%&\subseteq tF(x)+(1-t)F(y)+ A\big(2t(x-y)\big) 
			%+ \sum_{k=1}^{n}{\dfrac{1}{2^k}A\big(2 d_{\Z}(2^kt)(x-y)\big)}.
	%}
	
	
\end{frame}


\begin{frame}

\Thm{*}{
Sea $D\subseteq X$ un conjunto convexo no-vacío y $A,B:(D-D)\to\P_0(Y)$ multifunciones tales que
$0\in A(x)\cap B(x)$ para todo $x\in (D-D)$. Sea $K=\overline{\rec}(B)$, y 
sea $F:D\to\P_0(Y)$ una multifunción que satisface
para $x,y\in D$ la inclusión de tipo Jensen
\Eq{JCC}{
F\pr{\dfrac{x+y}{2}} + A(x-y) \subseteq \cl\bigg(\dfrac{F(x) + F(y)}{2} + B(x-y)\bigg).
}
Supongamos además, que $F$ es puntualmente semi-$K$-convexa y que
$F$ es localmente semi-$K$-acotada inferior.
Entonces, $F$ satisface para $x,y\in D$ y $t\in[0,1]$, 
la siguiente inclusión
\Eq{CC}{
 F(tx+(1-t)y)&+A^T(t,x-y) \\
&\subseteq \cl\big(tF(x)+(1-t)F(y)+B^T(t,x-y)\big)
}
}{\cite{GonNikPalRoa14}, Theorem 4.2}
\end{frame} 

%\begin{frame}
	%The next four corollaries generalizes some of the results obtained
	%by Averna, Cardinali,Nikodem, Papalini and Borwein
	%\cite{AveCar90,Bor77,CarNikPap93, Nik86, Nik87a,Nik87c,Nik89,Pap90}
	%related to $K$-convex set valued functions. Also, results from Az\'ocar, 
	%Gim\'enez, Nikodem and S\'anchez for strongly convex functions can be derived 
	%as a direct consecuence of them and for the set valued setting the results 
	%from Leiva, Merentes, Nikodem and S\'anchez related to strongly convex
	%set valued maps can be obtained too.
	%
%\end{frame}



\subsection{Corolarios.}


\begin{frame}
Sea $\varphi:D-D\to\R_+$ una función no-negativa localmente acotada superior,
$S_0\subseteq Y$ un conjunto convexo que contiene al origen $0\in Y$ y $K\subseteq Y$ 
un cono convexo y cerrado.
\Cor{*}{
Supongamos que $F:D\to\P_0(Y)$ es una multifunción puntualmente semi-$K$-acotada inferior y 
localmente débil-semi-$K$-acotada superior que satisface
\Eq{JCV+1}{
\dfrac{F(x) + F(y)}{2} \subseteq 
\cl\bigg(F\pr{\dfrac{x+y}{2}} + K + \varphi(x-y)S_0 \bigg) 
}
para todo $x,y\in D.$ Entonces
\Eq{CV+1}{
 tF(x)+(1-t)F(y) \subseteq \cl\big(F(tx+(1-t)y)+K+\varphi^T(t,x-y)S_0\big)
}
para todo $x,y\in D$ y para todo $t\in[0,1].$
}{\cite{GonNikPalRoa14}, Corollary 4.4}
\end{frame}

\begin{frame}
\Cor{*}{
Supongamos que $F:D\to\P_0(Y)$ es una multifunción puntualmente semi-$K$-acotada inferior y 
localmente débil-semi-$K$-acotada superior que satisface
\Eq{JCV+2}{
\dfrac{F(x) + F(y)}{2} + \varphi(x-y)S_0 \subseteq \cl\bigg(F\pr{\dfrac{x+y}{2}} + K \bigg) 
}
para todo $x,y\in D.$ Entonces
\Eq{CV+2}{
 tF(x)+(1-t)F(y) + \varphi^T(t,x-y)S_0 \subseteq \cl\big(F(tx+(1-t)y) + K \big)
}
para todo $x,y\in D$ y para todo $t\in[0,1].$
}{\cite{GonNikPalRoa14}, Corollary 4.5}
\end{frame}


\begin{frame}
\Cor{*}{
Supongamos que $F:D\to\P_0(Y)$ es una multifunción puntualmente semi-$K$-convexa y
localmente semi-$K$-acotada inferior que satisface
\Eq{JCC+1}{
F\pr{\dfrac{x+y}{2}} \subseteq \cl\bigg(\dfrac{F(x) + F(y)}{2} + K + \varphi(x-y)S_0 \bigg)
}
para todo $x,y\in D.$ Entonces
\Eq{CC+1}{
 F(tx+(1-t)y) \subseteq \cl\big(tF(x)+(1-t)F(y) + K + \varphi^T(t,x-y)S_0 \big)
}
para todo $x,y\in D$ y para todo $t\in[0,1].$
}{\cite{GonNikPalRoa14}, Corollary 4.6}
\end{frame}

\begin{frame}
\Cor{*}{
Supongamos que $F:D\to\P_0(Y)$ es una multifunción puntualmente semi-$K$-convexa y
localmente semi-$K$-acotada inferior que satisface
\Eq{JCC+2}{
F\pr{\dfrac{x+y}{2}} + \varphi(x-y)S_0 \subseteq \cl\bigg(\dfrac{F(x) + F(y)}{2} + K\bigg)
}
para todo $x,y\in D.$ Entonces
\Eq{CC+2}{
 F(tx+(1-t)y)+\varphi^T(t,x-y)S_0 \subseteq \cl\big(tF(x)+(1-t)F(y) + K \big)
}
para todo $x,y\in D$ y para todo $t\in[0,1].$
}{\cite{GonNikPalRoa14}, Corollary 4.7}
\end{frame}

\subsection{El Teorema de Bernstein--Doetsch.}

\begin{frame}{El Teorema de Bernstein--Doetsch.}
	\Thm{*}{
		Toda función Jensen-convexa $f:D\to\R$ en $D$, localmente acotada superior en un punto 
		$x_0\in D$ is continua y por lo tanto convexa en $D$.
	}{\cite{Kuc85}, Teorema 6.4.2}
	Este teorema teorema fue formulado por F. Bernstein and G. Doetsch en 1915 
	\cite{BerDoe15}, y desde entonces ha sido muy importante en la teoría de convexidad,
	razón por la cual ha sido generalizado de muchas maneras diferentes y por varios autores.
	Como consecuencia directa se tiene el siguiente
	\Cor{*}{
		Una función $f:D\to\R$ es convexa si y sólo si es 
		continua y Jensen-convexa.
	}{\cite{Kuc85}, Teorema 7.1.1}
\end{frame}

\subsection{Convexidad aproximada}

\begin{frame}{Funciones aproximadamente convexas.}
	Sea $\alpha:\R^+\to\R^+$, una función no-decreciente.
	\Defi{e-Cvx}{
		Se dice que la función, $f:D\to\R$ es \textbf{$\alpha$-Jensen-convexa}
		en $D,$ si para todo $x,y \in D$
		\Eq{e-Cvx}{
			f\bigg(\frac{x+y}2\bigg) \leq \frac{f(x)+f(y)}2 + \alpha(|x-y|).
		}
	}
	\Rem{*}{
		Cuando $\alpha(x)=\epsilon>0,$ entonces, $\alpha$-Jensen convexidad es simplemente,
		$\epsilon$-Jensen convexidad \cite{HyeUla52}. Resultados de tipo B-D para este tipo de convexidad
		fueron obtenidos por Ng y Nikodem en 1993 \cite{NgNik93}.
	}
\end{frame}

\begin{frame}
	\Thm{*}{
		Sea $D\subseteq \R$ un subconjunto abierto y convexo de la recta real.
		Si $f:D\to\R$ es una función localmente acotada superior en un punto y 
		$\alpha$-Jensen-convexa en $D$, entonces, para todo $x,y\in D$ y para todo
		$t\in[0,1]$ 
		$$
		f(tx+(1-t)y)\leq tf(x)+(1-t)f(y)+\T_\alpha(t,|x-y|),
		$$
		donde 
		$$
		\T_\alpha(t,u)=\sum_{n=0}^{\infty}\frac{1}{2^n}\alpha(2\dist_\Z(2^nt)u),\quad
		t\in[0,1], u\in D-D.
		$$
	}{\cite{HazPal04,MakPal10b}}
	\Rem{*}{
		Si $\alpha(u)=\epsilon |u|^p + \delta,$ con $\epsilon,\delta,p >0$ y $u\in\R$. Entonces 
		$$
		\T_\alpha(t,u)=\epsilon\Bigg(\sum_{n=0}^{\infty}2^{p-n}\dist_\Z^p(2^nt)\Bigg)|u|^p
		+2\delta = \epsilon T_p(t)|u|^p + 2\delta.
		$$
	}	
\end{frame}

\begin{frame}
	\Thm{*}{
		Sea $D\subseteq \R$ un subconjunto abierto y convexo de la recta real.
		Si $\sum_{n=0}^\infty\alpha(2^{-n})<\infty$ y
		$f:D\to\R$ es una función localmente acotada superior en un punto y 
		$\alpha$-Jensen-convexa en $D$, entonces, para todo $x,y\in D$ y para todo
		$t\in[0,1]$ 
		$$
		f(tx+(1-t)y)\leq tf(x)+(1-t)f(y)+\MS_\alpha(t,\|x-y\|),
		$$
		donde 
		$$
		\MS_\alpha(t,u)=\sum_{n=0}^{\infty}2\dist_\Z(2^nt)\alpha\Big(\frac{u}{2^{n+1}}\Big),\quad
		t\in[0,1], u\in D-D.
		$$
	}{\cite{TabTab09b}}
	\Rem{*}{
		Si $\alpha(u)=\epsilon |u|^p,$ con $\epsilon,p >0$ y $u\in\R$. Entonces 
		$$
		\MS_\alpha(t,u)=\epsilon\Bigg(\sum_{n=0}^{\infty}
		\frac{\dist_\Z(2^nt)}{2^{np+p+1}} \Bigg)|u|^p 
		= \epsilon S_p(t)|u|^p.
		$$
	}
\end{frame}

\begin{frame}{Funciones fuertemente convexas.}
	Sea $c$ un número real positivo. Siguiendo a Polyak, \cite{Pol66}
	\Defi{StrCvx}{
		Una función $f:D\to \R$ es \textbf{fuertemente convexa}
		con módulo $c$ si para todo $x,y\in D$ y para todo $t\in[0,1]$
		\Eq{StrCvx}{
			f(tx+(1-t)y)\leq tf(x) + (1-t)f(y) - ct(1-t)|x-y|^2 
		}
	}
	\Thm{*}{
		Si $f:D\to\R$ es fuertemente Jensen-convexa con módulo $c$,
		y localmente acotada superior en un punto de $D$ entonces 
		$f$ es continua y fuertemente convexa con módulo $c$.
	}{\cite{AzoGimNikSan11}, Teorema 2.3}
\end{frame}

\section{Multifunciones.}
\subsection{Terminología básica.}

\begin{frame}{Multifunciones $K$-Convexas.}
	Sean $X,Y$ espacios topológicos lineales, $K\subseteq Y$
	un cono convexo cerrado y $D\subseteq X$ un conjunto convexo
	y abierto.
	Denote por $\P(Y)$ a la clase de subconjuntos no-vacios de $Y$.
	\Defi{KCvxSvm}{
		Una multifunción $F:D\to \P(Y)$ 
		es \textbf{$K$-convexa} en $D$, si para todo $x,y\in D$ y todo 
		$t\in [0,1]$
		\Eq{KCvxSvm}{
			tF(x) + (1-t)F(y) \subseteq F(tx+(1-t)y) + K.
		}
	}
\end{frame}

\begin{frame}{Cono Recesión.}
	\Defi{RecCone}{
		Sea $H\subseteq X$ un conjunto no vacío. El \textbf{cono recesión} de $H$ 
		denotado por $\rec(H)$ es el conjunto
		\Eq{recH}{
			rec(H):=\{x\in X \,|\, tx + H\subseteq H,\mbox{ for all } t\geq0\}.
		}
	}
	\begin{block}{Propiedades.}
		\begin{enumerate}
			\item $\rec(H)$ es un cono convexo que contiene al origen;
			\item $K=\rec(H)$ es el cono más grande con la propiedad $K+H\subseteq H$;
			\item $\overline{\rec}(H)\subseteq\rec(\overline{H});$
			\item para todo $x\in X$, $t>0,$ $\rec(tx+H) = \rec(H);$
			\item %Para todo par de conjuntos $H_1,H_2\subseteq X$,
			$\rec(H_1)+\rec(H_2)\subseteq\rec(H_1+H_2),$ para todo $H_1,H_2\subseteq X$.
		\end{enumerate}
	\end{block}
\end{frame}

%\begin{frame}{Cono Recesión de una multifunción.}
%\Defi{recSvm}{
%Given a set valued map $S:D\to\P(Y)$, \textbf{the recession cone} of $S$,
%denoted by $\rec(S)$, is the set:
%\Eq{recSvm}{
%\rec(S) = \bigcap_{x\in D}\rec(S(x)).
%}
%}
%\begin{block}{Properties}
%\begin{enumerate}
%\item $\rec(S)\neq\emptyset$.
%\item If $S(x)$ is bounded for some $x\in D$, then $\rec(S)=\{0\}$.
%\item $\rec(S)+S(x) \subseteq S(x)$ for all $x\in D$.
%\end{enumerate}
%\end{block}
%\end{frame}

\begin{frame}{Multifunciones acotadas.}
	\Defi{bdd1}{
		Sea $S:D\to\P(Y)$ una multifunción. Se dice que $S$ es 
		\textbf{localmente semi-$K$-acotada inferior} si para todo $x\in D$ existe
		un entorno abierto $U\subseteq X$ de $x$ y un conjunto acotado $H\subseteq X$,
		tal que 
		\Eq{*}{
			S(u)\subseteq\cl(H+K), \qquad (u\in U\cap D). 
		}
	}
	\Defi{bdd2}{
		Sea $S:D\to\P(Y)$ una multifunción. Se dice que $S$ es 
		\textbf{localmente débil-semi-$K$-acotada superior} si para todo $x\in D$ existe
		un entorno abierto $U\subseteq X$ de $x$ y un conjunto acotado $H\subseteq X$,
		tal que
		\Eq{*}{
			0\in\cl(S(u)+H+K), \qquad (u\in U\cap D). 
		}
	}
\end{frame}

\begin{frame}{$K$-continuidad direccional.}
	%Sea $F:D\to\P(Y)$ una multifunción. 
	\Defi{*}{
		Decimos que $F$ es direccionalmente $K$-semicontinua superior en un punto 
		$p\in D$, si para toda dirección $h\in X$ y para todo entorno abierto 
		$U$ de $0\in Y$, existe un nùmero positivo $\delta$ tal que
		$$
		F(p+th)\subseteq F(p) + U +K,
		$$
		para todo $t\in(0,\delta)$ tal que $p+th\in D.$
	}
	\Defi{*}{
		Decimos que $F$ es direccionalmente $K$-semicontinua inferior en un punto 
		$p\in D$, si para toda dirección $h\in X$ y para todo entorno abierto 
		$U$ de $0\in Y$, existe un nùmero positivo $\delta$ tal que
		$$
		F(p)\subseteq F(p+th) + U +K,
		$$
		para todo $t\in(0,\delta)$ tal que $p+th\in D.$
	}
\end{frame}

\subsection{Transformación de Takagi-Tabor.}
\begin{frame}
	Asumamos que $D\subseteq X$ es un conjunto estrellado.
	\Defi{TakTabT}{
		Para una multifunción $S:D\to\P(Y)$, tal que $0\in S(x)$
		para todo $x\in D$, definimos la 
		\textbf{transformación de Takagi-Tabor} de $S$, como la multifunción
		$S^\perp:\R\times D\to\P(Y)$ tal que
		\Eq{TakagiT}{
			S^\perp(t,x):= \cl\Bigg(
			\bigcup_{n=0}^\infty\sum_{k=0}^n2\dist(2^kt,\Z)S\bigg(\frac{x}{2^{n+1}}\bigg)
			\Bigg).
		}
	}
	\begin{block}{Relación entre $S$ y $S^\perp$.}
		Sea $S:D\to\P(Y)$ una multifunción tal que $0\in S(x)$
		para todo $x\in D$. Entonces
		\Eq{S-St}{
			S^\perp\big(\tfrac12,x\big)=\cl(S(x))\qquad (x\in D).
		} 
	\end{block}
\end{frame}

\section{Resultados principales}

%\begin{frame}
	%\begin{block}{Observaciones.}
	%\begin{enumerate}
		%\item Ya vimos una posible solución a los Problemas 1 y 2 respectivamente. 
		%En este caso, consideramos las condiciones de regularidad mencionadas 
		%y se obtuvo el resultado deseado al considerar $\Phi_A:=A^T$ y
		%$\Phi_B:=B^T$.
		%
		%\item Bajo que condiciones de regularidad se obtendrá otra posible solución a los
		%Problemas 1 y 2 respectivamente, si ahora definimos para $t\in[0,1]$ y $u\in D-D$
		%$$
		%\Phi_A(t,u)=A^\perp(t,u):=\cl\Bigg(
		%\bigcup_{n=0}^\infty\sum_{k=0}^n2\dist(2^kt,\Z)A\bigg(\frac{u}{2^n}\bigg)
		%\Bigg).
		%$$
		%\cite{GilGonNikPal15}
	%\end{enumerate}
	%\end{block}
%\end{frame}

\section{Referencias}
\begin{frame}[allowframebreaks]
  \frametitle{References}
  \scriptsize{\bibliographystyle{amsalpha}}%amsalpha
  \bibliography{funcequ,publ}
  \beamertemplatearticlebibitems
\end{frame}

\end{document}
