\documentclass[notheorems,envcountsect,serif,mathserif,professionalfonts,spanish,10pt]{beamer}
%envcountsect,serif,mathserif,professionalfont
%\setbeamertemplate{theorems}[numbered]

\setbeamertemplate{bibliography item}[text]

\usepackage[english]{babel}
\usepackage[utf8]{inputenc}

\usepackage{times,ifthen,lmodern,pxfonts}
\usepackage[T1]{fontenc}
\usepackage{mathrsfs} 

\newcommand{\phii}{\varphi}
\newcommand{\R}{\mathbb{R}}
\newcommand{\Q}{\mathbb{Q}}
\newcommand{\N}{\mathbb{N}}
\newcommand{\Z}{\mathbb{Z}}
\renewcommand{\P}{\mathscr{P}}

\newcommand{\ds}{\displaystyle}
\newcommand{\implica}{\Rightarrow}
\newcommand{\suma}[2]{\ds\sum_{k = #1}^{#2} }
\newcommand{\pr}[1]{\left( #1\right) }
\newcommand{\ch}[1]{\left[ #1\right] }

\newcommand{\dist}{\mathop{\hbox{\rm dist}}\nolimits}
\newcommand{\conv}{\mathop{\hbox{\rm conv}}\nolimits}
\newcommand{\rec}{\mathop{\hbox{\rm rec}}\nolimits}
\newcommand{\rsg}{\mathop{\hbox{\rm rsg}}\nolimits} 
\newcommand{\cl}{\mathop{\hbox{\rm cl}}\nolimits}
\newcommand{\dx}{\mathop{\hbox{\rm dx}}\nolimits}
\newcommand{\Dom}{\mathop{\hbox{\rm Dom}}\nolimits}


\newtheorem{theorem}{Teorema}
\renewcommand{\thetheorem}{\textrm{\thesection.\arabic{theorem}}}
\newtheorem{theorem*}{Teorema}
\def\Thm#1#2#3{\ifthenelse{\equal{#1}{*}}{\begin{theorem*}[#3]#2\end{theorem*}}
  {\begin{theorem}[#3]\label{T#1}#2\end{theorem}}}
\newtheorem{Atheorem}{Theorem}
\renewcommand{\theAtheorem}{\Alph{Atheorem}}
\def\THM#1#2{\begin{Atheorem}\label{T#1}#2\end{Atheorem}}
\def\thm#1{Teorema~\ref{T#1}}
\newtheorem{proposition}{Proposición}
\newtheorem{proposition*}{Proposición}
\def\Prp#1#2{\ifthenelse{\equal{#1}{*}}{\begin{proposition*}#2\end{proposition*}}
             {\begin{proposition}\label{P#1}#2\end{proposition}}}
\def\prp#1{Proposición~\ref{P#1}}
\newtheorem{corollary}[theorem]{Corolario}
\newtheorem*{corollary*}{Corolario}
\def\Cor#1#2#3{\ifthenelse{\equal{#1}{*}}{\begin{corollary*}[#3]#2\end{corollary*}}
             {\begin{corollary}[#3]\label{C#1}#2\end{corollary}}}
\def\cor#1{Corolario~\ref{C#1}}
\newtheorem{lemma}{Lema}
\newtheorem{lemma*}{Lema}
\def\Lem#1#2{\ifthenelse{\equal{#1}{*}}{\begin{lemma*}#2\end{lemma*}}
             {\begin{lemma}\label{L#1}#2\end{lemma}}}
\def\lem#1{Lema~\ref{L#1}}
\newtheorem{example}{Ejemplo}
\newtheorem{example*}{Ejemplo}
\long\def\Exa#1#2{\ifthenelse{\equal{#1}{*}}{\begin{example*}\rm #2\end{example*}}
            {\begin{example}\label{Ex#1}\rm #2\end{example}}}
\def\exa#1{Example~\ref{E#1}}
\newtheorem{problem}[subsection]{Problem}
\def\Prob#1#2{\begin{problem}\label{Prob#1}\rm #2\end{problem}}
\def\prob#1{Problem~\ref{Prob#1}}
\theoremstyle{definition}
\newtheorem{definition}{Definición}
\newtheorem*{definition*}{Definición}
\def\Defi#1#2{\ifthenelse{\equal{#1}{*}}{\begin{definition*}#2\end{definition*}}
      {\begin{definition}\label{D#1}#2\end{definition}}}
\def\defi#1{Definición~\ref{D#1}}
\newtheorem{remark}{Observación}
\newtheorem{remark*}{Observación}
\long\def\Rem#1#2{\ifthenelse{\equal{#1}{*}}{\begin{remark*}#2\end{remark*}}
             {\begin{remark}\label{R#1}#2\end{remark}}}
\def\rem#1{Observación~\ref{R#1}}
\def\eq#1{{\rm(\ref{E#1})}}
\renewcommand{\theequation}{\thesection.\arabic{equation}}
\def\Eq#1#2{\ifthenelse{\equal{#1}{*}}
  {\begin{equation*}\begin{aligned}[]#2\end{aligned}\end{equation*}}
  {\begin{equation}\begin{aligned}[]\label{E#1}#2\end{aligned}\end{equation}}}

\mode<presentation>
{
  \usetheme{Warsaw}
	%\usetheme{progressbar}%Darmstadt}
%	\usecolortheme{crane}
  % or ...

  %\setbeamercovered{transparent}
  % or whatever (possibly just delete it)
}


% Or whatever. Note that the encoding and the font should match. If T1
% does not look nice, try deleting the line with the fontenc.


\title[Teoremas de tipo Bernstein--Doetsch.]% (optional, use only with long paper titles)
{Teoremas de tipo Bernstein-Doetsch para multifunciones 
convexas y c\'oncavas.}


\author[C. Gonz\'alez.]
{
Carlos~Gonz\'alez.
  }
\institute[Central University of Venezuela]
{
  Universidad Central de Venezuela.
}
\date[\today] % (optional, should be abbreviation of conference name)
{Presentaci\'on para optar al t\'itulo de Magister Scientiarium en Matem\'atica.}

\subject{Mathematical analysis, Inequalities.}
% This is only inserted into the PDF information catalog. Can be left
% out. 

\pgfdeclareimage[height=.5cm]{university-logo}{logo_ucv}
\logo{\pgfuseimage{university-logo}}

% If you wish to uncover everything in a step-wise fashion, uncomment
% the following command: 

%\beamerdefaultoverlayspecification{<+->}


\begin{document}

\begin{frame}
  \titlepage
\end{frame}

\begin{frame}{Contenido}
  \tableofcontents
  % You might wish to add the option [pausesections]
\end{frame}

\section{Introducción}

\subsection{Funciones convexas y cóncavas a valores reales.}
\begin{frame}{Funciones a valores reales.}
Sean $X$ un espacio normado real, $D\subseteq X$ un subconjunto abierto y convexo y 
$f:D\to\R$ una función.
\Defi{Cvx}{
Se dice que la función $f$ es \textbf{convexa} en $D,$ si para todo $x,y\in D$:
\Eq{Cvx}{
f(tx+(1-t)y)\leq tf(x)+(1-t)f(y), \quad t\in[0,1].
}
}
\Defi{Ccv}{
Se dice que la función $f$ es \textbf{cóncava} en $D,$ si para todo $x,y\in D$:
\Eq{Ccv}{
tf(x)+(1-t)f(y) \leq f(tx+(1-t)y),\quad t\in[0,1].
}
}
\begin{block}{Obervación.}
Es evidente que $f$ es cóncava si y sólo si $-f$ es convexa.
\end{block}
\end{frame}

\begin{frame}{Funciones Jensen-convexas a valores reales.}
\Defi{MidCvx}{
Se dice que la función $f$ es \textbf{Jensen-convexa} en $D,$ si para todo
$x,y\in D$:
\Eq{MidCvx}{
f\bigg(\frac{x+y}2\bigg)\leq\frac{f(x)+f(y)}2.
} 
}
\Thm{*}{
Sea $D\subseteq X$ un subconjunto abierto y convexo, y sea $f:D\to\R$ una
función Jensen-convexa. Entonces $f$ satisface la siguiente desigualdad para todo
$x,y\in D$ y para todo $q\in[0,1]\cap\Q$:
\Eq{cvx}{
f(qx+(1-q)y)\leq qf(x)+(1-q)f(y).
}
}{\cite{Kuc85}, Teorema 5.3.5.}
\end{frame}

\subsection{El Teorema de Bernstein--Doetsch.}

\begin{frame}{El Teorema de Bernstein--Doetsch.}
\Thm{*}{
Toda función Jensen-convexa $f:D\to\R$ en $D$, localmente acotada superior en un punto 
$x_0\in D$ is continua y por lo tanto convexa en $D$.
}{\cite{Kuc85}, Teorema 6.4.2}
Este teorema teorema fue formulado por F. Bernstein and G. Doetsch en 1915 
\cite{BerDoe15}, y desde entonces ha sido muy importante en la teoría de convexidad,
razón por la cual ha sido generalizado de muchas maneras diferentes y por varios autores.
Como consecuencia directa se tiene el siguiente
\Cor{*}{
Una función $f:D\to\R$ es convexa si y sólo si es 
continua y Jensen-convexa.
}{\cite{Kuc85}, Teorema 7.1.1}
\end{frame}

\subsection{Algunas generalizaciones}

\begin{frame}{Funciones aproximadamente convexas.}
Sean $\epsilon$ y $\delta$ números reales, no negativos.
\Defi{e-Cvx}{
Se dice que la función, $f:D\to\R$ es \textbf{$(\epsilon,\delta)$-convexa}
en $D,$ si para todo $x,y \in D$ y para todo $t\in[0,1]$:
\Eq{e-Cvx}{
f(tx+(1-t)y) \leq tf(x) + (1-t)f(y) + \epsilon t(1-t)||x-y|| + \delta.
}
}
\Defi{e-J-Cvx}{
Se dice que la función $f:D\to\R,$ es \textbf{$(\epsilon,\delta)$-Jensen-convex} 
en $D,$ si para todo $x,y \in D$:
\Eq{e-Cvx}{
f\bigg(\frac{x+y}2\bigg) \leq \frac{f(x)+f(y)}2 +\epsilon||x-y||+\delta.
}
}
\end{frame}

\begin{frame}{Resultados de tipo B--D.}
\Thm{*}{
Si $f:D\to \R$ es $(0,\delta)$-Jensen-convexa y localmente acotada superior
en algún punto de $D$, entonces $f$ es $(0,2\delta)$-convexa.  
}{\cite{NgNik93}, Teorema 1.}
\Thm{*}{
Si $f$ es $(\epsilon,0)$-Jensen convexa y localmente acotada superior en algún punto de $D$, 
entonces, $f$ satisface la siguiente desigualdad
\Eq{GACvx}{
f(tx+(1-t)y)\leq tf(x) + (1-t)f(y) + 2\epsilon T(t)\|x-y\|,
}
donde $T(\cdot):\R\to\R$ es la función de Takagi, que se define mediante
la fórmula
\Eq{Tak}{
T(t) = \sum_{k=0}^{\infty}{\frac1{2^k}\dist(2^kt,\Z)}
}
}{\cite{HazPal04}, Teorema 4.}
\end{frame}

\begin{frame}{Funciones fuertemente convexas.}
Sea $c$ un número real positivo. Siguiendo a Polyak, \cite{Pol66}
\Defi{StrCvx}{
Una función $f:D\to \R$ es \textbf{fuertemente convexa}
con módulo $c$ si para todo $x,y\in D$ y para todo $t\in[0,1]$
\Eq{StrCvx}{
f(tx+(1-t)y)\leq tf(x) + (1-t)f(y) - ct(1-t)||x-y||^2 
}
}
\Thm{*}{
Si $f:D\to\R$ es fuertemente Jensen-convexa con módulo $c$,
y localmente acotada superior en un punto de $D$ entonces 
$f$ es continua y fuertemente convexa con módulo $c$.
}{\cite{AzoGimNikSan11}, Teorema 2.3}
\end{frame}

\section{Multifunciones.}
\subsection{Terminología básica.}

\begin{frame}{Multifunciones $K$-Convexas.}
Sean $X,Y$ espacios topológicos lineales, $K\subseteq Y$
un cono convexo cerrado y $D\subseteq X$ un conjunto convexo
y abierto.
Denote por $\P(Y)$ a la clase de subconjuntos no-vacios de $Y$.
\Defi{KCvxSvm}{
Una multifunción $F:D\to \P(Y)$ 
es \textbf{$K$-convexa} en $D$, si para todo $x,y\in D$ y todo 
$t\in [0,1]$
\Eq{KCvxSvm}{
tF(x) + (1-t)F(y) \subseteq F(tx+(1-t)y) + K.
}
}
\end{frame}

\begin{frame}{Cono Recesión.}
\Defi{RecCone}{
Sea $H\subseteq X$ un conjunto no vacío. El \textbf{cono recesión} de $H$ 
denotado por $\rec(H)$ es el conjunto
\Eq{recH}{
rec(H):=\{x\in X \,|\, tx + H\subseteq H,\mbox{ for all } t\geq0\}.
}
}
\begin{block}{Propiedades.}
\begin{enumerate}
	\item $\rec(H)$ es un cono convexo que contiene al origen;
	\item $K=\rec(H)$ es el cono más grande con la propiedad $K+H\subseteq H$;
	\item $\overline{\rec}(H)\subseteq\rec(\overline{H});$
	\item para todo $x\in X$, $t>0,$ $\rec(tx+H) = \rec(H);$
	\item %Para todo par de conjuntos $H_1,H_2\subseteq X$,
	$\rec(H_1)+\rec(H_2)\subseteq\rec(H_1+H_2),$ para todo $H_1,H_2\subseteq X$.
\end{enumerate}
\end{block}
\end{frame}

%\begin{frame}{Cono Recesión de una multifunción.}
%\Defi{recSvm}{
%Given a set valued map $S:D\to\P(Y)$, \textbf{the recession cone} of $S$,
%denoted by $\rec(S)$, is the set:
%\Eq{recSvm}{
%\rec(S) = \bigcap_{x\in D}\rec(S(x)).
%}
%}
%\begin{block}{Properties}
%\begin{enumerate}
	%\item $\rec(S)\neq\emptyset$.
	%\item If $S(x)$ is bounded for some $x\in D$, then $\rec(S)=\{0\}$.
	%\item $\rec(S)+S(x) \subseteq S(x)$ for all $x\in D$.
%\end{enumerate}
%\end{block}
%\end{frame}

\begin{frame}{Acotación de multifunciones.}
\Defi{bdd1}{
Sea $S:D\to\P(Y)$ una multifunción. Se dice que $S$ es 
\textbf{localmente semi-$K$-acotada inferior} si para todo $x\in D$ existe
un entorno abierto $U\subseteq X$ de $x$ y un conjunto acotado $H\subseteq X$,
tal que 
\Eq{*}{
S(u)\subseteq\cl(H+K), \qquad (u\in U\cap D). 
}
}
\Defi{bdd2}{
Sea $S:D\to\P(Y)$ una multifunción. Se dice que $S$ es 
\textbf{localmente débil-semi-$K$-acotada superior} si para todo $x\in D$ existe
un entorno abierto $U\subseteq X$ de $x$ y un conjunto acotado $H\subseteq X$,
tal que
\Eq{*}{
0\in\cl(S(u)+H+K), \qquad (u\in U\cap D). 
}
}
\end{frame}

\subsection{Transformación de Takagi de una multifunción.}
\begin{frame}
Asumamos que $D\subseteq X$ es un conjunto estrellado.
\Defi{TakagiT}{
Para una multifunción $S:D\to\P(Y)$, tal que $0\in S(x)$
para todo $x\in D$, definimos la 
\textbf{transformación de Takagi} de $S$, como la multifunción
$S^T:\R\times D\to\P(Y)$ tal que
\Eq{TakagiT}{
S^T(t,x):= \cl\Bigg(
\bigcup_{n=0}^\infty\sum_{k=0}^n\frac1{2^n}S\big(2\dist(2^kt,\Z)x\big)
\Bigg).
}
}
\begin{block}{Relación entre $S$ y $S^T$.}
Sea $S:D\to\P(Y)$ una multifunción tal que $0\in S(x)$
para todo $x\in D$. Entonces
\Eq{S-St}{
\cl(S(x))\subseteq S^T\big(\tfrac12,x\big)\qquad (x\in D).
} 
Además si $S(0)\subseteq\overline{\rec}(S)$ entonces la inclusión \eq{S-St},
se convierte en una igualdad.
\end{block}
\end{frame}

\begin{frame}
\begin{block}{Algunas transformaciones de Takagi}
Sea $S_0\subseteq Y$ un conjunto convexo que contiene a 0 y sea $\varphi$
una función no negativa localmente acotada superior.
\begin{enumerate}
\item $S(x) = K + \varphi(x)S_0 \stackrel{T.T}{\longrightarrow} 
	       S^T(t,x) = \cl\Big(K+\varphi^T(t,x)S_0\Big),$ donde
\Eq{*}{
\varphi^T(t,x) =\sum_{n=0}^\infty\frac1{2^n}\varphi(2\dist(2^nt,\Z)x).
}	
\item Cuando, $\varphi(x) = ||x||^\alpha$, entonces:
$%\Eq{*}{
\varphi^T(t,x)=T_\alpha(t)||x||^\alpha,
$%}
donde la función $T_\alpha(\cdot):\R\to\R$ es la función de takagi
de orden $\alpha$, definida por
\Eq{a-tak}{
T_\alpha(t):=\sum_{n=0}^\infty2^{\alpha-n}(\dist(2^nt,\Z))^\alpha
}
\end{enumerate}
\end{block}
\end{frame}

\section{Resultados principales}

\subsection{Teorema tipo B--D.}
\begin{frame}
\Thm{*}{
Sea $D\subseteq X$ un subconjunto no-vacío y convexo. Sean $A,B:(D-D)\to\P(Y)$, 
multifunciones tales que $0\in A(x)\cap B(x)$ para todo $x\in (D-D)$, y consideremos
$K=\overline{\rec}(B).$ 
Sea $F:D\to\P(Y)$ una multifunción que satisface la siguiente inclusión de
convexidad tipo Jensen para todo $x,y\in D$ 
\Eq{JCvx}{
\frac{F(x)+F(y)}2 + A(x-y)\subseteq
			\cl\bigg(F\Big(\frac{x+y}2\Big)+B(x-y)\bigg).
}
Supongamos además que $F$ es puntualmente semi-$K$-acotada inferior
y localmente semi-débil-$K$-acotada superior en $D$. Entonces $F$ 
satisface para $x,y\in D$ y $t\in[0,1]$, la inclusión
\Eq{Cvx}{
tF(x)+(1-t)F(y) &+ A^T(t,x-y) \\
&\subseteq \cl\big( F(tx+(1-t)y) + B^T(t,x-y)\big). 
}
}{\cite{GonNikPalRoa14}, Theorem 4.1}
\end{frame}

\begin{frame}
Los siguientes corolarios generalizan algunos de los resultados 
obtenidos por Averna, Cardinali,Nikodem, Papalini and Borwein
	\cite{AveCar90,Bor77,CarNikPap93, Nik86, Nik87a,Nik87c,Nik89,Pap90}
relacionados con multifunciones $K$-convexas. Tambien, los resultados
de Az\'ocar, Gim\'enez, Nikodem and S\'anchez para funciones fuertemente
convexas se pueden obtener como consecuencia directa de los mismos, asi como 
también los resultados obtenidos por Leiva, Merentes, Nikodem and S\'anchez 
\cite{LeiMerNikSan13} relacionados con multifunciones fuertemente convexas.
\end{frame}

\subsection{Corolarios.}

\begin{frame}
Sea $\varphi:D-D\to\R_+$ una función no negativa, localmente acotada superior
y sea $S_0\subseteq Y$ un conjunto convexo que contiene a $0\in Y$.
\Cor{*}{
Supongamos que $F:D\to\P(Y)$ es una multifunción puntualmente semi-$K$-acotada 
superior y localmente semi-débil-$K$-acotada superior que satisface
\Eq{*}{
\frac{F(x)+F(y)}2\subseteq
			\cl\bigg(F\Big(\frac{x+y}2\Big)+K+\varphi(x-y)S_0\bigg)\quad(x,y\in D).
}
Entonces
\Eq{*}{
tF(x)+(1-t)F(y)\subseteq \cl\big(F(tx+(1-t)y)+K+ \varphi^T(t,x-y)S_0\big), 
}
para todo $x,y\in D$ y para todo $t\in[0,1].$
}{\cite{GonNikPalRoa14}, Corollary 4.4}
\end{frame}

\begin{frame}
\Cor{*}{
Supongamos que $F:D\to\P(Y)$ es una multifunción puntualmente semi-$K$-acotada 
superior y localmente semi-débil-$K$-acotada superior que satisface
\Eq{*}{
\frac{F(x)+F(y)}2+\varphi(x-y)S_0\subseteq
			\cl\bigg(F\Big(\frac{x+y}2\Big)+K\bigg)\quad(x,y\in D).
}
Entonces
\Eq{*}{
tF(x)+(1-t)F(y)+ \varphi^T(t,x-y)S_0\subseteq \cl\big(F(tx+(1-t)y)+K\big), 
}
para todo $x,y\in D$ y para todo $t\in[0,1].$
}{\cite{GonNikPalRoa14}, Corollary 4.5}
\end{frame}

\begin{frame}[allowframebreaks]
  \frametitle{References}
   
	\scriptsize{\bibliographystyle{amsalpha}}
  \bibliography{funcequ,publ}
	
  \beamertemplatearticlebibitems


\end{frame}



\end{document}

