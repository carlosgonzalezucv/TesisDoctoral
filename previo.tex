\chapter{El teorema de Bernstein--Doetsch}
\setcounter{theorem}{0}
\label{chapPrevio}

En este capítulo se presentarán las diferentes versiones 
del teorema de Bernstein--Doetsch que serán englobadas más 
adelante como consecuencia directa de los resultados principales que
obtuvimos al realizar este trabajo. Esta línea de investigación
comienza formalmente en el año 1905, con el artículo del matemático 
Danés, Johan Jensen \cite{Jen06}, luego en el año 1915, Bernstein y Doetsch 
publican un artículo donde demuestran el teorema que hoy en día lleva su nombre 
y que ha sido uno de los teoremas más importantes en la teoría de convexidad \cite{Kuc09}. 

Modificaciones 

Comenzaremos este capítulo dando dos definiciones que serán fundamentales
a lo largo del resto del trabajo.
\Defi{cvxFunc}{
Sea $X$ un espacio normado y sea $D$ un subconjunto convexo de $X$.
Una función $f:D\to\R$ es convexa si para todo $x,y\in D$ 
y para todo $t\in[0,1]$ se tiene que 
\Eq{cvxFunc}{
f(tx+(1-t)y)\leq tf(x)+(1-t)f(y).
}
}
\Defi{midCvx}{
Sea $X$ un espacio normado y sea $D$ un subconjunto convexo de $X$.
Una función $f:D\to\R$ es midconvexa si para todo $x,y\in D$ 
se tiene que
\Eq{midCvxFunc}{
f\bigg(\frac{x+y}{2}\bigg)\leq \frac{f(x)+f(y)}2.
}
}
Es elemental demostrar que si $f$ es una función midconvexa,
entonces $f$ satisface la desigualdad \eq{cvxFunc} para todo
$x,y\in D$ y para todo $t\in[0,1]\cap\Q$ (ver \cite{Kuc09}). 
Por lo tanto, en la clase de las funciones continuas, las nociones
de convexidad y midconvexidad son equivalentes entre sí.
%En este capítulo se pretende dar a conocer algunos resultados previos
%relacionados con funciones convexas, aproximadamente convexas y 
%fuertemente convexas los cuales han sido la motivación de nuestra investigación.

Antes de ir al teorema de Bernstein--Doetsch, no podemos dejar a un lado un resultado 
que se puede decir, sin dudas que dió comienzo a esta linea de investigación. 
El matemático Johan Jensen en 1905, 
publica un artículo en el cual demuestra el siguiente teorema. 

\begin{theorem}[\cite{Jen06}, pg. 188]
\label{TJensenFirst}
Sean $a,b\in\R$, $a<b$. 
Toda función $f:(a,b)\to\R$ acotada superiormente en $(a,b)$ que satisface 
la desigualdad
$$
f\bigg(\frac{x+y}{2}\bigg)\leq \frac{f(x)+f(y)}{2},
$$
para todo $x,y\in (a,b)$ necesariamente tiene que ser continua
y por lo tanto convexa.
\end{theorem}

\section{Versión Original.}
No es sino, diez años más tarde que Felix Bernstein y Gustav Doetsch publican,
en el año 1915, lo que será una generalización del resultado obtenido
por Jensen en 1905 y lo que hoy en día se conoce como el teorema de Bernstein--Doetsch. 
En dicho teorema, los autores debilitan la condición de que la función en cuestión
sea acotada en su dominio y la cambian por una condición más débil, a saber,
la condición de estar localmente acotada superior en solo un punto.
%
%En el libro verde del prof. Marek Kuczma \cite{Kuc09}, se pueden encontrar 
%dos demostraciones de dicho teorema. A continuación presentaremos una de ellas
%en la cual se usan tres resultados importantes que serán englobados en el siguiente 
%teorema
Debido a la importancia del teorema de Bernstein--Doetsch para el desarrollo
de esta investigación, haremos una demostración de dicho resultado. Para ello,
necesitaremos el siguiente lema auxiliar.
\begin{lemma}[\cite{Kuc09}, Teoremas 6.2.1, 6.2.2 y 6.2.3]
\label{LLocUpBdd}
Sean $D\subseteq\R^n$ un conjunto abierto y convexo. Sea $f:D\to\R$ una función 
midconvexa.
Si $f$ es localmente acotada superior en un punto $x_0\in D$, entonces
$f$ es localmente acotada en cada punto de $D$.
\end{lemma}

Vale la pena destacar que la demostración del lema anterior es la 
combinación de tres teoremas, cada uno de los cuales amerita
una demostración detallada. Con esta herramamienta, estaremos en la 
capacidad de demostrar el siguiente teorema.

\begin{theorem}[Teorema de Bernstein--Doetsch,\cite{BerDoe15}]
\label{TBD15}
Sea $N\in\N$ y sea $f:D\subseteq\R^N\to\R$ una función midconvexa. Si
$f$ es localmente acotada superior en un punto de $D$ entonces 
$f$ es continua en $D$.
\end{theorem}

\begin{proof}
El hecho de que la función midconvexa $f$ sea localmente acotada superior
en un punto, implica, según el \lem{LocUpBdd} que la función $f$ tiene que
ser localmente acotada en cada punto de $D$. Por lo tanto, a partir de $f$
podemos definir los siguientes números
\Eq{*}{
m_f(x_0)= \lim_{h\to0}\inf_{B(x_0,h)}{f(x)} \qquad\mbox{y}\qquad
M_f(x_0)= \lim_{h\to0}\sup_{B(x_0,h)}{f(x)}.
}
Note que tanto $m_f$ como $M_f$ son finitos y además, se tiene que para
todo $x\in D$
\Eq{mfM}{
m_f(x) \leq f(x) \leq M_f(x),
}
Ahora bien, sea $x\in D$ arbitrario. Podemos construir un par de sucesiones 
$(x_n)\subseteq D$ y $(z_n)\subseteq D$ tales que 
\Eq{limx}{
\lim_{n\to\infty}x_n=x, \qquad \lim_{n\to\infty}f(x_n) = m_f(x),
}
\Eq{limz}{
\lim_{n\to\infty}z_n=x, \qquad \lim_{n\to\infty}f(z_n) = M_f(x).
}
Para $n\in\N$, sea $y_n = 2z_n-x_n$. Es claro que $\lim y_n = x$, más aún
$z_n=(x_n+y_n)/2$ y por la midconvexidad de $f$ se tiene lo siguiente
\Eq{*}{
f(z_n) \leq \frac{f(x_n)+f(y_n)}2,
}
o equivalentemente 
$$
f(y_n) \geq 2f(z_n)-f(x_n).
$$
Si ahora tomamos el limite inferior en ambos lados de la desigualdad anterior, 
se obtiene lo siguiente
$$
\liminf_{n\to\infty} f(y_n) \geq 2M_f(x)-m_f(x),
$$
pero también,
$$
\liminf_{n\to\infty} f(y_n) \leq M_f(x).
$$
En consecuencia, se tiene que $M_f(x)\leq m_f(x)$. Al combinar esto 
con la desigualdad \eq{mfM} se tiene que $M_f(x)=m_f(x)$ y por lo tanto,
la función $f$ es continua en $x$. Como $x$ fue escogido de manera arbitraria
se tiene que la función $f$ es continua en $D$, y esto finaliza la demostración
del teorema.
\end{proof}



\section{Cambio en la estructura del espacio subyacente.}

Casi 50 años más tarde en 1964, el matemático, M.R. Mehdi \cite{Meh64} obtiene 
el siguiente resultado, que generaliza el Teorema \ref{TBD15}, 
al cambiar la estructura del espacio subyacente.

\Thm{Mehdi}{
Sean $X$ un espacio topológico lineal, $D\subseteq X$ un conjunto abierto
y convexo y sea $f:D\to\R$ una función midconvexa. Si $f$ es acotada superiormente
en un subconjunto abierto no vacío de $D$, entonces $f$ es una función continua.
}

Ahora el teorema de Bernstein--Doetsch es válido, para funciones cuyo
dominio esté sumergido en un espacio lineal topológico. Por otra parte, 
los matemáticos Z. Kominek y M. Kuczma, aseguran en su artículo \cite{KomKuc89b}
que los resultados obtenidos en \cite{Kom87a} y \cite{KomKuc89a} implican
el siguiente teorema y además de eso, ofrecen una generalización inmediata de 
éste y del Teorema \ref{TMehdi}.
\begin{theorem}[\cite{KomKuc89b}, Teorema C]
\label{KK1}
Sean $X$ un espacio lineal, $D\subseteq X$ un conjunto convexo algebraicamente abierto,
y sea $f:D\to\R$ una función midconvexa. Si $f$ es acotada superiormente, en un 
subconjunto no vacío y algebraicamente abierto de $D$, entonces $f$ es continua en $D$
con respecto a la topología de los conjuntos algebraicamente abiertos.
\end{theorem}
\begin{theorem}[\cite{KomKuc89b}, Teorema 1]
\label{KK2}
Sea $X$ un espacio lineal, dotado con una topología semilineal, sea $D\subseteq X$ un 
conjunto abierto y convexo. Sea $f:D\to\R$ una función midconvexa. Si $f$ es acotada
superiormente en un subconjunto abierto y no-vacío de $D$, entonces, $f$ es continua.
\end{theorem}

\section{Convexidad Aproximada.}

En el contexto de convexidad aproximada, D. H. Hyers y S. M. Ulam, en 1952
introducen la definición de $\varepsilon$-convexidad \cite{HyeUla52}, donde $\varepsilon$ es un número
real positivo. Allí los autores establecen que dados $n\in\N$, $\varepsilon>0$, y
una función $f$ a valores reales definida en un subconjunto $D\subseteq\R^n,$ se 
dice que $f$ es $\varepsilon$-convexa si y sólo si, para todo $x,y\in D$ se tiene que
\Eq{eCvx}{
f(tx+(1-t)y)\leq tf(x)+(1-t)f(y)+\varepsilon
}
para todo $t\in(0,1).$ En su artículo, demostraron que toda solución de la
desigualdad \eq{eCvx} está en correspondencia con una función convexa $g$,
tal que $|f-g|\leq\varepsilon$, este resultado es mejor conocido como el teorema
de estabilidad de Hyers y Ulam. 

Siguiendo esta dirección, K. Nikodem junto con Ng. en \cite{NgNik93}
demostraron la siguiente versión del teorema de Bernstein--Doetsch para
funciones $\varepsilon$-convexas que generaliza el Teorema \ref{TBD15}. 
Este resultado también fue establecido independientemente
por Lacskovich en \cite{Lac99}.
\begin{theorem}[\cite{NgNik93}, Teorema 1]
\label{TNgNik}
Sea $D\subseteq X$ un conjunto abierto y convexo de un espacio lineal topológico $X$. 
Si $f:D\to\R$ es una función localmente acotada superior en un punto de $D$, y
$\varepsilon$-midconvexa, i.e, para todo $x,y\in D$ se tiene que
$$
f\bigg(\frac{x+y}{2}\bigg)\leq\frac{f(x)+f(y)}2 + \varepsilon,
$$ 
entonces, $f$ es $2\varepsilon$-convexa.
\end{theorem}  

Diez años más tarde, Zolt Páles en \cite{Pal03a} formula la siguiente
definición: Una función $f$ definida en un subconjunto abierto
y convexo $D$ de un espacio normado real $X,$ es $(\delta,\varepsilon)$-convexa
si satisface
\Eq{edMCvx}{
f(tx+(1-t)y)\leq tf(x)+(1-t)f(y)+\delta t(1-t)\|x-y\|+\varepsilon
}
para todo $x,y\in D$ y $t\in(0,1)$. En su artículo, Z. Páles, obtiene
propiedades de estabilidad de tipo Hyers--Ulam asociadas
con la desigualdad \eq{edMCvx} y estudia las propiedades
que caracterizan este tipo de funciones. Un año depués,
en 2004, buscando más generalizaciones del \thm{BD15}, 
A. Házy y Z. Páles, en \cite{HazPal04} se plantean la siguiente interrogante:
¿Qué propiedades tendrán las funciones $(\delta,\varepsilon)$-midconvexas, 
localmente acotadas? La respuesta a esta interrogante, la encontramos
en los siguientes teoremas.

\begin{theorem}[\cite{HazPal04}, Teorema 3]
\label{THazPal1}
Sea $\delta$ un número no negativo. Si $f:D\to\R$ es $(\delta,0)$-midconvexa,
i.e, $f$ satisface la desigualdad
$$
f\bigg(\frac{x+y}{2}\bigg) \leq \frac{f(x)+f(y)}{2}+\delta\|x-y\|
$$
para todo $x,y\in D$, y si además $f$ es localmente acotada superior en un punto
de $D$. Entonces, $f$ es continua.
\end{theorem}

Además, llegan a la conclusión de que si $\varepsilon$ es un número positivo,
 no se puede garantizar que toda función $(\delta,\varepsilon)$-convexa
localmente acotada superior en un punto sea continua. Sin embargo, plantean
el siguiente teorema que generaliza el \thm{BD15} y el \thm{NgNik}
\begin{theorem}[\cite{HazPal04}, Teorema 4]
\label{THazPal2}
Sean $\delta$ y $\varepsilon$ dos números no negativos. Si $f:D\to\R$ es una función
$(\delta,\varepsilon)$-midconvexa acotada superiormente en un punto de $D$ entonces,
para todo $x,y\in D$ y para todo $t\in(0,1)$
\Eq{HazPal2}{
f(tx+(1-t)y) \leq tf(x)+(1-t)f(y)+2\delta \varphi(t)\|x-y\|+2\varepsilon
}
donde $\varphi$ es un punto fijo del operador $\mathcal{H}:\R^{[0,1]}\to\R^{[0,1]}$
definido por 
\Eq{HazPal3}{
(\mathcal{H}\varphi)(t):=
\left\{
\begin{array}{ll}
\dfrac{\varphi(2t)}{2} + t,& 0\leq t\leq \frac12, \\[.3cm]
\dfrac{\varphi(2t-1)}{2} + (1-t),& \frac12\leq t\leq1.
\end{array}
\right.
}
\end{theorem}
\Rem{takSol}{
Note que la función de Takagi, $T:\R\to[0,1]$ definida en la introducción
mediante la fórmula 
\Eq{Tak}{
T(t) = \sum_{n=0}^{\infty}\frac1{2^n}\dist(2^nt,\Z),\qquad(t\in\R),
}
satisface la siguiente cadena de igualdades
\Eq{*}{
T(2t) =\sum_{n=0}^{\infty}\frac1{2^n}\dist(2^{n+1}t,\Z)
			=2\sum_{n=1}^{\infty}\frac1{2^n}\dist(2^{n}t,\Z)
			=2(T(t)-\dist(t,\Z)).
}
Despejando $T(t)$ en la expresión anterior llegamos a la siguiene
relación
\Eq{Takrec}{
T(t)=\frac12T(2t)+\dist(t,\Z).
}
Ahora bien, si $t\in[0,1/2]$ entonces, $\dist(t,\Z)=t$, por lo tanto
\eq{Takrec} se convierte en 
\Eq{*}{
T(t)=\frac12T(2t)+ t,\qquad 0\leq t\leq\frac12
}
por otra parte, si $t\in(1/2,1]$ entonces, $\dist(t,\Z)=1-t$, y en este
caso \eq{Takrec} se convierte en
\Eq{*}{
T(t)=\frac12T(2t)+ 1-t,
}
además, como consecuencia de las propiedades elementales del ínfimo de un conjunto
se tiene que 
\Eq{*}{
\dist(2t,\Z)=\inf_{\alpha\in\Z}{|2t-\alpha|}
=\inf_{\alpha\in\Z}{|2t-1-(\alpha-1)|}
=\inf_{\beta\in\Z}{|2t-1-\beta|}
=\dist(2t-1,\Z)
}
lo que demuestra que la función de Takagi es 1-periódica y por lo tanto,
si $t\in(1/2,1],$ entonces
$$
T(t)=\frac12T(2t-1)+1-t.
$$ 
Esto significa que la función $T$ es solución
de la ecuación funcional $\mathcal{H}\varphi = \varphi$ y por lo tanto, 
ésta puede ser usada en la conclusión del \thm{HazPal2}.

Usualmente $T$ es conocida como función de ``Vander Waerden'' \cite{Wae30},
sin embargo Knoop \cite{Kno18} ya había descubierto que dicha función ya
había sido construida casi 30 años antes por T. Takagi
\cite{Tak03}. Para más detalles históricos, se pueden consultar los artículos de 
Billingsley \cite{Bil82}, Cater \cite{Cat84} y Kairies \cite{Kai98}.
}
Una pregunta que surge de manera natural luego de ver la conclusión
del \thm{HazPal2} es la siguiente, ¿Cúal será la mejor función que
puede ser colocada en el lugar de $\varphi$ en la desigualdad
\eq{HazPal2}? En respuesta a este planteamiento, A. Házy y Z. Páles en \cite{HazPal04}
conjeturan que la función óptima está dada por el límite de la
sucesión de funciones $\varphi_{n+1}=\mathcal{H}(\varphi_{n})$
$n\in\N$ con $\varphi_1(t)=1$ para todo $t\in(0,1)$, sin embargo,
no logran dar una demostración formal de ello para entonces. 

En 2008, Z. Boros logra resolver satisfactoriamente en su artículo \cite{Bor08},
el problema planteado por Z. Páles en \cite{ISFE41} sobre la 
$(1/2,0)$-midconvexidad de la función de Takagi. En el referido artículo
Boros demuestra que  
\Eq{*}{
T\bigg(\frac{x+y}2\bigg)\leq\frac{T(x)+T(y)}{2}+\frac12|x-y|,\qquad(x,y\in[0,1])
} 
y que por lo tanto, la función óptima que puede ser utilizada en la conclusión del
\thm{HazPal2} es la función de Takagi. 

Con base en lo que acabamos de desarrollar, el \thm{HazPal2}
puede ser reformulado de una manera un poco más simple
\begin{theorem}[\cite{HazPal04}, Teorema 4]
\label{THazPal3}
Sean $\delta$ y $\varepsilon$ dos números no negativos. Si $f:D\to\R$ es una función
$(\delta,\varepsilon)$-midconvexa acotada superiormente en un punto de $D$ entonces,
para todo $x,y\in D$ y para todo $t\in(0,1)$
\Eq{HazPal33}{
f(tx+(1-t)y) \leq tf(x)+(1-t)f(y)+2\delta T(t)\|x-y\|+2\varepsilon
}
donde $T$ es la función de Takagi definida en \eq{Tak}.
\end{theorem}

\subsection{$\alpha(\cdot)$-convexidad.}

A menos que se especifique otra cosa, a lo largo de esta sección
$D$ denotará un subconjunto abierto y convexo de un espacio normado
real $X$.

En el año 2005, A. Házy \cite{Haz05a} introduce el concepto de $(\delta,\varepsilon,p)$-convexidad.
Allí, el autor establece que
una función $f:D\to\R$ es $(\delta,\varepsilon,p)$-convexa, si 
\Eq{*}{
f(tx+(1-t)y)\leq tf(x)+(1-t)f(y)+\delta\|x-y\|^p+\varepsilon
}
para todo $x,y\in D$ y para todo $t\in[0,1]$. En su artículo, Házy
obtiene resultados análogos a los obtenidos por él y Z. Páles
en \cite{HazPal04} y que corresponden al \thm{HazPal1} y al \thm{HazPal3}
de esta sección.

Abriendo el abanico de posibilidades, Jacek Tabor y
Józef Tabor en \cite{TabTab09b}, generalizan las definiciones 
establecidas por Z. Páles y A. Házy anteriormente. En su artículo
ellos introducen la siguiente definición.
\Defi{alphaCvx}{
Dada una función 
$\alpha:[0,\infty)\to[0,\infty)$ 
no decreciente, se dice que una función $f:D\to\R$ es 
$\alpha(\cdot)$-midconvexa si
\Eq{*}{
f\bigg(\frac{x+y}{2}\bigg)\leq\frac{f(x)+f(y)}2+\alpha(\|x-y\|)
}
para todo $x,y\in D$.
}
Note que si $\alpha(u)=\varepsilon+\delta \|u\|^p$, la definición anterior
se reduce a la establecida por A. Házy en \cite{Haz05a}, mientras
que para $p=1$ se reduce a la definición establecida por Z. Páles 
en \cite{Pal03a}. De inmediato, Ja. Tabor y Jó. Tabor,
obtienen una adaptación del teorema de Bernstein--Doetsch
para esta nueva clase de funciones.

\begin{theorem}[\cite{TabTab09b}, Teorema 2.1]
\label{TTabTab1}
Sea $f:D\to\R$ una función $\alpha(\cdot)$-midconvexa y
localmente acotada superior en un punto. Entonces $f$ es localmente
acotada en cada punto de $\mbox{int}D$. Si además, 
$
\ds\lim_{r\to0^+}\alpha(r) = 0,
$
entonces $f$ es continua en $D$.
\end{theorem}
Motivado en los resultados obtenidos por Házy y Páles
en \cite{HazPal04}, Ja. Tabor y Jó. Tabor, establecieron los siguientes
resultados.

\begin{theorem}[\cite{TabTab09b}, Teorema 2.2]
\label{TTabTab2}
Sea $f:D\to\R$ una función $\alpha(\cdot)$-midconvexa. Entonces,
\Eq{TabTab2}{
f(tx+(1-t)y)\leq tf(x)+(1-t)f(y)
+\sum_{n=0}^\infty\frac{1}{2^k}\,\alpha\big(\dist(2^nt,\Z)\|x-y\|\big)
}
para todo $x,y\in D$, $t\in[0,1]\cap\Q$. Más aún, si $f$ es localmente 
acotada superior en un punto de $D$, entonces, la desigualdad
\eq{TabTab2} es válida para todo $t\in[0,1]$.
\end{theorem}

\begin{theorem}[\cite{TabTab09b}, Teorema 3.1]
\label{TTabTab3}
Sea $f:D\to\R$ una función $\alpha(\cdot)$-midconvexa. Entonces,
\Eq{TabTab3}{
f(tx+(1-t)y)\leq tf(x)+(1-t)f(y)
+\sum_{n=0}^\infty\dist(2^nt,\Z)\alpha\bigg(\frac{\|x-y\|}{2^k}\bigg)
}
para todo $x,y\in D$, $t\in[0,1]\cap\D$, donde $\D$ es el conjunto
de los racionales diádicos. Más aún, si $f$ es localmente 
acotada superior en un punto de $D$ y 
$$
\sum_{n=0}^\infty\alpha(1/2^n)<\infty
$$
entonces, $f$ es continua en $[0,1]$ y la desigualdad
\eq{TabTab3} es válida para todo $t\in[0,1]$.
\end{theorem}

Note que el \thm{TabTab2} generaliza el \thm{HazPal2}, mientras que
el \thm{TabTab3} introduce un nuevo término de error para la 
convexidad aproximada de la función $f$. Cabe destacar que 
J. Mako y Z. Páles en \cite[Teorema 26]{MakPal12a} establecen un teorema 
análogo al \thm{TabTab2} pero la demostración es completamente
diferente a la hecha por J. Tabor et. al. en \cite{TabTab09b}.

%la pregunta que surge de forma
%natural es: ¿Cúal de estas estimaciones es la más óptima?. 
%
%La respuesta a dicha interrogante, depende, por supuesto de la función 
%$\alpha$. Tabor et al. y Páles et al. en \cite{TabTab09a} y \cite{MakPal10b}
%respectivamente investigan al respecto enfocándose en el caso particular
%$\alpha(u)=\|u\|^p$ para $u\in D$. Por una parte, Ja. Tabor y Jó. Tabor
%en \cite{TabTab09a} demuestran que cuando $p\in[1,2]$, la estimación
%hecha en el \thm{TabTab2} es óptima. Por otra parte, J. Mako y Z. Páles 
%en \cite{MakPal10b} demuestran que en el caso $p\in(0,1)$

\section{Convexidad fuerte.}

En el año 1966, Boris Polyak en \cite{Pol66} establece la siguiente definición.
\Defi{strgPol}{
Sea $c$ una constante positiva. Se dice que una función $f:D\to\R$ 
es fuertemente convexa con módulo $c$, si satisface 
\Eq{strgPol}{
f(tx+(1-t)y)\leq tf(x)+(1-t)f(y)-ct(1-t)\|x-y\|^2
}
para todo $x,y\in D,$ $t\in[0,1]$.
}
Como consecuencia de la definición anterior, se tiene la siguiente
\Defi{StrMid}{
Sea $c$ una constante positiva. Se dice que una función $f:D\to\R$ 
es fuertemente midconvexa con módulo $c$, si satisface 
\Eq{strgMidPol}{
f\bigg(\frac{x+y}{2}\bigg)\leq \frac{f(x)+f(y)}2-\frac{c}4\|x-y\|^2
}
para todo $x,y\in D,$ $t\in[0,1]$.
}
En el año 2011, A. Azócar et. al. en \cite{AzoGimNikSan11} demuestran
que en la clase de funciones continuas, midconvexidad fuerte
y convexidad fuerte son equivalentes entre sí. 
Un resultado de tipo Bernstein--Doetsch para esta nueva 
clase de funciones fue establecido por A. Azócar et. al.
en \cite{AzoGimNikSan11}, allí los autores demostraron el 
siguiente teorema.
\begin{theorem}[\cite{AzoGimNikSan11}, Teorema 2.3]
\label{azoBD}
Sea $c>0$. Si $f:D\to\R$ es una función fuertemente midconvexa 
con módulo $c$ y acotada superiormente en un subconjunto de $D$ 
con interior no vacío, entonces, $f$ es una función continua
y además fuertemente convexa con módulo $c$.
\end{theorem}

Hasta ahora solo se han presentado resultados de tipo Bernstein--Doetsch
cuando el conjunto de llegada es $\R$. En el Capítulo 
\ref{ChapMultifunciones} mostraremos algunos resultados importantes
de este tipo cuando el codominio tiene una estructura más general.



%Resultados tipo B--D con convexidad aproximada. NgNik93
%Resultados tipo B--D con convexidad fuerte. AzoNikGimSan11
%Resultados tipo B--D para h-convexidad ??
%Resultados tipo B--D para convexidad aproximada generalizada HazPal, TabTab, 
%resultados de trudzik,
%quasy-convex
%algebraic version



