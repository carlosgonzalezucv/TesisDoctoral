\chapter*{Introducción}
\markright{INTRODUCCIÓN}


El Teorema de Bernstein y Doetsch \cite{BerDoe15} publicado hace 100 años, ha sido uno de los
resultados fundamentales en la teoría de convexidad
\cite{Kuc09}. Este teorema asegura que si la funci\'on $f:I\to\R$ es 
Jensen convexa (donde $I$ es un intervalo de la recta real), i.e.,
\Eq{JC}{
  f\Big(\frac{x + y}{2}\Big)\leq \frac{f(x) + f(y)}{2} \qquad(x,y\in I)
}
y también es localmente acotada superior, entonces $f$ debe ser continua
sobre $I$ y por lo tanto convexa en $I$. 
Si $-f$ es Jensen convexa, entonces se dice que $f$ es Jensen cóncava y los resultados obtenidos
para esta clase de funciones son análogos bajo la hipótesis de que $f$ debe ser localmente
acotada inferior. Este teorema es muy importante y ha sido aplicado y generalizado de muchas maneras
las cuales describiremos brevemente a continuación.

Cuando el co-dominio $Y$ de la función $f$ es un espacio vectorial ordenado, 
i.e, el conjunto $K$ de elementos
no-negativos de $Y$ forma un cono convexo, entonces se puede definir la convexidad de tipo Jensen 
con respecto al cono $K$ (frecuentemente conocida como Jensen $K$-convexidad) de la función
$f:I\to Y$ por
\Eq{JCK}{
  \frac{f(x) + f(y)}{2}\in f\Big(\frac{x + y}{2}\Big)+K \qquad(x,y\in I).
}
En particular, si $Y=\R$ y $K=\R_+$, entonces \eq{JCK} es equivalente a \eq{JC}. 
Las extensiones del teorema de Bernstein--Doetsch a esta clase de funciones fueron formuladas por
Trudzik \cite{Tru84}. Las funciones $f:I\to Y$ que satisfacen la inclusión
\Eq{JCVK}{
  f\Big(\frac{x + y}{2}\Big)\in \frac{f(x) + f(y)}{2} + K \qquad(x,y\in I)
}
se denominan $K$-Jensen cóncavas. Evidentemente, esta última relación es válida si y sólo
si $(-f)$ es $K$-Jensen convexa (ó si $f$ es $(-K)$-Jensen convexa). 
Por lo tanto, los resultados relacionados con funciones $K$-Jensen cóncavas siempre pueden
ser obtenidos directamente de los resultados establecidos para funciones $K$-Jensen convexas.
%(Esto sin embargo, no ocurrirá para el caso de multifunciones.)

Generalizado un poco más, podemos considerar el caso de las multifunciones.
Una multifunci\'on no es m\'as que una funci\'on cuyas im\'agenes
son subconjuntos de un conjunto $Y$ cualquiera.
Si $X$ es un espacio normado, $D\subseteq X$ es un conjunto convexo y abierto y
$Y$ es un espacio vectorial ordenado,
entonces se dice que una multifunción $F:D\to \P_0(Y)$ es $K$-Jensen convexa si para todo 
$x,y\in D$ se cumple lo siguiente
\Eq{SVJCK}{
  \frac{F(x) + F(y)}{2}\subseteq F\Big(\frac{x + y}{2}\Big)+K. 
}
Observe que si $F$ es de la forma $F(x)=\{f(x)\}$ para alguna función $f:D\to Y$, entonces \eq{SVJCK}
es equivalente a \eq{JCK}, y así se evidencia como la inclusión \eq{SVJCK} generaliza \eq{JCK}.
Los resultados de tipo Bernstein--Doetsch  para este tipo de multifunciones han sido obtenidos 
durante las últimas cinco décadas por los profesores 
Averna, Cardinali, Nikodem, Papalini \cite{AveCar90,CarNikPap93,Nik86,Nik87a,Nik87c,Nik89,Pap90} 
y Borwein \cite{Bor77}. La noción de $K$-Jensen concavidad para una multifunción 
$F:D\to \P_0(Y)$, análoga a la inclusión \eq{JCVK}, se define por 
\Eq{SVJCVK}{
  F\Big(\frac{x + y}{2}\Big)\subseteq \frac{F(x) + F(y)}{2} + K \qquad(x,y\in D).
}
En el desarrollo de este trabajo, veremos que, en general, la $K$-Jensen concavidad 
de $F$ \textit{no es} equivalente a la $K$-Jensen 
convexidad de $-F$. Esto trae como consecuencia, que al hablar de multifunciones, 
los resultados relacionados a convexidad y concavidad necesitan ser tratados por separado. 

Otra cadena de generalizaciones del teorema de Bernstein--Doetsch emerge del artículo del profesor
K. Nikodem junto con el prof. Ng \cite{NgNik93} en el contexto de convexidad aproximada.
Allí, ellos demuestran que si $f:D\to\R$ es $\varepsilon$-Jensen convexa para algún 
$\varepsilon\geq0$, i.e.,
\Eq{JCe}{
  f\Big(\frac{x + y}{2}\Big)\leq \frac{f(x) + f(y)}{2} + \varepsilon \qquad(x,y\in D),
}
y si además $f$ es localmente acotada superior, entonces $f$ es $2\varepsilon$-convexa, i.e.,
\Eq{Ce}{
  f(tx+(1-t)y)\leq tf(x)+(1-t)f(y)+2\varepsilon \qquad(x,y\in D,\,t\in[0,1]).
}
%La versión correspondiente del resultado anterior para el caso de multifunciones, será formulado 
%como una consecuencia de los resultados principales obtenidos en nuestra investigación.

Considerando un término de error más general, Házy y Páles \cite{HazPal04} investigaron la 
siguiente desigualdad de tipo Jensen aproximada
\Eq{JCHP}{
  f\Big(\frac{x + y}{2}\Big)\leq \frac{f(x) + f(y)}{2} + \varepsilon\|x-y\| \qquad(x,y\in D),
}
asumiendo que $D$ es un subconjunto de un espacio normado $X$ y $f$ es una función a valores reales.
En dicho artículo, los autores demostraron que bajo la hipótesis usual de que $f$ es localmente 
acotada superior, la desigualdad \eq{JCHP} implica
\Eq{CHP}{
  f(tx+(1-t)y)\leq tf(x)+(1-t)f(y)+2\varepsilon T(t)\|x-y\| \qquad(x,y\in D,\,t\in[0,1]),
}
donde la función $T:\R\to[0,1]$, es la conocida función de Takagi, y se define por
\Eq{*}{
  T(t):=\sum_{n=0}^\infty \frac1{2^n}\dist(2^nt,\Z).
}
Algunos resultados, que extienden este tipo de nociones a términos más generales de errores y también 
a conceptos de convexidad relacionados con sistemas de Chebyshev, han sido obtenidos recientemente
por Házy, Makó and Páles \cite{Haz05a,Haz07b,HazPal05,HazPal09,MakPal10b,MakPal12b,MakPal12c,MakPal13b}
y por Mureńko, Ja. Tabor, Jó. Tabor, and Żoldak 
\cite{MurTabTab12,TabTab09b,TabTab09a,TabTabZol10b,TabTabZol10a}.

%
%Para el caso de multifunciones $K$-Jensen convexas y $K$-Jensen cóncavas, haremos formulaciones
%más generales como consecuencia directa de los resultados principales de nuestra investigación.

Finalmente, haremos mención a la noción de convexidad fuerte, la cual en cierto sentido, es lo contrario 
a convexidad aproximada. Siguiendo a Polyak \cite{Pol66}, una función $f:D\to\R$ es 
fuertemente Jensen convexa con módulo $\varepsilon\geq0$ si
\Eq{SJC}{
  f\Big(\frac{x + y}{2}\Big)\leq \frac{f(x) + f(y)}{2} - \frac{\varepsilon}{4}\|x-y\|^2 \qquad(x,y\in D).
}
Asumiendo que $f$ es localmente acotada superior, los profesores Azócar, Gimenez, Nikodem y Sánchez en
 \cite{AzoGimNikSan11} demostraron que si $f$ es fuertemente Jensen convexa 
entonces $f$ es fuertemente convexa con módulo $\varepsilon$, i.e., 
\Eq{SC}{
  f(tx+(1-t)y)\leq tf(x)+(1-t)f(y) - \varepsilon t(1-t)\|x-y\|^2 \qquad(x,y\in D,\,t\in[0,1]).
}
La versión conjunto valuada de este resultado fue establecida por Leiva, Merentes, Nikodem, y Sánchez 
en \cite{LeiMerNikSan13}.

En este trabajo, mostraremos una generalización del teorema de Bernstein--Doetsch 
para multifunciones que satisfacen una inclusión general de tipo Jensen
de la forma
\Eq{intro1}{
\frac{F(x)+F(y)}2 + A(x-y) \subseteq F\bigg(\frac{x+y}{2}\bigg) + B(x-y)
\qquad (x,y\in D),
}
donde $A$ y $B$ son multifunciones definidas en $D-D$. Bajo ciertas condiciones
de regularidad sobre la multifunción $F$ demostraremos que ella satisface
la inclusión
\Eq{intro2}{
tF(x)+(1-t)F(y)+S_A(t,x-y)\subseteq F(tx+(1-t)y)+S_B(t,x-y)
} 
para todo $t\in [0,1]$ y para todo $x,y\in D$.
Donde $S_A$ y $S_B$ son ciertas multifunciones definidas en
$[0,1]\times D-D$ que surgen de forma natural como una 
generalización conjunto-valuada de la función de Takagui cuya 
importancia se ve reflejada en los artículos de Z. Páles, J. Makó,
A. Hazy, Jo. Tabor, Ja. Tabor \cite{MakPal10b,MakPal12a,TabTab09b,TabTab09a}
como veremos en el Capítulo 2 de esta monografía.
Los resultados que presentaremos aquí generalizan
 a la mayoría  de los resultados
obtenidos en esta línea de investigación desde el año 1915 y además
proporcionan nuevos Teoremas de tipo Bernstein--Doetsch
para multifunciones aproximadamente midconvexas y midcóncavas.

Note que la inclusión \eq{intro1} es una combinación de 
midconvexidad aproximada y midconvexidad fuerte para la multifunción
$F$, mientras que la inclusión \eq{intro2} combina convexidad fuerte y 
aproximada. De hecho, si $A=\{0\}$, entonces \eq{intro1} corresponde
a un tipo de midconvexidad aproximada mientras que por el contrario
si $B=\{0\}$ entonces la misma inclusión corresponde a un tipo de 
midconvexidad fuerte. Por otra parte, podemos decir que en este
trabajo no se demuestra que la inclusión \eq{intro2} es óptima
y por lo tanto queda como problema abierto.

Debido a la observación hecha previamente con respecto a la naturaleza
de las multifunciones convexas y cóncavas, estableceremos también, 
un Teorema de tipo Bernstein--Doetsch para multifunciones midcóncavas. Los resultados principales
de esta investigación que  serán presentados en el Capítulo 4 de esta monografía,
se pueden encontrar en \cite{GonNikPalRoa}.